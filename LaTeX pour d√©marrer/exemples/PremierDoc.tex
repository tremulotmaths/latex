\documentclass[12pt,french]{article}
\usepackage[utf8x]{inputenc}
\usepackage[T1]{fontenc}
\usepackage[upright]{fourier}
\usepackage[a4paper,margin=2cm]{geometry}
\usepackage{mathtools,amssymb}
\usepackage{babel}

\begin{document}
\textbf{Définition 1.} Pour tous réels $a$, $b$ et $c \neq 0$, on a :\par
$\dfrac{a}{c}+\dfrac{b}{c}=\dfrac{a+b}{c}$. % Facile !

\textbf{Définition 2.} Soit $x \geq 0$ et $A \geq 0$ :
$\sqrt{x}=A \Leftrightarrow A^2 = x$. % Evident !


$\sqrt[3]{x} =A\Leftrightarrow A^3 = x$. % marche aussi avec x < 0 !

Évidemment,
ce
n'est
pas                bien                  compliqué.
\end{document}
C'est vraiment trop bien !! 