\chapter{Arbres pondérés}

On a vu dans le chapitre \ref{graph} un exemple pour obtenir des arbres pondérés.

En utilisant Pstricks, on obtient une syntaxe souvent plus simple pour obtenir de tels arbres.

En voici quelques exemples :

\VerbatimInput[label={[Arbre 2x2]},gobble=0]{exemples/arbre1.tex}

Ce \jargon{code source} donne :

\psset{nodesep=3mm,levelsep=30mm,treesep=15mm}
\begin{center}
\pstree[treemode=R]{\TR{}}
{
   \pstree{\TR{$A$}\taput{$0,6$}}
      {
      \TR{$F$} \taput{$0,2$}
      \TR{$\overline{F}$} \tbput{$0,8$}
      }

   \pstree{\TR{$B$} \tbput{$0,4$}}
     {
     \TR{$F$} \taput{$0,4$}
     \TR{$\overline{F}$} \tbput{$0,6$}
     }
}
\end{center}


\VerbatimInput[label={[Arbre 2x3]},gobble=0]{exemples/arbre2.tex}

Ce \jargon{code source} donne :

\psset{nodesep=3mm,levelsep=30mm,treesep=15mm}
\begin{center}
\pstree[treemode=R]{\TR{}}
{
  \pstree{\TR{$G$}\taput{$0,3$}}
     {
     \TR{$A$} \taput{$0,1$}
     \TR{$B$} \tbput{$0,3$}
     \TR{$C$} \tbput{$0,6$}
     }
  \pstree{\TR{$\overline{G}$}\tbput{$0,7$}}
     {
     \TR{$A$} \taput{$0,1$}
     \TR{$B$} \tbput{$0,3$}
     \TR{$C$} \tbput{$0,6$}
     }
}
\end{center}


\VerbatimInput[label={[Arbre 3x2]},gobble=0]{exemples/arbre3.tex}

Ce \jargon{code source} donne :

\input{exemples/arbre3}


\VerbatimInput[label={[Arbres non symétriques]},gobble=0]{exemples/arbre4.tex}

Ce \jargon{code source} donne :

\begin{center}
\pstree[treemode=R]{\TR{}}
{
   \pstree{\TR{$C$}\taput{$0,3$}}
      {
      \TR{$B$} \taput{$0,2$}
      \TR{$\overline{B}$} \tbput{$0,8$}
      }
   \TR{$\overline{C}$} \tbput{$0,7$}
}
\end{center}


\begin{center}
\pstree[treemode=R]{\TR{}}
{
   \TR{$\overline{A}$} \taput{$0,7$}   
   \pstree{\TR{$B$}\tbput{$0,3$}}
      {
      \TR{$A$} \taput{$0,7$}
      \pstree{\TR{$B$}\tbput{$0,3$}}
         {
         \TR{$A$} \taput{$0,7$}
         \TR{$B$} \tbput{$0,3$}        
         }
      }
}
\end{center}

\VerbatimInput[label={[Arbre avec des points]},gobble=0]{exemples/arbre5.tex}

Ce \jargon{code source} donne :

\psset{nodesep=0mm,levelsep=20mm,treesep=10mm}
 \pstree[treemode=R]{\Tdot}
 {
 \pstree
 {\Tdot~[tnpos=a]{$N$}\taput{\small ${\blue 0,64}$}}
 {
 \Tdot~[tnpos=r]{$T$}\taput{\small ${\red 0,35}$}
 \Tdot~[tnpos=r]{$\overline{T}$}\tbput{\small ${\blue 0,65}$}
 }
 \pstree
 {\Tdot~[tnpos=a]{$\overline{N}$}\tbput{\small ${\red 0,36}$}}
 {
 \Tdot~[tnpos=r]{$T$ {\blue ---> 0,27} }\taput{\small ${\red 0,75}$}
 \Tdot~[tnpos=r]{$\overline{T}$}\tbput{\small ${\red 0,25}$}
 }
 }-



