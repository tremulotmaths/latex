\chapter{Premier document}

\section{Mon premier document \LaTeX{}}
\subsection{\'Ecrire un fichier source}

Dans la pratique, ce que l'on écrit avec \jargon{Texmaker} se nomme le \jargon{fichier source} (ou encore le \jargon{code source}). Ce fichier est compilé par \LaTeX{} (plusieurs fois si nécessaire) et les différentes lignes de code sont interprétées pour obtenir en bout de chaîne le document final, celui qui sera imprimé.\medskip

Voici un premier document que vous pouvez taper dans votre éditeur, enregistrer, puis compiler (touche \touche{F1} sur \jargon{TeXmaker}).

\VerbatimInput[label={[Premier document]},gobble=0]{exemples/PremierDoc.tex}

\begin{info}
En règle générale, dans les noms des documents, on évitera d'utiliser des espaces et des lettres accentuées.
\end{info}


\subsection{Premières remarques}

\begin{enumerate}
    \item Le symbole \ordi{\$} sert à ouvrir et fermer le mode mathématiques (qui permet d'écrire des formules mathématiques).
    \item Le symbole \ordi{\%} permet d'écrire des commentaires qui n'apparaîtront pas dans le document compilé.
    \item La commande \verb!\par! de la ligne 10 sert à effectuer un saut de ligne dans le document compilé.
    
    Une ligne vide a le même effet.
    
    Personnellement, je préfère la deuxième solution qui me permet d'aérer mon \jargon{code source}.
    
    \item Un simple changement de ligne dans le \jargon{code source} ne sert qu'à aérer le \jargon{code source} et n'a aucun effet sur le document final.
    \item Dans le \jargon{code source}, lorsque plusieurs espaces séparent deux mots, \LaTeX{} n'en prend qu'un seul en compte.
\end{enumerate}

\section{Explication rapide du code source}
\subsection{Structure du code source}

Le code source est divisé en plusieurs parties :
\begin{description}
    \item[La définition du document :] la première ligne permet de déterminer quel type de document est réalisé : on parle de la \jargon{classe} du document (d'où le nom \verb!documentclass!). 
    
    Ici, il s'agit d'un document de type \verb!article!. Les \jargon{options globales} sont également déterminées à ce moment : elles sont valables pour tout le document sauf indication contraire.
    \item[Le préambule :] il s'agit des lignes entre \textbackslash\verb!documentclass! et \verb!\begin{document}! (lignes 2 à 8). 
    
    Le préambule contient tous les package(s) utilisés ainsi que différentes \jargon{commandes} définies par l'utilisateur ou spécifiquement utilisées dans le préambule.
    \item[Le corps du document :] situé entre \verb!\begin{document}! et \verb!\end{document}!,
        il s'agit du contenu même du document qui sera alors formaté en fonction du contenu du préambule et des commandes utilisées dans le texte.
    \item[Après :] les lignes après \verb!\end{document}!
        ne sont pas interprétées par \LaTeX{} et peuvent donc contenir ce que l'on veut : des commentaires, des notes, des parties mises de côté\dots
\end{description}

\subsection{Explication du préambule}

\begin{description}
    \item[] \verb!\documentclass[12pt,french]{article}! : cette commande indique que le document est de classe \verb!article! et sera donc assez court. Il existe, en comparaison, la classe \verb!book! pour écrire des documents plus longs. Bien d'autres classes existent (\verb!letter!, \verb!beamer!,\dots).\par
        Ce document respectera la typographie française et la taille des fontes sera de \verb!12pt! (\verb!10pt! étant la taille par défaut).
    \item[] \verb!\usepackage[utf8x]{inputenc}! : cette commande permet de charger le package \verb!inputenc! avec l'option \verb!utf8x!. Nous n'expliquerons rien en détails ici mais cela gère le \jargon{codage} d'entrée des caractères du \jargon{code source} (d'où l'intérêt d'avoir configuré \jargon{Texmaker} en \ordi{utf8}).
    \item[] \verb!\usepackage[T1]{fontenc}! : cette commande permet de charger le package \verb!fontenc! avec l'option \verb!T1! permettant de gérer, entre autre, les caractères accentués et notamment les \og copiés-collés \fg{} à partir de fichiers \bsc{pdf}.
    \item[] \verb!\usepackage[upright]{fourier}! : charge le package \verb!fourier! avec son option \verb!upright! qui permet d'avoir d'avoir les majuscules droites dans les formules mathématiques.
   % \item[] \verb!\usepackage{kpfonts}! : charge un ensemble de fontes de la police \jargon{Kp-Fonts}. D'autres sont disponibles comme par exemple \verb!mathpazo! pour la police \jargon{Palatino} ou bien \verb!mathptmx! pour la police \jargon{Times}.
    \item[] \verb!\usepackage[a4paper,margin=2cm]{geometry}! : charge le package \verb!geometry! avec différentes options de mise en page.
    \item[] \verb!\usepackage{mathtools,amssymb}! : charge le package \verb!mathtools! qui est essentiel dans un document destiné à composer des textes scientifiques, avec un formalisme et donc une mise en page particulière. Le package \verb!amssymb! regroupe quant à lui quantité de symboles utilisés notamment en mathématiques et en physiques.
    \item[] \verb!\usepackage{babel}! : obligatoirement le dernier de la liste. Le package \verb!babel! permet d'assurer au rédacteur que le texte sera composé en respectant les usages propres à la langue de composition du document (ici en français). La langue peut être spécifiée en option de ce package en écrivant \verb!\usepackage[french]{babel}! mais il est préférable d'indiquer l'option de langue avec la \jargon{classe} du document (comme nous l'avons fait). Ainsi, cette langue sera utilisée de façon globale par tous les package(s) en ayant besoin.
\end{description}

\section{Commandes}
\subsection{Arguments d'une commande}
Les \jargon{commandes} permettent de structurer et de mettre en forme le document. Elles sont reconnaissables car elles commencent par le caractère \textbackslash{} suivi du nom de la commande (plus ou moins explicite). La commande se termine par tout caractère autre qu'une lettre (accolade, crochet, chiffre, espace, ponctuation\dots).\par

Différents types de commandes existent :
\begin{description}
    \item[Sans \jargon{paramètres} :] elles exécutent simplement une action : \verb!\neq!, \verb!\par!, \verb!\Leftrightarrow!
    \item[Avec \jargon{paramètres} :] il existe alors deux types de \jargon{paramètres}. Ils peuvent être :
        \begin{description}
            \item[Obligatoires :] ceux-là sont notés entre accolades et il peut y avoir plusieurs paramètres par commande (une paire d'accolades pour chacun d'entre eux). Par exemple, la commande \verb!\textbf! ne possède qu'un paramètre alors que la commande \verb!\frac! en possède deux. Il a été dit que la commande s'arrêtait par tout caractère autre qu'une lettre. 
            
Dans ce cas, il n'est pas toujours nécessaire d'utiliser les accolades. 

Ainsi, \verb!\frac{2}{3}! $\qLRq$ \verb!\frac23! $\qLRq$ \verb!\frac 2 3 ! : cela donne la fraction $\textstyle \frac23$. 

En revanche, écrire \verb!\fracab! ne donnera pas $\textstyle \frac a b$, car \LaTeX{} ne saura pas interpréter la commande \verb!\fracab! (à moins que vous ayez créé une commande portant ce nom).

            \item[Facultatifs :] ceux-là sont notés entre crochets avant le premier argument obligatoire (qui n'existe pas toujours d'ailleurs). Les \jargon{arguments} facultatifs ou \jargon{optionnels} permettent de modifier localement l'action d'une commande. Par exemple \verb!\sqrt[3]{x}!.
        \end{description}
\end{description}

En résumé, une commande peut avoir une des trois formes suivantes :

	\verb!\commande! \par
	\verb!\commande!\{<\verb!argument 1!>\}\{<\verb!argument 2!>\}\dots\par
	\verb!\commande![<\verb!argument optionnel!>]\{<\verb!argument 1!>\}\{<\verb!argument 2!>\}\dots


\begin{info}
    Les majuscules dans les noms de commandes sont importantes : ainsi \ordi{\textbackslash frac} n'est pas identique à \ordi{\textbackslash Frac}.
\end{info}

\subsection{Commandes semi-globales}

Les \jargon{commandes locales} permettent de modifier l'aspect du texte de façon locale.\par
Les \jargon{commandes semi-globales} n'ont pas d'argument et modifient tout le texte qui suit jusqu'à ce qu'une autre commande semi-globale ou qu'une commande locale ne modifie encore la mise en forme. On peut limiter l'action des commandes semi-globales à l'aide d'une paire d'accolades englobant le texte mais également la commande. Autrement dit :
\begin{description}
    \item[Commande locale :] \verb!\commande!\{<du texte>\}
    \item[Commande semi-globale :] \verb!{!\verb!\commande! <du texte>\verb!}!
\end{description}

La plupart du temps, on utilise les commandes locales sur des textes courts sans changement de paragraphes alors que les commandes semi-globales sont appliquées à des textes plus longs et acceptent les changements de paragraphes. Voici un exemple :\bigskip

\begin{SideBySideExample}
    Voici un \textbf{exemple :}
    tout va bien mais on peut vouloir
    continuer avec un texte
    \tiny plus petit jusqu'au bout.\par
    {\bfseries
        Cela est important !\par
        Comprenez-vous ?
    }
    C'est bien !
\end{SideBySideExample}


\subsection{Les packages}

Les package(s) sont chargés à l'aide de la commande \verb!\usepackage!. Chaque package est une \jargon{extension} de \LaTeX{} et c'est la création de ces milliers de package(s) qui fait que \LaTeX{} évolue de jours en jours. Le nom du package est l'argument obligatoire et les options sont spécifiées entre crochets si nécessaire :
\begin{center}
    \verb!\usepackage![<\verb!options!>]\{<\verb!nom du package!>\}
\end{center}
Si plusieurs package(s) doivent être appelés sans option particulière (ou avec la même option), alors on peut les lister au sein de la même commande \verb!\usepackage!. C'est le cas par exemple de la ligne 4 : \verb!\usepackage{mathtools,amssymb}! fait appel à deux package(s) liés aux mathématiques.

\begin{info}
    \textbf{Rappel :} Le package \verb!babel! est le package qui permet la gestion de la langue dans laquelle est écrite le document. C'est d'ailleurs la fonctionnalité de \verb!french! signalée dans la \jargon{commande} \textbackslash\verb!documentclass!.\par
    Le package \verb!babel! doit être, en général, le dernier de la liste des package(s) utilisés.
\end{info}

\section{Les caractères spéciaux}

On a vu que certains caractères avaient une utilisation spécifique dans le code source : c'est le cas par exemple des caractères \ordi{\%} et \ordi{\$} qui permettent respectivement d'entrer un commentaire et une formule mathématique. Comment faire cependant pour écrire -20\% sur une veste à 50\$ ou bien $S = \{ 1 ; 3 \}$ ?\par
Le \og texte\fg{} ci-dessous donne les caractères spéciaux réservés par \LaTeX{} et la syntaxe nécessaire pour les utiliser dans un texte classique :

Voici quelques exemples de caractères spéciaux, réservés par \LaTeX{} :

\verb!\{! et \verb!\}!, utiles pour les ensembles, donnent les accolades ouvrante et fermante,

\verb!%! qui sert à écrire des commentaires dans le code source

\verb!$! qui sert à ouvrir ou fermer le mode \og mathématiques\fg{}

\verb!^! qui sert pour les exposants

\verb!_! qui sert pour les indices

\verb!&! qui sert pour les tableaux (séparateur de colonnes)

\verb!#! qui sert lorsque l'on définit des commandes personnelles

\verb!~! qui produit un espace insécable


\begin{info}
	Le caractère \verb!@! n'est pas un caractère spécial et permet d'obtenir simplement @.\par
	Cependant, dans certains cas, il est utilisé de façon particulière : pour les tableaux, les commandes personnelles...
\end{info}

