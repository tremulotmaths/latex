\documentclass[french]{beamer}
\usepackage{diapo-formation-tice}

% Redéfinition ici pour enlever le \smaller.
\renewcommand{\verbtexte}[1]{%
  \begingroup%
  \bfseries%
  \color{teal}
  %\smaller%
  \texttt{#1}%
  \endgroup%
}


%%% url faisant office de lien :
%\url{http://www.google.fr}
%%% Lien mais ce n'est pas forcément l'adresse qu'on affiche :
%\href{http://www.google.fr}{Google}
%%% Texte verbatim (en fait les caractères spéciaux doivent être échappés).
% \verbtexte{blabla}

\newcommand{\email}[1]{\href{mailto:#1}{\smaller\nolinkurl{#1}}}

\title[J3 d'accomp. des PR]{Journée 3 d'accompagnement des personnes ressources et référents numériques}
\author{Les conseillers Tice du 91} 
\date{Mardi 21 mai 2013}




\begin{document}%%%%%%%%%%%%%%%%%%%%%%%%%%%%%%%%%%%%%%%%%%%%%%%%%%%%

\begin{frame}
\titlepage
\end{frame}


\section{Bascule de l'Etherpad}

\begin{frame}
\frametitle{L'Etherpad du CRDP amené à être remplacé...}
%\stepcounter{beamerpauses}

Si vous utilisez régulièrement Etherpad,
il faudra sans doute basculer dans un avenir proche vers l'Etherpad hébergé par la
DSI du rectorat, sur son \alert{Édu portail pédagogique}.
\bigskip

%\onslide<+->
Pour l'instant, il n'existe pas encore de plateforme de production.

\end{frame}




\section{L'IFIC}


\begin{frame}%[label=main]
\frametitle{\href{http://www.premiumorange.com/uasenver/elus/ific/cr_ific_25_04_2013.pdf}{L'IFIC}
(indemnité pour fonctions d'intérêt collectif)}
%\stepcounter{beamerpauses}


Une somme variant entre 400 et 2400 euros est attribuée à certains établissements.
La répartition de la somme est votée au CA.

\begin{itemize}%[<+->]
\item Au niveau des Tice cette somme est réservée aux référents
pour les usages pédagogiques numériques.
\item En principe, il ne faut pas toucher d'heures supplémentaires
pour prétendre à l'IFIC.
\end{itemize}


\end{frame}




\section{Le PAF}

\begin{frame}%[label=main]
\frametitle{Les nouveautés du PAF 2013-2014}

\begin{onlyenv}<+>
Un \alert{stage tablette} est créé sur 2 jours. C'est une formation
collective en établissement.
\bigskip

Pour les collèges ayant été dotés en tablettes par le conseil général,
en amont du stage (en principe peu de temps
après la livraison des tablettes), une prise en main dans le collège sera assurée par
le CDDP91 sur une demi journée pour quelques professeurs désignés par le chef
d'établissement.
\end{onlyenv}

\begin{onlyenv}<+>
Le module \og généralités sur les réseaux \fg{} disparaît
et est \alert{remplacé par 3 petits modules} hybrides
(dont 3 jours de présentiel chacun) :

\begin{enumerate}
\item Le rôle du référent numérique;
\item Usages citoyens et gestion des services en ligne;
\item Architecture et usage du réseau de l'EPLE.
\end{enumerate}

Il est possible de ne s'inscrire qu'à une partie des 3 modules
durant une année.
\medskip

Plus d'aménagement d'emploi du temps obligatoire pour suivre ces modules.

\end{onlyenv}

\begin{onlyenv}<+>
Dans la famille des formations individuelles \og Culture numérique pour enseigner \fg{},
deux nouveautés (entre autres) :
\begin{itemize}
\item Utiliser la messagerie et les services académiques;
\item Droits et usages numériques : études de cas.
\end{itemize}
\bigskip

Le dispositif \og Culture numérique pour enseigner \fg{}
est \alert{destiné à tout le personnel de l'académie} (pas seulement personnes
ressources et aux référents numériques).
\end{onlyenv}


\end{frame}


\section{Le projet Numéritab91}


\begin{frame}%[label=main]
\frametitle{Le projet Numéritab91 toujours d'actualité}
%\stepcounter{beamerpauses}

Il reste encore \alert{20 lots de 30 tablettes} non distribués
par le conseil général. Les établissements non retenus de la vague 2
sont prioritaires, ne pas hésiter à renvoyer une nouvelle version du projet.
Il doit :

\begin{itemize}%[<+->]
\item être pluridisciplinaire;
\item être essentiellement axé sur les classes de sixième;
\item s'intégrer dans le socle commum;
\item présenter quelques indicateurs permettant d'évaluer son efficacité.
\end{itemize}


\end{frame}



\section{Les VPI}

\begin{frame}%[label=main]
\frametitle{Les VPI}

\begin{itemize}
\item Pour les lycées : campagne de VPI au niveau de la région.
\item Pour les collèges : le conseil général équipe en VPI
numéro 5 et 6 sur projet. Bien joindre le projet à la demande
sur Equip'clg, en précisant la ou les matières concernées et surtout la salle ainsi que le besoin,
si nécessaire, d'un tableau blanc.
\end{itemize}


\end{frame}



\section[SE3]{SE3, migration vers Squeeze}

\begin{frame}%[label=main]
\frametitle{SE3, migration vers Squeeze}

La migration Squeeze SE3 est maintenant envisageable voire conseillée
(choisir un moment opportun car ça n'est quand même pas une opération anodine).

La procédure est décrite \href{http://wwdeb.crdp.ac-caen.fr/mediase3/index.php/Faqlenny2squeeze}{ici}.

La version Lenny de SE3 n'est plus mise à jour.

\end{frame}



\section[La Mission Tice recrute]{Devenir formateur ou/et conseiller Tice}

\begin{frame}%[label=main]
\frametitle{Toi aussi, deviens formateur ou/et conseiller Tice}

La Mission Tice recherche des formateurs et/ou conseillers Tice de bassin.
La procédure à suivre est \href{http://www.tice.ac-versailles.fr/Recrutement-2013}{là}
(bien que la date butoir indiquée soit dépassée, l'annonce est toujours d'actualité).
%Certains ont-ils postulés ?

\end{frame}



\section[Changements de PR]{Changements de personnes ressources ?}


\begin{frame}%[label=main]
\frametitle{Changements de personnes ressources ?}

N'hésitez pas à \alert{nous informer par mail de tout changement au
niveau des personnes ressources et référents numériques} dans
votre établissement afin que nous puissions mettre à jour les
liste de diffusion de bassin.




\end{frame}



\section{Fin}

\begin{frame}%[label=main]
\frametitle{Bonne fin d'année scolaire !}

Les conseillers Tice de bassin de l'Essonne :
\begin{itemize}
\item Jérôme Beaudet (\email{jerome.beaudet@crdp.ac-versailles.fr})\\
Conseiller Tice du bassin de Savigny-sur-Orge

\item Nour-Eddine El Yazghi (\email{elyazghi@crdp.ac-versailles.fr})\\
Raoul Pernot (\email{raoul.pernot@crdp.ac-versailles.fr}) \\
Conseillers Tice du bassin d’Étampes

\item François Lafont (\email{francois.lafont@crdp.ac-versailles.fr})\\
Conseiller Tice du bassin de Massy

\item Didier Percevault ({\smaller\email{didier.percevault@crdp.ac-versailles.fr}})\\
Conseiller Tice du bassin de Montgeron

\item Cyrille Bertrand (\email{cyrille.bertrand@crdp.ac-versailles.fr})\\
Conseiller Tice du bassin d’Évry
\end{itemize}

\end{frame}





\end{document}
