\documentclass[12pt]{article}

\usepackage{fourier}
\usepackage[latin1]{inputenc}    
\usepackage[T1]{fontenc}
\usepackage[francais]{babel}
\usepackage{enumerate} % pour pouvoir changer la num�rotation des listes
\usepackage[scaled=0.875]{helvet}
\renewcommand{\ttdefault}{lmtt}
\usepackage{amsmath,amssymb,makeidx}
\usepackage{fancybox}
\usepackage{fancyhdr}

\usepackage{graphicx} %inclure des graphiques
\usepackage{picins} %grahique avec texte entourant avec la commande \parpic[r]{\includegraphics[width=5cm]{ex41p218.eps}} par exemple

%liens hypertexte
\usepackage[colorlinks=true,linkcolor=blue,filecolor=blue,urlcolor=blue]{hyperref}


\usepackage{tabularx}
\usepackage[normalem]{ulem}
\usepackage{soul} %pour utiliser \ul pour souligner
\usepackage{pifont}
\usepackage{slashbox}
\usepackage{textcomp}
\usepackage{pst-plot}

\usepackage[np]{numprint}
\usepackage[top=1cm, bottom=3cm, left=2.5cm, right=2.5cm]{geometry}


%diminuer la taille des caract�res dans un tableau
\let\oldtabular=\tabular
\def\tabular{\small\oldtabular}

\newcommand{\euro}{\texteuro{}}

%simplification notation norme \norme{}
\newcommand{\norme}[1]{\left\Vert #1\right\Vert}

%Ensembles de nombres
\newcommand{\R}{\mathbb{R}}
\newcommand{\N}{\mathbb{N}}
\newcommand{\D}{\mathbb{D}}
\newcommand{\Z}{\mathbb{Z}}
\newcommand{\Q}{\mathbb{Q}}
\newcommand{\C}{\mathbb{C}}


%simplification de la notation de vecteur \vect{}
\newcommand{\vect}[1]{\mathchoice%
{\overrightarrow{\displaystyle\mathstrut#1\,\,}}%
{\overrightarrow{\textstyle\mathstrut#1\,\,}}%
{\overrightarrow{\scriptstyle\mathstrut#1\,\,}}%
{\overrightarrow{\scriptscriptstyle\mathstrut#1\,\,}}}

\renewcommand{\theenumi}{\textbf{\arabic{enumi}}}
\renewcommand{\labelenumi}{\textbf{\theenumi.}}
\renewcommand{\theenumii}{\textbf{\alph{enumii}}}
\renewcommand{\labelenumii}{\textbf{\theenumii.}}

%Rep�res
\def\Oij{$\left(\text{O}~;~\vect{\imath},~\vect{\jmath}\right)$}
\def\Oijk{$\left(\text{O}~;~\vect{\imath},~ \vect{\jmath},~ \vect{k}\right)$}
\def\Ouv{$\left(\text{O}~;~\vect{u},~\vect{v}\right)$}

%\setlength{\voffset}{-1,5cm}
%\setlength{\textheight}{23,5cm}

\begin{document}

\setlength\parindent{0mm}
\rhead{\small 19/07/12}
\chead{\small Liens utiles pour \LaTeX}
%\lhead{\small Pr�paration de l'ann�e 2012-2013}
%\lfoot{\small{Br�tigny sur Orge}}
%\rfoot{\small{06/03/12}}
\renewcommand \footrulewidth{.2pt}%
\pagestyle{fancy}

%\thispagestyle{empty} %annuler le style pour la page en cours, soit la premi�re page, ici

%%%%%%%%%%%%%%%%%%%%%%%Page 1%%%%%%%%%%%%%%%%%%%%%%%%%%%%%%%%%%%%%%%%

Voici quelques liens utiles pour certaines fonctionnalit�es de \LaTeX :


\begin{enumerate}

\item Un site � conna�tre : \url{http://fr.wikibooks.org/wiki/LaTeX}

\item Pour ins�rer des figures (ou images), avec ou sans bord, avec ou sans l�gende, avec le texte entourant, etc... : 

\url{http://en.wikibooks.org/wiki/LaTeX/Floats,_Figures_and_Captions} 

et \url{http://www.grappa.univ-lille3.fr/FAQ-LaTeX/8.17.html}.

\item Package hyperref : \href{http://www-irem.univ-fcomte.fr/download/irem/document/stages/hyperref-intro.pdf}{http://www-irem.univ-fcomte.fr/...}

\item Ecrire des QCM avec Alterqcm : \url{http://mathsp.tuxfamily.org/spip.php?article64}

\item Installation manuelle d'un package sous MikTeX : \href{http://forum.mathematex.net/latex-f6/installation-manuelle-d-un-package-sous-miktex-t2121.html}{http://forum.mathematex.net/latex-f6/...}

\item Autour des tableaux

		\begin{itemize}
		\item \url{http://latex.developpez.com/faq/?page=LATEX_TABLE#LATEX_TABLE_FONT}
		\item \url{http://fr.wikibooks.org/wiki/LaTeX/Faire_des_tableaux}
		\item \url{http://fr.wikibooks.org/wiki/Programmation_LaTeX/Tableaux}
		\end{itemize}
		
\item Pour les tableaux de signes et de variations, utiliser le logiciel \href{http://www.xm1math.net/pstplus/index.html}{PstPlus} et exporter le r�sultat sous forme d'image.

\item Param�trages du style de pages avec \href{http://pfercour.free.fr/stock/fancyhdr_fr.pdf}{fancyhdr}

\end{enumerate}

%%%%%%%%%%%%%%%%%%%%%%%%%%%%%%%%%%%%%%%%%%%%%%%%%%%%%%%%%%%%
\end{document}