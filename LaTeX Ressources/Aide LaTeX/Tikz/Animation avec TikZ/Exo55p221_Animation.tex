\documentclass[10pt,french]{article}
\input preambule_2013
\pagestyle{empty}

\usepackage{animate}

\begin{document}

\newcounter{pos} % création du compteur "pos" utiliser dans les coordonnées du point M. Le compteur augmentera au fur et à mesure ce qui créera le déplacement du point M dans l'animation.
\setcounter{pos}{0} % initialisation du compteur

\begin{center}
\begin{animateinline}[loop, poster = first, controls,palindrome]{5}%
% loop : redémarre l'animation automatiquement
% poster = first : affiche la première position lorsque l'animation est à l'arrêt
% controls : affiche les boutons de contrôle. Sans bouton, il suffit de cliquer sur la figure pour lancer l'animation
% palindrome : l'animation commence la boucle du début à la fin puis reprend de la fin vers le début
% {5} : 5 images par seconde
%
\whiledo{\thepos<61}{
%
    \begin{tikzpicture}
        \coordinate (A) at (0,0); \draw (A) node [below left] {$A$};
        \coordinate (I) at (1,0);
        \coordinate (J) at (0,1);
        \coordinate (B) at (6,0); \draw (B) node [below right] {$B$};
        \coordinate (C) at (6,6); \draw (C) node [above right] {$C$};
        \coordinate (D) at (0,6); \draw (D) node [above left] {$D$};
        \coordinate (O) at (3,3); %milieu du segment [AC]
        \draw (O) node {$+$} node[above left] {$O$};
        \draw (A) -- (B) -- (C) -- (D) -- cycle;
        \draw (A) -- (C);
         \coordinate (M) at ({\thepos/10},{\thepos/10}); %point M sur le segment [AC]
        \draw (M) node {$\times$} node[above left] {$M$};
        \draw[fill=red!10] let \p1=(M) in (\x1,0) -- (O) -- (6,\y1) -- (B) -- cycle;
        \draw[dashed]  let \p1=(M) in (M) -- (\x1,0) node[below] {$E$} node {$+$};
        \draw[dashed]  let \p1=(M) in (M) -- (6,\y1) node[right] {$F$} node {$+$};
    \end{tikzpicture}
    %
    \stepcounter{pos}
    \ifthenelse{\thepos<61}{
            \newframe
    }{
            \end{animateinline}
    }
}
\end{center}


\end{document} 