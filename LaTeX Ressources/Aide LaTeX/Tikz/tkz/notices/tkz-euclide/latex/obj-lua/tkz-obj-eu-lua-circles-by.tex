% tkz-obj-eu-circles.tex
% Copyright 2024  Alain Matthes
% This work may be distributed and/or modified under the
% conditions of the LaTeX Project Public License, either version 1.3
% of this license or (at your option) any later version.
% The latest version of this license is in
%   http://www.latex-project.org/lppl.txt
% and version 1.3 or later is part of all distributions of LaTeX
% version 2005/12/01 or later.
% This work has the LPPL maintenance status “maintained”.
% The Current Maintainer of this work is Alain Matthes.

\def\fileversion{6.00}
\def\filedate{2024/04/19} 
\typeout{2024/04/19 6.00 tkz-obj-eu-circles.tex} 
\makeatletter
%<--------------------------------------------------------------------------–>
%            tkzCircle center and one point 
%<--------------------------------------------------------------------------–>
% attention radius circle is defined by center and a point on the circle
% R defined by center and the value of the radius
% no need to define a circle with R tikz uses this method.
% through instead of radius
\def\tkz@numcby{0}
\pgfkeys{/tkzDefCircleBy/.cd,
  translation/.code args = {from #1 to #2} {             \def\tkz@numcby{0}
                                                         \def\tkzfrom{#1}
                                                         \def\tkzto{#2}},
  homothety/.code args = {center #1 ratio #2}{           \def\tkz@numcby{1}
                                                         \def\tkzcenter{#1}
                                                         \def\tkzratio{#2}},
  reflection/.code args = {over #1--#2}{                 \def\tkz@numcby{2}
                                                         \def\tkzdeb{#1}
                                                         \def\tkzfin{#2}},
  symmetry/.code args = {center #1}{                     \def\tkz@numcby{3}
                                                         \def\tkzcenter{#1}},
  rotation/.code args = {center #1 angle #2}{            \def\tkz@numcby{4}
                                                         \def\tkzcenter{#1}
                                                         \def\tkzangle{#2}},
  inversion/.code args={center #1 through #2}{           \def\tkz@numcby{5}
                                                         \def\tkzcenter{#1}
                                                         \def\tkzpoint{#2}},
  inversion negative/.code args = {center #1 through #2}{\def\tkz@numcby{6}
                                                         \def\tkzcenter{#1}
                                                         \def\tkzpoint{#2}}
      } 
\def\tkzDefCircleBy{\pgfutil@ifnextchar[{\tkz@DefCircleBy}{\tkz@DefCircleBy[]}}
\def\tkz@DefCircleBy[#1](#2){% 
\begingroup  
\pgfqkeys{/tkzDefCircleBy}{#1}  
\ifcase\tkz@numcby%
  \tkzDefCircleTranslation(#2)%
  \or% 1 
  \tkzDefCircleHomothety(#2)  
  \or% 2
  \tkzDefCircleReflection(#2) 
  \or% 3
  \tkzDefCircleSymmetry(#2) 
  \or% 4
  \tkzDefCircleRotation(#2) 
  \or% 5
  \tkzDefInversionCircle(#2,\tkzcenter,\tkzpoint) 
  \or% 6
  \tkzDefInversionNegativeCircle(#2,\tkzcenter,\tkzpoint) 
\fi    
\endgroup
}
%<--------------------------------------------------------------------------–>
\def\tkzDefCircleTranslation(#1,#2){%
\begingroup 
   \tkzUTranslation(\tkzfrom,\tkzto)(#1)
   \pgfnodealias{tkzFirstPointResult}{tkzPointResult}
   \tkzUTranslation(\tkzfrom,\tkzto)(#2)
   \pgfnodealias{tkzSecondPointResult}{tkzPointResult}
\endgroup
}  
%<--------------------------------------------------------------------------–>
\def\tkzDefCircleHomothety(#1,#2){%
\begingroup 
   \tkzUHomo(\tkzcenter,\tkzratio)(#1)
   \pgfnodealias{tkzFirstPointResult}{tkzPointResult}
   \tkzUHomo(\tkzcenter,\tkzratio)(#2)
   \pgfnodealias{tkzSecondPointResult}{tkzPointResult}
\endgroup
}
%<--------------------------------------------------------------------------–>
\def\tkzDefCircleReflection(#1,#2){%
\begingroup 
   \tkzUSymOrth(\tkzdeb,\tkzfin)(#1) 
   \pgfnodealias{tkzFirstPointResult}{tkzPointResult}
   \tkzUSymOrth(\tkzdeb,\tkzfin)(#2) 
   \pgfnodealias{tkzSecondPointResult}{tkzPointResult}
\endgroup
}
%<--------------------------------------------------------------------------–>
\def\tkzDefCircleSymmetry(#1,#2){%
\begingroup 
   \tkzUCSym(\tkzcenter)(#1)
   \pgfnodealias{tkzFirstPointResult}{tkzPointResult}
   \tkzUCSym(\tkzcenter)(#2)
   \pgfnodealias{tkzSecondPointResult}{tkzPointResult}
\endgroup
}
%<--------------------------------------------------------------------------–>
\def\tkzDefCircleRotation(#1,#2){%
\begingroup 
   \tkzURotateAngle(\tkzcenter,\tkzangle)(#1)
    \pgfnodealias{tkzFirstPointResult}{tkzPointResult}
    \tkzURotateAngle(\tkzcenter,\tkzangle)(#2)
    \pgfnodealias{tkzSecondPointResult}{tkzPointResult}
\endgroup
}
%<--------------------------------------------------------------------------–>
% #3,#4 cercle d'inversion centre #3. through #4
% Si le cercle passe par le pôle l'image est une droite
\def\tkzDefInversionCircle(#1,#2,#3,#4){%  
\begingroup
    \tkz@@CalcLengthcm(#1,#2){tkz@lna}% 
    \tkz@@CalcLengthcm(#1,#3){tkz@lnb}% 
    \edef\tkzMathResult{\tkz@Dec{\tkz@Round{\tkz@Abs{\tkz@lnb - \tkz@lna}}{6}}}
    \ifdim\tkzMathResult pt < 0.00001 pt\relax%
    \tkzURotateAngle(#1,-90)(#2)
    \pgfnodealias{tkz@a}{tkzPointResult}
    \tkzUInversePoint(#3,#4)(tkz@a)
    \pgfnodealias{tkzFirstPointResult}{tkzPointResult}
    \tkzUInversePoint(#3,#4)(#2)
    \pgfnodealias{tkzSecondPointResult}{tkzPointResult}
     \else
    \tkzURotateAngle(#1,90)(#2)
    \pgfnodealias{tkz@a}{tkzPointResult}
    \tkzURotateAngle(#1,-90)(#2)
    \pgfnodealias{tkz@b}{tkzPointResult}
    \tkzUInversePoint(#3,#4)(tkz@a)
    \pgfnodealias{tkz@p1}{tkzPointResult}
    \tkzUInversePoint(#3,#4)(tkz@b)
    \pgfnodealias{tkz@p2}{tkzPointResult}
    \tkzUInversePoint(#3,#4)(#2)    
    \pgfnodealias{tkz@p3}{tkzPointResult}
    \tkzDefCircle[circum](tkz@p3,tkz@p1,tkz@p2)
    \fi
\endgroup
} 
%<--------------------------------------------------------------------------–>
\def\tkzDefInversionNegativeCircle(#1,#2,#3,#4){%
\begingroup
   \tkzDefInversionCircle(#1,#2,#3,#4)
   \tkzUCSym(\tkzcenter)(tkzFirstPointResult)
   \pgfnodealias{tkzFirstPointResult}{tkzPointResult}
   \tkzUCSym(\tkzcenter)(tkzSecondPointResult)
   \pgfnodealias{tkzSecondPointResult}{tkzPointResult}
\endgroup
}
%<--------------------------------------------------------------------------–>
%              End Def Circle
%<--------------------------------------------------------------------------–>  
\makeatother 
\endinput