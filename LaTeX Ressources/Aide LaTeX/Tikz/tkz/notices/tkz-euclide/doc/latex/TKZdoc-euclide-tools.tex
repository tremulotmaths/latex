\section{Miscellaneous tools and mathematical tools}
\subsection{Duplicate a segment} 
This involves constructing a segment on a given half-line of the same length as a given segment.

\begin{NewMacroBox}{tkzDuplicateSegment}{\parg{pt1,pt2}\parg{pt3,pt4}\marg{pt5}}%
This involves creating a segment on a given half-line of the same length as a given segment . It is in fact the definition of a point.
\tkzcname{tkzDuplicateSegment} is the new name of \tkzcname{tkzDuplicateLen}.

\medskip  
\begin{tabular}{lll}%
\toprule
arguments             & example & explanation                         \\ 

\midrule
\TAline{(pt1,pt2)(pt3,pt4)\{pt5\}} {\tkzcname{tkzDuplicateSegment}(A,B)(E,F)\{C\}}{AC=EF and $C \in [AB)$} \\  
\bottomrule
\end{tabular}

\medskip
\emph{The macro \tkzcname{tkzDuplicateLength} is identical to this one. }
\end{NewMacroBox}

\subsubsection{Use of\tkzcname{tkzDuplicateSegment}} 

\begin{tkzexample}[latex=6cm,small]
\begin{tikzpicture}[scale=.5]
 \tkzDefPoints{0/0/A,2/-3/B,2/5/C}
 \tkzDuplicateSegment(A,B)(A,C)  
 \tkzGetPoint{D}
 \tkzDrawSegments[new](A,B A,C)
 \tkzDrawSegment[teal](A,D)
 \tkzDrawPoints[new](A,B,C,D) 
 \tkzLabelPoints[above right=3pt](A,B,C,D)
\end{tikzpicture} 
\end{tkzexample} 

\subsubsection{Proportion of gold with \tkzcname{tkzDuplicateSegment}} 
\begin{tkzexample}[latex=7cm,small]
\begin{tikzpicture}[rotate=-90,scale=.4]
 \tkzDefPoints{0/0/A,10/0/B}
 \tkzDefMidPoint(A,B)   
 \tkzGetPoint{I}
 \tkzDefPointWith[orthogonal,K=-.75](B,A)
 \tkzGetPoint{C}
 \tkzInterLC(B,C)(B,I)  \tkzGetSecondPoint{D}
 \tkzDuplicateSegment(B,D)(D,A) \tkzGetPoint{E}
 \tkzInterLC(A,B)(A,E)   \tkzGetPoints{N}{M}
 \tkzDrawArc[orange,delta=10](D,E)(B)
 \tkzDrawArc[orange,delta=10](A,M)(E)
 \tkzDrawLines(A,B B,C A,D)
 \tkzDrawArc[orange,delta=10](B,D)(I)
 \tkzDrawPoints(A,B,D,C,M,I)
 \tkzLabelPoints[below left](A,B,D,C,M,I)
\end{tikzpicture}
\end{tkzexample}

\subsubsection{Golden triangle or sublime triangle} 
\begin{tkzexample}[latex=7cm,small]
\begin{tikzpicture}[scale=.75] 
  \tkzDefPoints{0/0/A,5/0/C,0/5/B} 
  \tkzDefMidPoint(A,C)\tkzGetPoint{H} 
  \tkzDuplicateSegment(H,B)(H,A)\tkzGetPoint{D} 
  \tkzDuplicateSegment(A,D)(A,B)\tkzGetPoint{E} 
  \tkzDuplicateSegment(A,D)(B,A)\tkzGetPoint{G} 
  \tkzInterCC(A,C)(B,G)\tkzGetSecondPoint{F}
  \tkzDrawLine(A,C)
  \tkzDrawArc(A,C)(B) 
  \begin{scope}[arc style/.style={color=gray,%
                               style=dashed}]
    \tkzDrawArc(H,B)(D) 
    \tkzDrawArc(A,D)(B) 
    \tkzDrawArc(B,G)(F) 
  \end{scope}
  \tkzDrawSegment[dashed](H,B) 
  \tkzCompass(B,F) 
  \tkzDrawPolygon[new](A,B,F)
  \tkzDrawPoints(A,...,H)
  \tkzLabelPoints[below left](A,...,H)
\end{tikzpicture}
\end{tkzexample}

\subsection{Segment length \tkzcname{tkzCalcLength}}
There's an option in \TIKZ\  named \tkzname{veclen}. This option
 is used to calculate AB if A and B are two points.

The only problem for me is that the version of \TIKZ\ is not accurate enough in some cases. My version uses the \tkzNamePack{xfp} package and is slower, but more accurate.

\begin{NewMacroBox}{tkzCalcLength}{\oarg{local options}\parg{pt1,pt2}}%
You can store the result with the macro \tkzcname{tkzGetLength} for example \tkzcname{tkzGetLength\{dAB\}} \\
defines the macro \tkzcname{dAB}.

\medskip
\begin{tabular}{lll}%
\toprule
arguments    & example & explanation       \\
\midrule
\TAline{(pt1,pt2)\{name of macro\}} {\tkzcname{tkzCalcLength}(A,B)}{\tkzcname{dAB} gives $AB$ in cm}
\bottomrule
\end{tabular}

\medskip
Only one option

\begin{tabular}{lll}%
   
\toprule
 options    & default & example       \\
\midrule
\TOline{cm}  {true}{\tkzcname{tkzCalcLength}(A,B) After \tkzcname{tkzGetLength\{dAB\}} \tkzcname{dAB} gives $AB$ in cm}
\end{tabular}
\end{NewMacroBox}

\subsubsection{Compass square construction}

\begin{tkzexample}[latex=7cm,small]
\begin{tikzpicture}[scale=1]
  \tkzDefPoint(0,0){A} \tkzDefPoint(4,0){B}
  \tkzCalcLength(A,B)\tkzGetLength{dAB}
  \tkzDefLine[perpendicular=through A](A,B)
  \tkzGetPoint{D}
  \tkzDefPointWith[orthogonal,K=-1](B,A)    
    \tkzGetPoint{F}
  \tkzGetPoint{C}
  \tkzDrawLine[add= .6 and .2](A,B)
  \tkzDrawLine(A,D) 
  \tkzShowLine[orthogonal=through A,gap=2](A,B)
  \tkzMarkRightAngle(B,A,D)
  \tkzCompasss(A,D D,C)
  \tkzDrawArc[R](B,\dAB)(80,110)
  \tkzDrawPoints(A,B,C,D)
  \tkzDrawSegments[color=gray,style=dashed](B,C C,D)
  \tkzLabelPoints[below left](A,B,C,D)
\end{tikzpicture}
\end{tkzexample}


\subsubsection{Example}
The macro \tkzcname{tkzDefCircle[radius](A,B)} defines the radius that we retrieve with \tkzcname{tkzGetLength},  this result is in \tkzname{cm}.

\begin{tkzexample}[latex=6cm,small]
\begin{tikzpicture}[scale=.5]
 \tkzDefPoint(0,0){A}
 \tkzDefPoint(3,-4){B}
 \tkzDefMidPoint(A,B) \tkzGetPoint{M}
 \tkzCalcLength(M,B)\tkzGetLength{rAB}
 \tkzDrawCircle(A,B)
 \tkzDrawPoints(A,B)
 \tkzLabelPoints(A,B)
 \tkzDrawSegment[dashed](A,B)
 \tkzLabelSegment(A,B){$\pgfmathprintnumber{\rAB}$}
\end{tikzpicture}
\end{tkzexample}


\subsection{Transformation from pt to cm or cm to pt}
Not sure if this is necessary and it is only a division by 28.45274 and a multiplication by the same number. The macros are:

\begin{NewMacroBox}{tkzpttocm}{\parg{number}\marg{name of macro}}%
The result is stored in a macro.

\medskip
\begin{tabular}{lll}%
\toprule
arguments             & example & explanation                         \\
\midrule
\TAline{(number)\{name of macro\}} {\tkzcname{tkzpttocm}(120)\{len\}}{\tkzcname{len} gives a number of tkzname{cm}}
\bottomrule
\end{tabular}

\medskip
You'll have to use \tkzcname{len} along with \tkzname{cm}.
\end{NewMacroBox}

\subsection{Change of unit} 
\begin{NewMacroBox}{tkzcmtopt}{\parg{number}\marg{name of macro}}%
The result is stored in a macro.

\medskip
\begin{tabular}{lll}
\toprule
arguments             & example & explanation                         \\
\midrule
\TAline{(number)\{name of macro\}}{\tkzcname{tkzcmtopt}(5)\{len\}}{\tkzcname{len} length in \tkzname{pts}}
\bottomrule
\end{tabular}

\medskip
\emph{The result can be used with \tkzcname{len}\ \tkzname{pt}}
\end{NewMacroBox}


\subsection{Get point coordinates}
%<--------------------------------------------------------------------------–>
%                    Coordonnées d'un point 
%    result in #2x and #2y    #1 is the point and we get its coordinates
% use either $A$ one point \tkzGetPointCoord(A){V} then \Vx = xA and \Vy = yA
% in cm 
% tkzGetPointCoord with [#1] cm or  pt ?? todo
%<--------------------------------------------------------------------------–>
\begin{NewMacroBox}{tkzGetPointCoord}{\parg{$A$}\marg{name of macro}}%
\begin{tabular}{lll}%
arguments             & example & explanation                         \\
\midrule
\TAline{(point)\{name of macro\}} {\tkzcname{tkzGetPointCoord}(A)\{A\}}{\tkzcname{Ax} and \tkzcname{Ay} give coordinates for $A$}
\end{tabular}

\medskip
\emph{Stores in two macros the coordinates of a point. If the name of the macro is \tkzname{p}, then \tkzcname{px} and \tkzcname{py} give the coordinates of the chosen point with the cm as unit.}
\end{NewMacroBox}

\subsubsection{Coordinate transfer with \tkzcname{tkzGetPointCoord}}

\begin{tkzexample}[width=8cm,small]
\begin{tikzpicture}
 \tkzInit[xmax=5,ymax=3]
 \tkzGrid[sub,orange]
 \tkzDrawX \tkzDrawY
 \tkzDefPoint(1,0){A}
 \tkzDefPoint(4,2){B}
 \tkzGetPointCoord(A){a}
 \tkzGetPointCoord(B){b}
 \tkzDefPoint(\ax,\ay){C}
 \tkzDefPoint(\bx,\by){D}
 \tkzDrawPoints[color=red](C,D)
\end{tikzpicture}
\end{tkzexample}

\subsubsection{Sum of vectors with \tkzcname{tkzGetPointCoord}}
\begin{tkzexample}[width=6cm,small]
\begin{tikzpicture}[>=latex]
  \tkzDefPoint(1,4){a}
  \tkzDefPoint(3,2){b}
  \tkzDefPoint(1,1){c}
  \tkzDrawSegment[->,red](a,b)
  \tkzGetPointCoord(c){c}
  \draw[color=blue,->](a) -- ([shift=(b)]\cx,\cy) ;
  \draw[color=purple,->](b) -- ([shift=(b)]\cx,\cy) ;
  \tkzDrawSegment[->,blue](a,c)
  \tkzDrawSegment[->,purple](b,c)
\end{tikzpicture}
\end{tkzexample}

\subsection{Swap labels of points}

\begin{NewMacroBox}{tkzSwapPoints}{\parg{$pt1$,$pt2$}}%
\begin{tabular}{lll}%
arguments             & example & explanation                         \\
\midrule
\TAline{(pt1,pt2)} {\tkzcname{tkzSwapPoints}(A,B)}{now $A$ has the coordinates of $B$ }
\end{tabular}

\emph{The points have exchanged their coordinates.}
\end{NewMacroBox}

\subsubsection{Use of \tkzcname{tkzSwapPoints}}

\begin{tkzexample}[width=6cm,small]
\begin{tikzpicture}
  \tkzDefPoints{0/0/O,5/-1/A,2/2/B}
   \tkzSwapPoints(A,B)
   \tkzDrawPoints(O,A,B)
   \tkzLabelPoints(O,A,B)
\end{tikzpicture}
\end{tkzexample}

\subsection{Dot Product}
In Euclidean geometry, the dot product of the Cartesian coordinates of two vectors is widely used.

\begin{NewMacroBox}{tkzDotProduct}{\parg{$pt1$,$pt2$,$pt3$}}%
  The dot product of two vectors $\overrightarrow{u} = [a,b]$ and  $\overrightarrow{v} = [a',b']$ is defined as: $\overrightarrow{u}\cdot \overrightarrow{v} = aa' + bb'$

$\overrightarrow{u} = \overrightarrow{pt1pt2}$ $\overrightarrow{v} = \overrightarrow{pt1pt3}$
  
\begin{tabular}{lll}%
arguments             & example & explanation                         \\
\midrule
\TAline{(pt1,pt2,pt3)} {\tkzcname{tkzDotProduct}(A,B,C)}{the result is $\overrightarrow{AB}\cdot \overrightarrow{AC}$}
\end{tabular}

\emph{The result is a number that can be retrieved with \tkzcname{tkzGetResult}.}
\end{NewMacroBox}

\subsubsection{Simple example} % (fold)
\label{ssub:simple_example}

\begin{tkzexample}[small,latex=7cm]
\begin{tikzpicture}
  \tkzDefPoints{-2/-3/A,4/0/B,1/3/C}
  \tkzDefPointBy[projection= onto A--B](C)  
  \tkzGetPoint{H}
  \tkzDrawSegment(C,H)
  \tkzMarkRightAngle(C,H,A)
  \tkzDrawSegments[vector style](A,B A,C)
  \tkzDrawPoints(A,H) \tkzLabelPoints(A,B,H)
  \tkzLabelPoints[above](C)
  \tkzDotProduct(A,B,C) \tkzGetResult{pabc}
 % \pgfmathparse{round(10*\pabc)/10}
  \let\pabc\pgfmathresult
  \node at (1,-3) {$\overrightarrow{PA}\cdot \overrightarrow{PB}=\pabc$};
  \tkzDotProduct(A,H,B) \tkzGetResult{phab}
 % \pgfmathparse{round(10*\phab)/10}
  \let\phab\pgfmathresult
  \node at (1,-4) {$PA \times PH = \phab $};
\end{tikzpicture}
\end{tkzexample}
% subsubsection simple_example (end)


\subsubsection{Cocyclic points} % (fold)
\label{ssub:cocyclicpts}

\begin{tkzexample}[small,latex=7cm]
\begin{tikzpicture}[scale=.75]
  \tkzDefPoints{1/2/O,5/2/B,2/2/P,3/3/Q}
  \tkzInterLC[common=B](O,B)(O,B) \tkzGetFirstPoint{A}
  \tkzInterLC[common=B](P,Q)(O,B) \tkzGetPoints{C}{D}
  \tkzDrawCircle(O,B)
  \tkzDrawSegments(A,B C,D)
  \tkzDrawPoints(A,B,C,D,P)
  \tkzLabelPoints(P)
  \tkzLabelPoints[below left](A,C)
  \tkzLabelPoints[above right](B,D)
  \tkzDotProduct(P,A,B) \tkzGetResult{pab}
  \pgfmathparse{round(10*\pab)/10}
  \let\pab\pgfmathresult
  \tkzDotProduct(P,C,D) \tkzGetResult{pcd}
  \pgfmathparse{round(10*\pcd)/10}
  \let\pcd\pgfmathresult
  \node at (1,-3) {%
  $\overrightarrow{PA}\cdot \overrightarrow{PB} =
   \overrightarrow{PC}\cdot \overrightarrow{PD}$};
  \node at (1,-4)%
  {$\overrightarrow{PA}\cdot \overrightarrow{PB}=\pab$};
 \node at (1,-5){%
 $\overrightarrow{PC}\cdot \overrightarrow{PD} =\pcd$};
\end{tikzpicture}
\end{tkzexample}
% subsubsection cocyclicpts (end)

\newpage
\subsection{Power of a point with respect to a circle}

\begin{NewMacroBox}{tkzPowerCircle}{\parg{$pt1$}\parg{$pt2$,$pt3$}}%
\begin{tabular}{lll}%
arguments             & example & explanation                         \\
\midrule
\TAline{(pt1)(pt2,pt3)} {\tkzcname{tkzPowerCircle}(A)(O,M)}{power of $A$ with respect to the circle (O,A)}
\end{tabular}

\emph{The result is a number that represents the power of a point with respect to a circle.}
\end{NewMacroBox}

\subsubsection{Power from the radical axis} % (fold)
\label{ssub:power}

In this example, the radical axis $(EF)$ has been drawn. A point $H$ has been chosen on $(EF)$ and the power of the point $H$ with respect to the circle of center $A$ has been calculated as well as $PS^2$. You can check that the power of $H$ with respect to the circle of center $C$ as well as $HS'^2, HT^2, HT'^2$ give the same result.  

\begin{tkzexample}[small,latex=7cm]
\begin{tikzpicture}[scale=.5]
  \tkzDefPoints{-1/0/A,0/5/B,5/-1/C,7/1/D}
  \tkzDrawCircles(A,B C,D)
  \tkzDefRadicalAxis(A,B)(C,D) \tkzGetPoints{E}{F}
  \tkzDrawLine[add=1 and 2](E,F)
  \tkzDefPointOnLine[pos=1.5](E,F) 
  \tkzGetPoint{H}
  \tkzDefLine[tangent from = H](A,B)
  \tkzGetPoints{T}{T'}
  \tkzDefLine[tangent from = H](C,D)
  \tkzGetPoints{S}{S'}
  \tkzDrawSegments(H,T H,T' H,S H,S')
  \tkzDrawPoints(A,B,C,D,E,F,H,T,T',S,S')
  \tkzPowerCircle(H)(A,B) \tkzGetResult{pw}
  \tkzDotProduct(H,S,S) \tkzGetResult{phtt}
  \node {Power $\approx \pw \approx \phtt$};
\end{tikzpicture}
\end{tkzexample}
% subsubsection power (end)

\subsection{Radical axis}

In geometry, the radical axis of two non-concentric circles is the set of points whose power with respect to the circles are equal. Here |\tkzDefRadicalAxis(A,B)(C,D)| gives the radical axis of the two circles $\mathcal{C}(A,B)$ and $\mathcal{C}(C,D)$. 

\begin{NewMacroBox}{tkzDefRadicalAxis}{\parg{$pt1$,$pt2$}\parg{$pt3$,$pt4$}}%
\begin{tabular}{lll}%
arguments             & example & explanation                         \\
\midrule
\TAline{(pt1,pt2)(pt3,pt4)} {\tkzcname{tkzDefRadicalAxis}(A,B)(C,D)}{Two circles with centers $A$ and $C$}
\midrule
\end{tabular}


\emph{The result is two points of the radical axis.}
\end{NewMacroBox}

\subsubsection{Two circles disjointed} % (fold)
\label{ssub:two_circles_disjointed}


\begin{tkzexample}[small,latex=8cm]
\begin{tikzpicture}[scale=.75]
    \tkzDefPoints{-1/0/A,0/2/B,4/-1/C,4/0/D}
    \tkzDrawCircles(A,B C,D)
    \tkzDefRadicalAxis(A,B)(C,D)
     \tkzGetPoints{E}{F}
    \tkzDrawLine[add=1 and 2](E,F)
    \tkzDrawLine[add=.5 and .5](A,C)
\end{tikzpicture}
\end{tkzexample}
% subsubsection two_circles_disjointed (end)

\subsection{Two intersecting circles} % (fold)
\label{sub:two_intersecting_circles}
\begin{tkzexample}[small,latex=8cm]
\begin{tikzpicture}[scale=.5]
  \tkzDefPoints{-1/0/A,0/2/B,3/-1/C,3/-2/D}
  \tkzDrawCircles(A,C B,D)
  \tkzDefRadicalAxis(A,C)(B,D)
  \tkzGetPoints{E}{F}
  \tkzDrawPoints(A,B,C,D,E,F)
    \tkzLabelPoints(A,B,C,D,E,F)
  \tkzDrawLine[add=.25 and .5](E,F)
  \tkzDrawLine[add=.25 and .25](A,B)
\end{tikzpicture}
\end{tkzexample}
% subsection two_intersecting_circles (end)


\subsection{Two externally tangent circles} % (fold)
\label{sub:two_externally_tangent_circles}
\begin{tkzexample}[small,latex=8cm]
\begin{tikzpicture}[scale=.5]
  \tkzDefPoints{0/0/A,4/0/B,6/0/C}
  \tkzDrawCircles(A,B C,B)
  \tkzDefRadicalAxis(A,B)(C,B)
  \tkzGetPoints{E}{F}
  \tkzDrawPoints(A,B,C,E,F)
    \tkzLabelPoints(A,B,C,E,F)
  \tkzDrawLine[add=1 and 1](E,F)
  \tkzDrawLine[add=.5 and .5](A,B)
\end{tikzpicture}
\end{tkzexample}
% subsection two_externally_tangent_circles (end)


\subsection{Two circles tangent internally} % (fold)
\label{sub:deux_cercles_tangents_interieurement}
\begin{tkzexample}[small,latex=8cm]
\begin{tikzpicture}[scale=.5]
  \tkzDefPoints{0/0/A,3/0/B,5/0/C}
  \tkzDrawCircles(A,C B,C)
  \tkzDefRadicalAxis(A,C)(B,C)
  \tkzGetPoints{E}{F}
  \tkzDrawPoints(A,B,C,E,F)
  \tkzLabelPoints[below right](A,B,C,E,F)
  \tkzDrawLine[add=1 and 1](E,F)
  \tkzDrawLine[add=.5 and .5](A,B)
\end{tikzpicture}
\end{tkzexample}
% subsection deux_cercles_tangents_interieurement (end)

\subsubsection{Three circles} % (fold)
\label{ssub:threecircles}



\begin{tkzexample}[small,latex=7cm]
\begin{tikzpicture}[scale=.4]
  \tkzDefPoints{0/0/A,5/0/a,7/-1/B,3/-1/b,5/-4/C,2/-4/c}
  \tkzDrawCircles(A,a B,b C,c)
  \tkzDefRadicalAxis(A,a)(B,b) \tkzGetPoints{i}{j}
  \tkzDefRadicalAxis(A,a)(C,c) \tkzGetPoints{k}{l}
  \tkzDefRadicalAxis(C,c)(B,b) \tkzGetPoints{m}{n}
  \tkzDrawLines[new](i,j k,l m,n)
\end{tikzpicture}
\end{tkzexample}
% subsubsection threecircles (end)
  
\subsection{\tkzcname{tkzIsLinear}, \tkzcname{tkzIsOrtho}}
 \begin{NewMacroBox}{tkzIsLinear}{\parg{$pt1$,$pt2$,$pt3$}}%
 \begin{tabular}{lll}%
 arguments             & example & explanation                         \\
 \midrule
 \TAline{(pt1,pt2,pt3)} {\tkzcname{tkzIsLinear}(A,B,C)}{$A,B,C$ aligned ?}
 \midrule
 \end{tabular}
 
 \emph{\tkzcname{tkzIsLinear} allows to test the alignment of the three points $pt1$,$pt2$,$pt3$. }
 \end{NewMacroBox}
 
 \begin{NewMacroBox}{tkzIsOrtho}{\parg{$pt1$,$pt2$,$pt3$}}%
 \begin{tabular}{lll}%
 arguments             & example & explanation                         \\
 \midrule
 \TAline{(pt1,pt2,pt3)} {\tkzcname{tkzIsOrtho}(A,B,C)}{$(AB)\perp (AC)$ ? }
 \midrule
 \end{tabular}
 
 \emph{\tkzcname{tkzIsOrtho} allows to test the orthogonality of lines $(pt1pt2)$ and $(pt1pt3)$. }
 \end{NewMacroBox}
 
 \subsubsection{Use of \tkzcname{tkzIsOrtho} and \tkzcname{tkzIsLinear}}
   
\begin{tkzexample}[small,latex=7cm]
  \begin{tikzpicture}
  \tkzDefPoints{1/-2/A,5/0/B}
  \tkzDefCircle[diameter](A,B) \tkzGetPoint{O}
  \tkzDrawCircle(O,A)
  \tkzDefPointBy[rotation= center O angle 60](B) 
  \tkzGetPoint{C}
  \tkzDefPointBy[rotation= center O angle 60](A) 
  \tkzGetPoint{D}
  \tkzDrawCircle(O,A)
  \tkzDrawPoints(A,B,C,D,O)
  \tkzIsOrtho(C,A,B)
  \iftkzOrtho
    \tkzDrawPolygon[blue](A,B,C)
  \tkzDrawPoints[blue](A,B,C,D)
  \else
  \tkzDrawPoints[red](A,B,C,D)
  \fi
   \tkzIsLinear(O,C,D)
    \iftkzLinear
    \tkzDrawSegment[orange](C,D)
    \fi
\end{tikzpicture}
  
\end{tkzexample}

  
\endinput