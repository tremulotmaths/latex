\section{Angles} 
\subsection{Definition and usage with \tkzname{tkz-euclide}}
In Euclidean geometry, an angle is the figure formed by two rays, called the sides of the angle, sharing a common endpoint, called the vertex of the angle.[Wikipedia]. A ray with \tkzname{tkz-euclide} is defined by two points also each angle is defined with three points like $\widehat{AOB}$. The vertex $O$ is the second point. Their order is important because it is assumed that the angle is specified in the direct order (counterclockwise). 
In trigonometry and mathematics in general, plane angles are conventionally measured counterclockwise, starting with $0^\circ$  pointing directly to the right (or east), and $90^\circ$ pointing straight up (or north)[Wikipedia]. 
Let us agree that an angle measured counterclockwise is positive.

  \begin{center}
    \begin{tikzpicture}[scale=.75]
      \node {clockwise};
      \tkzDefPoint(0,0){O} \tkzDefPoint(90:2){A}\tkzDefPoint(180:2){B}   
      \tkzDrawArc[black,line width=2pt,arrows = {Stealth-}](O,B)(A)
    \end{tikzpicture}
    \begin{tikzpicture}[scale=.75]
          \node {counterclockwise};
      \tkzDefPoint(0,0){O} \tkzDefPoint(90:2){A}\tkzDefPoint(0:2){B}   
      \tkzDrawArc[black,line width=2pt,arrows = {-Stealth}](O,A)(B)
    \end{tikzpicture}  
  \end{center}
  
 \tkzname{Angles} are involved in several macros like \tkzcname{tkzDefPoint},\tkzcname{tkzDefPointBy[rotation = \dots]}, \tkzcname{tkzDrawArc}
 and the next one  \tkzcname{tkzGetAngle}. With the exception of the last one, all these macros accept negative angles.
 
 \begin{figure}[!ht]
 \centering
 \begin{tabular}{|c|c|}
 \hline
 \tkzsubf{\begin{tikzpicture}
 \tkzDefPoint(0,0){O}    \tkzDefPoint(0:2){A}
 \tkzDefPointBy[rotation=center O angle 80](A)  \tkzGetPoint{B}
 \tkzDrawSegments[-{Stealth}](O,A O,B)
 \tkzMarkAngles[size=2,-Stealth,teal](A,O,B)
 \tkzFindAngle(A,O,B)   \tkzGetAngle{an}
 \tkzLabelAngle[pos=1,teal](A,O,B){$ \pgfmathprintnumber{\an}^\circ$}
 \tkzLabelPoints(A)  \tkzLabelPoints[above](B)
 \end{tikzpicture}}
      {Rotation $80^\circ$ from $(O,A)$ to $(O,B)$\\ 
    {\textbackslash}tkzDefPointBy[rotation=center O angle 80]}
 &
 \tkzsubf{\begin{tikzpicture}
 \tkzDefPoint(0,0){O}    \tkzDefPoint(0:2){A}
 \tkzDefPointBy[rotation=center O angle -80](A)  \tkzGetPoint{B}
 \tkzDrawSegments[-{Stealth}](O,A O,B)
 \tkzMarkAngles[size=2,Stealth-,red](B,O,A)
 \tkzFindAngle(B,O,A)   \tkzGetAngle{an}
 \tkzLabelAngle[pos=1,red](B,O,A){$-\pgfmathprintnumber{\an}^\circ$}
\tkzLabelPoints[right](A)  \tkzLabelPoints[below](B)
 \end{tikzpicture}}
  {Rotation $-80^\circ$ from $(O,A)$ to $(O,B)$\\ 
     {\textbackslash}tkzDefPointBy[rotation=center O angle -80]}
 \\ \hline
 \tkzsubf{\begin{tikzpicture}
 \tkzDefPoint(0,0){O}    \tkzDefPoint(0:2){A}
 \tkzDefPointBy[rotation=center O angle 80](A)  \tkzGetPoint{B}
 \tkzDrawSegments[-{Stealth}](O,A O,B)
 \tkzMarkAngles[size=1.5,-Stealth,teal](A,O,B)
 \tkzFindAngle(A,O,B)   \tkzGetAngle{an}
 \tkzLabelAngle[pos=1,teal](A,O,B){$ \pgfmathprintnumber{\an}^\circ$}
\tkzLabelPoints(A)  \tkzLabelPoints[above](B)
 \end{tikzpicture}}
      { {\textbackslash}tkzFindAngle(A,O,B) gives $80$}
 &
 \tkzsubf{\begin{tikzpicture}
 \tkzDefPoint(0,0){O}    \tkzDefPoint(0:2){A}
 \tkzDefPointBy[rotation=center O angle -80](A)  \tkzGetPoint{B}
 \tkzDrawSegments[-{Stealth}](O,A O,B)
 \tkzMarkAngles[size=1,-Stealth,red](A,O,B)
 \tkzFindAngle(A,O,B)   \tkzGetAngle{an}
 \tkzLabelAngle[pos=.75,red](A,O,B){$\pgfmathprintnumber{\an}^\circ$}
\tkzLabelPoints[right](A)  \tkzLabelPoints[below](B)
 \end{tikzpicture}}
  {{\textbackslash}tkzFindAngle(A,O,B) gives $\pgfmathprintnumber{\an}^\circ$}
 \\\hline
 \end{tabular}
 \end{figure}

As we can see, the $-80^\circ$ rotation defines a clockwise angle but the macro 
\tkzcname{tkzFindAngle} recovers a counterclockwise angle.

\subsection{Recovering an angle \tkzcname{tkzGetAngle}}
\begin{NewMacroBox}{tkzGetAngle}{\parg{name of macro}}%
Assigns the value in degree of an angle to a macro. The value is positive and between $0^\circ$ and $360^\circ$.  This macro retrieves \tkzcname{tkzAngleResult} and stores the result in a new macro.

\medskip

\begin{tabular}{lll}%
\toprule
arguments             & example & explanation             \\
\midrule
\TAline{name of macro} {\tkzcname{tkzGetAngle}\{ang\}}{\tkzcname{ang} contains the value of the angle.}
\end{tabular}
\end{NewMacroBox}

This is an auxiliary macro that allows you to retrieve the result of the following macro \tkzcname{tkzFindAngle}.

\subsection{Angle formed by three points}

\begin{NewMacroBox}{tkzFindAngle}{\parg{pt1,pt2,pt3}}%
The result is stored in a macro \tkzcname{tkzAngleResult}.

\medskip

\begin{tabular}{lll}%
\toprule
arguments     & example & explanation     \\
\midrule
\TAline{(pt1,pt2,pt3)} {\tkzcname{tkzFindAngle}(A,B,C)}{\tkzcname{tkzAngleResult} gives the angle ($\overrightarrow{BA},\overrightarrow{BC}$)}
\bottomrule
\end{tabular}

\medskip
The measure is always positive and between $0^\circ$  and $360^\circ$. With the usual conventions, a counterclockwise angle smaller than a straight angle has always a measure between $0^\circ$ and $180^\circ$, while a clockwise angle smaller than a straight angle will have a measurement greater than $180^\circ$. \tkzcname{tkzGetAngle} can retrieve the angle.
\end{NewMacroBox}
 
\subsubsection{Verification of angle measurement}
    
\begin{tkzexample}[latex=7cm,small]
\begin{tikzpicture}[scale=.75]
  \tkzDefPoint(-1,1){A}
  \tkzDefPoint(5,2){B}
  \tkzDefEquilateral(A,B)
  \tkzGetPoint{C}
  \tkzDrawPolygon(A,B,C)
  \tkzFindAngle(B,A,C) \tkzGetAngle{angleBAC}
  \edef\angleBAC{\fpeval{round(\angleBAC)}}
  \tkzDrawPoints(A,B,C) 
  \tkzLabelPoints(A,B)
  \tkzLabelPoint[right](C){$C$}
  \tkzLabelAngle(B,A,C){\angleBAC$^\circ$}
  \tkzMarkAngle[size=1.5](B,A,C)
\end{tikzpicture}
\end{tkzexample}

\subsubsection{Determination of the three angles of a triangle}

\begin{tkzexample}[latex=6cm,small]
\begin{tikzpicture}
\tikzset{label angle style/.append style={pos=1.4}}
\tkzDefPoints{0/0/a,5/3/b,3/6/c}
\tkzDrawPolygon(a,b,c)
\tkzFindAngle(c,b,a)\tkzGetAngle{angleCBA}
\pgfmathparse{round(1+\angleCBA)}
\let\angleCBA\pgfmathresult
\tkzFindAngle(a,c,b)\tkzGetAngle{angleACB}
\pgfmathparse{round(\angleACB)}
\let\angleACB\pgfmathresult
\tkzFindAngle(b,a,c)\tkzGetAngle{angleBAC}
\pgfmathparse{round(\angleBAC)}
\let\angleBAC\pgfmathresult
\tkzMarkAngle(c,b,a)
\tkzLabelAngle(c,b,a){\tiny $\angleCBA^\circ$}
\tkzMarkAngle(a,c,b)
\tkzLabelAngle(a,c,b){\tiny $\angleACB^\circ$}
\tkzMarkAngle(b,a,c)
\tkzLabelAngle(b,a,c){\tiny $\angleBAC^\circ$}
\end{tikzpicture}
\end{tkzexample}

\subsubsection{Angle between two circles}
We are looking for the angle formed by the tangents at a point of intersection

\begin{tkzexample}[latex=7cm,small]
\begin{tikzpicture}[scale=.4]
\pgfkeys{/pgf/number format/.cd,%
          fixed,precision=1}
\tkzDefPoints{0/0/A,6/0/B,4/2/C}
\tkzDrawCircles(A,C B,C)
\tkzDefLine[tangent at=C](A) \tkzGetPoint{a}
\tkzDefPointsBy[symmetry = center C](a){d}
\tkzDefLine[tangent at=C](B) \tkzGetPoint{b}
\tkzDrawLines[add=1 and 4](a,C  C,b)
\tkzFillAngle[fill=teal,opacity=.2%
                        ,size=2](b,C,d)
\tkzFindAngle(b,C,d)\tkzGetAngle{bcd}
\tkzLabelAngle[pos=1.25](b,C,d){%
  \tiny $\pgfmathprintnumber{\bcd}^\circ$}
\end{tikzpicture}
\end{tkzexample}

\subsection{Angle formed by a straight line with the horizontal axis \tkzcname{tkzFindSlopeAngle}}
Much more interesting than the last one. The result is between -180 degrees and +180 degrees.

\begin{NewMacroBox}{tkzFindSlopeAngle}{\parg{A,B}}%
Determines the slope of the straight line (AB). The result is stored in a macro \tkzcname{tkzAngleResult}.

\medskip
\begin{tabular}{lll}%
\toprule
arguments  & example & explanation     \\
\midrule
\TAline{(pt1,pt2)} {\tkzcname{tkzFindSlopeAngle}(A,B)}{}
\bottomrule
\end{tabular}

\medskip
\tkzcname{tkzGetAngle} can retrieve the result. If retrieval is not necessary, you can use \tkzcname{tkzAngleResult}.
\end{NewMacroBox}

\subsubsection{How to use  \tkzcname{tkzFindSlopeAngle}}

 The point here is that $(AB)$ is the bisector of $\widehat{CAD}$, such that the $AD$ slope is zero. We recover the slope of $(AB)$ and then rotate twice.

\begin{tkzexample}[latex=7cm,small]
\begin{tikzpicture}
 \tkzDefPoint(1,5){A} \tkzDefPoint(5,2){B}  
 \tkzFindSlopeAngle(A,B)\tkzGetAngle{tkzang}
 \tkzDefPointBy[rotation= center A angle \tkzang ](B)
 \tkzGetPoint{C}
 \tkzDefPointBy[rotation= center A angle -\tkzang ](B) 
 \tkzGetPoint{D}
 \tkzDrawSegment(A,B)
 \tkzDrawSegments[new](A,C A,D)
 \tkzDrawPoints(A,B,C,D)
 \tkzCompass[length=1](A,C)
 \tkzCompass[delta=10,brown](B,C)  
 \tkzLabelPoints(B,C,D)  
 \tkzLabelPoints[above left](A)
\end{tikzpicture}
\end{tkzexample}

\subsubsection{Use of \tkzcname{tkzFindSlopeAngle} and \tkzcname{tkzGetAngle}}
Here is another version of the construction of a mediator

\begin{tkzexample}[latex=6cm,small]
\begin{tikzpicture}
 \tkzInit
 \tkzDefPoint(0,0){A}        \tkzDefPoint(3,2){B}
 \tkzDefLine[mediator](A,B)  \tkzGetPoints{I}{J}
 \tkzCalcLength(A,B)         \tkzGetLength{dAB}
 \tkzFindSlopeAngle(A,B)     \tkzGetAngle{tkzangle}
 \begin{scope}[rotate=\tkzangle]
   \tkzSetUpArc[color=gray,line width=0.2pt,%
     /tkzcompass/delta=10]
   \tkzDrawArc[R,arc](B,3/4*\dAB)(120,240)
   \tkzDrawArc[R,arc](A,3/4*\dAB)(-45,60)
   \tkzDrawLine(I,J)         \tkzDrawSegment(A,B)
  \end{scope}
  \tkzDrawPoints(A,B,I,J)    \tkzLabelPoints(A,B)
   \tkzLabelPoints[right](I,J)
\end{tikzpicture}
\end{tkzexample}

\subsubsection{Another use of \tkzcname{tkzFindSlopeAngle}}

\begin{tkzexample}[latex=7cm,small]
\begin{tikzpicture}[scale=1.5]
  \tkzDefPoint(1,2){A}    \tkzDefPoint(3,4){B}
  \tkzDefPoint(3,2){C}    \tkzDefPoint(3,1){D}
  \tkzDrawSegments(A,B A,C A,D)
  \tkzDrawPoints[color=red](A,B,C,D)  
  \tkzLabelPoints(A,B,C,D)
  \tkzFindSlopeAngle(A,B)\tkzGetAngle{SAB} 
  \tkzFindSlopeAngle(A,C)\tkzGetAngle{SAC}
  \tkzFindSlopeAngle(A,D)\tkzGetAngle{SAD}
  \pgfkeys{/pgf/number format/.cd,fixed,precision=2}
  \tkzText(1,5){The slope of (AB) is : 
     $\pgfmathprintnumber{\SAB}^\circ$}     
  \tkzText(1,4.5){The slope of (AC) is : 
     $\pgfmathprintnumber{\SAC}^\circ$}    
  \tkzText(1,4){The slope of (AD) is : 
     $\pgfmathprintnumber{\SAD}^\circ$}
\end{tikzpicture}
\end{tkzexample}

\endinput