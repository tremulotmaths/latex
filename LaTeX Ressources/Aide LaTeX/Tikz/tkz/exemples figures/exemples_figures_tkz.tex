\documentclass[12pt,french,oneside]{report}
%%%%%%%%%%%%%%%%%%%%%%%%%%%%%%%%%%%%%%%%%%%%%%%%%%%%%%%%%%%%%%%%%%%%%%%%%%%%%%%
%%%%%%%%%%%%%%%%
%%%Modifications par rapport au précédent
%Suppression du package eurosym incompatible avec le package marvosym (pour certains symboles) à cause de \EUR{}
%Ajout du package marvosym
%voir le fichier des symboles utiles.
%Ajout du package tikzsymbols
%a nécessité la mise à jour du package l3kernel
%Ajout de la commande \pfr{} pour encadrer un résultat en rouge
%Modification de \pv


%___________________________
%===    Configurations 09.06.2016
%------------------------------------------------------
%packages permettant d'augmenter le nombre de registres de dimension et donc d'éviter les erreurs de compilation dûs aux packages tikz, pstricks and compagnie
\usepackage{etex}
%___________________________
%===    Pour le français
%------------------------------------------------------
\usepackage[utf8x]{inputenc}
\usepackage[T1]{fontenc}
\usepackage[english,french]{babel}
\FrenchFootnotes
\usepackage{tipa}%alphabet phonétique internationnal
%___________________________
%===    Polices d'écriture
%------------------------------------------------------
%\usepackage{mathpazo}
\usepackage{frcursive} % Pour l'écriture cursive
\usepackage[upright]{fourier}% l'option permet d'avoir les majuscules droites dans les formules mathématiques
\usepackage[scaled=0.875]{helvet}

%___________________________
%===    Les couleurs
%------------------------------------------------------
\usepackage[dvipsnames,table]{xcolor}
%
\newcommand{\rouge}[1]{{\color{red} #1}}
\definecolor{midblue}{rgb}{0.145,0.490,0.882}
\newcommand\MaCouleur{midblue}

%___________________________
%===   Entête, pied de page
%------------------------------------------------------

\usepackage{lscape} %permet le format paysage du document
\usepackage{xspace} % création automatique d'espaces dans les commandes
\setlength{\parindent}{0pt}

\usepackage{fancyhdr}
%
\renewcommand{\headrulewidth}{0pt}% pas de trait en entête
\newcommand\RegleEntete[1][0.4pt]{\renewcommand{\headrulewidth}{#1}}%commande pour ajouter un trait horizontal en entête

\newcommand{\entete}[3]{\lhead{#1} \chead{#2} \rhead{#3}}
\newcommand{\pieddepage}[3]{\lfoot{#1} \cfoot{#2} \rfoot{#3}}

\renewcommand{\chaptermark}[1]{\markboth{#1}{}} % enregistre le titre courant du chapitre 
%en-tete droite page [paire] et {impaire}
\rhead[]{\textbf{\leftmark.}}
%en-tete gauche page [paire] et {impaire}
\lhead[\textbf{\chaptername~\thechapter.}]{}


\usepackage{enumerate} %permet la modif de la numérotation et de poursuivre une numérotation en cours avec \begin{enumerate}[resume]
\usepackage{enumitem}
\frenchbsetup{StandardLists=true}%frenchb ne s'occupera pas des listes
\setenumerate[1]{font=\bfseries,label=\arabic*.} % numérotation 1. 2. ...
%\setenumerate[2]{font=\itshape,label=(\alph*)} % sous-numérotation (a) (b) ...
\setenumerate[2]{font=\bfseries,label=\alph*)} % sous-numérotation a) b) ...

\usepackage{lastpage} % permet d'afficher le nombre total de pages après DEUX compilations.

%___________________________
%===    Raccourcis classe
%------------------------------------------------------
\newcommand\seconde{2\up{nde}\xspace}
\newcommand\premiere{1\up{ère}\xspace}
\newcommand\terminale{T\up{le}\xspace}
\newcommand\stmg{\bsc{Stmg}}
\newcommand\sti{\bsc{Sti2d}}
\newcommand\bat{BAT 1\xspace}
\newcommand\BAT{BAT 2\xspace}
\newcommand\tesspe{TES Spécialité\xspace}


%___________________________
%===    Réglages et Commandes Maths
%------------------------------------------------------
%les commandes suivantes évitent le message "too many math alphabets"...
\newcommand\hmmax{0}
\newcommand\bmmax{0}

%redéfinition de fractions, limites, sommes, intégrales, coefficients binomiaux en displaystyle, limites de suites
\usepackage{amssymb,mathtools}
\let\binomOld\binom
\renewcommand{\binom}{\displaystyle\binomOld}
\let\limOld\lim
\renewcommand{\lim}{\displaystyle\limOld}
\newcommand{\limn}{\lim_{n\to +\infty}} %limite lorsque n tend vers + infini
\newcommand{\limm}{\lim_{x\to -\infty}} %limite lorsque x tend vers - infini
\newcommand{\limp}{\lim_{x\to +\infty}} %limite lorsque x tend vers + infini
\newcommand{\limz}{\lim_{x\to 0}} %limite lorsque x tend vers 0
\newcommand{\limzm}{\lim_{\substack{x \to 0\\ x < 0}}} %limite lorsque x tend vers 0-
\newcommand{\limzp}{\lim_{\substack{x \to 0\\ x > 0}}} %limite lorsque x tend vers 0+
\let\sumOld\sum
\renewcommand{\sum}{\displaystyle\sumOld}
\let\intOld\int
\renewcommand{\int}{\displaystyle\intOld}

%\usepackage{yhmath}%permet les arcs de cercles
%\usepackage[euler-digits]{eulervm} %-> police maths
%
\usepackage{stmaryrd}%\llbracket et \rrbracket % crochets doubles pour intervalles d'entier
%symbole parallèle avec \sslash

\newcommand{\crochets}[2]{\ensuremath{\llbracket #1 ; #2 \rrbracket}}

\newcommand{\intervalleff}[2]{\left[#1\,;#2\right]}
\newcommand{\intervallefo}[2]{\left[#1\,;#2\right[}
\newcommand{\intervalleof}[2]{\left]#1\,;#2\right]}
\newcommand{\intervalleoo}[2]{\left]#1\,;#2\right[}

\usepackage{xlop}%pour écrire des opérations posées (LaTeX effectue les calculs lui-même)

\usepackage{bm} % pour l'écriture en gras des formules mathématiques avec \bm

\usepackage{cancel} % pour les simplifications de fractions
\renewcommand\CancelColor{\color{red}}
%\usepackage{siunitx} % écriture de nombres et d'unités
%\sisetup{output-decimal-marker={,},detect-all}
\usepackage[autolanguage,np]{numprint}
%permet les espacement pour les nombres décimaux avec \np{3,12456} en environnement maths ou pas
\DecimalMathComma %supprime l'espace après la virgule dans un nombre

%
\usepackage{dsfont} %écriture des ensemble N, R, C ...
\newcommand{\C}{\mathds C}
\newcommand{\R}{\mathds R}
\newcommand{\Q}{\mathds Q}
\newcommand{\D}{\mathds D}
\newcommand{\Z}{\mathds Z}
\newcommand{\N}{\mathds N}
\newcommand\Ind{\mathds 1} %= fonction indicatrice
\newcommand\p{\mathds P} %= probabilité
\newcommand\E{\mathds E} % Espérance
\newcommand\V{\mathds V} % Variance
\newcommand{\e}{\text{e}}
\newcommand{\dd}{\,\text{d}}

%Nombres complexes
\let\Reold\Re
\renewcommand{\Re}{~\text{Re}~}
\let\Imold\Im
\renewcommand{\Im}{~\text{Im}~}
\newcommand{\ii}{\,\text{i}}
% Exponentielle complexe
\newcommand{\ei}[2]{\,\e^{\dfrac{#1\ii\pi}{#2}}}


%
\usepackage{mathrsfs}   % Police de maths jolie caligraphie
\newcommand{\calig}[1]{\ensuremath{\mathscr{#1}}}
\newcommand\mtc[1]{\ensuremath{\mathcal{#1}}}


%Gestion des espaces
%
%\newcommand{\pv}{\ensuremath{\: ; \,}}
\newcommand{\pv}{\ensuremath{\: ;}}
\newlength{\EspacePV}
\setlength{\EspacePV}{1em plus 0.5em minus 0.5em}
\newcommand{\qq}{\hspace{\EspacePV} ; \hspace{\EspacePV}}
\newcommand{\qetq}{\hspace{\EspacePV} \text{et} \hspace{\EspacePV}}
\newcommand{\qouq}{\hspace{\EspacePV} \text{ou} \hspace{\EspacePV}}
\newcommand{\qLq}{\hspace{\EspacePV} \Leftarrow \hspace{\EspacePV}}
\newcommand{\qRq}{\hspace{\EspacePV} \Rightarrow \hspace{\EspacePV}}
\newcommand{\qLRq}{\hspace{\EspacePV} \Leftrightarrow \hspace{\EspacePV}}

%simplification notation norme \norme{}
\newcommand{\norme}[1]{\left\Vert #1\right\Vert}


%simplification de la notation de vecteur \vect{}
\newcommand{\vect}[1]{\mathchoice%
{\overrightarrow{\displaystyle\mathstrut#1\,\,}}%
{\overrightarrow{\textstyle\mathstrut#1\,\,}}%
{\overrightarrow{\scriptstyle\mathstrut#1\,\,}}%
{\overrightarrow{\scriptscriptstyle\mathstrut#1\,\,}}}



%Repères
\def\Oij{$\left(\text{O}\pv\vect{\imath},~\vect{\jmath}\right)$\xspace}
\def\Oijk{$\left(\text{O}\pv\vect{\imath},~ \vect{\jmath},~ \vect{k}\right)$\xspace}
\def\Ouv{$\left(\text{O}\pv\vect{u},~\vect{v}\right)$\xspace}
\def\OIJ{$\left(O\pv I\:,\,J\right)$\xspace}

\newcommand\abs[1]{\ensuremath{\left\vert #1 \right\vert}}%valeur absolue
\newcommand\Arc[1]{\ensuremath{\wideparen{#1}}}%arc de cercle


%symbole pour variable aléatoire qui suit une loi
\newcommand{\suit}{\hookrightarrow}

%encadrer un résultat en rouge
\newcommand\pfr[1]{\psframebox[linecolor=red]{#1}}

%___________________________
%===    Pour les tableaux
%------------------------------------------------------
\usepackage{array}
\usepackage{longtable}
\usepackage{tabularx,tabulary}
\usepackage{multirow}
\usepackage{multicol}
%exemple
%\begin{multicols}{3}[Titre sur une seule colonne.]
%   3~colonnes équilibrées, 3~colonnes équilibrées, 3~colonnes équilibrées, 3~colonnes équilibrées
%\end{multicols}
%\begin{multicols}{2}[\section{Titre numéroté.}]
%   blabla sur deux colonnes, c'est plus sérieux. C'est le style qui est généralement utilisé pour écrire des articles.
%saut de colonne forcé :
%\columnbreak
%djhskjdhjsq
%sdkksqjhd
%\end{multicols}
%Pour ajouter un titre numéroté qui apparaisse sur toute la largeur de la page, il faut utiliser l'option [\section{Titre.}] juste après \begin{multicols}{nb-col}.
%Remarques :
%Pour qu'une ligne de séparation apparaisse entre les colonnes, il faut utiliser : \setlength{\columnseprule}{1pt}.

%Pour redéfinir la largeur de l'espace inter-colonnes, il faut utiliser \setlength{\columnsep}{30pt}.

%Pour remonter le texte, dans chaque colonne vers le haut : \raggedcolumns qui se tape :\begin{multicols}{2}\raggedcolumns...\columnbreak...\columnbreak\end{multicols}

%Pour supprimer les traits verticaux : \setlength{\columnseprule}{0pt} avant \begin{multicols}{3}...\end{multicols}
\setlength\columnseprule{0.4pt}
\renewcommand{\arraystretch}{1.5}%augmente la hauteur des lignes des tableaux
%colonnes centrées verticalement et horizontalement permettant d'écrire des paragraphes de largeur fixée du type M{3cm}
\newcolumntype{M}[1]{>{\centering\arraybackslash}m{#1}}%cellule centrée horizontalement et verticalement
\newcolumntype{R}[1]{>{\raggedleft\arraybackslash}m{#1}}%cellule alignée à droite et centrée verticalement
%\arraybackslash permet de continuer à utiliser \\ pour le changement de ligne

\usepackage{arydshln}% permet des filets horizontaux ou verticaux en pointillés avec
%pour les filets horizontaux \hdashline ou \cdashline qui s'utilisent comme \hline ou \cline
% pour les filets verticaux les deux points :


%___________________________
%===    Divers packages
%------------------------------------------------------
\usepackage{textcomp}

\usepackage{soul} % Pour souligner : \ul
\usepackage{ulem} % Pour souligner double : \uuline
                      % Pour souligner ondulé : \uwave
                      % Pour barrer horizontal : \sout
                      % Pour barrer diagonal : \xout
\usepackage{tikz,tikz-3dplot}
\usetikzlibrary{calc,shapes,arrows,plotmarks,lindenmayersystems,decorations,decorations.markings,decorations.pathmorphing,
decorations.pathreplacing,patterns,positioning,decorations.text}
\usetikzlibrary{shadows,trees}
\usepackage{pstricks,pst-plot,pst-text,pstricks-add,pst-eucl,pst-all}

\usepackage{pgfplots}

\usepackage{tkz-base,tkz-fct,tkz-euclide,tkz-tab,tkz-graph}
\usetkzobj{all}


%INTERLIGNES
\usepackage{setspace}
%s'utilise avec \begin{spacing}{''facteur''}
%   […]
%\end{spacing}

%Pointillés sur toute la ligne
\usepackage{multido}
\newcommand{\Pointilles}[1][1]{%
\multido{}{#1}{\makebox[\linewidth]{\dotfill}\\[1.5\parskip]
}}
%commandes : \Pointilles ou \Pointilles[4] pour 4 lignes


%textes à trous
\newlength\lgtrou
\newcommand*\trou[1]{%
\settowidth\lgtrou{#1}%
\makebox[2\lgtrou]{\dotfill}
\setlength\baselineskip{1.2\baselineskip}}
%Commande à utiliser : \trou{texte qui sera remplacé par des pointillés}

%divers cadres
\usepackage{fancybox} % par exemple \ovalbox{}

%caractères spéciaux avec la commande \ding{230} par exemple
\usepackage{pifont}


%autres symboles
\usepackage{marvosym}
\usepackage{bclogo}
\usepackage{tikzsymbols}

%___________________________
%===    Quelques raccourcis perso
%------------------------------------------------------

%checked box
\newcommand{\checkbox}{
\makebox[0pt][l]{$\square$}\raisebox{.15ex}{\hspace{0.1em}$\checkmark$}
}

%QCM
%\dingsquare %carré avant V ou F
%\dingchecksquare %carré validé devant V ou F


%QRcode généré par le package qrcode
\usepackage{qrcode}


%Texte en filigrane
\usepackage{watermark}
%On utilise ensuite les commandes \watermark, \leftwatermark, \rightwatermark ou \thiswatermark qui permettent de définir un filigrane sur toutes les pages, les pages paires, les pages impaires ou juste une page
%Exemple : \thiswatermark {
%\begin{minipage}{0.95\linewidth}
%\vspace{25cm}
%\begin{center}
%\rotatebox{55}{\scalebox{8}{\color[gray]{0.7}\LaTeX}}
%\end{center}
%\end{minipage}
%}

%Rond entourant une lettre avec pour arguments la couleur de fond, puis la lettre
\newcommand\rond[2][red!20]{\tikz[baseline]{\node[fill=#1,anchor=base,circle]{\bf #2};}}


%Ecrire card en écriture normale :
\newcommand{\card}{\text{card}\xspace}


%___________________________
%===    ALGORITHMES
%------------------------------------------------------

%Autres packages de Stéphane Pasquet
\usepackage{pas-algo}
\usepackage{tcolorbox}

%exemple :
%\begin{center}
%\textbf{À compiler en pdfLaTeX}
%\end{center}
%
%
%\begin{center}
%\begin{algo}[somsuitar]{Calcul d'une somme}
%\begin{algovar}
%r est un nombre réel \\
%u est un nombre réel \\
%n est un entier naturel \\
%i est un entier naturel \\
%S est un nombre réel
%\end{algovar}
%\begin{algoentries}
%r est un nombre réel \\
%u est un nombre réel \\
%n est un entier naturel \\
%i est un entier naturel \\
%S est un nombre réel
%\end{algoentries}
%\begin{algoinit}
%Affecter à S la valeur u\\
%Entrer la valeur de r ( raison de la suite arithmétique )\\
%Entrer la valeur de u ( premier terme de la somme )\\
%Entrer la valeur de n ( nombre de termes dans la somme )
%\end{algoinit}
%\begin{algobody}
%\begin{algofor}{i}{1}{n-1}
%Affecter à S la valeur S+(u+i*r)
%\end{algofor}
%\end{algobody}
%\begin{algoend}
%Afficher S
%\end{algoend}
%\end{algo}
%\end{center}
%
%
%
%L'algorithme \ref{algo:somsuitar} permet de calculer la somme 
%$u_p+u_{p +1}+ u_{p +2}+\cdots +u_{p+n -1}$ , où $(u)$ est une suite
%arithmétique de raison $r$. La valeur de $u$ saisie lors de l'initialisation est la valeur de $u_p$.

%Algorithme sur Casio
\newcommand{\RetourChariot}{\Pisymbol{psy}{191}}


%Nécessaire pour l'environnement lslisting
\usepackage{listings}
%\begin{lstlisting}[language=Python]
%# Calcul de la factorielle
%def factorielle(x):
%	if x < 2:
%		return 1
%	else:
%		return x * factorielle(x-1)
%str(5) + "! = " + str(factorielle(5))
%\end{lstlisting}

%___________________________
%===    MISE EN FORME EXERCICES
%------------------------------------------------------
\usepackage{slashbox}

\newcounter{exo}
\newenvironment{exo}{%
  \refstepcounter{exo}\Writinghand\ \textbf{Exercice \theexo.}\par
  \medskip}%
{\[*\]}


%___________________________
%===    HYPERLIENS
%------------------------------------------------------
\usepackage[colorlinks=true,linkcolor=black,filecolor=blue,urlcolor=blue,bookmarksnumbered]{hyperref} 


%___________________________
%===    SOMMAIRE DANS LES CHAPITRES
%------------------------------------------------------

\usepackage{minitoc}

%___________________________
%===    TABLEUR
%------------------------------------------------------
\usepackage{pas-tableur}%package de Stéphane Pasquet
\usepackage{xstring}
\usepackage{xkeyval}


%___________________________
%===    touches calculatrices
%------------------------------------------------------

\newcommand{\touche}[1]{\begin{pspicture}(0,0)(0.9,0.4)\psframe[framearc=0.5,shadow=true,shadowcolor=gray!50](0,0)(0.8,0.45)\rput[cc](0.4,
0.225){#1}\end{pspicture}} %touche calculatrice
\newcommand{\gtouche}[1]{\begin{pspicture}(0,0)(1.8,0.4)\psframe[framearc=0.5,shadow=true,shadowcolor=gray!50](0,0)(1.6,0.45)\rput[cc](0.8,
0.225){#1}\end{pspicture}} %grande touche calculatrice
\newcommand{\ggtouche}[1]{\begin{pspicture}(0,0)(2.4,0.4)\psframe[framearc=0.5,shadow=true,shadowcolor=gray!50](0,0)(2.4,0.45)\rput[cc](1.2,
0.225){#1}\end{pspicture}} %grande touche calculatrice

%\usepackage{tipfr}


%___________________________
%===   Redéfinition des marges par défaut
%------------------------------------------------------
%\usepackage[textwidth=18.6cm]{geometry}%à mettre dans le preambule perso
%\pagestyle{fancy}%à mettre dans le preambule perso


\setlength\paperheight{297mm}
\setlength\paperwidth{210mm}
\setlength{\evensidemargin}{0cm}% Marge gauche sur pages paires
\setlength{\oddsidemargin}%{0cm}%
{-0.5cm}% Marge gauche sur pages impaires
\setlength{\topmargin}{-2cm}% Marge en haut
\setlength{\headsep}{0.5cm}% Entre le haut de page et le texte
\setlength{\headheight}{0.7cm}% Haut de page
\setlength{\textheight}{25.2cm}% Hauteur de la zone de texte
\setlength{\textwidth}{17cm}% Largeur de la zone de texte


% Environnement enumerate
\renewcommand{\theenumi}{\bf\textsf{\arabic{enumi}}}
\renewcommand{\labelenumi}{\bf\textsf{\theenumi.}}
\renewcommand{\theenumii}{\bf\textsf{\alph{enumii}}}
\renewcommand{\labelenumii}{\bf\textsf{\theenumii.}}
\renewcommand{\theenumiii}{\bf\textsf{\roman{enumiii}}}
\renewcommand{\labelenumiii}{\bf\textsf{\theenumiii.}}


%definition des couleurs
\definecolor{fondpaille}{cmyk}{0,0,0.1,0}%\pagecolor{fondpaille}
\definecolor{gris}{rgb}{0.7,0.7,0.7}
\definecolor{rouge}{rgb}{1,0,0}
\definecolor{bleu}{rgb}{0,0,1}
\definecolor{vert}{rgb}{0,1,0}
\definecolor{deficolor}{HTML}{2D9AFF}
\definecolor{backdeficolor}{HTML}{EDEDED}%{036DD0}%dégradé bleu{666666}%dégradé gris
\definecolor{theocolor}{HTML}{C10CC7}%{HTML}{036DD0}%F4404D%rouge
\definecolor{backtheocolor}{HTML}{D3D3D3}
\definecolor{methcolor}{HTML}{008800}%12BB05}
\definecolor{backmethcolor}{HTML}{FFFACD}
\definecolor{backilluscolor}{HTML}{EDEDED}
\definecolor{sectioncolor}{HTML}{C10CC7}%{B2B2B2}%vert : {HTML}{008800}%{HTML}{2D9AFF}
\definecolor{subsectioncolor}{HTML}{C10CC7}%{B2B2B2}%vert : {HTML}{008800}%{rgb}{0.5,0,0}
\definecolor{subsubsectioncolor}{HTML}{C10CC7}
\definecolor{engcolor}{HTML}{D4D7FE}
\definecolor{exocolor}{rgb}{0,0.6,0}
\definecolor{exosoltitlecolor}{rgb}{0,0.6,0}
\definecolor{titlecolor}{rgb}{1,1,1}

%commande pour enlever les couleurs avant impression
\newcommand{\nocolor}
{\pagecolor{white}
\definecolor{gris}{rgb}{0.7,0.7,0.7}
\definecolor{rouge}{rgb}{0,0,0}
\definecolor{bleu}{rgb}{0,0,0}
\definecolor{vert}{rgb}{0,0,0}
\definecolor{deficolor}{HTML}{B2B2B2}
\definecolor{backdeficolor}{HTML}{EEEEEE}%{036DD0}%dégradé bleu{666666}%dégradé gris
\definecolor{theocolor}{HTML}{B2B2B2}
\definecolor{backtheocolor}{HTML}{EEEEEE}
\definecolor{methcolor}{HTML}{B2B2B2}
\definecolor{backmethcolor}{HTML}{EEEEEE}
\definecolor{backilluscolor}{HTML}{EEEEEE}
\definecolor{sectioncolor}{HTML}{B2B2B2}
\definecolor{subsectioncolor}{HTML}{B2B2B2}
\definecolor{subsubsectioncolor}{HTML}{B2B2B2}
\definecolor{engcolor}{HTML}{EEEEEE}
\definecolor{exocolor}{HTML}{3B3838}
\definecolor{exosoltitlecolor}{rgb}{0,0,0}
\definecolor{titlecolor}{rgb}{0,0,0}
}



%%%%%%%%%%%%%%%%%%%%%%%%%%%%%%%%%%%%%%%%%%%%%%%%%%%%%%%%%%%%%%%%%%%%%%%%%%%%%%%
%Encadrés pour Propriétés, Théorème, Définitions, exemples, exercices

\usepackage{environ}%pour pouvoir utiliser la commande \NewEnviron

%___________________________
%===    Propriété avec ou sans s et avec ou sans titre
%------------------------------------------------------
%
\NewEnviron{Prop}[2][]{
\begin{tikzpicture}[node distance=0 cm]
\node[fill=theocolor,rounded corners=5pt,anchor=south west] (theorem) at (0,0)
{\textcolor{titlecolor}{Propriété#1~:~#2}};
\node[draw,drop shadow,color=theocolor,very thick,fill=backtheocolor,rounded corners=5pt,anchor=north west] at(0,-0.02)
{\black\parbox{\columnwidth-12pt}{\BODY}};
\end{tikzpicture}
\bigskip
}


%___________________________
%===    Conséquence avec ou sans s et avec ou sans titre
%------------------------------------------------------
%
\NewEnviron{Cons}[2][]{
\begin{tikzpicture}[node distance=0 cm]
\node[fill=theocolor,rounded corners=5pt,anchor=south west] (theorem) at (0,0)
{\textcolor{titlecolor}{Conséquence#1~:~#2}};
\node[draw,drop shadow,color=theocolor,very thick,fill=backtheocolor,rounded corners=5pt,anchor=north west] at(0,-0.02)
{\black\parbox{\columnwidth-12pt}{\BODY}};
\end{tikzpicture}
\bigskip
}


%___________________________
%===    Corolaire avec ou sans titre
%------------------------------------------------------
%
\NewEnviron{Cor}[1][]{
\begin{tikzpicture}[node distance=0 cm]
\node[fill=theocolor,rounded corners=5pt,anchor=south west] (theorem) at (0,0)
{\textcolor{titlecolor}{Corollaire~:~#1}};
\node[draw,drop shadow,color=theocolor,very thick,fill=backtheocolor,rounded corners=5pt,anchor=north west] at(0,-0.02)
{\black\parbox{\columnwidth-12pt}{\BODY}};
\end{tikzpicture}
\bigskip
}


%___________________________
%===    Théorème avec ou sans titre
%------------------------------------------------------
%
\NewEnviron{Thm}[1][]{
\begin{tikzpicture}[node distance=0 cm]
\node[fill=theocolor,rounded corners=5pt,anchor=south west] (theorem) at (0,0)
{\textcolor{titlecolor}{Théorème~:~#1}};
\node[draw,drop shadow,color=theocolor,very thick,fill=backtheocolor,rounded corners=5pt,anchor=north west] at(0,-0.02)
{\black\parbox{\columnwidth-12pt}{\BODY}};
\end{tikzpicture}
\medskip
}


%___________________________
%===    Cadre arrondi coloré
%------------------------------------------------------
%
\NewEnviron{CadreColor}{
\medskip
\begin{tikzpicture}[node distance=0 cm]
\node[draw,drop shadow,color=theocolor,very thick,fill=backtheocolor,rounded corners=5pt,anchor=north west] at(0,-0.02)
{\black\parbox{\linewidth-12pt}{\BODY}};
\end{tikzpicture}
%\medskip
}


%___________________________
%===    Cadre arrondi blanc
%------------------------------------------------------
%
\NewEnviron{Cadre}{
\medskip
\begin{tikzpicture}[node distance=0 cm]
\node[draw,very thick,rounded corners=5pt,anchor=north west] at(0,-0.02)
{\black\parbox{\linewidth-12pt}{\BODY}};
\end{tikzpicture}
%\medskip
}



%___________________________
%===    Règle(s) avec ou sans s et avec sans titre
%------------------------------------------------------
%
\NewEnviron{Regle}[2][]{
\begin{tikzpicture}[node distance=0 cm]
\node[fill=theocolor,rounded corners=5pt,anchor=south west] (theorem) at (0,0)
{\textcolor{titlecolor}{Règle#1~:~#2}};
\node[draw,drop shadow,color=deficolor,very thick,fill=backdeficolor,rounded corners=5pt,anchor=north west] at(0,-0.02)
{\black\parbox{\columnwidth-12pt}{\BODY}};
\end{tikzpicture}
\medskip
}

%___________________________
%===    Définition avec ou sans s et avec sans titre
%------------------------------------------------------
%
\NewEnviron{Defi}[2][]{
\begin{tikzpicture}[node distance=0 cm]
\node[fill=theocolor,rounded corners=5pt,anchor=south west] (theorem) at (0,0)
{\textcolor{titlecolor}{Définition#1~:~#2}};
\node[draw,drop shadow,color=deficolor,very thick,fill=backdeficolor,rounded corners=5pt,anchor=north west] at(0,-0.02)
{\black\parbox{\columnwidth-12pt}{\BODY}};
\end{tikzpicture}
\medskip
}

%___________________________
%===    Méthode avec ou sans s et avec sans titre
%------------------------------------------------------
%
\NewEnviron{Methode}[2][]{
\begin{tikzpicture}[node distance=0 cm]
\node[fill=theocolor,rounded corners=5pt,anchor=south west] (theorem) at (0,0)
{\textcolor{titlecolor}{Méthode#1~:~#2}};
\node[draw,drop shadow,color=methcolor,very thick,fill=backmethcolor,rounded corners=5pt,anchor=north west] at(0,-0.02)
{\black\parbox{\columnwidth-12pt}{\BODY}};
\end{tikzpicture}
\medskip
}


%___________________________
%===    Redéfinition de la commande \chapter{•}
%------------------------------------------------------
%
\makeatletter

\renewcommand{\@makechapterhead}[1]{
\begin{tikzpicture}
\node[fill=theocolor,rectangle,rounded corners=5pt]{%
\begin{minipage}{\linewidth}
\begin{center}
\vspace*{9pt}
\textcolor{titlecolor}{\Large \textsc{\textbf{Chapitre \thechapter \ : \ #1}}}
\vspace*{9pt}
\end{center}
\end{minipage}
};\end{tikzpicture}
}

\makeatother


%___________________________
%===    Exemple avec ou sans s et avec ou sans titre
%------------------------------------------------------
%
\NewEnviron{Exemple}[2][]{
\begin{tikzpicture}[node distance=0 cm]
\node[draw,drop shadow,color=methcolor,very thick,fill=backmethcolor,rounded corners=5pt,anchor=north west] at(0,-0.02)
{\black\parbox{\columnwidth-12pt}{\textbf{Exemple#1~:~#2}\\
\BODY}};
\end{tikzpicture}
\medskip
}

%___________________________
%===    Remarque avec ou sans s
%------------------------------------------------------
%
\NewEnviron{Rmq}[1][]{
\textbf{\large{Remarque#1 :}}\par
\BODY
\medskip
}

%___________________________
%===    Remarques numérotées R1, R2, etc...
%------------------------------------------------------
%
\newcounter{rem}\newcommand{\rem}{\refstepcounter{rem}\textbf{R \therem \ :}\xspace}

%___________________________
%===    Exercices du contrôle numérotés
%------------------------------------------------------
%
\newcounter{exercice}
\NewEnviron{Exercice}[1][]{
\refstepcounter{exercice}\textbf{\large{Exercice \theexercice \ :}}\hfill \textbf{#1}\par
\BODY
\medskip
}

%___________________________
%===    Exercices non numérotés
%------------------------------------------------------
%
\NewEnviron{Exo}[1][]{
\textbf{\large{Exercice #1 \ :}}\par
\BODY
\medskip
}

%___________________________
%===    Démonstration
%------------------------------------------------------
\NewEnviron{Demo}[1][]{%
\textit{\textbf{Démonstration #1}}\par
\BODY
\strut\hfill$\square$
\medskip
}

%___________________________
%===    Commandes perso
%------------------------------------------------------
%
%Ancienne commande chapitre
\newcommand{\chapitre}[1]{
\begin{tikzpicture}
\node[fill=sectioncolor,rectangle,rounded corners=5pt]{%
\begin{minipage}{\linewidth}
\begin{center}
\vspace*{9pt}
\textcolor{titlecolor}{\Large \textsc{\textbf{#1}}}
\vspace*{9pt}
\end{center}
\end{minipage}
};
\end{tikzpicture}
\bigskip
}

%Pour les fiches : commande de Cécile
\newcommand{\Fiche}[2]{%
\begin{tikzpicture}
	\node[draw, color=blue,fill=white,rectangle,rounded corners=5pt]{%
	\begin{minipage}{\linewidth}
		\begin{center}
			\vspace*{9pt}
			\textcolor{blue}{\Large \textsc{\textbf{Fiche~#1\ :\ #2}}}
			\vspace*{7pt}
		\end{center}
	\end{minipage}
	};
\end{tikzpicture}
}%

\pagecolor{white}%couleur du fond de page


%centrer du texte ou une formule avec moins d'espace autour
\newcommand{\centrer}[1]
{
\vspace*{-12pt}
\begin{center}
#1
\end{center}
\vspace*{-12pt}
}

%Pour pouvoir utiliser l'environnement verbatim
\usepackage{verbatim}

%Panneau danger (nécessite le package pstricks)
\def\danger{\begingroup
\psset{unit=1ex}%
\begin{pspicture}(0,0)(3,3)
\pspolygon[linearc=0.2,linewidth=0.12,linecolor=red](0,0)(1.5,2.6)(3,0)
\psellipse*(1.5,1.33)(0.14,0.75)\pscircle*(1.5,0.3){0.15}\end{pspicture}
\endgroup}%

\newcommand{\cad}{c'est-à-dire }


%___________________________
%===    Nouvelles commandes pour documents venant de Sesamath
%------------------------------------------------------
%
\definecolor{CyanTikz40}{cmyk}{.4,0,0,0}
\definecolor{CyanTikz20}{cmyk}{.2,0,0,0}
\tikzstyle{general}=[line width=0.3mm, >=stealth, x=1cm, y=1cm,line cap=round, line join=round]
\tikzstyle{quadrillage}=[line width=0.3mm, color=CyanTikz40]
\tikzstyle{quadrillageNIV2}=[line width=0.3mm, color=CyanTikz20]
\tikzstyle{quadrillage55}=[line width=0.3mm, color=CyanTikz40, xstep=0.5, ystep=0.5]
\tikzstyle{cote}=[line width=0.3mm, <->]
\tikzstyle{epais}=[line width=0.5mm, line cap=butt]
\tikzstyle{tres epais}=[line width=0.8mm, line cap=butt]
\tikzstyle{axe}=[line width=0.3mm, ->, color=Noir, line cap=rect]
\newcommand{\quadrillageSeyes}[2]{\draw[line width=0.3mm, color=A1!10, ystep=0.2, xstep=0.8] #1 grid #2;
\draw[line width=0.3mm, color=A1!30, xstep=0.8, ystep=0.8] #1 grid #2; }
\newcommand{\axeX}[4][0]{\draw[axe] (#2,#1)--(#3,#1); \foreach \x in {#4} {\draw (\x,#1) node {\small $+$}; \draw (\x,#1) node[below] {\small $\x$};}}
\newcommand{\axeY}[4][0]{\draw[axe] (#1,#2)--(#1,#3); \foreach \y in {#4} {\draw (#1, \y) node {\small $+$}; \draw (#1, \y) node[left] {\small $\y$};}}
\newcommand{\axeOI}[3][0]{\draw[axe] (#2,#1)--(#3,#1);  \draw (1,#1) node {\small $+$}; \draw (1,#1) node[below] {\small $I$};}
\newcommand{\axeOJ}[3][0]{\draw[axe] (#1,#2)--(#1,#3); \draw (#1, 1) node {\small $+$}; \draw (#1, 1) node[left] {\small $J$};}
\newcommand{\axeXgraduation}[2][0]{\foreach \x in {#2} {\draw (\x,#1) node {\small $+$};}}
\newcommand{\axeYgraduation}[2][0]{\foreach \y in {#2} {\draw (#1, \y) node {\small $+$}; }}
\newcommand{\origine}{\draw (0,0) node[below left] {\small $0$};}
\newcommand{\origineO}{\draw (0,0) node[below left] {$O$};}
\newcommand{\point}[4]{\draw (#1,#2) node[#4] {$#3$};}
\newcommand{\pointGraphique}[4]{\draw (#1,#2) node[#4] {$#3$};
\draw (#1,#2) node {$+$};}
\newcommand{\pointFigure}[4]{\draw (#1,#2) node[#4] {$#3$};
\draw (#1,#2) node {$\times$};}
\newcommand{\pointC}[3]{\draw (#1) node[#3] {$#2$};}
\newcommand{\pointCGraphique}[3]{\draw (#1) node[#3] {$#2$};
\draw (#1) node {$+$};}
\newcommand{\pointCFigure}[3]{\draw (#1) node[#3] {$#2$};
\draw (#1) node {$\times$};}


\definecolor{B1prime}                {cmyk}{0.00, 1.00, 0.00, 0.50}
\definecolor{H1prime}                {cmyk}{0.50, 0.00, 1.00, 0.00}

\definecolor{FootFonctionColor}{cmyk}{0.50, 0.00, 0.00, 0.00}
\definecolor{FootGeometrieColor}{cmyk}{0.40, 0.40, 0.00, 0.00}
\definecolor{FootStatistiqueColor}{cmyk}{0.30, 0.48, 0.00, 0.10}
\definecolor{FootStatistiqueOLDColor}{cmyk}{0.48, 0.30, 0.10, 0.00}
\definecolor{FootStatistique*Color}{cmyk}{0.20, 0.00, 0.00, 0.00}
\definecolor{ActiviteFootColor}{cmyk}{0.50, 0.00, 0.25, 0.00}
\definecolor{CoursFootColor}{cmyk}{0.15, 0.00, 0.00, 0.03}
\definecolor{ExoBaseFootColor}{cmyk}{0.00, 0.25, 0.50, 0.00}
\definecolor{ExoApprFootColor}{cmyk}{0.00, 0.25, 0.50, 0.00}
%\colorlet{ConnFootColor}{F2}
\definecolor{TPFootColor}{cmyk}{0.00, 0.30, 0.00, 0.10}
\definecolor{RecreationFootColor}{cmyk}{0.20, 0.00, 0.50, 0.05}

\definecolor{Blanc}             {cmyk}{0.00, 0.00, 0.00, 0.00}
\definecolor{Gris1}             {cmyk}{0.00, 0.00, 0.00, 0.20}
\definecolor{Gris2}             {cmyk}{0.00, 0.00, 0.00, 0.40}
\definecolor{Gris3}             {cmyk}{0.00, 0.00, 0.00, 0.50}
\definecolor{Noir}              {cmyk}{0.00, 0.00, 0.00, 1.00}
\definecolor{A1}              {cmyk}{0.33, 1.00, 0.00, 0.40}
\definecolor{F1}              {cmyk}{0.00, 1.00, 1.00, 0.00}
\definecolor{C1}              {cmyk}{0.00, 1.00, 0.00, 0.50}
\definecolor{G1}              {cmyk}{0.00, 0.00, 0.00, 0.20}
\definecolor{D1}              {cmyk}{0.00, 0.22, 0.49, 0.69}%bitume
\definecolor{J1}              {cmyk}{0.00, 0.34, 1.00, 0.02}%orangé


%augmenter l'espace au-dessus ou en-dessous d'une fraction
\makeatletter
\newcommand*\Strut[1][1]{%
  \leavevmode
  \vrule \@height #1\ht\strutbox
         \@depth #1\dp\strutbox
         \@width\z@
}
\newcommand*\TopStrut[1][1]{%
  \leavevmode
  \vrule \@height #1\ht\strutbox
         \@depth \z@
         \@width \z@
}
\newcommand*\BotStrut[1][1]{%
  \leavevmode
  \vrule \@height \z@
         \@depth #1\dp\strutbox
         \@width \z@
}
\makeatother

%exemple
%Résoudre: $  \dfrac{ \TopStrut 3 x -1} { \BotStrut x + 2} < 3$.
%%%%%%%%%%%%%%%%%%%%%%%%%%%%%%%%%%%%%%%%%%%%%%%%%%%%%%%%%%%%%%%%%%%%%%%%%%%%%%%
%%%%%%%%%%%%%%%%%%%%%%%%%%%%%%%%%%%%%%%%%%%%%%%%%%%%%%%%%%%%%%%%%%%%%%%%%%%%%%%
%%%%%%%%%%%%%%%%%%%%%%%%%%%%%%%%%%%%%%%%%%%%%%%%%%%%%%%%%%%%%%%%%%%%%%%%%%%%%%%

%--------------------- PRESENTATION SECTIONS
\usepackage{titlesec}
\setcounter{secnumdepth}{3}
\makeatletter

% couleurs section
\definecolor{section@title@color}{cmyk}{1,0.2,0.3,0.1}
\definecolor{subsection@title@color}{cmyk}{0,0.6,0.9,0}
\definecolor{subsubsection@title@color}{cmyk}{1,0.2,0.3,0.1}
\definecolor{shadow@color}{cmyk}{.07,0,0,0.49}

% fontes section
\def\sectiontitle@font{\fontfamily{ppl}\fontseries{bx}\selectfont}
%\def\subsectiontitle@font{\fontfamily{ppl}\fontseries{bx}\selectfont}
%\def\subsubsectiontitle@font{\fontfamily{ppl}\fontseries{bx}\selectfont}

% Décalages numéro de sections / titres des sections
\newlength\decalnumsec
\newlength\decalnumsubsec
\newlength\decalnumsubsubsec
\setlength{\decalnumsec}{-0.5em}
\setlength{\decalnumsubsec}{-0.5em}
\setlength{\decalnumsubsubsec}{-0.5em}
\newlength\decalxtitlesec
\newlength\decalxtitlesubsec
\newlength\decalxtitlesubsubsec
\setlength{\decalxtitlesec}{-2.45em}
\setlength{\decalxtitlesubsec}{-1em}
\setlength{\decalxtitlesubsubsec}{0.45em}

% Espace entre le numéro de section et le titre
\newlength\spacetitlesec
\newlength\spacetitlesubsec
\newlength\spacetitlesubsubsec
\setlength{\spacetitlesec}{0.2em}
\setlength{\spacetitlesubsec}{0.2em}
\setlength{\spacetitlesubsubsec}{0.2em}

%%%%%%%%%%%%% Titre de section

\renewcommand{\thesection}{\Roman{section}}
\titleformat{\section}[block]
{%
	\hspace*{\decalxtitlesec}
	\bfseries\large
	\color{section@title@color}
	\sectiontitle@font
}
{
\raisebox{\decalnumsec}
{%
\begin{tikzpicture}
\node (numsec) {\sectiontitle@font\thesection};
\fill[rounded corners=2pt,fill=shadow@color] ($(numsec.north west)+(2pt,-2pt)$) -- ($(numsec.north east)+(1mm,0mm)+(2pt,-2pt)$) -- ($(numsec.south east)+(2pt,-2pt)$) -- ($(numsec.south west)+(-1mm,0)+(2pt,-2pt)$) -- cycle;
\fill[rounded corners=2pt,fill=section@title@color] (numsec.north west) -- ($(numsec.north east)+(1mm,0mm)$) -- (numsec.south east) -- ($(numsec.south west)+(-1mm,0)$) -- cycle;
\node[white] at (numsec) {\sectiontitle@font\thesection};
\end{tikzpicture}
}
}
{\spacetitlesec}
{}

%%%%%%%%%%%%% Titre de subsection

\renewcommand{\thesubsection}{\Alph{subsection}}
\titleformat{\subsection}[block]
{%
	\hspace*{\decalxtitlesubsec}
	\bfseries
	\color{subsection@title@color}
	\sectiontitle@font
}
{
	\raisebox{\decalnumsubsec}
	{%
	\begin{tikzpicture}
	\node (numsubsec) {\sectiontitle@font\thesubsection};
	\fill[rounded corners=2pt,fill=shadow@color] ($(numsubsec.north west)+(2pt,-2pt)$) -- ($(numsubsec.north east)+(1mm,0mm)+(2pt,-2pt)$) -- ($(numsubsec.south east)+(2pt,-2pt)$) -- ($(numsubsec.south west)+(-1mm,0)+(2pt,-2pt)$) -- cycle;
	\fill[rounded corners=2pt,fill=subsection@title@color] (numsubsec.north west) -- ($(numsubsec.north east)+(1mm,0mm)$) -- (numsubsec.south east) -- ($(numsubsec.south west)+(-1mm,0)$) -- cycle;
	\node[white] at (numsubsec) {\sectiontitle@font\thesubsection};
	\end{tikzpicture}
	}
}
{%
	\spacetitlesubsec
}
{} 

%%%%%%%%%%%%% Titre de subsubsection

\renewcommand{\thesubsubsection}{\arabic{subsubsection}}
\titleformat{\subsubsection}[block]
{%
	\hspace*{\decalxtitlesubsubsec}
	\bfseries
	\color{section@title@color}
	\sectiontitle@font
}
{
	\raisebox{\decalnumsubsubsec}
	{%
	\begin{tikzpicture}
	\node (numsubsubsec) {\sectiontitle@font\thesubsubsection};
	\fill[rounded corners=2pt,fill=shadow@color] ($(numsubsubsec.north west)+(2pt,-2pt)$) -- ($(numsubsubsec.north east)+(1mm,0mm)+(2pt,-2pt)$) -- ($(numsubsubsec.south east)+(2pt,-2pt)$) -- ($(numsubsubsec.south west)+(-1mm,0)+(2pt,-2pt)$) -- cycle;
	\fill[rounded corners=2pt,fill=subsubsection@title@color] (numsubsubsec.north west) -- ($(numsubsubsec.north east)+(1mm,0mm)$) -- (numsubsubsec.south east) -- ($(numsubsubsec.south west)+(-1mm,0)$) -- cycle;
	\node[white] at (numsubsubsec) {\sectiontitle@font\thesubsubsection};
	\end{tikzpicture}
	}
}
{\spacetitlesubsubsec}
{} 

\makeatother


%%%%%%%%%%%%%%%%%%%%%%%%%%%%%%%%%%%%%

%%___________________________
%===   Redéfinition des marges par défaut
%------------------------------------------------------
%\usepackage[textwidth=18.6cm]{geometry}%à mettre dans le preambule perso
%\pagestyle{fancy}%à mettre dans le preambule perso


\setlength\paperheight{297mm}
\setlength\paperwidth{210mm}
\setlength{\evensidemargin}{0cm}% Marge gauche sur pages paires
\setlength{\oddsidemargin}%{0cm}%
{-0.5cm}% Marge gauche sur pages impaires
\setlength{\topmargin}{-2cm}% Marge en haut
\setlength{\headsep}{0.5cm}% Entre le haut de page et le texte
\setlength{\headheight}{0.7cm}% Haut de page
\setlength{\textheight}{25.2cm}% Hauteur de la zone de texte
\setlength{\textwidth}{17cm}% Largeur de la zone de texte


% Environnement enumerate
\renewcommand{\theenumi}{\bf\textsf{\arabic{enumi}}}
\renewcommand{\labelenumi}{\bf\textsf{\theenumi.}}
\renewcommand{\theenumii}{\bf\textsf{\alph{enumii}}}
\renewcommand{\labelenumii}{\bf\textsf{\theenumii.}}
\renewcommand{\theenumiii}{\bf\textsf{\roman{enumiii}}}
\renewcommand{\labelenumiii}{\bf\textsf{\theenumiii.}}


\usetikzlibrary{shadows,trees}


%definition des couleurs
\definecolor{fondpaille}{cmyk}{0,0,0.1,0}%\pagecolor{fondpaille}
\definecolor{gris}{rgb}{0.7,0.7,0.7}
\definecolor{rouge}{rgb}{1,0,0}
\definecolor{bleu}{rgb}{0,0,1}
\definecolor{vert}{rgb}{0,1,0}
\definecolor{deficolor}{HTML}{2D9AFF}
\definecolor{backdeficolor}{HTML}{EDEDED}%{036DD0}%dégradé bleu{666666}%dégradé gris
\definecolor{theocolor}{HTML}{036DD0}%F4404D%rouge
\definecolor{backtheocolor}{HTML}{D3D3D3}
\definecolor{methcolor}{HTML}{008800}%12BB05}
\definecolor{backmethcolor}{HTML}{FFFACD}
\definecolor{backilluscolor}{HTML}{EDEDED}
\definecolor{sectioncolor}{HTML}{221E1E}%{B2B2B2}%vert : {HTML}{008800}%{HTML}{2D9AFF}
\definecolor{subsectioncolor}{HTML}{221E1E}%{B2B2B2}%vert : {HTML}{008800}%{rgb}{0.5,0,0}
\definecolor{engcolor}{HTML}{D4D7FE}
\definecolor{exocolor}{rgb}{0,0.6,0}
\definecolor{exosoltitlecolor}{rgb}{0,0.6,0}
\definecolor{titlecolor}{rgb}{1,1,1}

%commande pour enlever les couleurs avant impression
\newcommand{\nocolor}
{\pagecolor{white}
\definecolor{gris}{rgb}{0.7,0.7,0.7}
\definecolor{rouge}{rgb}{0,0,0}
\definecolor{bleu}{rgb}{0,0,0}
\definecolor{vert}{rgb}{0,0,0}
\definecolor{deficolor}{HTML}{B2B2B2}
\definecolor{backdeficolor}{HTML}{EEEEEE}%{036DD0}%dégradé bleu{666666}%dégradé gris
\definecolor{theocolor}{HTML}{B2B2B2}
\definecolor{backtheocolor}{HTML}{EEEEEE}
\definecolor{methcolor}{HTML}{B2B2B2}
\definecolor{backmethcolor}{HTML}{EEEEEE}
\definecolor{backilluscolor}{HTML}{EEEEEE}
\definecolor{sectioncolor}{HTML}{B2B2B2}
\definecolor{subsectioncolor}{HTML}{B2B2B2}
\definecolor{engcolor}{HTML}{EEEEEE}
\definecolor{exocolor}{HTML}{3B3838}
\definecolor{exosoltitlecolor}{rgb}{0,0,0}
\definecolor{titlecolor}{rgb}{0,0,0}
}



%___________________________
%===    Exercice résolu
%------------------------------------------------------
%
%#1 : énoncé
%#2 : solution
\newcounter{exosol}
\newcommand{\exosol}[2]{
\stepcounter{exosol}
\begin{tikzpicture}[node distance=0 cm]
\node[fill=backilluscolor,rounded corners=2pt,anchor=south west] (illus) at (0,-0.02)
{\it \textbf{\textcolor{exosoltitlecolor}{Exercice résolu \arabic{exosol}~:~}}};
\node[fill=backilluscolor,rounded corners=2pt,anchor=north west]at(0,0)
{\parbox{\columnwidth-10pt}{#1\par\medskip{\it \textbf{\textcolor{exosoltitlecolor}{Solution~:~}}}\par#2 }};
\end{tikzpicture}
\bigskip
}

\newcommand{\suite}[1]{
\begin{tikzpicture}[node distance=0 cm]
\node[fill=backilluscolor,rounded corners=2pt,anchor=north west]at(0,0)
{\parbox{\columnwidth-10pt}{{\it \textbf{\textcolor{exosoltitlecolor}{Suite de la solution~:}}}\par#1}};
\end{tikzpicture}
\bigskip
}



%%%%%%%%%%%%%%%%%%%%%%%%%%%%%%%%%%%%%%%%%%%%%%%%%%%%%%%%%%%%%%%%%%%%%%%%%%%%%%%
%Encadrés pour Propriétés, Théorème, Définitions, exemples, exercices

\usepackage{environ}%pour pouvoir utiliser la commande \NewEnviron

%___________________________
%===    Propriété avec ou sans s et avec ou sans titre
%------------------------------------------------------
%
\NewEnviron{Prop}[2][]{
\begin{tikzpicture}[node distance=0 cm]
\node[fill=theocolor,rounded corners=5pt,anchor=south west] (theorem) at (0,0)
{\textcolor{titlecolor}{Propriété#1~:~#2}};
\node[draw,drop shadow,color=theocolor,very thick,fill=backtheocolor,rounded corners=5pt,anchor=north west] at(0,-0.02)
{\black\parbox{\columnwidth-12pt}{\BODY}};
\end{tikzpicture}
\bigskip
}


%___________________________
%===    Conséquence avec ou sans s et avec ou sans titre
%------------------------------------------------------
%
\NewEnviron{Cons}[2][]{
\begin{tikzpicture}[node distance=0 cm]
\node[fill=theocolor,rounded corners=5pt,anchor=south west] (theorem) at (0,0)
{\textcolor{titlecolor}{Conséquence#1~:~#2}};
\node[draw,drop shadow,color=theocolor,very thick,fill=backtheocolor,rounded corners=5pt,anchor=north west] at(0,-0.02)
{\black\parbox{\columnwidth-12pt}{\BODY}};
\end{tikzpicture}
\bigskip
}


%___________________________
%===    Théorème avec ou sans titre
%------------------------------------------------------
%
\NewEnviron{Thm}[1][]{
\begin{tikzpicture}[node distance=0 cm]
\node[fill=theocolor,rounded corners=5pt,anchor=south west] (theorem) at (0,0)
{\textcolor{titlecolor}{Théorème~:~#1}};
\node[draw,drop shadow,color=theocolor,very thick,fill=backtheocolor,rounded corners=5pt,anchor=north west] at(0,-0.02)
{\black\parbox{\columnwidth-12pt}{\BODY}};
\end{tikzpicture}
\medskip
}


%___________________________
%===    Cadre arrondi coloré
%------------------------------------------------------
%
\NewEnviron{CadreColor}{
\begin{tikzpicture}[node distance=0 cm]
\node[draw,drop shadow,color=theocolor,very thick,fill=backtheocolor,rounded corners=5pt,anchor=north west] at(0,-0.02)
{\black\parbox{\columnwidth-12pt}{\BODY}};
\end{tikzpicture}
\medskip
}


%___________________________
%===    Cadre arrondi blanc
%------------------------------------------------------
%
\NewEnviron{Cadre}{
\begin{tikzpicture}[node distance=0 cm]
\node[draw,very thick,rounded corners=5pt,anchor=north west] at(0,-0.02)
{\black\parbox{\columnwidth-12pt}{\BODY}};
\end{tikzpicture}
\medskip
}



%___________________________
%===    Règle(s) avec ou sans s et avec sans titre
%------------------------------------------------------
%
\NewEnviron{Regle}[2][]{
\begin{tikzpicture}[node distance=0 cm]
\node[fill=theocolor,rounded corners=5pt,anchor=south west] (theorem) at (0,0)
{\textcolor{titlecolor}{Règle#1~:~#2}};
\node[draw,drop shadow,color=deficolor,very thick,fill=backdeficolor,rounded corners=5pt,anchor=north west] at(0,-0.02)
{\black\parbox{\columnwidth-12pt}{\BODY}};
\end{tikzpicture}
\medskip
}

%___________________________
%===    Définition avec ou sans s et avec sans titre
%------------------------------------------------------
%
\NewEnviron{Defi}[2][]{
\begin{tikzpicture}[node distance=0 cm]
\node[fill=theocolor,rounded corners=5pt,anchor=south west] (theorem) at (0,0)
{\textcolor{titlecolor}{Définition#1~:~#2}};
\node[draw,drop shadow,color=deficolor,very thick,fill=backdeficolor,rounded corners=5pt,anchor=north west] at(0,-0.02)
{\black\parbox{\columnwidth-12pt}{\BODY}};
\end{tikzpicture}
\medskip
}

%___________________________
%===    Méthode avec ou sans s et avec sans titre
%------------------------------------------------------
%
\NewEnviron{Methode}[2][]{
\begin{tikzpicture}[node distance=0 cm]
\node[fill=theocolor,rounded corners=5pt,anchor=south west] (theorem) at (0,0)
{\textcolor{titlecolor}{Méthode#1~:~#2}};
\node[draw,drop shadow,color=methcolor,very thick,fill=backmethcolor,rounded corners=5pt,anchor=north west] at(0,-0.02)
{\black\parbox{\columnwidth-12pt}{\BODY}};
\end{tikzpicture}
\medskip
}


%___________________________
%===    Redéfinition de la commande \chapter{•}
%------------------------------------------------------
%
\makeatletter

\renewcommand{\@makechapterhead}[1]{
\begin{tikzpicture}
\node[fill=theocolor,rectangle,rounded corners=5pt]{%
\begin{minipage}{\linewidth}
\begin{center}
\vspace*{9pt}
\textcolor{titlecolor}{\Large \textsc{\textbf{Chapitre \thechapter \ : \ #1}}}
\vspace*{9pt}
\end{center}
\end{minipage}
};\end{tikzpicture}
}

\makeatother


%___________________________
%===    Exemple avec ou sans s et avec ou sans titre
%------------------------------------------------------
%
\NewEnviron{Exemple}[2][]{
\begin{tikzpicture}[node distance=0 cm]
\node[draw,drop shadow,color=methcolor,very thick,fill=backmethcolor,rounded corners=5pt,anchor=north west] at(0,-0.02)
{\black\parbox{\columnwidth-12pt}{\textbf{Exemple#1~:~#2}\\
\BODY}};
\end{tikzpicture}
\medskip
}

%___________________________
%===    Remarque avec ou sans s
%------------------------------------------------------
%
\NewEnviron{Rmq}[1][]{
\textbf{\large{Remarque#1 :}}\par
\BODY
\medskip
}

%___________________________
%===    Remarques numérotées R1, R2, etc...
%------------------------------------------------------
%
\newcounter{rem}\newcommand{\rem}{\refstepcounter{rem}\textbf{R \therem \ :}\xspace}

%___________________________
%===    Exercices du contrôle numérotés
%------------------------------------------------------
%
\newcounter{exercice}
\NewEnviron{Exercice}[1][]{
\refstepcounter{exercice}\textbf{\large{Exercice \theexercice \ :}}\hfill \textbf{#1}\par
\BODY
\medskip
}

%___________________________
%===    Exercices non numérotés
%------------------------------------------------------
%
\NewEnviron{Exo}[1][]{
\textbf{\large{Exercice #1 \ :}}\par
\BODY
\medskip
}

%___________________________
%===    Démonstration
%------------------------------------------------------
\NewEnviron{Demo}{%
\textit{\textbf{Démonstration.}}\par
\BODY
\strut\hfill$\square$
\medskip
}

%___________________________
%===    Commandes perso
%------------------------------------------------------
%
%\Leftrightarrow
\newcommand{\Lr}{\Leftrightarrow}

%Ancienne commande chapitre
\newcommand{\chapitre}[1]{
\begin{tikzpicture}
\node[fill=theocolor,rectangle,rounded corners=5pt]{%
\begin{minipage}{\linewidth}
\begin{center}
\vspace*{9pt}
\textcolor{titlecolor}{\Large \textsc{\textbf{#1}}}
\vspace*{9pt}
\end{center}
\end{minipage}
};
\end{tikzpicture}
\bigskip
}

%Pour les fiches : commande de Cécile
\newcommand{\Fiche}[2]{%
\begin{tikzpicture}
	\node[draw, color=blue,fill=white,rectangle,rounded corners=5pt]{%
	\begin{minipage}{\linewidth}
		\begin{center}
			\vspace*{9pt}
			\textcolor{blue}{\Large \textsc{\textbf{Fiche~#1\ :\ #2}}}
			\vspace*{7pt}
		\end{center}
	\end{minipage}
	};
\end{tikzpicture}
}%

\pagecolor{white}%couleur du fond de page

\renewcommand{\Pointilles}{%
\makebox[\linewidth]{\dotfill}
}


%centrer du texte ou une formule avec moins d'espace autour
\newcommand{\centrer}[1]
{
\smallskip
\centerline{#1}
\smallskip
}


%QRcode généré par le package qrcode
\usepackage{qrcode}


%Pour pouvoir utiliser l'environnement verbatim
\usepackage{verbatim}

%Panneau danger (nécessite le package pstricks)
\def\danger{\begingroup
\psset{unit=1ex}%
\begin{pspicture}(0,0)(3,3)
 
\pspolygon[linearc=0.2,linewidth=0.12,linecolor=red](0,0)(1.5,2.6)(3,0)
 
\psellipse*(1.5,1.33)(0.14,0.75)\pscircle*(1.5,0.3){0.15}\end{pspicture}
%
\endgroup}%

%___________________________
%===    Nouvelles commandes pour documents venant de Sesamath
%------------------------------------------------------
%
\definecolor{CyanTikz40}{cmyk}{.4,0,0,0}
\definecolor{CyanTikz20}{cmyk}{.2,0,0,0}
\tikzstyle{general}=[line width=0.3mm, >=stealth, x=1cm, y=1cm,line cap=round, line join=round]
\tikzstyle{quadrillage}=[line width=0.3mm, color=CyanTikz40]
\tikzstyle{quadrillageNIV2}=[line width=0.3mm, color=CyanTikz20]
\tikzstyle{quadrillage55}=[line width=0.3mm, color=CyanTikz40, xstep=0.5, ystep=0.5]
\tikzstyle{cote}=[line width=0.3mm, <->]
\tikzstyle{epais}=[line width=0.5mm, line cap=butt]
\tikzstyle{tres epais}=[line width=0.8mm, line cap=butt]
\tikzstyle{axe}=[line width=0.3mm, ->, color=Noir, line cap=rect]
\newcommand{\quadrillageSeyes}[2]{\draw[line width=0.3mm, color=A1!10, ystep=0.2, xstep=0.8] #1 grid #2;
\draw[line width=0.3mm, color=A1!30, xstep=0.8, ystep=0.8] #1 grid #2; }
\newcommand{\axeX}[4][0]{\draw[axe] (#2,#1)--(#3,#1); \foreach \x in {#4} {\draw (\x,#1) node {\small $+$}; \draw (\x,#1) node[below] {\small $\x$};}}
\newcommand{\axeY}[4][0]{\draw[axe] (#1,#2)--(#1,#3); \foreach \y in {#4} {\draw (#1, \y) node {\small $+$}; \draw (#1, \y) node[left] {\small $\y$};}}
\newcommand{\axeOI}[3][0]{\draw[axe] (#2,#1)--(#3,#1);  \draw (1,#1) node {\small $+$}; \draw (1,#1) node[below] {\small $I$};}
\newcommand{\axeOJ}[3][0]{\draw[axe] (#1,#2)--(#1,#3); \draw (#1, 1) node {\small $+$}; \draw (#1, 1) node[left] {\small $J$};}
\newcommand{\axeXgraduation}[2][0]{\foreach \x in {#2} {\draw (\x,#1) node {\small $+$};}}
\newcommand{\axeYgraduation}[2][0]{\foreach \y in {#2} {\draw (#1, \y) node {\small $+$}; }}
\newcommand{\origine}{\draw (0,0) node[below left] {\small $0$};}
\newcommand{\origineO}{\draw (0,0) node[below left] {$O$};}
\newcommand{\point}[4]{\draw (#1,#2) node[#4] {$#3$};}
\newcommand{\pointGraphique}[4]{\draw (#1,#2) node[#4] {$#3$};
\draw (#1,#2) node {$+$};}
\newcommand{\pointFigure}[4]{\draw (#1,#2) node[#4] {$#3$};
\draw (#1,#2) node {$\times$};}
\newcommand{\pointC}[3]{\draw (#1) node[#3] {$#2$};}
\newcommand{\pointCGraphique}[3]{\draw (#1) node[#3] {$#2$};
\draw (#1) node {$+$};}
\newcommand{\pointCFigure}[3]{\draw (#1) node[#3] {$#2$};
\draw (#1) node {$\times$};}


\definecolor{B1prime}                {cmyk}{0.00, 1.00, 0.00, 0.50}
\definecolor{H1prime}                {cmyk}{0.50, 0.00, 1.00, 0.00}

\definecolor{FootFonctionColor}{cmyk}{0.50, 0.00, 0.00, 0.00}
\definecolor{FootGeometrieColor}{cmyk}{0.40, 0.40, 0.00, 0.00}
\definecolor{FootStatistiqueColor}{cmyk}{0.30, 0.48, 0.00, 0.10}
\definecolor{FootStatistiqueOLDColor}{cmyk}{0.48, 0.30, 0.10, 0.00}
\definecolor{FootStatistique*Color}{cmyk}{0.20, 0.00, 0.00, 0.00}
\definecolor{ActiviteFootColor}{cmyk}{0.50, 0.00, 0.25, 0.00}
\definecolor{CoursFootColor}{cmyk}{0.15, 0.00, 0.00, 0.03}
\definecolor{ExoBaseFootColor}{cmyk}{0.00, 0.25, 0.50, 0.00}
\definecolor{ExoApprFootColor}{cmyk}{0.00, 0.25, 0.50, 0.00}
%\colorlet{ConnFootColor}{F2}
\definecolor{TPFootColor}{cmyk}{0.00, 0.30, 0.00, 0.10}
\definecolor{RecreationFootColor}{cmyk}{0.20, 0.00, 0.50, 0.05}

\definecolor{Blanc}             {cmyk}{0.00, 0.00, 0.00, 0.00}
\definecolor{Gris1}             {cmyk}{0.00, 0.00, 0.00, 0.20}
\definecolor{Gris2}             {cmyk}{0.00, 0.00, 0.00, 0.40}
\definecolor{Gris3}             {cmyk}{0.00, 0.00, 0.00, 0.50}
\definecolor{Noir}              {cmyk}{0.00, 0.00, 0.00, 1.00}
\definecolor{A1}              {cmyk}{0.33, 1.00, 0.00, 0.40}
\definecolor{F1}              {cmyk}{0.00, 1.00, 1.00, 0.00}
\definecolor{C1}              {cmyk}{0.00, 1.00, 0.00, 0.50}
\definecolor{G1}              {cmyk}{0.00, 0.00, 0.00, 0.20}
\definecolor{D1}              {cmyk}{0.00, 0.22, 0.49, 0.69}%bitume
\definecolor{J1}              {cmyk}{0.00, 0.34, 1.00, 0.02}%orangé


%augmenter l'espace au-dessus ou en-dessous d'une fraction
\usepackage{fixltx2e}
\makeatletter
\newcommand*\Strut[1][1]{%
  \leavevmode
  \vrule \@height #1\ht\strutbox
         \@depth #1\dp\strutbox
         \@width\z@
}
\newcommand*\TopStrut[1][1]{%
  \leavevmode
  \vrule \@height #1\ht\strutbox
         \@depth \z@
         \@width \z@
}
\newcommand*\BotStrut[1][1]{%
  \leavevmode
  \vrule \@height \z@
         \@depth #1\dp\strutbox
         \@width \z@
}
\makeatother


% coordonnées vecteurs dans le plan
%\newcommand{\covec2}[2]{\begin{pmatrix}#1\\#2\end{pmatrix}}


% coordonnées dans l'espace
%\newcommand{\covec3}[3]{\begin{pmatrix}#1\\#2\\#3\end{pmatrix}}

% QCM



%%%%%%%%%%%%%%%%%%%%%%%%%%%%%%%%%%%%%%%%%%%%%%%%%%%%%%%%%%%%%%%%%%%%%%%%%%%%%%%
%% Numéro de section dans la marge
\renewcommand\thesection{\arabic{section}}
\makeatletter
\def\section{\@ifstar\unnumberedsection\numberedsection}
\def\numberedsection{\@ifnextchar[%]
  \numberedsectionwithtwoarguments\numberedsectionwithoneargument}
\def\unnumberedsection{\@ifnextchar[%]
  \unnumberedsectionwithtwoarguments\unnumberedsectionwithoneargument}
\def\numberedsectionwithoneargument#1{\numberedsectionwithtwoarguments[#1]{#1}}
\def\unnumberedsectionwithoneargument#1{\unnumberedsectionwithtwoarguments[#1]{#1}}
\def\numberedsectionwithtwoarguments[#1]#2{%
  \ifhmode\par\fi
  \removelastskip
  \vskip 3ex\goodbreak
  \refstepcounter{section}%
           \hspace{-30pt}
           \begin{tikzpicture}[node distance=0 cm]
	    \node[fill=sectioncolor,rectangle,rounded corners=5pt,anchor=south west] (sectionnumber) at (0,0)
	    {\bfseries\Large\textcolor{white}{\thesection.}};
	    \node[anchor=south west] (sectiontitle) [right = of sectionnumber]
	    {\bfseries\Large\textsc{#1}};
	    \end{tikzpicture}
	    %petite ligne en dessous
	    %\black \hspace{-30pt}\hrule height 1pt depth 0pt width \hsize \black
%   \vskip 2mm\nobreak
  \addcontentsline{toc}{section}{\protect\numberline{\thesection}#1}%
  \ignorespaces
 % \medskip
  }
\def\unnumberedsectionwithtwoarguments[#1]#2{%
    \ifhmode\par\fi
  \removelastskip
  \vskip 3ex\goodbreak
  \refstepcounter{section}%
           \hspace{-30pt}
           \begin{tikzpicture}[node distance=0 cm]
	    \node[fill=sectioncolor,rectangle,rounded corners=5pt,anchor=south west] (sectionnumber) at (0,0)
	    {\bfseries\Large\textcolor{white}{\thesection.}};
	    \node[anchor=south west] (sectiontitle) [right = of sectionnumber]
	    {\bfseries\Large\textsc{#1}};
	    \end{tikzpicture}
	    %petite ligne en dessous
	    %\black \hrule height 1pt depth 0pt width \hsize \black
%   \vskip 2mm\nobreak
   \ignorespaces
  }
\makeatother


% redefinition soussection
\makeatletter
\def\subsection{\@ifstar\unnumberedsubsection\numberedsubsection}
\def\numberedsubsection{\@ifnextchar[%]
  \numberedsubsectionwithtwoarguments\numberedsubsectionwithoneargument}
\def\unnumberedsubsection{\@ifnextchar[%]
  \unnumberedsubsectionwithtwoarguments\unnumberedsubsectionwithoneargument}
\def\numberedsubsectionwithoneargument#1{\numberedsubsectionwithtwoarguments[#1]{#1}}
\def\unnumberedsubsectionwithoneargument#1{\unnumberedsubsectionwithtwoarguments[#1]{#1}}
\def\numberedsubsectionwithtwoarguments[#1]#2{%
  \ifhmode\par\fi
  \removelastskip
  \vskip 3ex\goodbreak
  \refstepcounter{subsection}%
            \begin{tikzpicture}[node distance=0 cm]
	    \node[fill=subsectioncolor,rectangle,rounded corners=5pt,anchor=south west] (subsectionnumber) at (0,0)
	    {\bfseries\large\textcolor{white}{\thesubsection.}};
	    \node[anchor=south west] (subsectiontitle) [right = of subsectionnumber]
	    {\bfseries\large #1};
	    \end{tikzpicture}
	    %petite ligne en dessous
	    %\black \hrule height 1pt depth 0pt width \hsize \black
%  \vskip 2mm\nobreak
  \addcontentsline{toc}{subsection}{\protect\numberline{\thesubsection}#1}%
  \ignorespaces
  }
\def\unnumberedsubsectionwithtwoarguments[#1]#2{%
   \ifhmode\par\fi
  \removelastskip
  \vskip 3ex\goodbreak
  \refstepcounter{subsection}%
            \begin{tikzpicture}[node distance=0 cm]
	    \node[fill=subsectioncolor,rectangle,rounded corners=5pt,anchor=south west] (subsectionnumber) at (0,0)
	    {\bfseries\large\textcolor{white}{\thesubsection.}};
	    \node[anchor=south west] (subsectiontitle) [right = of subsectionnumber]
	    {\bfseries\large #1};
	    \end{tikzpicture}
	    %petite ligne en dessous
	    %\black \hrule height 1pt depth 0pt width \hsize \black
%   \vskip 2mm\nobreak
   \ignorespaces
  }
\makeatother


% redefinition sous-sous-section
\makeatletter
\def\subsubsection{\@ifstar\unnumberedsubsubsection\numberedsubsubsection}
\def\numberedsubsubsection{\@ifnextchar[%]
  \numberedsubsubsectionwithtwoarguments\numberedsubsubsectionwithoneargument}
\def\unnumberedsubsubsection{\@ifnextchar[%]
  \unnumberedsubsubsectionwithtwoarguments\unnumberedsubsubsectionwithoneargument}
\def\numberedsubsubsectionwithoneargument#1{\numberedsubsubsectionwithtwoarguments[#1]{#1}}
\def\unnumberedsubsubsectionwithoneargument#1{\unnumberedsubsubsectionwithtwoarguments[#1]{#1}}
\def\numberedsubsubsectionwithtwoarguments[#1]#2{%
  \ifhmode\par\fi
  \removelastskip
  \vskip 3ex\goodbreak
  \refstepcounter{subsubsection}%
            \begin{tikzpicture}[node distance=0 cm]
	    \node[fill=subsectioncolor,rectangle,rounded corners=5pt,anchor=south west] (subsubsectionnumber) at (0,0)
	    {\bfseries\large\textcolor{white}{\thesubsubsection.}};
	    \node[anchor=south west] (subsubsectiontitle) [right = of subsubsectionnumber]
	    {\bfseries\large #1};
	    \end{tikzpicture}
	    %petite ligne en dessous
	    %\black \hrule height 1pt depth 0pt width \hsize \black
%  \vskip 2mm\nobreak
  \addcontentsline{toc}{subsubsection}{\protect\numberline{\thesubsubsection}#1}%
  \ignorespaces
  }
\def\unnumberedsubsubsectionwithtwoarguments[#1]#2{%
   \ifhmode\par\fi
  \removelastskip
  \vskip 3ex\goodbreak
  \refstepcounter{subsubsection}%
            \begin{tikzpicture}[node distance=0 cm]
	    \node[fill=subsubsectioncolor,rectangle,rounded corners=5pt,anchor=south west] (subsubsectionnumber) at (0,0)
	    {\bfseries\large\textcolor{white}{\thesubsubsection.}};
	    \node[anchor=south west] (subsubsectiontitle) [right = of subsubsectionnumber]
	    {\bfseries\large #1};
	    \end{tikzpicture}
	    %petite ligne en dessous
	    %\black \hrule height 1pt depth 0pt width \hsize \black
%   \vskip 2mm\nobreak
   \ignorespaces
  }
\makeatother

%%%%%%%%%%%%%%%%%%%%%%%%%%%%%%%%%%%%%%%%%%%%%%%%%%%%%%%%%%%%%%%%%%%%%%%%%%%%%%%

%entête classique

\fancypagestyle{garde_tete}{% 
%\fancyhead[C]{\small\textbf{\seconde} \hfill \small \textbf{Année 2014-2015}}
\renewcommand{\headrulewidth}{0cm}}

\newcommand{\tete}{
\thispagestyle{garde_tete}
\chapitre{Exemples figures avec TkZ}

\noindent 
\vspace{-24pt}
}

%%%%%%%%%%%%%%%%%%%%%%%%%%%%%%%%%%%%%%%%%%%%%%%%%%%%%%%%%%%%%%%%%%%%%%%%%%%%%%%
%%%%%%%%%%%%%%%%%%%%%%%%%%%%%%%%%%%%%%%%%%%%%%%%%%%%%%%%%%%%%%%%%%%%%%%%%%%%%%%
\usetikzlibrary{spy}%exemple 18
%%%%%%%%%%%%%%%%%%%%%%%
%% DEBUT DU DOCUMENT %%
%%%%%%%%%%%%%%%%%%%%%%%

\begin{document}
\selectlanguage{french}
\setlength\parindent{0mm}
\tete 		%entête classique

\renewcommand \footrulewidth{0.2pt}%
\renewcommand \headrulewidth{0pt}%
\pagestyle{fancy}
\fancyhf{}
\pieddepage{}{\thepage / \pageref{LastPage}}{}

%%%%%%%%%%%%%%%%%%%%%%%%%%%%%%%%%%%%%%%%%%%%%%%%%%%%%%%%%%%%
\begin{spacing}{1.2}
%%%%%%%%%%%%%%%%%%%%%%%%%%%%%%%%%%%%%%%%%%%%%%%%%%%%%%%%%%%%

\textbf{Exemple 1 :}

\begin{center}
	\begin{tikzpicture}
	\tkzTabInit[lgt=3,espcl=3]
	{$x$/1,$x+2$/1,$x+4$/1,$\dfrac{x+2}{x+4}$/1.5}
	{$-\infty$,$-4$,$-2$,$+\infty$}
	\tkzTabLine{,-,t,-,z,+,}
	\tkzTabLine{,-,z,+,t,+,}
	\tkzTabLine{,+,d,-,z,+,}
	\end{tikzpicture}
\end{center}


%%%%%%%%%%%%%%%%%%%%%%%%%%%%%%%%%%%%%%%%%%%%%%%%%%%%%%%%%%%%
%%%%%%%%%%%%%%%%%%%%%%%%%%%%%%%%%%%%%%%%%%%%%%%%%%%%%%%%%%%%
\textbf{Exemple 2 :}

\begin{center}
\begin{tikzpicture}[scale=0.8]
\tkzInit[xmin=-1,xmax=7,ymin=-1,ymax=8]
\tkzClip
\tkzDefPoint(0,0){D}\tkzLabelPoints[font=\boldmath,below left,color=blue](D)
\tkzDefPoint(5,0){C}\tkzLabelPoints[font=\boldmath,below right,color=blue](C)
\tkzDefPoint(0,5){A}\tkzLabelPoints[font=\boldmath,above left,color=blue](A)
\tkzDefPoint(5,5){B}\tkzLabelPoints[font=\boldmath,below right,color=blue](B)
\tkzDefPoint(1,2){H}\tkzLabelPoints[font=\boldmath,above left,color=blue](H)
\tkzDefPoint(6,2){G}\tkzLabelPoints[font=\boldmath,above right,color=blue](G)
\tkzDefPoint(1,7){E}\tkzLabelPoints[font=\boldmath,above left,color=blue](E)
\tkzDefPoint(6,7){F}\tkzLabelPoints[font=\boldmath,above right,color=blue](F)
\tkzDefPoint(1,2){H}\tkzLabelPoints[font=\boldmath,above left,color=blue](H)

\tkzDrawPolygon[line width=1.5pt,color=blue](A,B,C,D)
\tkzDrawSegment[line width=1.5pt,color=blue,style=dotted](E,H)
\tkzDrawSegment[line width=1.5pt,color=blue,style=dotted](H,D)
\tkzDrawSegment[line width=1.5pt,color=blue,style=dotted](H,G)
\tkzDrawSegment[line width=1.5pt,color=blue](E,A)
\tkzDrawSegment[line width=1.5pt,color=blue](E,F)
\tkzDrawSegment[line width=1.5pt,color=blue](B,F)
\tkzDrawSegment[line width=1.5pt,color=blue](G,F)
\tkzDrawSegment[line width=1.5pt,color=blue](G,C)

\tkzDefMidPoint(A,E)\tkzGetPoint{J}
\tkzLabelPoints[font=\boldmath,above left,color=blue](J)
\tkzDefMidPoint(A,B)\tkzGetPoint{I}
\tkzLabelPoints[font=\boldmath,below,yshift=-3pt,color=blue](I)
\tkzDefMidPoint(A,C)\tkzGetPoint{A'}
\tkzLabelPoints[font=\boldmath,above,yshift=3pt,color=blue](A')
\tkzDefMidPoint(E,G)\tkzGetPoint{E'}
\tkzLabelPoints[font=\boldmath,below,yshift=-3pt,color=blue](E')
\tkzDrawPoints[size=10,color=blue](A',E',I,J)

\end{tikzpicture}
\end{center}

%%%%%%%%%%%%%%%%%%%%%%%%%%%%%%%%%%%%%%%%%%%%%%%%%%%%%%%%%%%%
%%%%%%%%%%%%%%%%%%%%%%%%%%%%%%%%%%%%%%%%%%%%%%%%%%%%%%%%%%%%
\textbf{Exemple 3 :}

\begin{center}
\begin{tikzpicture}[scale=1]
\tkzDefPoint(0,0){A}\tkzLabelPoints[font=\boldmath,below left,color=blue](A)
\tkzDefPoint(3,0){B}\tkzLabelPoints[font=\boldmath,below right,color=blue](B)
\tkzDefPoint(3,5){C}\tkzLabelPoints[font=\boldmath,above right,color=blue](C)
\tkzDefPoint(0,1.5){D}\tkzLabelPoints[font=\boldmath,above left,color=blue](D)
\tkzMarkRightAngle[fill=red!50](B,A,D)
\tkzMarkRightAngle[fill=red!50](C,B,A)
\tkzMarkAngle[fill=red!50,size=0.8cm](D,C,B)
\tkzLabelAngle[pos=1.2](D,C,B){50\degres}
\tkzDrawPolygon[line width=1.5pt,color=blue](A,B,C,D)
\end{tikzpicture}
\end{center}

%%%%%%%%%%%%%%%%%%%%%%%%%%%%%%%%%%%%%%%%%%%%%%%%%%%%%%%%%%%%
%%%%%%%%%%%%%%%%%%%%%%%%%%%%%%%%%%%%%%%%%%%%%%%%%%%%%%%%%%%%
\newpage\textbf{Exemple 4 :}

\begin{center}
\begin{tikzpicture}
\tkzTabInit[lgt=2.5,espcl=3]
{$x$/1,$1-x$/1,$\e^x$/1,$f'(x)$/1,Variations\\de $f$/2}
{$-\infty$,$1$,$+\infty$}
\tkzTabLine{,+,z,-,}
\tkzTabLine{,+,t,+,}
\tkzTabLine{,+,z,-,}
\tkzTabVar{-/$0$,+/$\e$,-/$-\infty$}
\end{tikzpicture}
\end{center}

\begin{center}
\begin{tikzpicture}
\tkzTabInit[lgt=2.5,espcl=3]
{$x$/1,$f'(x)$/1,Variations\\de $f$/2}
{$0$,$4$,$8$}
\tkzTabLine{,+,d,+,}
\tkzTabVar{-/$3$,+V-/$\approx 4.14$/$2.9$,+/$6.9$}
\end{tikzpicture}
\end{center}

\begin{center}
\begin{tikzpicture}
\tkzTabInit[lgt=2.5,espcl=3]
{$x$/1,Variations\\de $f$/2}
{$-\infty$,$-2$,$+\infty$}
\tkzTabVar{+/$0$,-D+/$-\infty$ /$+\infty$,-/$0$}
\end{tikzpicture}
\end{center}

\begin{center}
\begin{tikzpicture}
\tkzTabInit[lgt=2.5,espcl=2.5]
{$x$/1,$f'(x)$/1,Variations\\de $f$/2}
{$0$,$0.5$,$3.5$,$6$}
\tkzTabLine{,-,z,+,z,-,}
\tkzTabVar{+/$-9$,-/$-14$,+/$40$,-/$-135$}
\tkzTabVal[draw]{2}{3}{0.4}{$1.5$}{$0$}
\tkzTabVal[draw]{3}{4}{0.6}{$\alpha$}{$0$}
\end{tikzpicture}
\end{center}

\begin{center}
\begin{tikzpicture}
\tkzTabInit[lgt=3,espcl=2]
{$x$/1,$2x^2$/1,$x^2-8x+15$/1,$\left(2x^2-8x+10\right)^2$/1,$f'(x)$/1,Variations\\de $f$/2}
{$-\infty$,$0$,$3$,$5$,$+\infty$}
\tkzTabLine{,+,z,+,t,+,t,+,}
\tkzTabLine{,+,t,+,z,-,z,+,}
\tkzTabLine{,+,t,+,t,+,t,+,}
\tkzTabLine{,+,z,+,z,-,z,+,}
\tkzTabVar{-/ , R , +/$\dfrac{29}{4}$,-/$\dfrac{27}{4}$,+/ }
\end{tikzpicture}
\end{center}

\begin{center}
\begin{tikzpicture}
\tkzTabInit[color,lgt=2.5,espcl=3,colorC=blue!20,colorV=blue!20]
{$x$/1,$f'(x)$/1,Variations\\de $f$/2}
{$-\infty$,$-7$,$-3$,$2$,$+\infty$}
\tkzTabLine{,-,z,+,d,-,z,+,}
\tkzTabVar{+/$-4$,-/$-10$,+D+/$+\infty$/$+\infty$,-/$3$,+/$5$}
\end{tikzpicture}
\end{center}

\begin{center}
\begin{tikzpicture}
\tkzTabInit[color,lgt=2.5,espcl=3,colorC=blue!20,colorV=blue!20]
{$x$ / 1 ,Variations\\de $f$ / 2}
{$-\infty$, $-5$, $-3$, 0, $+\infty$}
\tkzTabVar{-/ $-\infty$, +CD-/ $0$/ $2$, +D+/ $0$ /$0$, -V-/ $-2$ / $3$, +/ $+\infty$}
\end{tikzpicture}
\end{center}

\textbf{Pour avoir des nombres décimaux écrits avec une virgule mettre le nombre entre accolades}

\begin{center}
\begin{tikzpicture}
\tkzTabInit[color,lgt=2.5,espcl=6,colorC=blue!20,colorV=blue!20]
{$x$/1,$f'(x)$/1,Variations\\de $f$/1.5}
{$0$,{$1,5$},$+\infty$}
\tkzTabLine{,-,z,+,}
\tkzTabVar{+/,-/$0$,+/}
\end{tikzpicture}
\end{center}

\textbf{Réglage des espaces avant le premier antécédent et après le dernier}

\begin{center}
\begin{tikzpicture}
\tkzTabInit[color,lgt=2.5,espcl=4,deltacl=0.7,colorC=blue!20,colorV=blue!20]
{$x$/1,$f'(x)$/1,Variations\\de $f$/2}
{$1$,$5+5\ln(4)$,$18$}
\tkzTabLine{,-,z,+,}
\tkzTabVar{+/{$96,02$},-/{$38,86$},+/{$43,97$}}
\end{tikzpicture}
\end{center}

\begin{center}
\textbf{Avec l'instruction} \verb~\tkzTabIma~

\danger Pour placer une image entre deux autres, il faut que les deux images extrêmes existent\dots

Il ne faut donc pas utiliser une image qui a été remplacée par R\dots

\medskip

\begin{tikzpicture}
\tkzTabInit[color,lgt=2.5,espcl=3,colorC=blue!20,colorV=blue!20]
{$t$/1,Variations\\ de $f$/2,Variations\\ de $g$/2}
{$-1$,$0$,$1$,$2$}
\tkzTabVar{-/$-5$,+/$0$,-/$-1$,+/$4$}
\tkzTabVar{-/$-5$,R/,R/,+/$4$}
\tkzTabIma[draw]{1}{4}{2}{$0$}
\tkzTabIma[draw]{1}{4}{3}{$3$}
\end{tikzpicture}
\end{center}

%%%%%%%%%%%%%%%%%%%%%%%%%%%%%%%%%%%%%%%%%%%%%%%%%%%%%%%%%%%%
%%%%%%%%%%%%%%%%%%%%%%%%%%%%%%%%%%%%%%%%%%%%%%%%%%%%%%%%%%%%
\textbf{Exemple 5 :}

\begin{center}
\begin{tikzpicture}[>=stealth,scale=0.8]
\tkzInit[xmin=-1,xmax=7,ymin=-1,ymax=4]
\tkzClip
\tkzDefPoint(0,0){A}
\tkzDefPoint(0,-0.2){A'}
\tkzDefPoint(5,0){B}
\tkzDefPoint(5,-0.2){B'}
\tkzDefPoint(4,3){C}
\tkzDefPoint(4,3.2){C'}
\tkzDefPoint(2,3){D}
\tkzDefPoint(2,3.2){D'}
\tkzDefPoint(3,3){E}
\tkzDefPoint(3,0){F}

\tkzMarkRightAngle[fill=red!50](C,E,F)
\tkzMarkRightAngle[fill=red!50](E,F,B)
\tkzDrawPolygon[line width=1.5pt,color=blue](A,B,C,D)
\tkzDrawSegment[line width=1.5pt,color=blue,style=dotted](E,F)
\tkzDrawSegment[<->,line width=1.5pt,color=blue](A',B')
\tkzDrawSegment[<->,line width=1.5pt,color=blue](C',D')
\tkzLabelSegment[left,pos=0.5,color=blue](E,F){$h$}
\tkzLabelSegment[below,pos=0.5,color=blue](A',B'){B}
\tkzLabelSegment[above,pos=0.5,color=blue](C',D'){$b$}

\end{tikzpicture}
\end{center}




%%%%%%%%%%%%%%%%%%%%%%%%%%%%%%%%%%%%%%%%%%%%%%%%%%%%%%%%%%%%
%%%%%%%%%%%%%%%%%%%%%%%%%%%%%%%%%%%%%%%%%%%%%%%%%%%%%%%%%%%%
%\newpage
\bigskip
\textbf{Exemple 6 :}

\begin{center}
\begin{tikzpicture}[scale=0.8]
%\tkzInit[xmin=-1,xmax=7,ymin=-6,ymax=6]
%\tkzClip
\tkzDefPoint(0,0){A}\tkzLabelPoints[font=\boldmath,below left,color=blue](A)
\tkzDefPoint(5,0){B}\tkzLabelPoints[font=\boldmath,right,color=blue](B)
\tkzDefPoint(5,5){C}\tkzLabelPoints[font=\boldmath,above right,color=blue](C)
\tkzDefPoint(5,-5){F}\tkzLabelPoints[font=\boldmath,below right,color=blue](F)

\tkzCalcLength[cm](A,F)\tkzGetLength{dAF}

\tkzDrawArc[R,line width=1.5pt,color=red,style=dashed](A,\dAF cm)(90,120)
\tkzDrawArc[R,line width=1.5pt,color=red,style=dashed](C,\dAF cm)(150,180)

\tkzInterCC[R](A,\dAF cm)(C,\dAF cm) \tkzGetPoints{G}{H}


\tkzMarkRightAngle[fill=red!50](A,B,F)
\tkzMarkRightAngle[fill=blue!50](C,B,A)

\tkzDrawPolygon[line width=1.5pt,color=blue](A,B,C)
\tkzDrawPolygon[line width=1.5pt,color=blue](A,B,F)
\tkzDrawPolygon[line width=1.5pt,color=blue](B,F,C)

\tkzDrawPolygon[line width=1.5pt,color=blue](A,C,G)

\tkzMarkSegments[mark=||,size=4pt](A,C A,F A,G C,G)
\tkzMarkSegments[mark=o,size=4pt](A,B B,F B,C)


\end{tikzpicture}
\end{center}


\begin{center}
\begin{tikzpicture}[scale=0.8]
%\tkzInit[xmin=-1,xmax=7,ymin=-6,ymax=6]
%\tkzClip
\tkzDefPoint(0,0){A}\tkzLabelPoints[font=\boldmath,below left,color=blue](A)
\tkzDefPoint(4,-1){B}\tkzLabelPoints[font=\boldmath,below,color=blue](B)
\tkzDefPointBy[rotation= center A angle 60](B)
\tkzGetPoint{C}\tkzLabelPoints[font=\boldmath,above right,color=blue](C)
\tkzDefPointBy[rotation= center C angle -90](A)
\tkzGetPoint{D}\tkzLabelPoints[font=\boldmath,above left,color=blue](D)
\tkzDefPointBy[rotation= center B angle -90](C)
\tkzGetPoint{E}\tkzLabelPoints[font=\boldmath,right,color=blue](E)

\tkzMarkRightAngle[fill=red!50](E,B,C)
\tkzMarkRightAngle[fill=blue!50](D,C,A)

\tkzDrawPolygon[line width=1.5pt,color=blue](A,B,C)
\tkzDrawPolygon[line width=1.5pt,color=blue](B,C,E)
\tkzDrawPolygon[line width=1.5pt,color=blue](A,C,D)

\tkzMarkSegments[mark=||,size=4pt](A,C C,D B,C B,E A,B)
\end{tikzpicture}
\end{center}
%%%%%%%%%%%%%%%%%%%%%%%%%%%%%%%%%%%%%%%%%%%%%%%%%%%%%%%%%%%%
%%%%%%%%%%%%%%%%%%%%%%%%%%%%%%%%%%%%%%%%%%%%%%%%%%%%%%%%%%%%

\textbf{Exemple 7 :}

\begin{center}
\begin{tikzpicture}[scale=1]

\tkzDefPoint(0,0){A}\tkzLabelPoints[font=\boldmath,below left,color=blue](A)
\tkzDefPoint(2,0){C}\tkzLabelPoints[font=\boldmath,below right,color=blue](C)
\tkzDefPoint(5,0){B}\tkzLabelPoints[font=\boldmath,below right,color=blue](B)
\tkzDefPoint(0,4){D}\tkzLabelPoints[font=\boldmath,above left,color=blue](D)
\tkzDefPoint(2,4){F}\tkzLabelPoints[font=\boldmath,above right,color=blue](F)
\tkzDefPoint(5,4){E}\tkzLabelPoints[font=\boldmath,above right,color=blue](E)
\tkzDefPoint(2,7){E_1}\tkzLabelPoints[font=\boldmath,above right,color=blue](E_1)
\tkzDefPoint(2,-3){B_1}\tkzLabelPoints[font=\boldmath,below right,color=blue](B_1)

\tkzCalcLength[cm](A,B_1)\tkzGetLength{dAB}

\tkzDefPoint(-\dAB,0){B_2}\tkzLabelPoints[font=\boldmath,below left,color=blue](B_2)
\tkzDefPoint(-\dAB,4){E_2}\tkzLabelPoints[font=\boldmath,above left,color=blue](E_2)

\tkzDrawArc[delta=10,line width=1.5pt,color=red,style=dashed](A,B_2)(B_1)
\tkzDrawArc[delta=10,line width=1.5pt,color=red,style=dashed](D,E_1)(E_2)

\tkzMarkRightAngle[fill=red!50](E_1,F,D)
\tkzMarkRightAngle[fill=red!50](F,C,A)
\tkzMarkRightAngle[fill=red!50](A,D,F)
\tkzMarkRightAngle[fill=red!50](D,A,B_2)
\tkzMarkRightAngle[fill=red!50](B_2,E_2,D)
\tkzMarkRightAngle[fill=red!50](C,F,E)
\tkzMarkRightAngle[fill=red!50](E,B,C)

\tkzMarkRightAngle[fill=blue!50](D,F,C)
\tkzMarkRightAngle[fill=blue!50](A,C,B_1)
\tkzMarkRightAngle[fill=blue!50](C,A,D)
\tkzMarkRightAngle[fill=blue!50](E_2,D,A)
\tkzMarkRightAngle[fill=blue!50](A,B_2,E_2)
\tkzMarkRightAngle[fill=blue!50](B,C,F)
\tkzMarkRightAngle[fill=blue!50](F,E,B)

\tkzDrawPolygon[line width=1.5pt,color=blue](B_2,A,D,E_2)
\tkzDrawPolygon[line width=1.5pt,color=blue](A,C,F,D)
\tkzDrawPolygon[line width=1.5pt,color=blue](C,B,E,F)
\tkzDrawPolygon[line width=1.5pt,color=blue](A,C,B_1)
\tkzDrawPolygon[line width=1.5pt,color=blue](D,F,E_1)

\tkzMarkSegments[mark=||,size=4pt](B_1,C C,B F,E F,E_1)
\tkzMarkSegments[mark=o,size=4pt](E_2,B_2 A,D C,F B,E)
\tkzMarkSegments[mark=|||,size=4pt](A,B_1 A,B_2 D,E_1 D,E_2)
\tkzMarkSegments[mark=oo,size=8pt](A,C D,F)


\end{tikzpicture}
\end{center}

%%%%%%%%%%%%%%%%%%%%%%%%%%%%%%%%%%%%%%%%%%%%%%%%%%%%%%%%%%%%
%%%%%%%%%%%%%%%%%%%%%%%%%%%%%%%%%%%%%%%%%%%%%%%%%%%%%%%%%%%%

\textbf{Exemple 8 : Section d'un cube}

\begin{center}
\begin{tikzpicture}[scale=0.6]

\tkzDefPoint(0,0){A}\tkzLabelPoints[font=\boldmath,below left,color=blue](A)
\tkzDefPoint(10,-1){B}\tkzLabelPoints[font=\boldmath,below right,color=blue](B)
\tkzDefPoint(10,9){F}\tkzLabelPoints[font=\boldmath,below right,color=blue](F)
\tkzDefPoint(0,10){E}\tkzLabelPoints[font=\boldmath,below left,color=blue](E)
\tkzDefPoint(3,2){D}\tkzLabelPoints[font=\boldmath,above left,color=blue](D)
\tkzDefPoint(13,4){I}\tkzLabelPoints[font=\boldmath,above right,color=red](I)

\tkzDefPoint(13,1){C}\tkzLabelPoints[font=\boldmath,above right,color=blue](C)
\tkzDefPoint(13,11){G}\tkzLabelPoints[font=\boldmath,above right,color=blue](G)
\tkzDefPoint(3,12){H}\tkzLabelPoints[font=\boldmath,above left,color=blue](H)

\tkzDrawPolygon[line width=1.5pt,color=blue](A,B,F,E)

\tkzDrawSegments[line width=1.5pt,color=blue,style=dashed](H,D D,C A,D)

\tkzDrawSegments[line width=1.5pt,color=blue](E,H H,G G,C C,B F,G)

\tkzDefLine[parallel=through I](B,E)\tkzGetPoint{I'}
%\tkzShowLine[parallel=through I](B,E)
%\tkzDrawLine[line width=1pt](I,I')

\tkzInterLL(H,G)(I,I')\tkzGetPoint{J}
\tkzLabelPoints[font=\boldmath,above right,color=red](J)

\tkzDrawSegments[line width=1.5pt,color=red](E,B B,I E,J)
\tkzDrawSegments[line width=1.5pt,color=red,style=dashed](I,J)

\end{tikzpicture}
\end{center}

%%%%%%%%%%%%%%%%%%%%%%%%%%%%%%%%%%%%%%%%%%%%%%%%%%%%%%%%%%%%
%%%%%%%%%%%%%%%%%%%%%%%%%%%%%%%%%%%%%%%%%%%%%%%%%%%%%%%%%%%%

\textbf{Exemple 9 : Intersection deux droites dans l'espace avec un tétraèdre}

\textbf{Utilisation des coordonnées barycentriques}

\begin{center}
\begin{tikzpicture}[scale=0.6]

\tkzDefPoint(0,0){A}\tkzLabelPoints[font=\boldmath,left,color=blue](A)
\tkzDefPoint(10,1){C}\tkzLabelPoints[font=\boldmath,right,color=blue](C)
\tkzDefPoint(8,-3){B}\tkzLabelPoints[font=\boldmath,below left,color=blue](B)
\tkzDefPoint(3,4){D}\tkzLabelPoints[font=\boldmath,above left,color=blue](D)

\tkzDrawPolygon[line width=1.5pt,color=blue](A,B,D)

\tkzDrawSegments[line width=1.5pt,color=blue,style=dashed](A,C)

\tkzDrawSegments[line width=1.5pt,color=blue](B,C C,D)

\tkzDefBarycentricPoint(D=1,B=4)\tkzGetPoint{N}
\tkzLabelPoints[font=\boldmath,below left,color=red](N)
\tkzDefBarycentricPoint(D=3,A=2)\tkzGetPoint{M}
\tkzLabelPoints[font=\boldmath,above left,color=red](M)

\tkzInterLL(A,B)(M,N)\tkzGetPoint{I}
\tkzLabelPoints[font=\boldmath,below right](I)

\tkzDrawSegments[line width=1.5pt,color=red](M,N N,C)
\tkzDrawSegments[line width=1.5pt,color=red,style=dashed](M,C)

\tkzDrawSegments[line width=1.5pt](N,I B,I)
\tkzDrawLine[line width=1pt](I,C)

\end{tikzpicture}
\end{center}

%%%%%%%%%%%%%%%%%%%%%%%%%%%%%%%%%%%%%%%%%%%%%%%%%%%%%%%%%%%%
%%%%%%%%%%%%%%%%%%%%%%%%%%%%%%%%%%%%%%%%%%%%%%%%%%%%%%%%%%%%

\newpage\textbf{Exemple 10 : Avec des vecteurs}


\begin{center}
\begin{tikzpicture}[scale=2,>=stealth]
\tkzInit[xmin=-2,xmax=5,ymin=-2,ymax=4]
\tkzGrid
\tkzDefPoint(2,3){A}\tkzDefPoint(4,2){B}
\tkzDefPointWith[orthogonal,K=-2](A,B)\tkzGetPoint{C}%AC=2*AB et (AB,AC)=-pi/2
\tkzDefPointWith[linear,K=2/3](A,C)\tkzGetPoint{D}%\vect{AD}=2/3*\vect{AC}
\tkzDefPointWith[colinear=at B,K=1/2](A,C)\tkzGetPoint{E}%\vect{BE}=1/2\vect{AC}
\tkzDrawPoints[shape=cross out,color=red,size=16pt](A,B,C)
\tkzDrawPoints[shape=cross,color=red,size=16pt](D)
\tkzDrawPoints[color=red,size=16pt](E)
\tkzLabelPoints[above right=3pt,font=\boldmath](A,B,C,D)
\tkzLabelPoints[below right=3pt,font=\boldmath,color=blue](E)
\tkzDrawSegment[->,color=blue,line width=1.5pt](B,E)
\end{tikzpicture}
\end{center}

%%%%%%%%%%%%%%%%%%%%%%%%%%%%%%%%%%%%%%%%%%%%%%%%%%%%%%%%%%%%
%%%%%%%%%%%%%%%%%%%%%%%%%%%%%%%%%%%%%%%%%%%%%%%%%%%%%%%%%%%%

\textbf{Exemple 11 :}


\begin{center}
\begin{tikzpicture}[scale=1,>=stealth]
\tkzInit[xmin=-6,xmax=8,ymin=-2,ymax=9]
\tkzGrid[color=gray!50]
\tkzDrawX[line width=1.5pt,color=blue,label=]
\tkzDrawY[line width=1.5pt,color=blue,label=]
\tkzRep[xnorm=1,ynorm=1,color=red,line width=2pt]
\tkzDefPoint(-1.6,-0.8){N}\tkzDefPoint(-4,2.4){E}
\tkzDefPoint(2.4,7.2){Z}\tkzDefPoint(0,0){O}
\tkzDefMidPoint(N,Z)\tkzGetPoint{K}
\tkzDefPointWith[linear,K=2](E,K)\tkzGetPoint{A}%\vect{EA}=2*\vect{EK}

\tkzDrawPolygon[style=dashed,color=red,line width=2 pt](N,E,Z)
\tkzDrawPolygon[style=dotted,color=blue,line width=2 pt](N,A,Z,E)

%droite perpendiculaire à (NZ) passant par E
\tkzDefLine[orthogonal=through E](Z,N)\tkzGetPoint{e}
\tkzDrawLine[style=dashed,line width=1.5pt,add=0.2 and -0.2](E,e)

%intersection de (Ee) et (NZ) qui se nomme M
\tkzInterLL(E,e)(Z,N) \tkzGetPoint{M}
\tkzMarkRightAngle[fill=blue!50](Z,M,E)

%intersection de (EM) et (AN) qui se nomme U
\tkzInterLL(E,M)(A,N) \tkzGetPoint{U}

\tkzDrawPoints[shape=cross out,color=red,size=3pt](N,E,Z,K,A,M,U,O)
\tkzLabelPoints[above right=3pt,font=\boldmath](Z)
\tkzLabelPoints[below right=3pt,font=\boldmath](K)
\tkzLabelPoints[below left=3pt,font=\boldmath](N)
\tkzLabelPoints[below left=3pt,font=\boldmath,color=red](O)
\tkzLabelPoints[above=3pt,font=\boldmath](E)
\tkzLabelPoints[above=6pt,font=\boldmath](M)
\tkzLabelPoints[right=3pt,font=\boldmath](A)
\tkzLabelPoints[right=6pt,font=\boldmath](U)
\end{tikzpicture}
\end{center}

%%%%%%%%%%%%%%%%%%%%%%%%%%%%%%%%%%%%%%%%%%%%%%%%%%%%%%%%%%%%
%%%%%%%%%%%%%%%%%%%%%%%%%%%%%%%%%%%%%%%%%%%%%%%%%%%%%%%%%%%%

\textbf{Exemple 12 :}


\begin{center}
\begin{tikzpicture}[scale=1,>=stealth]
\tkzInit[xmin=-4,xmax=6,ymin=-2,ymax=6]
\tkzGrid[color=gray!50]
\tkzDrawX[line width=1.5pt,color=blue,label=]
\tkzDrawY[line width=1.5pt,color=blue,label=]

\tkzDefPoint(-1.5,2){P}
\tkzDefPoint(3.5,2){T}
\tkzDefPoint(2.5,4){L}
\tkzDefPoint(0,0){O}
\tkzDefPoint(1,0){I}
\tkzDefPoint(0,1){J}
\tkzDefMidPoint(P,T)\tkzGetPoint{A}
\tkzDefMidPoint(O,L)\tkzGetPoint{N}
\tkzDefPointWith[linear,K=2](P,N)\tkzGetPoint{U}%\vect{PU}=2*\vect{PN}
\tkzDefPointWith[orthogonal,K=-1](A,L)\tkzGetPoint{S}%AS=AL et (AL,AS)=-pi/2
\tkzDefPointWith[linear,K=2](L,A)\tkzGetPoint{E}%\vect{LE}=2*\vect{LA}

\tkzDrawPolygon[style=dashed,color=red,line width=2 pt](P,L,T)
\tkzDrawPolygon[style=dashed,color=OliveGreen,line width=2 pt](A,L,S)
\tkzMarkRightAngle[fill=blue!50](S,A,L)
\tkzDrawPolygon[style=dotted,color=blue,line width=2 pt](P,O,U,L)
\tkzDrawPolygon[style=dotted,line width=2 pt](P,L,T,E)
\tkzDrawPolygon[line width=2 pt](E,A,T)

%cercle de diamètre [TP]
\tkzDrawCircle[diameter](T,P)

%marques sur les points
\tkzDrawPoints[shape=cross out,color=red,size=3pt](P,T,L,O,I,J,A,U,S,E)

%repère (O;I,J)
\tkzLabelPoints[below left=3pt,font=\boldmath,color=red](O)
\tkzLabelPoints[below right=3pt,font=\boldmath,color=red](I)
\tkzLabelPoints[above left=3pt,font=\boldmath,color=red](J)

%autres points
\tkzLabelPoints[above right=3pt,font=\boldmath](L,U)
\tkzLabelPoints[above left=3pt,font=\boldmath](T,P,A)
\tkzLabelPoints[below right=3pt,font=\boldmath](S)
\tkzLabelPoints[below left=3pt,font=\boldmath](E)

\end{tikzpicture}
\end{center}

%%%%%%%%%%%%%%%%%%%%%%%%%%%%%%%%%%%%%%%%%%%%%%%%%%%%%%%%%%%%

%%%%%%%%%%%%%%%%%%%%%%%%%%%%%%%%%%%%%%%%%%%%%%%%%%%%%%%%%%%%

\textbf{Exemple 13 : Droite d'Euler}

\begin{center}
\begin{tikzpicture}[scale=0.8]
\tkzInit[xmin=-10,xmax=10,ymin=-8,ymax=10]
\tkzGrid[color=gray!50]
\tkzDrawX[line width=2pt,color=blue,label=,>=stealth]
\tkzDrawY[line width=2pt,color=blue,label=,>=stealth]
\clip(-10,-8)rectangle(10,10);

%repère
\tkzDefPoint(0,0){O}\tkzLabelPoints[font=\boldmath,below left,color=red](O)
\tkzDefPoint(1,0){I}\tkzLabelPoints[font=\boldmath,below right,color=red](I)
\tkzDefPoint(0,1){J}\tkzLabelPoints[font=\boldmath,above left,color=red](J)

\tkzDefPoint(1,7){A}\tkzLabelPoints[font=\boldmath,above right,color=blue](A)
\tkzDefPoint(-5,-5){B}\tkzLabelPoints[font=\boldmath,below=4pt,color=blue](B)
\tkzDefPoint(7,-1){C}\tkzLabelPoints[font=\boldmath,above right,color=blue](C)

\tkzDefPoint(3,1){H}\tkzLabelPoints[font=\boldmath,below left,color=blue](H)
\tkzDefPoint(4,-2){A_1}\tkzLabelPoints[font=\boldmath,below right,color=blue](A_1)
\tkzDefPoint(-1,3){C_1}\tkzLabelPoints[font=\boldmath,above left,color=blue](C_1)

\tkzDefMidPoint(B,C)\tkzGetPoint{A'}
\tkzDefMidPoint(B,A)\tkzGetPoint{C'}
\tkzDefMidPoint(A,C)\tkzGetPoint{B'}
\tkzLabelPoints[font=\boldmath,below right,color=blue](A')
\tkzLabelPoints[font=\boldmath,right=4pt,color=blue](B')
\tkzLabelPoints[font=\boldmath,above left=4pt,color=blue](C')

%intersection (AA') et (BB')
\tkzInterLL(A,A')(B,B') \tkzGetPoint{K}
\tkzLabelPoints[font=\boldmath,above left,color=blue](K)

%droites (AA'), (BB') et (CC')
\tkzDrawLine[style=dashed,line width=1.5pt,add=1 and 1](A,A')
\tkzDrawLine[style=dashed,line width=1.5pt,add=1 and 1](B,B')
\tkzDrawLine[style=dashed,line width=1.5pt,add=1 and 1](C,C')

%droite (OH)
\tkzDrawLine[style=dashed,line width=2pt,color=red,add=4 and 3](O,H)

%triangles
\tkzDrawPolygon[line width=2 pt](A,B,C)
\tkzDrawPolygon[style=dashed,color=OliveGreen,line width=2 pt](A,A_1,C)
\tkzDrawPolygon[style=dashed,color=orange,line width=2 pt](A,C,C_1)

%angles droits
\tkzMarkRightAngle[fill=blue!50](C,A_1,A)
\tkzMarkRightAngle[fill=blue!50](C,C_1,A)

%cercle de rayon [OA]
\tkzDrawCircle[style=dotted,color=red,line width=2 pt](O,A)

\tkzDrawPoints[size=2,color=blue](A,B,C,A',B',C',K,A_1,C_1)
\tkzDrawPoints[size=2,color=red](O,I,J)
\end{tikzpicture}
\end{center}

%%%%%%%%%%%%%%%%%%%%%%%%%%%%%%%%%%%%%%%%%%%%%%%%%%%%%%%%%%%%

%%%%%%%%%%%%%%%%%%%%%%%%%%%%%%%%%%%%%%%%%%%%%%%%%%%%%%%%%%%%

\textbf{Exemple 14 : quadrillage Seyes + somme vecteurs}

Les commandes utilisées ici sont définies dans l'entête.

\begin{center}
\begin{tikzpicture}[general, scale=1]
\quadrillageSeyes{(-3.2,-2.4)}{(6.4,3.2)} 
\tkzDefPoint(0.8,1.6){A}\tkzLabelPoints[font=\boldmath,above left,color=blue](A)
\tkzDefPoint(2.4,0.8){B}\tkzLabelPoints[font=\boldmath,below right,color=blue](B)
\tkzDefPoint(0,0){C}\tkzLabelPoints[font=\boldmath,above=4pt,color=blue](C)
\tkzDefPointWith[colinear=at C,K=1](B,A)\tkzGetPoint{E}%\vect{CE}=\vect{BA}
\tkzLabelPoints[font=\boldmath,above left,color=blue](E)
\tkzDefPointWith[colinear=at B,K=1](C,B)\tkzGetPoint{F}%\vect{BF}=\vect{CB}
\tkzLabelPoints[font=\boldmath,above right,color=blue](F)
\tkzDrawPoints[size=2,color=blue](A,B,C,E,F)
\tkzDrawVectors[color=blue,line width=1.5pt,>=stealth](C,E B,A)
\tkzDrawVectors[color=red,line width=1.5pt,>=stealth](B,F C,B)

\tkzDefPointWith[colinear=at C,K=1](A,C)\tkzGetPoint{G}%\vect{CG}=\vect{AC}
\tkzDrawVectors[color=OliveGreen,line width=1.5pt,style=dotted,>=stealth](B,C C,G)
\tkzDrawVector[color=OliveGreen,line width=1.5pt,>=stealth](B,G)
\tkzLabelSegment[below right,color=OliveGreen,font=\boldmath](B,G){$\vect{u}$}
\tkzLabelSegment[left=4pt,color=OliveGreen](C,G){$\vect{AC}$}

\end{tikzpicture}
\end{center}

%%%%%%%%%%%%%%%%%%%%%%%%%%%%%%%%%%%%%%%%%%%%%%%%%%%%%%%%%%%%

%%%%%%%%%%%%%%%%%%%%%%%%%%%%%%%%%%%%%%%%%%%%%%%%%%%%%%%%%%%%

\textbf{Exemple 15 :}

\begin{center}
\begin{tikzpicture}[scale=2]
\tkzInit[xmin=-3,xmax=3,ymin=-2,ymax=2]
\tkzGrid[sub,subxstep=0.5,subystep=0.5]
\tkzAxeXY[>=stealth,color=blue,line width=1.5pt]
\tkzFct[samples=200,color=red,line width=2pt,domain=-3:-0.5]{1/x}
\tkzFct[samples=200,color=red,line width=2pt,domain=0.5:3]{1/x}
\end{tikzpicture}
\end{center}


\begin{center}
\begin{tikzpicture}[scale=2]
\tkzInit[xmax=2,ymax=1.5]
\tkzGrid[sub,subxstep=0.5,subystep=0.5]
\tkzAxeXY[>=stealth,color=blue,line width=1.5pt]
\tkzFct[samples=2,color=red,line width=2pt,domain=0:1]{x}
\tkzFct[samples=2,color=red,line width=2pt,domain=1:2]{2-x}
\end{tikzpicture}
\end{center}
%%%%%%%%%%%%%%%%%%%%%%%%%%%%%%%%%%%%%%%%%%%%%%%%%%%%%%%%%%%%

%%%%%%%%%%%%%%%%%%%%%%%%%%%%%%%%%%%%%%%%%%%%%%%%%%%%%%%%%%%%

\textbf{Exemple 16 :}

\begin{center}
\begin{tikzpicture}[scale=0.65,>=stealth]
\tkzInit[xmin=14.90,xmax=15.10,xstep=0.01,ymin=0,ymax=500,ystep=50]
\tkzGrid
\tkzDrawX[line width=1.5pt,color=blue,below right,label=Diamètres,>=stealth]
\tkzLabelX[label options={below=12pt,rotate=-45}]
\tkzDrawY[line width=1.5pt,color=blue,above left,label=E.C.C.,>=stealth]
\tkzLabelY[label options={left=6pt}]
\tkzDefPoint(14.9,0){A}
\tkzDefPoint(14.92,5){B}
\tkzDefPoint(14.94,42){C}
\tkzDefPoint(14.96,106){D}
\tkzDefPoint(14.98,204){E}
\tkzDefPoint(15,329){F}
\tkzDefPoint(15.02,431){G}
\tkzDefPoint(15.04,479){H}
\tkzDefPoint(15.06,493){I}
\tkzDefPoint(15.08,498){J}
\tkzDefPoint(15.1,500){K}
\tkzDrawPoints[color=red,size=4pt](B,C,D,E,F,G,H,I,J,K)
\tkzDrawSegments[color=red,line width=1.5pt](A,B B,C C,D D,E E,F F,G G,H H,I I,J J,K)
%Médiane
\tkzDefPoint(14.9,250){L}
\tkzDefPoint(14.9874,250){M}
\tkzDefPoint(14.9874,0){N}
\tkzDrawSegment[color=OliveGreen,line width=1.5pt,style=dashed](L,M)
\tkzDrawSegment[->,>=stealth,color=OliveGreen,line width=1.5pt,style=dashed](M,N)
\tkzText[below=30pt,color=OliveGreen](N){Med}
%1er quartile
\tkzDefPoint(14.9,125){R}
\tkzDefPoint(14.9639,125){S}
\tkzDefPoint(14.9639,0){T}
\tkzDrawSegment[color=OliveGreen,line width=1.5pt,style=dashed](R,S)
\tkzDrawSegment[->,>=stealth,color=OliveGreen,line width=1.5pt,style=dashed](S,T)
\tkzText[below=30pt,color=OliveGreen](T){$Q_1$}
%3eme quartile
\tkzDefPoint(14.9,375){U}
\tkzDefPoint(15.009,375){V}
\tkzDefPoint(15.009,0){W}
\tkzDrawSegment[color=OliveGreen,line width=1.5pt,style=dashed](U,V)
\tkzDrawSegment[->,>=stealth,color=OliveGreen,line width=1.5pt,style=dashed](V,W)
\tkzText[below=30pt,color=OliveGreen](W){$Q_3$}
\end{tikzpicture}
\end{center}

%%%%%%%%%%%%%%%%%%%%%%%%%%%%%%%%%%%%%%%%%%%%%%%%%%%%%%%%%%%%

%%%%%%%%%%%%%%%%%%%%%%%%%%%%%%%%%%%%%%%%%%%%%%%%%%%%%%%%%%%%

\textbf{Exemple 17 :}

Voici l'histogramme de la série :

\begin{center}
\begin{tikzpicture}[scale=0.8]
\tkzInit[xmin=270,xmax=330,xstep=5,ymin=-1,ymax=10,ystep=1]
\tkzGrid[color=gray!50]
\tkzDrawX[line width=2pt,color=blue,>=stealth,label=Prix relevés,below right]
\tkzLabelX%[label options={below=12pt,rotate=-45}]

%Unité d'aire
\tkzDefPoint(275,6){Q}
\tkzDefPoint(280,6){R}
\tkzDefPoint(280,7){S}
\tkzDefPoint(275,7){T}
\tkzDrawPolygon[fill=gray!50](Q,R,S,T)
\tkzText(277.5,5.5){2 artisans}

%Premier rectangle
\tkzDefPoint(280,0){A}
\tkzDefPoint(290,0){B}
\tkzDefPoint(290,0.5){C}
\tkzDefPoint(280,0.5){D}
\tkzDrawPolygon[fill=gray!50](A,B,C,D)

%Deuxième rectangle
\tkzDefPoint(300,0){E}
\tkzDefPoint(300,3){F}
\tkzDefPoint(290,3){G}
\tkzDrawPolygon[fill=gray!50](B,E,F,G)

%Troisième rectangle
\tkzDefPoint(305,0){H}
\tkzDefPoint(305,9){I}
\tkzDefPoint(300,9){J}
\tkzDrawPolygon[fill=gray!50](E,H,I,J)

%Quatrième rectangle
\tkzDefPoint(310,0){K}
\tkzDefPoint(310,5){L}
\tkzDefPoint(305,5){M}
\tkzDrawPolygon[fill=gray!50](H,K,L,M)

%Ciquième rectangle
\tkzDefPoint(320,0){N}
\tkzDefPoint(320,1.25){O}
\tkzDefPoint(310,1.25){P}
\tkzDrawPolygon[fill=gray!50](K,N,O,P)

\end{tikzpicture}
\end{center}

%%%%%%%%%%%%%%%%%%%%%%%%%%%%%%%%%%%%%%%%%%%%%%%%%%%%%%%%%%%%

%%%%%%%%%%%%%%%%%%%%%%%%%%%%%%%%%%%%%%%%%%%%%%%%%%%%%%%%%%%%

\textbf{Exemple 18 : Tangentes et effet loupe}

% requires \usetikzlibrary{spy}

\begin{center}
\begin{tikzpicture}[scale=1,spy using outlines={circle, magnification=3, connect spies}]
\tkzInit[xmin=-6,xmax=9,ymin=-2,ymax=8]
\tkzGrid[sub,subxstep=0.5,subystep=0.5]
\tkzAxeXY[>=stealth,color=blue,line width=1.5pt]
\tkzClip
%Courbe de la fonction
\tkzFct[color=red,samples=200,line width=1.5pt]
{(\x**3+\x**2-4*\x+5)/(2*\x**2-8*\x+10)}
\tkzDefPoint(7,7){F}
\tkzText[font=\boldmath,color=red,below right](F){$\calig{C}_f$}

%Points du graphique
\tkzDefPoint(0,0.5){A}\tkzLabelPoints[font=\bfseries,above right,color=blue](A)
\tkzDefPoint(2.5,6.75){B}\tkzLabelPoints[font=\bfseries,left,color=blue](B)
\tkzDrawPoints[size=4,color=red](A,B)

%Tangentes tracées
\tkzDefPoint(3,8){C}
\tkzDefPointWith[linear,K=2](C,B)\tkzGetPoint{G}%\vect{CG}=2*\vect{CB}
\tkzDrawSegment[<->,>=stealth,line width=1.5pt](C,G)
\tkzDefPoint(-1.5,0.5){D}
\tkzDefPoint(1.5,0.5){E}
\tkzDrawSegment[<->,>=stealth,line width=1.5pt](D,E)

%effet loupe
\tkzDefPoint(6,3){magnifyglass}
\spy [blue, size=5cm] on (B) in node[fill=white] at (magnifyglass);
\end{tikzpicture}
\end{center}

%%%%%%%%%%%%%%%%%%%%%%%%%%%%%%%%%%%%%%%%%%%%%%%%%%%%%%%%%%%%

%%%%%%%%%%%%%%%%%%%%%%%%%%%%%%%%%%%%%%%%%%%%%%%%%%%%%%%%%%%%
\newpage
\textbf{Exemple 19 :}

\begin{center}
\begin{tikzpicture}[xscale=1.5,yscale=0.8,>=stealth]
\tkzInit[xmin=80,xmax=160,xstep=10,ymax=1.1,ystep=0.1]
\tkzGrid
\tkzDrawX[line width=1.5pt,color=blue,below right,label=Poids,>=stealth]
\tkzLabelX%[label options={below=12pt,rotate=-45}]
\tkzDrawY[line width=1.5pt,color=blue,above left,label=F.C.C.,>=stealth]
\tkzLabelY%[label options={left=6pt}]
\tkzDefPoint(80,0){A}
\tkzDefPoint(90,0.161){B}
\tkzDefPoint(100.94,0.516){C}
\tkzDefPoint(110,0.645){D}
\tkzDefPoint(120,0.871){E}
\tkzDefPoint(130,0.968){F}
\tkzDefPoint(150,0.968){G}
\tkzDefPoint(160,1){H}
\tkzDrawSegments[color=red,line width=1.5pt](A,B B,C C,D D,E E,F F,G G,H)
\tkzDrawPoints[shape=rectangle,color=red,size=4pt](B,C,D,E,F,G,H)
\end{tikzpicture}
\end{center}


%%%%%%%%%%%%%%%%%%%%%%%%%%%%%%%%%%%%%%%%%%%%%%%%%%%%%%%%%%%%
%\newpage
\medskip
\textbf{Exemple 20 :}


\begin{center}
\begin{tikzpicture}[scale=1,>=stealth]
\tkzInit[xmin=-4,xmax=6,ymin=-2,ymax=8]
\tkzGrid[color=gray!50]
\tikzset{xaxe style/.style={-}}
\tikzset{yaxe style/.style={-}}
\tkzDrawX[line width=1.5pt,color=blue,label=,right space=0]
\tkzDrawY[line width=1.5pt,color=blue,label=,up space=0]
\tkzRep[xnorm=1,ynorm=1,color=red,line width=2pt]

%Points
\tkzDefPoint(-2,5){M}
\tkzDefPoint(2,-1){N}
\tkzDefPoint(5,1){P}
\tkzDefPoint(0,0){O}
\tkzDefPoint(1,7){Q}


%polygone
\tkzDrawPolygon[style=dashed,color=red,line width=2 pt](M,N,P,Q)

%marques sur les points
\tkzDrawPoints[shape=cross out,size=3pt](M,N,P,Q)

%repère (O;I,J)
\tkzLabelPoints[below left=3pt,font=\boldmath,color=red](O)
%\tkzLabelPoints[below right=3pt,font=\boldmath,color=red](I)
%\tkzLabelPoints[above left=3pt,font=\boldmath,color=red](J)

%autres points
\tkzLabelPoints[above right=3pt,font=\boldmath](P,Q)
\tkzLabelPoints[above left=3pt,font=\boldmath](M)
%\tkzLabelPoints[below right=3pt,font=\boldmath](N)
\tkzLabelPoints[below left=3pt,font=\boldmath](N)

\end{tikzpicture}
\end{center}

%%%%%%%%%%%%%%%%%%%%%%%%%%%%%%%%%%%%%%%%%%%%%%%%%%%%%%%%%%%%
\newpage
%\medskip
\textbf{Exemple 21 :}


\textbf{Point aléatoire sur un cercle}

\begin{center}
\begin{tikzpicture}[scale=1]
\tkzDefPoint(0,0){A}
\tkzGetRandPointOn[circle=center A radius 2cm]{B}
%cercle de rayon [AB]
\tkzDrawCircle[color=red,line width=2 pt](A,B)
\tkzDrawPoint[size=2,color=red](A)
\tkzLabelPoints[font=\boldmath,below left,color=red](A)
\tkzLabelPoints[font=\boldmath,below right,color=red](B)
%rayon
\tkzDrawSegments[style=dotted,line width=1.5pt](A,B)
\tkzLabelSegment[above,pos=0.5,color=blue](A,B){R}
%tangente
\tkzTangent[at=B](A)
\tkzGetPoint{h}
\tkzDrawLine[style=dashed,color=blue,line width=1.5pt,add=4 and 3](B,h)
\tkzMarkRightAngle[fill=red!50](A,B,h)
\end{tikzpicture}
\end{center}

%%%%%%%%%%%%%%%%%%%%%%%%%%%%%%%%%%%%%%%%%%%%%%%%%%%%%%%%%%%%
%\newpage
\medskip
\textbf{Exemple 22 :}



\begin{tikzpicture}[scale=1]
\tkzDefPoint(0,0){O}
\tkzDefPoint(-30:3){A}
\tkzDefPointBy[rotation = center O angle -60](A)
\tkzDrawSector[fill=red!50](O,A)(tkzPointResult)
\begin{scope}[shift={(-60:1cm)}]
\tkzDefPoint(0,0){O}
\tkzDefPoint(-30:3){A}
\tkzDefPointBy[rotation = center O angle -60](A)
\tkzDrawSector[fill=blue!50](O,tkzPointResult)(A)
\end{scope}
\end{tikzpicture}

%%%%%%%%%%%%%%%%%%%%%%%%%%%%%%%%%%%%%%%%%%%%%%%%%%%%%%%%%%%%
\medskip
\textbf{Exemple 23 : Projeté orthogonal}

\begin{tikzpicture}
\tkzDefPoint(0,0){A}
\tkzLabelPoints[font=\boldmath,above=3pt,color=red](A)
\tkzDefPoint(5,3){B}
\tkzLabelPoints[font=\boldmath,above=3pt,color=red](B)
\coordinate (C) at ($(A)!1.5!(B)$); %vect(AC) = 1.5\vect(AB)
\coordinate (D) at ($(B)!1.5!(A)$); %vect(BD) = 1.5\vect(BA)
\tkzDrawSegments[line width=1.5pt,color=blue](C,D)

%% pour les projetés orthogonaux
\tkzDefPoint(4,0){E}
\tkzLabelPoints[font=\boldmath,below=3pt,color=red](E)
\coordinate (F) at ($(A)!(E)!(B)$);
\tkzDrawSegments[line width=1.5pt,color=blue,style=dashed](E,F)

\tkzMarkRightAngle[fill=blue!50](E,F,B)

\tkzDrawPoints[color=red,size=12pt](A,B,E)
\end{tikzpicture}
%%%%%%%%%%%%%%%%%%%%%%%%%%%%%%%%%%%%%%%%%%%%%%%%%%%%%%%%%%%%
\end{spacing}
%%%%%%%%%%%%%%%%%%%%%%%%%%%%%%%%%%%%%%%%%%%%%%%%%%%%%%%%%%%%
%%%%%%%%%%%%%%%%%%%%%
%% FIN DU DOCUMENT %%
%%%%%%%%%%%%%%%%%%%%%
\end{document}