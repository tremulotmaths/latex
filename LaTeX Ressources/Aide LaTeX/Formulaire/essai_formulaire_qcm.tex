\documentclass[12pt,oneside]{report}
%%%%%%%%%%%%%%%%%%%%%%%%%%%%%%%%%%%%%%%%%%%%%%%%%%%%%%%%%%%%%%%%%%%%%%%%%%%%%%%
\input{preambule_2020}
%%%%%%%%%%%%%%%%%%%%%%%%%%%%%%%%%%%%%%%%%%%%%%%%%%%%%%%%%%%%%%%%%%%%%%%%%%%%%%%
\portrait
%\paysage
%%%%%%%%%%%%%%%%%%%%%%%%%%%%%%%%%%%%%%%%%%%%%%%%%%%%%%%%%%%%%%%%%%%%%%%%%%%%%%%

%À modifier !!!!!!!!!!!!!!!!!!!!!!!!!!!!!!!!
\newcommand{\classe}{TS 2}
\newcommand{\anneescol}{Année 2019-2020}

%entête classique

\fancypagestyle{garde_tete}{% 
%\fancyhead[C]{\small\textbf{\seconde} \hfill \small \textbf{Année 2015-2016}}
\renewcommand{\headrulewidth}{0cm}}

\newcommand{\tete}{
\thispagestyle{garde_tete}
\chapitre{QCM 1 - Les fonctions logarithmes}
\noindent 
\vspace{-1em}
}

%%%%%%%%%%%%%%%%%%%%%%%%%%%%%%%%%%%%%%%%%%%%%%%%%%%%%%%%%%%%%%%%%%%%%%%%%%%%%%%
%%%%%%%%%%%%%%%%%%%%%%%%%%%%%%%%%%%%%%%%%%%%%%%%%%%%%%%%%%%%%%%%%%%%%%%%%%%%%%%
%\tikzset{domaine/.style 2 args={domain=#1:#2}}
%%%%%%%%%%%%%%%%%%%%%%%%%%%%%%%%%%%%%%%%%%%%%%%%%%%%%%%%%%%%%%%%%%%%%%%%%%%%%%%
%%%%%%%%%%%%%%%%%%%%%%%%%%%%%%%%%%%%%%%%%%%%%%%%%%%%%%%%%%%%%%%%%%%%%%%%%%%%%%%

%%%%%%%%%%%%%%%%%%%%%%%
%% DEBUT DU DOCUMENT %%
%%%%%%%%%%%%%%%%%%%%%%%

\begin{document}
%\selectlanguage{english}
\selectlanguage{french}
\setlength\parindent{0mm}
\tete 		%entête classique

\renewcommand \footrulewidth{0.2pt}%
\renewcommand \headrulewidth{0pt}%
\pagestyle{fancy}
\fancyhf{}
\pieddepage{\classe}{\thepage / \pageref{LastPage}}{\anneescol}

%%%%%%%%%%%%%%%%%%%%%%%%%%%%%%%%%%%%%%%%%%%%%%%%%%%%%%%%%%%%
\begin{spacing}{1.2}
%%%%%%%%%%%%%%%%%%%%%%%%%%%%%%%%%%%%%%%%%%%%%%%%%%%%%%%%%%%%

%\section*{Documentation}
%
%\url{http://www.tug.org/applications/hyperref/manual.html#x1-200006}

\section*{Nom, prénom}

\begin{Cadre}
\begin{Form}
\TextField[name=name,width=15em]{Prénom :}
\TextField[name=surname,width=15em]{NOM :}
\end{Form}
\medskip
\end{Cadre}

\section*{Question 1}

La fonction ln est
\begin{Form}
\ChoiceMenu[combo,name=niveau1,default=dérivable sur $\R$,charsize=10pt,align=0]{}
{
dérivable sur $\intervalleoo{0}{+\infty}$,
dérivable sur $\R$
}\\
\ChoiceMenu[radio,default=Male,name=sex,multiline=true]{Sex:}{Male,Female}
\end{Form}



\section*{Feedback}

Vous avez trouvé ce QCM\dots

\begin{Cadre}
\begin{Form}
\ChoiceMenu[combo,name=niveau,default=Très facile,charsize=10pt,align=0]
           {Niveau :}{Très facile,Facile,Un peu compliqué,Très compliqué}
\end{Form}
\end{Cadre}


  

%%%%%%%%%%%%%%%%%%%%%%%%%%%%%%%%%%%%%%%%%%%%%%%%%%%%%%%%%%%%
\end{spacing}
%%%%%%%%%%%%%%%%%%%%%%%%%%%%%%%%%%%%%%%%%%%%%%%%%%%%%%%%%%%%
%%%%%%%%%%%%%%%%%%%%%
%% FIN DU DOCUMENT %%
%%%%%%%%%%%%%%%%%%%%%
\end{document}