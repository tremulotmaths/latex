\documentclass{article}

\usepackage{lmodern}
\usepackage[T1]{fontenc}
\usepackage[scale=.9]{geometry}
\usepackage{amssymb}
\usepackage{tikz}
\usepackage{setspace}
\usepackage{hyperref}
\renewcommand{\LayoutChoiceField}[2]{%
\leavevmode #2 #1%
}
\renewcommand{\LayoutCheckField}[2]{%
\leavevmode #2 #1%
}

\newcommand*{\mtcode}{123456}

%%%%%%%%%%%%%%%%%%%%%%%%%%%%%%%%%%%%%%%%%%%%%%%%%%%%%%%%%%%%%%%%%%%%%
\begin{document}
%%%%%%%%%%%%%%%%%%%%%%%%%%%%%%%%%%%%%%%%%%%%%%%%%%%%%%%%%%%%%%%%%%%%%
\begin{spacing}{1.2}
%%%%%%%%%%%%%%%%%%%%%%%%%%%%%%%%%%%%%%%%%%%%%%%%%%%%%%%%%%%%%%%%%%%%


\begin{Form}
\TextField[bordercolor=1 0 0,width=10em,charsize=0pt,name=box]{NOM : }
\vspace{1cm}

\begin{enumerate}
\item 
Une primitive de $x \mapsto x^2$ est :\\
\ChoiceMenu[radio,name=Q1,default=-0,radiosymbol=\ding{52}]
{$x \mapsto x^3$=0,\\[0.5em]
$\displaystyle x \mapsto \frac{1}{3}x^3$=1,\\[0.5em]
$x \mapsto 2x$=0}{=q1}


\item Ma question 2 :

\makeatletter\Fld@checkedfalse\makeatother 
\CheckBox[name=Q2.1,width=0.4cm,height=0.4cm,bordercolor=black]{Oui}{}\\
\CheckBox[name=Q2.2,width=0.4cm,height=0.4cm,bordercolor=black]{Non}{}


\item Ma question 3 :

\ChoiceMenu[radio,name=Q3,default=-0,radiosymbol=\ding{52}]{}{
réponse à 2 points\hfill=2,\\
réponse à 1 point\hfill=1,\\
Réponse fausse\hfill=0}


\ChoiceMenu[name=Q3,radio,default=-0,radiosymbol=\ding{52}]{Ma Question 3 :}{ réponse à 2 points=2,réponse à 1 point=1,Réponse fausse=0}

\item La fonction ln est

\ChoiceMenu[radio,name=Q4,default=-0,align=0,radiosymbol=\ding{52}]{}
{dérivable sur $]0;+\infty[$=1,\\
dérivable sur $\mathbb{R}$=0}


\item La fonction ln est

\ChoiceMenu[radio,name=Q5,default=-0,align=0,radiosymbol=\ding{108},height=0.4cm,bordercolor=black]{}
{dérivable sur $]0;+\infty[$=1,\\
dérivable sur $\mathbb{R}$=0}

\item La fonction ln est

\ChoiceMenu[combo,name=Q6,default=-0,align=0,radiosymbol=\ding{108},height=0.4cm,bordercolor=black]{}
{dérivable sur $]0;+\infty[$=1,\\
dérivable sur $\mathbb{R}$=0}

\item La fonction ln est

\ChoiceMenu[radio,name=Q5,default=-0,align=0,radiosymbol=\ding{108},height=0.4cm,bordercolor=black]{}
{dérivable sur $]0;+\infty[$=1,
\hspace{0.2\linewidth}dérivable sur $\mathbb{R}$=0}

\end{enumerate}

\vspace{2cm}
\TextField[password,bordercolor=1 0 0,width=10em,charsize=0pt,name=boxcode]{code : }
%
%
\PushButton[name=verificationcode,bordercolor=1 1 1,
onclick=
{
if (this.getField("boxcode").value==\mtcode)
{
this.getField("obtenirscore").value=(this.getField("Q1").value+this.getField("Q2").value+this.getField("Q3").value);
}
else
{
this.getField("obtenirscore").value="Code incorrect";
}
this.getField("boxcode").value="";
}
]
{\tikz\node[rounded corners, draw=black!80, fill=black!20]  {\bf score final :};}
\TextField[name=obtenirscore,bordercolor=1 1 1,width=10em,charsize=0pt,readonly=true]{}
\end{Form}

%%%%%%%%%%%%%%%%%%%%%%%%%%%%%%%%%%%%%%%%%%%%%%%%%%%%%%%%%%%%%%%%%%%%
\end{spacing}
%%%%%%%%%%%%%%%%%%%%%%%%%%%%%%%%%%%%%%%%%%%%%%%%%%%%%%%%%%%%%%%%%%%%
\end{document}