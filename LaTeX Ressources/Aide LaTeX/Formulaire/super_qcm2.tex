\documentclass{article}

\usepackage{lmodern}
\usepackage[T1]{fontenc}
\usepackage[scale=.9]{geometry}
\usepackage{amssymb}
\usepackage{tikz}
\usepackage{setspace}
\usepackage{hyperref}
\renewcommand{\LayoutChoiceField}[2]{%
\leavevmode #2 #1%
}
\newcommand*{\mtcode}{123456}

%%%%%%%%%%%%%%%%%%%%%%%%%%%%%%%%%%%%%%%%%%%%%%%%%%%%%%%%%%%%%
\begin{document}
%%%%%%%%%%%%%%%%%%%%%%%%%%%%%%%%%%%%%%%%%%%%%%%%%%%%%%%%%%%%%

\begin{Form}
Choose wisely:\\ \\
\ChoiceMenu[radio,name=pokemon,radiosymbol=\ding{52}]{Charmander is good.}{=ch1}\\
\ChoiceMenu[radio,name=pokemon,radiosymbol=\ding{52}]{Bulbausaur is good.}{=ch2}\\
\ChoiceMenu[radio,name=pokemon,radiosymbol=\ding{52}]{Squirtle is good.}{=ch3}

\TextField[password,bordercolor=1 0 0,width=10em,charsize=0pt,name=boxcode]{code : }
%
%
\PushButton[name=verificationcode,bordercolor=1 1 1,
onclick=
{
if (this.getField("boxcode").value==\mtcode)
{
%this.getField("obtenirscore").value=(this.getField("pokemon1").value+this.getField("pokemon2").value+this.getField("pokemon3").value);
this.getField("obtenirscore").value=(this.getField("pokemon").value);
}
else
{
this.getField("obtenirscore").value="Code incorrect";
}
this.getField("boxcode").value="";
}
]
{\tikz\node[rounded corners, draw=black!80, fill=black!20]  {\bf score final :};}
\TextField[name=obtenirscore,bordercolor=1 1 1,width=10em,charsize=0pt,readonly=true]{}
\end{Form}

%%%%%%%%%%%%%%%%%%%%%%%%%%%%%%%%%%%%%%%%%%%%%%%%%%%%%%%%%%%%%
\end{document}

