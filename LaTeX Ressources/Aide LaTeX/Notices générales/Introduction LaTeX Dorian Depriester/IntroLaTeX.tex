\documentclass[svgnames,smaller]{beamer}
\usetheme{Warsaw}
%\useoutertheme{infolines}
%\usepackage[final]{movie15}
%\usepackage{default}
\usepackage[utf8]{inputenc}
\usepackage[francais]{babel}
\usepackage{etex}
\usepackage[babel=true,kerning=true]{microtype}
\usepackage{multimedia}
\usepackage[squaren,Gray]{SIunits}
\usepackage{multirow}
\usepackage{colortbl}
%\usepackage[pdftex]{graphicx}
\usepackage{graphicx}
\usepackage{multicol}
\usepackage[version=3]{mhchem}
\usepackage{caption}
\usepackage{pdfpages}
\usepackage{pgfplots}
\usepackage{tikz}
\pgfplotsset{compat=1.3}
\definecolor{lightgray}{gray}{0.95}
\usepackage{listings}
	\lstset{backgroundcolor=\color{lightgray},language=[LaTeX]TeX,texcsstyle=\color{blue},commentstyle=\color{gray},
	literate=
	{à}{{\`a}}1
	{é}{{\'e}}1
	{è}{{\`e}}1}
\usepackage{wrapfig}
\usepackage{amsmath}
\usepackage{verbatim}
\usepackage{breakcites}
\usepackage[normalem]{ulem}
\usepackage{nicefrac}
\usepackage[sort&compress,comma]{natbib}
\def\newblock{}
\setlength{\unitlength}{1mm}
\date{janvier 2012}
\author{Dorian Depriester}

\newcommand{\backupbegin}{
   \newcounter{framenumberappendix}
   \setcounter{framenumberappendix}{\value{framenumber}}
}
\newcommand{\backupend}{
   \addtocounter{framenumberappendix}{-\value{framenumber}}
   \addtocounter{framenumber}{\value{framenumberappendix}} 
}




\title{Introduction à \LaTeX}

\AtBeginSection[]{
   \begin{frame}
\begin{block}{}
   %%% affiche en début de chaque section, les noms de sections et
   %%% noms de sous-sections de la section en cours.
   \tableofcontents[currentsection,hideothersubsections]
\end{block}
   \end{frame} 
}





    \setbeamertemplate{caption}[numbered]
\addtobeamertemplate{footline}{\insertframenumber/\inserttotalframenumber}


\begin{document}
\maketitle



\section{Historique}
\frame{
		\begin{block}{Fin des années 70}
			\begin{columns}
				\begin{column}{0.2\linewidth}
	    			\includegraphics[width=\linewidth]{Donald.jpg}
				\end{column}
				\begin{column}{0.7\linewidth}
	    			Langage \TeX{} par Donald Knuth : invention du principe du \textit{What-you-see-is-what-you-mean} (WYSIWYM)
				\end{column}
			\end{columns}
		\end{block}
		\begin{block}{Années 80}
			\begin{columns}
				\begin{column}{0.7\linewidth}
	    			Leslie Lamport propose une surcouche simplifiée : \LaTeX
				\end{column}
				\begin{column}{0.2\linewidth}
	    			\includegraphics[width=\linewidth]{Leslie.jpg}
				\end{column}
			\end{columns}
		\end{block}
}



\section{Principe}
\frame{
\begin{block}{Principe du WYSIWYM}
		L'utilisateur se concentre sur le contenu, et laisse le logiciel gérer la mise en page. Le contenu est écrit dans un fichier texte standard (\texttt{.tex}) puis ce contenu est lu par le compilateur pour générer le document.
\end{block}

\begin{block}{Qu'est-ce que \LaTeX}<2->
		\LaTeX{} est un compilateur. Depuis le \texttt{.tex}, il va générer le document avec la mise en page souhaitée.
\end{block}
}

\frame{
\frametitle{Obtenir un fichier pdf}
\begin{block}{Avec le compilateur \texttt{latex}}
\centering
	\texttt{.tex}$\xrightarrow{\text{\texttt{latex}}}$\texttt{.dvi}$\xrightarrow{\text{\texttt{dvips}}}$\texttt{.ps}
		\visible<2->{$\left( \xrightarrow{\text{\texttt{pstopdf}}}\text{\texttt{.pdf}}\right)$}
\end{block}

\begin{block}{Avec le compilateur \texttt{pdflatex}}<3->
	\centering
	\texttt{.tex}$\xrightarrow{\text{\texttt{pdflatex}}}$\texttt{.pdf}
\end{block}


\begin{block}{Fichiers supplémentaires (créés lors de la compilation)}<4->
	\begin{description}
		\item[\texttt{.aux}] Fichier de configuration. Sert aux références croisées
		\item[\texttt{.log}] Fichier de journal
		\item[\texttt{.toc}] Table des matières
		\item[...] 
	\end{description}
\end{block}
}


\section{Pourquoi \LaTeX}
\frame{
	\begin{block}{Avantages}
		\begin{description}
			\item[Typographique] \LaTeX{} respecte des normes typographiques très strictes, propres à chaque langue.
			\item[Extensibilité] De très nombreuses \textit{packages} disponibles
			\item[Pérennité] Toutes les versions de \LaTeX{} sont rétrocompatibles
			\item[Modularité] Gestion aisée des très gros documents, possibilité de ne compiler que certaines parties
			\item[Portabilité] Compatible tout OS (Linux, Windows, Mac)
		\end{description}
	\end{block}
	
	\begin{alertblock}{Inconvénients}<2->
		\begin{itemize}
			\item Nécessite un apprentissage
			\item Résultat moins direct
			\item Personnalisation de la mise en page qui peut sembler complexe
		\end{itemize}
	\end{alertblock}	
	}
	
	
	
\section{Utilisation}
	\subsection{Édition du fichier source}	
\frame{
\begin{block}{Éditeur graphique (IDE)}
	\centering\includegraphics[height=0.5\textheight]{texmaker.jpg}
\end{block}
}

\subsection{Exemple minimal}
\begin{frame}[fragile]
\begin{lstlisting}
	\documentclass{article}       % Mise en page
	
	\usepackage[francais]{babel}  % Règles typo.
	\usepackage[latin1]{inputenc} % Accents
	\usepackage[T1]{fontenc}      % Fontes EC
	
	\begin{document}
		  Bonjour tout le monde !
	\end{document}
\end{lstlisting}
\end{frame}


\subsection{Commandes habituelles}
\begin{frame}[fragile]
	\begin{block}{Dans le préambule}
		\begin{lstlisting}
			\title{Introduction à \LaTeX}
			\date{janvier 2012}
			\author{Dorian Depriester}
		\end{lstlisting}
	\end{block}
	
	\begin{block}{Dans le document}
		\begin{lstlisting}
			\maketitle
			\tableofcontents
			\newpage
			\section{Utilisation}
			  \subsection{Exemple minimal}
		\end{lstlisting}
	\end{block}
\end{frame}

	
\subsection{L'environnement mathématique}
\begin{frame}[fragile]
	\begin{block}{Packages nécessaires}
	  amsmath, amsfonts et mathrsfs
	\end{block}

	\begin{block}{Utilisation ``inline''}
  		\begin{lstlisting}
			On sait que $\sin(\pi)=0$ donc tout va bien.
		\end{lstlisting}
			On sait que $\sin(\pi)=0$ donc tout va bien.
	\end{block}
	
	\begin{block}{Équation centrée}
  		\begin{lstlisting}
			De plus : 
			\begin{equation}
			  \sin(\frac{\pi}{2})=1
			\end{equation}
		\end{lstlisting}
			De plus : 
			\begin{equation}
			  \sin(\frac{\pi}{2})=1
			\end{equation}
	\end{block}
\end{frame}





\section{Les figures}
	\subsection{Les flottants}
	\begin{frame}[fragile]
		\begin{block}{Utilité}
		  Les éléments flottants servent à optimiser l'espace, tout en rendant la lecture plus continue grâce aux rassemblement des figures sur une même page.
		\end{block}
		
		\begin{block}{Sous \LaTeX}
		  Les éléments flottants sont définis dans les environnements tels que \texttt{figure} ou \texttt{table}. On spécifie en option la position \textbf{souhaitée}.
		\end{block}
		
		\begin{exampleblock}{Exemple}<2->
			\begin{lstlisting}
				\begin{figure}[htbp]
				  % Ici le contenu de la figure flottante
				  \caption{Légende de la figure}
				\end{figure}				 
			\end{lstlisting}
		\end{exampleblock}
	\end{frame}


	\subsection{Images}
		\begin{frame}[fragile]
			\begin{block}{package nécessaire}
				graphicx
			\end{block}

			\begin{block}{Insertion d'une image}
				\begin{lstlisting}
					\begin{figure}[htbp]
						  \includegraphics[width=3cm]{monimage.jpg}
						  \caption{Légende de la figure}
					\end{figure}		
				\end{lstlisting}
				\visible<2->{
					{\centering
					\includegraphics[width=0.2\textwidth]{guy-fawkes.jpg}
					\captionof{figure}{Légende de la figure}
					
					}
							}
			\end{block}
		\end{frame}

\begin{frame}
	\frametitle{Formats pris en charge}
	\begin{block}{Avec \texttt{latex}}
		PostScript Encapsulé (\texttt{.eps})
	\end{block}
	\begin{block}{Avec \texttt{pdflatex}}
		\begin{itemize}
			\item JPEG
			\item PNG
			\item PDF
		\end{itemize}
	\end{block}
\end{frame}



	\subsection{Tableaux}
		\begin{frame}[fragile]
			\begin{block}{environnement \texttt{tabular}}
				\begin{lstlisting}
					\begin{tabular}{|l|c|r|}
						  \hline
						  Colonne 1 & Colonne 2 & Colonne 3\\
						  \hline
						  à gauche  & centré    & à droite\\
						  \hline
					\end{tabular}
				\end{lstlisting}
					\visible<2->{
					\begin{tabular}{|l|c|r|}
						\hline
						Colonne 1	&	Colonne 2	& Colonne 3\\
						\hline
						à gauche	&	centré		& à droite\\
						\hline
					\end{tabular}
					}
			\end{block}
		\end{frame}			
			
		\begin{frame}[fragile]
		\frametitle{Quid des tableaux flottants ?}
			\begin{block}{environnement \texttt{table}}
				\begin{lstlisting}
					\begin{table}
						  \caption{Légende du tableau}
						  \begin{tabular}{c}
						  	  \hline Un tableau quelconque\\ \hline
						  \end{tabular}
						  \label{tab:montableau}
					\end{table}
				On voit dans le tableau~\ref{tab:montableau} que...
				\end{lstlisting}
			\end{block}
			
					\begin{table}
						  \caption{Légende du tableau}
						  \begin{tabular}{c}
						  	  \hline Un tableau quelconque\\ \hline
						  \end{tabular}
					\end{table}
		\end{frame}



\section{Mise en page}
	\subsection{Sectionnement}
	\begin{frame}[fragile]
		\begin{exampleblock}{Exemple}
				\begin{lstlisting}
					\section{Mise en page}
						  \subsection{Sectionnement}
						    Je parle du sectionnement sous \LaTeX.
						  \subsection{Les références croisées}
						    Après, je parlerai des références croisées.
					\section{Conclusion}
					  Enfin je conclurai.
				\end{lstlisting}
		\end{exampleblock}
		
		\begin{block}{Différents niveaux de sectionnement}<2->
		  Dépend de la classe choisie :
		  	\begin{itemize}
		  		\item \texttt{part}
		  		\item \texttt{chapter}
		  		\item \texttt{section/subsection/subsubsection}
		  		\item \texttt{paragraph/subparagraph}
		  	\end{itemize}
		\end{block}
	\end{frame}

	\subsection{Environnements spécifiques}	
		\begin{frame}[fragile]
			\begin{columns}
				\begin{column}{0.4\textwidth}
					\begin{block}{\texttt{itemize}}
						\begin{lstlisting}
							\begin{itemize}
							  \item un
							  \item deux
							  \item trois
							\end{itemize}
						\end{lstlisting}
						\begin{itemize}
							\item un
							\item deux
							\item trois
						\end{itemize}
					\end{block}
				\end{column}
				\begin{column}{0.4\textwidth}
					\begin{block}{\texttt{enumerate}}
						\begin{lstlisting}
							\begin{enumerate}
							  \item un
							  \item deux
							  \item trois
							\end{enumerate}
						\end{lstlisting}
						\begin{enumerate}
							\item un
							\item deux
							\item trois
						\end{enumerate}
					\end{block}	
				\end{column}
			\end{columns}
			
			\begin{exampleblock}{Astuce}<2->
				Il est possible d'imbriquer ces environnements pour créer différents niveaux.
			\end{exampleblock}
		\end{frame}
	
	
	
	
\subsection{Références croisées}
\begin{frame}[fragile]
	\begin{block}{Qu'est-ce ?}
	  Elles servent à faire référence à un objet (partie, équation, figure etc.) sans avoir à se soucier de son numéro ou de sa position.
	\end{block}
	
	\begin{block}{Méthode}
	  On donne un nom à chaque élément (commande \lstinline!\label{}!) lors de sa définition, puis on fait référence à ce label (commande \lstinline!\ref{}!).
	\end{block}
	
\end{frame}
\begin{frame}[fragile]
\frametitle{Exemple}	
	%\begin{block}{Source}
	\begin{lstlisting}
		\begin{figure}[h]
	  	    \includegraphics[width=2cm]{Tux-DJ.png}
	  	    \caption{Une image au hasard}
	  	    \label{fig:hasard}
		\end{figure}
	  Je peux faire référence à ma figure~\ref{fig:hasard} 
	  sans me soucier de son numéro.
	  \end{lstlisting}
	  		\begin{figure}[h]
	  	    \includegraphics[width=2cm]{Tux-DJ.png}
	  	    \caption{Une image au hasard}
	  	    \label{fig:hasard}
		\end{figure}
		\vspace{-2em}
	  Je peux faire référence à ma figure~\ref{fig:hasard} 
	  sans me soucier de son numéro.
	%\end{block}
\end{frame}

\begin{frame}[fragile]
\frametitle{Notes}

\begin{block}{Commandes supplémentaires}
	\begin{description}
		\item[Afficher la page de la référence] \lstinline!\pageref{}!
		\item[Référence à une équation] \lstinline!\eqref{}!, met le numéro entre parenthèse
	\end{description}
\end{block}

\begin{block}{Les bonnes habitudes}
	\begin{itemize}
		\item Utiliser une espace insécable avant la référence : tilde (\lstinline!~!) sous \LaTeX
		\item Préciser pour chaque objet sa nature (figure, tableau, etc.) : \lstinline!\label{fig:hasard}!
	\end{itemize}
\end{block}


\end{frame}




\end{document}
