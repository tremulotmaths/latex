\documentclass[svgnames,smaller]{beamer}
\usetheme{Warsaw}
%\useoutertheme{infolines}
%\usepackage[final]{movie15}
%\usepackage{default}
\usepackage[utf8]{inputenc}
\usepackage[francais]{babel}
\usepackage{etex}
\usepackage[babel=true,kerning=true]{microtype}
\usepackage{multimedia}
\usepackage[squaren,Gray]{SIunits}
\usepackage{multirow}
\usepackage{colortbl}
%\usepackage[pdftex]{graphicx}
\usepackage{graphicx}
\usepackage{multicol}
\usepackage[version=3]{mhchem}
\usepackage{caption}
\usepackage{pdfpages}
\usepackage{pgfplots}
\usepackage{tikz}
\pgfplotsset{compat=1.3}
\definecolor{lightgray}{gray}{0.95}
\usepackage{listings}
	\lstset{backgroundcolor=\color{lightgray},language=[LaTeX]TeX,texcsstyle=\color{blue},commentstyle=\color{gray},
	literate=
	{à}{{\`a}}1
	{é}{{\'e}}1
	{è}{{\`e}}1}
\usepackage{wrapfig}
\usepackage{amsmath}		% Permet de taper des formules mathématiques
\usepackage{amssymb}		% Permet d'utiliser des symboles mathématiques
\usepackage{amsfonts}		% Permet d'utiliser des polices mathématiques
\usepackage{verbatim}
\usepackage{breakcites}
\usepackage[normalem]{ulem}
\usepackage{nicefrac}
\usepackage[sort&compress,comma]{natbib}
\usepackage{dtklogos}



\def\newblock{}
\setlength{\unitlength}{1mm}
\date{février 2012}
\author{Dorian Depriester}

\newcommand{\backupbegin}{
   \newcounter{framenumberappendix}
   \setcounter{framenumberappendix}{\value{framenumber}}
}
\newcommand{\backupend}{
   \addtocounter{framenumberappendix}{-\value{framenumber}}
   \addtocounter{framenumber}{\value{framenumberappendix}} 
}




\title{Les mathématiques sous \LaTeX}

\AtBeginSection[]{
   \begin{frame}
\begin{block}{}
   %%% affiche en début de chaque section, les noms de sections et
   %%% noms de sous-sections de la section en cours.
   \tableofcontents[currentsection,hideothersubsections]
\end{block}
   \end{frame} 
}





    \setbeamertemplate{caption}[numbered]
\addtobeamertemplate{footline}{\insertframenumber/\inserttotalframenumber}


\begin{document}
\maketitle

\section{Les différents environnements}
\begin{frame}[fragile]
	\begin{block}{Équations \textit{inline} : \lstinline!$ $!}<1->
	  On bascule en mode mathématique dans un texte en ouvrant et en fermant avec le symbole \lstinline!$!%$
	\end{block}
	\begin{block}{Équations centrées numérotées : environnement \textit{equation}}<2->
	  \begin{lstlisting}
	  	\begin{equation}
	  		  % Equation ici
	  	\end{equation}
	  \end{lstlisting}
	\end{block}
	\begin{block}{Équations centrées non numérotées : environnement \textit{equation$\ast$}}<3->
	  \begin{lstlisting}
	  	\begin{equation*}
	  		  % Equation ici
	  	\end{equation*}
	  \end{lstlisting}
	\end{block}
\end{frame}


\section{Symboles mathématiques}
\begin{frame}[fragile]
	\begin{block}{Caractères}
		Les lettres latines s'écrivent normalement (\lstinline!$a,b,c...z$!), les caractères grecques s'appellent par leur noms (\lstinline!$\alpha...\Omega,\omega $!) :
		
		$$a,b,c...z$$\\
		$$\alpha...\Omega, \omega$$
	\end{block}

\begin{block}{Opérateurs}<2->
	Les packages \texttt{amsmath} et \texttt{amssymbol} fournissent des milliers d'opérateurs: \lstinline!$\pm\in\bigotimes\lll\because\oint\wedge\sum\uplus\propto$!
	$$\pm\in\bigotimes\lll\because\oint\wedge\sum\uplus\propto$$
\end{block}
\end{frame}



\section{Règles de syntaxe}
\begin{frame}[fragile]
	\begin{block}{Fonctions mathématiques}<1->
		Les fonctions mathématiques sont écrites en texte \og roman \fg : \lstinline!\mathrm{exp}!\\
		Les fonctions usuelles sont implantées dans des macros : \lstinline!\exp()\sin()\cosh()!
	\end{block}
	
	\begin{block}{Variables}<2->
		En italique, par défaut en mode math.
	\end{block}
	
	\begin{block}{Constantes, points et vecteurs}<3->
		Généralement en romain
	\end{block}
\end{frame}



\section{Mise en forme}
	\subsection{Espacements et texte}
\begin{frame}[fragile]
	\begin{block}{Espacements}
		\begin{description}
			\item[Indentation] un cadratin : \lstinline!\quad!
			\item[Double indentation] double cadratin : \lstinline!\qquad!
			\item[Espace fine] 3/18 de cadratin : \lstinline!\,!
		\end{description}
	\end{block}
	
	\begin{block}{Texte dans une équation}<2->
		\begin{description}
			\item[Sans respect de la typographie (mot seul) :] \lstinline!\textnormal{mon texte}!
			\item[Respect de la typographie (paragraphe) :] \lstinline!\text{ma phrase}!
		\end{description}
	\end{block}
\end{frame}



	\subsection{Parenthèses}
\begin{frame}[fragile]
	\begin{exampleblock}{Exemple de parenthèse laide}
		\begin{lstlisting}
			\Phi=\sum_{i=1}^n(\frac{x_i-y_i}{y_i})^2
		\end{lstlisting}
		\begin{equation}
			\Phi=\sum_{i=1}^n(\frac{x_i-y_i}{y_i})^2
		\end{equation}
		\visible<2->{\alert{Les parenthèses sont trop petites pour la fraction.}}
	\end{exampleblock}
	\begin{block}{Syntaxe correcte}<3->
		\begin{lstlisting}
			\Phi=\sum_{i=1}^n\left(\frac{x_i-y_i}{y_i}\right)^2
		\end{lstlisting}
		\begin{equation}
			\Phi=\sum_{i=1}^n\left(\frac{x_i-y_i}{y_i}\right)^2
		\end{equation}
	\end{block}
\end{frame}

	\subsection{Matrices}
\begin{frame}[fragile]
	\begin{block}{Même syntaxe que dans l'environnement \texttt{tabular}}
		\begin{lstlisting}
			\begin{bmatrix}
				  1	& 0\\
				  0	& 1
			\end{bmatrix}
			\begin{pmatrix}
			  a\\b
			\end{pmatrix}
		\end{lstlisting}
		\begin{equation}
			\begin{bmatrix}
				1	& 0\\
				0	& 1
			\end{bmatrix}
			\begin{pmatrix}
				a\\b
			\end{pmatrix}
		\end{equation}
	\end{block}
	
	\begin{alertblock}{Noms des matrices}<2->
		\texttt{\uline{b}matrix} pour ``brackets'' (crochets), \texttt{\uline{p}matrix} pour ``parenthesis'', \texttt{\uline{v}matrix} pour ``vertical lines''
	\end{alertblock}
\end{frame}

	\subsection{Systèmes d'équation}
\begin{frame}[fragile]
	\begin{exampleblock}{Objectif}
		\begin{equation}
			\left\lbrace
			\begin{aligned}
				a.&x	&= y\\
				a.b.&x+c	&= z
			\end{aligned}			 
			\right.
		\end{equation}
	\end{exampleblock}
	
	\begin{block}{Solution}<2->
	Utiliser l'environnement \texttt{aligned} pour forcer les alignements :
		\begin{lstlisting}
			\left\lbrace	% Accolade à gauche
			  \begin{aligned} % Les & seront alignées
				  a.  &x    &= y\\
				  a.b.&x+c  &= z
			  \end{aligned}			 
			\right.			% Ferme le \left
		\end{lstlisting}
	\end{block}
\end{frame}
\end{document}
