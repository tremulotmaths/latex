\documentclass[svgnames,smaller]{beamer}
\usetheme{Warsaw}
%\useoutertheme{infolines}
%\usepackage[final]{movie15}
%\usepackage{default}
\usepackage[utf8]{inputenc}
\usepackage[francais]{babel}
\usepackage{etex}
\usepackage[babel=true,kerning=true]{microtype}
\usepackage{multimedia}
\usepackage[squaren,Gray]{SIunits}
\usepackage{multirow}
\usepackage{colortbl}
%\usepackage[pdftex]{graphicx}
\usepackage{graphicx}
\usepackage{multicol}
\usepackage[version=3]{mhchem}
\usepackage{caption}
\usepackage{pdfpages}
\usepackage{pgfplots}
\usepackage{tikz}
\pgfplotsset{compat=1.3}
\definecolor{lightgray}{gray}{0.95}
\usepackage{listings}
	\lstset{backgroundcolor=\color{lightgray},language=[LaTeX]TeX,texcsstyle=\color{blue},commentstyle=\color{gray},
	literate=
	{à}{{\`a}}1
	{é}{{\'e}}1
	{è}{{\`e}}1}
\usepackage{wrapfig}
\usepackage{amsmath}
\usepackage{verbatim}
\usepackage{breakcites}
\usepackage[normalem]{ulem}
\usepackage{nicefrac}
\usepackage[sort&compress,comma]{natbib}
\usepackage{dtklogos}



\def\newblock{}
\setlength{\unitlength}{1mm}
\date{février 2012}
\author{Dorian Depriester}

\newcommand{\backupbegin}{
   \newcounter{framenumberappendix}
   \setcounter{framenumberappendix}{\value{framenumber}}
}
\newcommand{\backupend}{
   \addtocounter{framenumberappendix}{-\value{framenumber}}
   \addtocounter{framenumber}{\value{framenumberappendix}} 
}




\title{Utilisation de \BibTeX}

\AtBeginSection[]{
   \begin{frame}
\begin{block}{}
   %%% affiche en début de chaque section, les noms de sections et
   %%% noms de sous-sections de la section en cours.
   \tableofcontents[currentsection,hideothersubsections]
\end{block}
   \end{frame} 
}





    \setbeamertemplate{caption}[numbered]
\addtobeamertemplate{footline}{\insertframenumber/\inserttotalframenumber}


\begin{document}
\maketitle

\section{Qu'est-ce ?}
\frame{
	\begin{block}{Principe}
		Permet une utilisation rapide et automatisée des références bibliographiques dans un document \LaTeX
	\end{block}
	\begin{block}{Historique}<2->
		Conçu en 1985 par Oren Patashnik et Leslie Lamport.
	\end{block}
}

\section{Utilisation}
\frame{
\begin{block}{Principe}
	L'ensemble des références bibliographiques est enregistré dans un fichier annexe (\texttt{.bib}). La totalités des informations relatives à chaque entrée est renseignée (auteurs, journal, année etc.).
\end{block}
\begin{block}{Utilisation}
	Dans la source \texttt{.tex}, chaque référence est appelée par sa clé. La mise en forme des citations et des références est gérée par \LaTeX.
\end{block}
}




\begin{frame}[fragile]

\frametitle{Exemple minimal}
	\begin{block}{Dans le \texttt{.bib}}
		\begin{lstlisting}
			@book{lamport1994latex,
			  title={LaTeX$\}$:$\{$A$\}$ Document},
			  author={Lamport, Leslie},
			  volume={14},
			  year={1994},
			  publisher={pub-AW}
			}
		\end{lstlisting}
	\end{block}
	\begin{block}{Dans le \texttt{.tex}}
		\begin{lstlisting}
		Comme dit dans~\cite{lamport1994latex}, blabla...

			\bibliographystyle{plain}	% Style de biblio
			\bibliography{biblio}		% Nom du .bib
		\end{lstlisting}
	\end{block}
\end{frame}


\frame{
\frametitle{Compilation}
\begin{block}{1. \texttt{pdflatex}}
	Génération du \texttt{.aux} : liste des références nécessaires
\end{block}
\begin{block}{2. \texttt{bibtex}}<2->
	Génération du \texttt{.bbl} : enregistrement de la mise en forme des références
\end{block}
\begin{block}{3. \texttt{pdflatex}}<3->
	Modification du \texttt{.aux} : enregistrement de la mise en forme des citations,\\ inclusion des références dans le pdf
\end{block}
\begin{block}{4. \texttt{pdflatex}}<4->
	Mise à jour des citations dans le pdf
\end{block}
\begin{alertblock}{C'est toujours le \texttt{.tex} qui est compilé}<5->
	Même avec \texttt{bibtex}
\end{alertblock}
}


\section{Bien remplir les champs pour le \texttt{.bib}}
\frame{
\begin{block}{Utiliser un logiciel de gestion bibliographique}<1->
	Zotero, Jabref etc.
\end{block}



\visible<2->{
Directement depuis les sites internet :
\begin{columns}
	\begin{column}{0.4\textwidth}
		\begin{block}{Google Scholar}
			\includegraphics[width=\textwidth]{GoogleCitation.png}\\
			\includegraphics[width=0.8\textwidth]{GoogleBibTeX.png}
		\end{block}
	\end{column}
	\begin{column}{0.4\textwidth}
		\begin{block}{ScienceDirect}
			\includegraphics[width=\textwidth]{ScienceDirectCite.png}\\
			\includegraphics[width=0.8\textwidth]{ScienceDirectBibtex.png}
		\end{block}
	\end{column}
\end{columns}}
}
\end{document}
