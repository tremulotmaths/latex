\documentclass[xcolor={dvipsnames,svgnames,table}]{beamer}

\input philippe2013_beamer

\title[\'Equations et inéquations]{{\small Chapitre 1}\\[0.5cm] \textbf{\'Equations et inéquations}\\ \textbf{Résolution de problèmes}}
\date{}
\author{}

\begin{document}

\begin{frame}
\titlepage
\end{frame}


\section{Résolution d'équations}
\subsection{Développement et factorisation}

\begin{frame}
    \begin{definition}
        \alert{Développer} un produit revient à l'écrire sous forme d'une somme.
    \end{definition}

\pause

    \begin{Prop}[démontrée géométriquement]
        Pour tous nombres $k$, $a$, $b$, $c$ et $d$, on a :
        \invisible{\[k(a + b) = ka + kb \qetq (a + b)(c + d) = ac + ad + bc+ bd.\]}
    \end{Prop}
\end{frame}

\begin{frame}
    \begin{Example}
        Développer l'expression $A(x) = 5(x - 1) - (3x - 2)(-4 + x)$.\par
        \rule{0pt}{6cm}
    \end{Example}
\end{frame}

\begin{frame}
    \begin{definition}
        \alert{Factoriser} une somme revient à l'écrire sous forme d'un produit.
    \end{definition}
\pause
    \begin{Prop}[démontrée géométriquement]
        Pour tous nombres $k$, $a$, $b$, $c$ et $d$, on a :
        \invisible{\[ka + kb = k(a + b) \qetq a(c + d) + b(c + d) = (a + b)(c + d).\]}
    \end{Prop}
\end{frame}

\begin{frame}
    \begin{Example}
        Factoriser l'expression $B(x) = 2(x+4)+2(x-1) - (2x + 3)(4x + 6)$.\par\smallskip
        \rule{0pt}{5.5cm}
    \end{Example}
\end{frame}

\begin{frame}
    \begin{Prop}[Les identités remarquables]
        Soient $a$ et $b$ deux nombres quelconques. On a alors les \alert{identités remarquables} suivantes :
        
        \rule{0pt}{5cm}
    \end{Prop}
\end{frame}

\begin{frame}%{Démonstration}
    \begin{Proof}
        \rule{0pt}{7cm}
    \end{Proof}
\end{frame}

\begin{frame}
    \begin{Examples}
        \[C = \np{1001}^2 \qq D = 99^2 \qq E = 49 \times 51\]
        \rule{0pt}{6cm}
    \end{Examples}
\end{frame}

\begin{frame}
    \begin{Examples}
        \[F(x) = x^2 + 2x + 1 \qq G(a) = 9a^2 - 24a + 16 \qq H(y) = y^2 - 25\]
        \rule{0pt}{5.5cm}
    \end{Examples}
\end{frame}

\begin{frame}{Remarque}
    Les identités remarquables sont utiles pour gagner du temps dans un développement et, en cas d'oubli, on peut les retrouver en quelques lignes.\par
    En revanche, elles sont très pratiques pour factoriser dans certains cas où il n'y a pas de facteurs communs.
\end{frame}

\begin{frame}
    \begin{Example}
        Factoriser $I(t) = 4t^2 - 20t + 9$.\par
        \rule{0pt}{6cm}
    \end{Example}
\end{frame}

\subsection{Les équations}

\begin{frame}
    \begin{definition}
        Une \alert{équation} est une égalité où figure une inconnue.\par
        \alert{Résoudre} une équation revient à trouver la (ou les) valeur(s) de l'inconnue pour laquelle (ou lesquelles) l'égalité est vérifiée.
    \end{definition}
\pause
    \begin{Prop}
    On peut ajouter un même nombre à chaque membre d'une égalité pour obtenir ainsi une égalité équivalente :
        \[a = b \qLRq a \rouge{\ +\ c } = b \rouge{\ +\ c}\]
    \end{Prop}
\end{frame}

\begin{frame}
    \begin{Proof}
        \rule{0pt}{5cm}
    \end{Proof}
\end{frame}

\begin{frame}
    \begin{Prop}
        On peut multiplier par un même nombre non nul les deux membres d'une égalité pour obtenir ainsi une égalité équivalente :
        \[\text{Avec } c\neq 0,\quad a = b \qLRq a \rouge{\,\times\,c} = b \rouge{\,\times\,c}\]
    \end{Prop}
\pause
    \begin{Proof}
        \rule{0pt}{4cm}
    \end{Proof}
\end{frame}

\subsubsection{\'Equation du premier degré}

\begin{frame}
    \begin{definition}
        Une \alert{équation à une inconnue du premier degré} est une équation de la forme $ax + b = 0$ où $x$ est l'inconnue et $a$ et $b$ sont des paramètres donnés tels que $a \neq 0$.\par
    \end{definition}
\end{frame}

\begin{frame}
    \begin{Example}
        Résolution de l'équation : $8x - 3 = -2x + 6$.\par
        \rule{0pt}{6cm}
    \end{Example}
\end{frame}

\begin{frame}
    \begin{Examples}
        Résoudre les équations suivantes :
        \[3x - 4 = 3 \qq 2x + 2 = 5x - 4 \qq 1 - 2(2 - x) = 2x - 3\]
        \rule{0pt}{4cm}
    \end{Examples}
\end{frame}

\subsubsection{\'Equation produit}

\begin{frame}
    \begin{Prop}[admise]
        Un produit de facteur est nul si et seulement si l'un des facteurs est nul :
        \[A(x) \times B(x) = 0 \qLRq A(x) = 0 \text{ ou } B(x) = 0\]
    \end{Prop}
\end{frame}

\begin{frame}
    \begin{Example}
       Résolution de l'équation $(2x + 3)(6x - 8) = 0$.\par
       \rule{0pt}{5cm}
    \end{Example}
\end{frame}

\subsubsection{Résolution d'un problème}

\begin{frame}
    \begin{Example}
        Lors d'une séance de cinéma, on a accueilli $56$ spectateurs. Certains ont payé le tarif réduit ($5$\EUR), les autres le tarif normal ($8$\EUR). La recette de cette séance se monte à $376$\EUR.\par Combien de spectateurs ont payé le tarif réduit ?
        
        \rule{0pt}{4cm}
    \end{Example}
\end{frame}

\begin{frame}{En résumé}
    Voici les étapes de la résolution d'un problème en utilisant les équations :\pause
    \begin{enumerate}
        \item choix de l'inconnue ;\pause
        \item mise en équation du problème ;\pause
        \item résolution de l'équation ;\pause
        \item réponse au problème.
    \end{enumerate}
\end{frame}

\section{Les ensembles de nombres en résumé}

\begin{frame}
    \begin{definition}\pause
        \begin{enumerate}
            \item L'ensemble des entiers \alert{naturels} $\N$ est constitué des nombres entiers positifs.\pause
            \item L'ensemble des entiers \alert{relatifs} $\Z$ est constitué des entiers naturels ainsi que de nombres entiers négatifs.\pause
            \item L'ensemble des nombres \alert{rationnels} $\Q$ est constitué de tous les nombres qui peuvent s'écrire sous forme de fractions, ce qui inclut les ensembles $\N$ et $\Z$ mais aussi les nombres décimaux.\pause
            \item L'ensemble des nombres \alert{irrationnels} est constitué des nombres qui ne peuvent pas s'écrire sous forme de fractions.\pause
            \item L'ensemble des nombres \alert{réels} $\R$ est constitué de tous les nombres rationnels et des nombres irrationnels.
        \end{enumerate}
    \end{definition}
\end{frame}

\begin{frame}{Les ensembles de nombres}
    \[\N \subset \Z \subset \Q \subset R\]
    \rule{0pt}{6cm}
\end{frame}

\section{Résolution d'inéquations}
\subsection{Intervalles de nombres}
\subsubsection{Intervalles bornés}

\begin{frame}
    \begin{definition}
        Un \alert{intervalle borné} par deux nombres réels est constitué de tous les nombres réels compris entre ces deux nombres.\par
        Par exemple, l'intervalle $\intervalleff  a b$ est l'ensemble des nombres $x$ tels que $x \geq a$ et $x \leq b$.
    \end{definition}
\pause
    \begin{center}
    \renewcommand\arraystretch{1.7}
    \begin{tabular}{|>\bfseries c|>{\bfseries} c|>{\centering\bfseries\arraybackslash} p{6.25cm}|}
        \hline
            Inégalité & Notation & Représentation \\
        \hline
            $a \leq x \leq b$ & \invisible{$x\in \intervalleff a b$} &\\
    \hline
            $a < x \leq b$ & \invisible{$x\in \intervalleof a b$} &\\
    \hline
            $a \leq x < b$ & \invisible{$x\in \intervallefo a b$} &\\
    \hline
            $a < x < b$ & \invisible{$x\in \intervalleoo a b$} &\\
    \hline
    \end{tabular}
    \end{center}
\end{frame}

\subsubsection{Intervalles non bornés}

\begin{frame}
    \begin{definition}
        Un \alert{intervalle non borné} est constitué de tous les nombres réels supérieurs ou inférieurs à un nombre réel.\par
        Par exemple, l'intervalle $\intervalleoo{a}{+\infty}$ est l'ensemble des nombres $x$ tels que $x > a$.
    \end{definition}
\pause
    \begin{center}
    \renewcommand\arraystretch{1.7}
    \begin{tabular}{|>\bfseries c|>{\bfseries} c|>{\centering\bfseries\arraybackslash} p{5cm}|}
        \hline
            Inégalité & Notation & Représentation \\
        \hline
            $x \geq a $ & \invisible{$x\in \intervallefo{a}{+\infty}$} &\\
    \hline
            $x > a $ & \invisible{$x\in \intervalleoo{a}{+\infty}$} &\\
    \hline
            $x \leq a $ & \invisible{$x\in \intervalleof{-\infty}{a}$} &\\
    \hline
            $x < a $ & \invisible{$x\in \intervalleoo{-\infty}{a}$} &\\
    \hline
    \end{tabular}
    \end{center}
\end{frame}

\subsubsection{Réunion et intersection}

\begin{frame}
    \begin{definition}
        \begin{enumerate}[<+->]
            \item La \alert{réunion} de deux intervalles $I$ et $J$, notée $I \cup J$, est l'ensemble des nombres réels appartenant à $I$ \textbf{ou} à $J$.
            \item L'\alert{intersection} de deux intervalles $I$ et $J$, notée $I \cap J$, est l'ensemble des nombres réels appartenant à $I$ \textbf{et} à $J$.
        \end{enumerate}
    \end{definition}
\end{frame}

\begin{frame}
    \begin{Examples}
    $\intervalleff{-2}{5} \cup \intervalleoo{0}{+\infty} = $ \hspace{2cm} $\intervalleff{-2}{5} \cap \intervalleoo{0}{+\infty} =$ \vspace{2cm}

    $\intervalleff{-2}{5} \cup \intervalleof{8}{15} = $ \hspace{3cm} $\intervalleff{-2}{5} \cap \intervalleof{8}{15} = $\vspace{2cm}
    \end{Examples}
\end{frame}

\subsection{Résolution d'inéquations}
\subsubsection{Inéquations du premier degré à une inconnue}

\begin{frame}
    \begin{Prop}
        On peut ajouter un même nombre à chaque membre d'une inégalité pour obtenir ainsi une inégalité équivalente :
        \[a < b \qLRq a \rouge{\ +\ c } < b \rouge{\ +\ c}\]
    \end{Prop}
\pause
    \begin{Examples}
        \[x + 4 \leq 5 \hspace{2.5cm} x - 3 > 0\]
        \rule{0pt}{3cm}
    \end{Examples}
\end{frame}

\begin{frame}
    \begin{Proof}
        \rule{0pt}{6cm}
    \end{Proof}
\end{frame}

\begin{frame}
    \begin{Prop}
        On multiplie ou on divise les deux membres d'une inégalité par un même nombre $k$ non nul :
        \begin{itemize}
            \item si $k > 0$, alors : \quad $a < b \qLRq ka < kb$ ;
            \item si $k < 0$, alors : \quad $a < b \qLRq ka > kb$.
        \end{itemize}
    \end{Prop}
\pause
    \begin{Examples}
        \[3x + 4 < 2 \hspace{2.5cm} \frac{x}{-2} + 6 \geq 0\]
        \rule{0pt}{2.5cm}
    \end{Examples}
\end{frame}

\begin{frame}
    \begin{Proof}
        \rule{0pt}{6cm}
    \end{Proof}
\end{frame}

\subsubsection{Résolution de problèmes}

\begin{frame}
    \begin{Example}
        Dans un club de gym, deux formules sont proposées :
        \begin{description}
            \item[Formule A :] abonnement mensuel de $18$\EUR et $5$\EUR la séance.
            \item[Formule B :] abonnement mensuel de $30$\EUR et $3$\EUR la séance.
        \end{description}
        Déterminer par le calcul le nombre de séances minimum pour lequel la formule B est plus avantageuse.
        
    \rule{0pt}{4cm}
    \end{Example}
\end{frame}

\begin{frame}{En résumé}
    Voici les étapes de la résolution d'un problème en utilisant les inéquations :\pause
    \begin{enumerate}
        \item choix de l'inconnue ;\pause
        \item trouver l'inéquation correspondant au problème ;\pause
        \item résolution de l'inéquation ;\pause
        \item réponse au problème.
    \end{enumerate}
\end{frame}

\end{document}
