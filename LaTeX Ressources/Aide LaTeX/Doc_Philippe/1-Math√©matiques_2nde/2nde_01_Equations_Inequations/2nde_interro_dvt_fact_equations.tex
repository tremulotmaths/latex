\documentclass[10pt,french]{book}

\input philippe2013
\usepackage{marvosym,slashbox}
\RegleEntete


\newcommand\competences{
\setcounter{exo}{0}
\begin{tabular}{ll} Nom : \\[5pt] Prénom : \end{tabular}
\hfill
\textbf{Note :}\renewcommand\arraystretch{2.3}
\begin{tabular}{|c|}
\hline
\slashbox{\Huge\bfseries\phantom{10}}{\Huge\bfseries 10}\\
\hline
\end{tabular}\renewcommand\arraystretch{1}\medskip
}

\entete{\seconde 7}{Calcul littéral}{A}
\pieddepage{}{}{}


\begin{document}
%--------------------------------------------------------------------------------------------------------------------------------------------------------------------------
%                           SUJET A
%--------------------------------------------------------------------------------------------------------------------------------------------------------------------------
\competences

{\bfseries \bsc{Attention aux signes !!!}}\medskip

On rappelle les identités remarquables :
\begin{center}
    \begin{tabular}{rcl}
        \underbar{Forme factorisée} && \underbar{Forme développée} \\[7.5pt]
        $(a + b)^2$ & $=$ & $a^2 + 2ab + b^2$ \\
       $(a - b)^2$ & $=$ & $a^2 - 2ab + b^2$ \\
        $(a + b)(a - b)$ & $=$ & $a^2 - b^2$
    \end{tabular}
\end{center}\medskip

\exo 
\begin{enumerate}
    \item Développer les expressions suivantes en utilisant une identité remarquable si nécessaire :
        \[A(x) = 3(2x - 4) \qq B(x) = (6x + 2)(-2x - 3) \qq C(x) = (5 - 2x)^2\]
    \item Factoriser les expressions suivantes en utilisant une identité remarquable si nécessaire :
        \[D(x) = 4x + 8 \qq E(x) = (3x + 1)(-2x - 1) + (3x + 1)(x + 3) \qq F(x) = (x - 4)^2 - (2x + 3)^2\]
\end{enumerate}\medskip

\exo Résoudre les deux équations suivantes :
    \[4x - 3 = 7 \qq (x + 2)(3x - 1) = 3x^2 + 4x \quad \text{(penser à développer)}\]
\clearpage

%--------------------------------------------------------------------------------------------------------------------------------------------------------------------------
%                           SUJET B
%--------------------------------------------------------------------------------------------------------------------------------------------------------------------------

\entete{\seconde 7}{Calcul littéral}{B}
\competences

{\bfseries \bsc{Attention aux signes !!!}}\medskip

On rappelle les identités remarquables :
\begin{center}
    \begin{tabular}{rcl}
        \underbar{Forme factorisée} && \underbar{Forme développée} \\[7.5pt]
        $(a + b)^2$ & $=$ & $a^2 + 2ab + b^2$ \\
       $(a - b)^2$ & $=$ & $a^2 - 2ab + b^2$ \\
        $(a + b)(a - b)$ & $=$ & $a^2 - b^2$
    \end{tabular}
\end{center}\medskip

\exo
\begin{enumerate}
    \item Développer les expressions suivantes en utilisant une identité remarquable si nécessaire :
        \[A(x) = 4(3x - 2) \qq B(x) = (2x + 6)(-4x - 1) \qq C(x) = (4 - 3x)^2\]
    \item Factoriser les expressions suivantes en utilisant une identité remarquable si nécessaire :
        \[D(x) = 6x + 12 \qq E(x) = (3x - 1)(2x - 4) + (3x - 1)(x + 3) \qq F(x) = (2x - 4)^2 - (x + 3)^2\]
\end{enumerate}\medskip

\exo Résoudre les deux équations suivantes :
    \[5x - 7 = 9 \qq (2x + 1)(3x - 1) = 6x^2 + 4x \quad \text{(penser à développer)}\]

\end{document} 