\documentclass[10pt,openright,twoside,french]{book}

\input philippe2013
\input philippe2013_cours
\input philippe2013_sections
\input philippe2013_chapitre


\begin{document}
%\small
\pagestyle{empty}

\subsubsection{Inéquations du premier degré à une inconnue}

\begin{Prop}
    On peut ajouter un même nombre à chaque membre d'une inégalité pour obtenir ainsi une inégalité équivalente :
    \[a < b \qLRq a \rouge{\ +\ c } < b \rouge{\ +\ c}\]
\end{Prop}

\begin{Exemple}[s]
$x + 4 \leq 5 \hspace{1cm} x - 3 > 0$.
\end{Exemple}

\begin{Demo}
    $a$, $b$ et $c$ sont trois nombres tels que $a < b$. Donc :
    \[a - b < 0 \Leftrightarrow a - b + \underbrace{c - c}_{= 0} < 0 \Leftrightarrow a + c - (b + c) < 0 \Leftrightarrow \pfr{a + c < b + c}\]
\end{Demo}

\begin{Prop}
    On multiplie ou on divise les deux membres d'une inégalité par un même nombre $k$ non nul :
    \begin{itemize}
        \item si $k > 0$, alors : \quad $a < b \qLRq ka < kb$ ;
        \item si $k < 0$, alors : \quad $a < b \qLRq ka > kb$.
    \end{itemize}
\end{Prop}

\begin{Exemple}[s]
$3x + 4 < 2 \hspace{1cm} \dfrac{x}{-2} +6\geq 0$
\end{Exemple}

\begin{Demo}
    On considère trois nombres $a$, $b$ et $k$ tels que $a > b$.\par
    $a - b$ est donc un nombre positif.\par
    On rappelle que le produit de deux nombres de même signe est positif, négatif sinon.
    \[\begin{array}{c@{\hspace*{3cm}}c}
        \underbar{\text{Si } k > 0} & \underbar{\text{Si } k < 0} \\[5pt]
        k(a - b) > 0 & k(a - b) < 0 \\
        \Leftrightarrow ka - kb > 0 & \Leftrightarrow ka - kb < 0 \\
        \Leftrightarrow \pfr{ka > kb} & \Leftrightarrow \pfr{ka < kb}
    \end{array}\]
\end{Demo}\medskip

\subsubsection{Résolution de problèmes}

\begin{Exemple}
    Dans un club de gym, deux formules sont proposées :
    \begin{description}
        \item[Formule A :] abonnement mensuel de \EUR{$18$} et \EUR{$5$} la séance.
        \item[Formule B :] abonnement mensuel de \EUR{$30$} et \EUR{$3$} la séance.
    \end{description}
    Déterminer par le calcul le nombre de séances minimum pour lequel la formule B est plus avantageuse.
\end{Exemple}\medskip

Voici les étapes de la résolution d'un problème en utilisant les inéquations :
\begin{enumerate}
    \item choix de l'inconnue ;
    \item trouver l'inéquation correspondant au problème ;
    \item résolution de l'inéquation ;
    \item réponse au problème.
\end{enumerate}\medskip

\end{document} 