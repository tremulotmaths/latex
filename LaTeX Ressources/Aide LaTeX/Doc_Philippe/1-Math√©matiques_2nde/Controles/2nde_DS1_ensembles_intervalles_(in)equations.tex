\documentclass[11pt,french]{book}
\input preambule_2013

\newcounter{exoc}
\newenvironment{exoc}[1]{%
  \refstepcounter{exoc}\textbf{Exercice \theexoc} :\hfill {\textbf{(#1)}}\par
  \medskip}%
{\medskip}

\pagestyle{empty}

\begin{document}

\begin{center}
\begin{tabularx}{\textwidth}{|>\centering m{2.5cm}|>\centering X|>{\centering\arraybackslash} m{2.5cm}|}
	\hline
		\seconde 7 &  Mardi 8 octobre \np{2013} & \textbf{\'Equations Inéquations} \\
	\hline
		\multicolumn{3}{|c|}{\bsc{Contrôle de mathématiques}} \\
	\hline
        \multicolumn{1}{|r}{\bsc{Nom}:} & \multicolumn{2}{l|}{} \\
		\multicolumn{1}{|r}{Prénom:} & \multicolumn{2}{l|}{} \\
	\hline
        \multicolumn{3}{|l|}{\bfseries Note et observations :} \\[1cm]
    \hline
\end{tabularx}\bigskip

{\itshape
La qualité et la précision de la rédaction seront prises en compte dans l'appréciation des copies.\par
Le barème est indicatif.}
\end{center}

\begin{exoc}{4 points}
\begin{enumerate}
    \item Compléter en utilisant les symboles $\in$ et $\notin$.
    \begin{multicols}{4}
        \begin{enumerate}
            \item $\sqrt 3 \ldots \Q$
            \item $-2 \ldots \N$
            \item $\frac 2 3 \ldots \Q$
            \item $3 \ldots \R$
        \end{enumerate}
    \end{multicols}
    \item Compléter les pointillés avec V ou F pour indiquer si les affirmations sont vraie (V) ou fausse (F).
    \begin{multicols}{2}
        \begin{enumerate}
            \item $1 \in \intervalleoo{1}{+\infty}$\quad \ldots
            \item $1 \in \intervallefo{-2}{4}$ \quad \ldots
            \item $x \geqslant 10 \Leftrightarrow  x \in \intervallefo{10}{+\infty}$ \quad \ldots
            \item $3 < x < 9 \Leftrightarrow x \in \intervalleff 1 9$ \quad \ldots
        \end{enumerate}
    \end{multicols}
\end{enumerate}
\end{exoc}

\begin{exoc}{6 points}
    \begin{enumerate}
        \item Résoudre les équations suivantes.
        \item Pour chaque solution, donner les ensembles de nombres auxquelles elles appartiennent.
    \end{enumerate}
    \[2x - 4 = 8 \qq  (x + 2)(3x - 1) = 0 \qq 3x - 2(1 - 2x) = 6x - (7x - 2)\]
\end{exoc}

\begin{exoc}{3 points}
    \begin{enumerate}
        \item Résoudre les inéquations suivantes.
        \item Donner les solutions sous forme d'un intervalle.
        \item Représenter l'intervalle sur une droite graduée.
    \end{enumerate}
    \[-2x + 6 \geqslant 8 \qq 8x - 48 < 6 - 3x\]
\end{exoc}

\begin{exoc}{7 points}
Un opérateur de téléphone portable propose trois formules. L'unité de durée des communications est la minute.
    \begin{description}
        \item[Formule \og~libre~\fg :] pas d'abonnement : \EUR{$0,50$} par minute.
        \item[Formule \og~éco~\fg :] forfait 2 heures : \EUR{$15$} par mois, puis chaque minute supplémentaire est facturée \EUR{$0,30$}.
        \item[Formule \og~pro~\fg :] tout illimité : \EUR{$54$} par mois.
    \end{description}
    
    \begin{enumerate}
        \item Calculer le prix de chaque formule pour trois heures de communication.
        \item Combien de temps doit durer la communication pour le prix de la formule \og~libre~\fg coûte le même prix que la formule \og~pro~\fg ?
        \item Un artisan hésite entre la formule \og~éco~\fg et la formule \og~pro~\fg.\par
        À partir de combien de minutes de communication la formule \og~pro~\fg est la plus intéressante ?
    \end{enumerate}
\end{exoc}
\end{document} 