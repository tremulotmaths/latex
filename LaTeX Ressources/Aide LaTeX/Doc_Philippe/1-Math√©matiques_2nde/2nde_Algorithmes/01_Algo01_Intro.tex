\documentclass[10pt,openright,twoside,french]{book}

\usepackage{marvosym}
\input philippe2013
\input philippe2013_activites

\pagestyle{empty}

\begin{document}

\TitreAlgo{i.1}{Comprendre un algorithme}

\exo

On donne ci-dessous un algorithme écrit en langage courant :

\begin{center}
\small
    \psframebox{
    \parbox{0.4\linewidth}{
        \textbf{Variables}

            \quad $n$ : un nombre réel

            \quad $q$ : un nombre réel

        \textbf{Entrée}

            \quad Saisir $n$

        \textbf{Traitement}

            \quad Affecter à $q$ la valeur $(n+2) \times (n+2)$

            \quad Affecter à $q$ la valeur $q - (n+4)$

            \quad Affecter à $q$ la valeur $q/(n+3)$

        \textbf{Sortie}

            \quad Afficher $q$
    }}
\end{center}

\begin{enumerate}
    \item Tester cet algorithme pour $n = 4$ puis pour $n = -7$.
    \item Un élève a saisi $n = -3$. Que se passe-t-il ? Pourquoi ?
    \item \'Emettre une conjecture sur le résultat fourni par cet algorithme.
    \item Démontrer cette conjecture.
\end{enumerate}\[*\]

\exo

\begin{enumerate}
    \item Dans l'algorithme suivant, montrer que l'on pourrait exprimer $y$ directement en fonction de $x$ avec une instruction ne comportant qu'une seule opération.\par
    \'Ecrire alors cette instruction sous la forme d'une expression mathématique $A(x)$ dépendant de $x$ et le plus simplement possible.

\begin{center}
\small
    \psframebox{
    \parbox{0.4\linewidth}{
        \textbf{Variables}

            \quad $a$ : un nombre réel

            \quad $b$ : un nombre réel

            \quad $c$ : un nombre réel

            \quad $d$ : un nombre réel

            \quad $x$ : un nombre réel

            \quad $y$ : un nombre réel

        \textbf{Entrée}

            \quad Saisir $x$

        \textbf{Traitement}

            \quad Affecter à $a$ la valeur $x+2$

            \quad Affecter à $b$ la valeur $a \times a$

            \quad Affecter à $c$ la valeur $x-2$

            \quad Affecter à $d$ la valeur $c \times c$

            \quad Affecter à $y$ la valeur $(b - d)/4$

        \textbf{Sortie}

            \quad Afficher $y$
    }}
\end{center}

    \item Modifier alors la partie \textbf{Traitement} de l'algorithme.
    \item Quelles sont les seules variables nécessaires ?
\end{enumerate}


\end{document} 