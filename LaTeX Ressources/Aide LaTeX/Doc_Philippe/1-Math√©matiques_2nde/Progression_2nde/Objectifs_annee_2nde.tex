\documentclass[10pt,openright,twoside]{book}
\usepackage{etex}

\input philippe2011_complet
\usepackage{lscape}


\entete{}{}{\footnotesize\itshape Programme de \seconde}
\pieddepage{}{\footnotesize - \thepage/\pageref{LastPage} -}{}
\renewcommand\headrulewidth{0pt}

\newcounter{chapp}
\newcommand\chapitre[1]{%
\refstepcounter{chapp}%
\begin{center}
\bfseries
\underbar{\bsc{Chapitre \thechapp}}\par\large\textit{#1}
\end{center}\nopagebreak[4]
}


\newcounter{fct}
\newcommand\FCT{%
\refstepcounter{fct}%
\item[\pfr{Fct.\thefct}]
}

\newcounter{geo}
\newcommand\GEO{%
\refstepcounter{geo}%
\item[\pfr{Géo.\thegeo}]
}

\newcounter{stp}
\newcommand\STP{%
\refstepcounter{stp}%
\item[\pfr{StP.\thestp}]
}

\newcommand\EnPlus[1]{\footnotesize\textit{\underbar{Permet de travailler en parallèle :} #1}\par}

\begin{document}

\begin{center}
{\fontfamily{augie}\fontsize{9}{11}\selectfont
{\Large \pfr{\begin{tabular}{cc}
                            Cours de Mathématiques\\
                            Classe de Seconde Générale et Technologique
                        \end{tabular}}}}
\end{center}\bigskip


\noindent\textbf{Légende des trois parties du programme :}
    \begin{enumerate}
        \item[\quad $\checkmark$] \bsc{Fct.} Fonctions ;
        \item[\quad $\checkmark$] \bsc{Geo.} Géométrie ;
        \item[\quad $\checkmark$] \bsc{StP.} Statistiques et Probabilités.
    \end{enumerate}\[*\]

\chapitre{Résolution de problèmes à l'aide d'équations et d'inéquations}
\EnPlus{les ensembles de nombres, les différents types d'intervalles, le développement et la factorisation.}
    \begin{enumerate}
        \FCT Mettre un problème en équation ;
        \FCT Résoudre une équation se ramenant au premier degré ;
        \FCT Modéliser un problème par une inéquation ;
        \FCT Résoudre algébriquement les inéquations nécessaires à la résolution d'un problème.
    \end{enumerate}\[*\]

\chapitre{Coordonnées d'un point dans le plan}
\EnPlus{configurations du plan, algorithmes possibles.}
    \begin{enumerate}
        \GEO Repérer un point donné du plan, placer un point connaissant ses coordonnées ;
        \GEO Calculer la distance de deux points connaissant leurs coordonnées ;
        \GEO Calculer les coordonnées du milieu d'un segment ;
        \GEO Pour résoudre des problèmes :
            \begin{itemize}
                \item utiliser les propriétés des triangles, des quadrilatères, des cercles ;
                \item utiliser les propriétés des symétries axiale ou centrale.
            \end{itemize}
    \end{enumerate}\[*\]

\chapitre{Généralités sur les fonctions}
\EnPlus{les intervalles, résolutions graphiques d'équations, algorithme.}
    \begin{enumerate}
        \FCT Traduire le lien entre deux quantités par une formule ;
        \FCT Pour une fonction définie par une courbe, un tableau de données ou une formule :
            \begin{itemize}
                \item identifier la variable et, éventuellement, l'ensemble de définition ;
                \item déterminer l'image d'un nombre ;
                \item rechercher des antécédents d'un nombre.
            \end{itemize}
    \end{enumerate}\[*\]

\chapitre{\'Equations de droites}
\EnPlus{les configurations du plan.}
    \begin{enumerate}
        \GEO Tracer une droite dans le plan repéré ;
        \GEO Interpréter graphiquement le coefficient directeur d'une droite ;
        \GEO Caractériser analytiquement une droite ;
        \GEO Établir que trois points sont alignés, non alignés ;
        \GEO Reconnaître que deux droites sont parallèles, sécantes ;
        \GEO Déterminer les coordonnées du point d'intersection de deux droites sécantes.
    \end{enumerate}\[*\]

\chapitre{Variations de fonctions}
    \begin{enumerate}
        \FCT Décrire, avec un vocabulaire adapté ou un tableau de variations, le comportement d'une fonction définie par une courbe ;
        \FCT Dessiner une représentation graphique compatible avec un tableau de variations ;
        \FCT Lorsque le sens de variation est donné, par une phrase ou un tableau de variations :
            \begin{itemize}
                \item comparer les images de deux nombres d'un intervalle ;
                \item déterminer tous les nombres dont l'image est supérieure (ou inférieure) à une image donnée.
            \end{itemize}
        \FCT Résoudre graphiquement des inéquations de la forme $f(x) < k$ ; $f(x) < g(x)$ ;
        \FCT $\blacktriangle$ Encadrer une racine d'une équation grâce à un algorithme de dichotomie.
    \end{enumerate}\[*\]

\chapitre{Fonctions de référence \bsc i : fonctions affines}
\EnPlus{intervalles de nombres, résolution graphique d'équations.}
    \begin{enumerate}
        \FCT Donner le sens de variation d'une fonction affine ;
        \FCT Donner le tableau de signes de $ax + b$ pour des valeurs numériques données de $a$ et $b$ ;
        \FCT Résoudre graphiquement des inéquations de la forme $f(x) < k$ ; $f(x) < g(x)$ (cf chap.5 également).
    \end{enumerate}\[*\]

\chapitre{Statistiques descriptive, analyse de données}
\EnPlus{médiane, quartile, moyenne.}
    \begin{enumerate}
        \STP Utiliser un logiciel (par exemple un tableur) ou une calculatrice pour étudier une série statistique ;
        \STP Passer des effectifs aux fréquences, calculer les caractéristiques d'une série définie par effectifs ou fréquences ;
        \STP Calculer des effectifs cumulés, des fréquences cumulées ;
        \STP Représenter une série statistique graphiquement (nuage de points, histogramme, courbe des fréquences cumulées).
    \end{enumerate}\[*\]

\chapitre{\'Etudes de signes}
\EnPlus{tableaux de signes, intervalles, factorisation.}
    \begin{enumerate}
        \FCT Modéliser un problème par une inéquation (cf chap.1 également) ;
        \FCT Résoudre une inéquation à partir de l'étude du signe d'une expression produit ou quotient de facteurs du premier degré ;
        \FCT Résoudre algébriquement les inéquations nécessaires à la résolution d'un problème (cf chap.1 également).
    \end{enumerate}\[*\]

\chapitre{Fonctions de référence \bsc{ii} : fonction carré, fonction inverse}
\EnPlus{ensembles de nombres, résolutions graphiques d'équations et d'inéquations.}
    \begin{enumerate}
        \FCT Connaître les variations des fonctions carré et inverse ;
        \FCT Représenter graphiquement les fonctions carré et inverse.
    \end{enumerate}\[*\]\clearpage

\chapitre{Probabilités}
\EnPlus{statistiques, fréquence.}
    \begin{enumerate}
        \STP Déterminer la probabilité d'événements dans des situations d'équiprobabilité ;
        \STP Utiliser des modèles définis à partir de fréquences observées ;
        \STP Connaître et exploiter la formule $p(A \cup B) + p(A \cap B) = p(A) + p(B)$.
    \end{enumerate}\[*\]

\chapitre{Géométrie dans l'espace}
\EnPlus{configurations du plan.}
    \begin{enumerate}
        \GEO Manipuler, construire, représenter en perspective des solides ;
        \GEO Pour résoudre des problèmes (cf chap.2 également) :
            \begin{itemize}
                \item utiliser les propriétés des triangles, des quadrilatères, des cercles ;
                \item utiliser les propriétés des symétries axiale ou centrale.
            \end{itemize}
    \end{enumerate}\[*\]

\chapitre{Fonctions polynômes de degré 2, fonctions homographiques}
\EnPlus{développement, factorisation, algorithmes.}
    \begin{enumerate}
        \FCT Associer à un problème une expression algébrique ;
        \FCT Identifier la forme la plus adéquate (factorisée, développée) d'une expression en vue de la résolution du problème donné ;
        \FCT Développer, factoriser des expressions polynomiales simples ; transformer des expressions rationnelles simples ;
        \FCT Connaître les variations des fonctions polynômes de degré 2 (monotonie, extremum) et la propriété de symétrie de leurs courbes ;
        \FCT Identifier l'ensemble de définition d'une fonction homographique.
    \end{enumerate}\[*\]

\chapitre{\'Echantillonage}
    \begin{enumerate}
        \STP Concevoir, mettre en œuvre et exploiter des simulations de situations concrètes à l'aide du tableur ou d'une calculatrice ;
        \STP Exploiter et faire une analyse critique d'un résultat d'échantillonnage.
    \end{enumerate}\[*\]

\chapitre{Vecteurs}
\EnPlus{configurations du plan, algorithme.}
    \begin{enumerate}
        \GEO Savoir que $\vect{AC} = \vect{CD}$ équivaut à $ABDC$ est un parallélogramme, éventuellement aplati ;
        \GEO Connaître les coordonnées $(x_B - x_A \pv y_B - y_A)$ du vecteur $\vect{AB}$ ;
        \GEO Calculer les coordonnées de la somme de deux vecteurs dans un repère ;
        \GEO Utiliser la notation $\lambda \vect u$ ;
        \GEO Établir la colinéarité de deux vecteurs ;
        \GEO Construire géométriquement la somme de deux vecteurs ;
        \GEO Caractériser alignement et parallélisme par la colinéarité de vecteurs ;
        \GEO Pour résoudre des problèmes (cf chap.2 également) :
            \begin{itemize}
                \item utiliser les propriétés des triangles, des quadrilatères, des cercles ;
                \item utiliser les propriétés des symétries axiale ou centrale.
            \end{itemize}
    \end{enumerate}\[*\]

\chapitre{Trigonométrie}
\EnPlus{étude graphique de fonctions.}
    \begin{enumerate}
        \FCT On fait le lien avec les valeurs des sinus et cosinus des angles de $0\degres$, $30\degres$, $45\degres$, $60\degres$, $90\degres$.
    \end{enumerate}\[*\]

\begin{center}
{\fontfamily{augie}\fontsize{9}{11}\selectfont
{\Large \pfr{Algorithmique (objectifs pour le lycée)}}}
\end{center}\bigskip


\noindent\textbf{Instructions élémentaires (affectation, calcul, entrée, sortie)}\par
Les élèves, dans le cadre d'une résolution de problèmes, doivent être capables :
\begin{itemize}
    \item d'écrire une formule permettant un calcul ;
    \item d'écrire un programme calculant et donnant la valeur d'une fonction, ainsi que les instructions d'entrées et sorties nécessaires au traitement.
\end{itemize}

\noindent\textbf{Boucle et itérateur, instruction conditionnelle}\par
Les élèves, dans le cadre d'une résolution de problèmes, doivent être capables :
\begin{itemize}
    \item de programmer un calcul itératif, le nombre d'itérations étant donné ;
    \item de programmer une instruction conditionnelle, un calcul itératif, avec une fin de boucle conditionnelle.
\end{itemize}




\end{document}
