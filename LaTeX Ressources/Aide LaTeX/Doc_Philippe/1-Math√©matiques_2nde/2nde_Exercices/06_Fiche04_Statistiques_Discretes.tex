\documentclass[10pt,openright,twoside,french]{book}

\usepackage{marvosym}
\input philippe2013
\input philippe2013_activites

\pagestyle{empty}

\begin{document}

\TitreExo{4}{Statistiques discrètes \\ Bilan de ce qu'il faut savoir}

\exo

On demande à vingt adolescents le nombre de minutes dans la semaine durant laquelle ils font du sport. Voici les résultats obtenus :
\[30 \qq 60 \qq 120 \qq 60 \qq 60 \qq 240 \qq 180 \qq 0 \qq 0 \qq 60\]\[30 \qq 30 \qq 120 \qq 0 \qq 30 \qq 240 \qq 30 \qq 60 \qq 240 \qq 60\]

\begin{enumerate}
    \item Construire un tableau contenant trois lignes avec les informations suivantes :
        \begin{itemize}
            \item[$\star$] le temps en minutes ;
            \item[$\star$] l'effectif de chaque valeur ;
            \item[$\star$] les effectifs cumulés croissants.
        \end{itemize}
    \item Construire le diagramme en bâton représentant cette série statistique en choisissant comme unités 2 carreaux pour 30 min en abscisse et 1 carreau pour 1 adolescent en ordonnée.
    \item Calculer la moyenne de cette série statistique.
    \item Calculer l'étendue de cette série statistique. Comment interpréter ce résultat ?
    \item Calculer la médiane de cette série statistique. Comment interpréter ce résultat ?
    \item Calculer le premier et le troisième quartile de cette série statistique. Comment interpréter ces résultats ?
\end{enumerate}\bigskip

\exo
Dans un club de tennis, on s'intéresse aux adhérents âgés de moins de 20 ans. Le graphique ci-dessous donne la répartition des adhérents en fonction de leur âge.\par
\[\includegraphics[scale=1.25]{stats_fig.1}\]
\begin{enumerate}
    \item Calculer l'effectif total d'adhérents âgés de moins de 20 ans.
    \item Calculer le pourcentage d'adhérents de moins de 14 ans.
    \item Calculer et interpréter l'étendue de cette série.
    \item Calculer la moyenne d'âge des adhérents de moins de 20 ans (arrondir au dixième).
    \item Calculer la médiane de cette série statistique et interpréter le résultat.
    \item Calculer le premier et le troisième quartile de cette série statistique et interpréter les résultats.
\end{enumerate}
\end{document} 