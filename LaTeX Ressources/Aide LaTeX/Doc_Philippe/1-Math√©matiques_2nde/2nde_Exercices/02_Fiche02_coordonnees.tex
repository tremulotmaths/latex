\documentclass[10pt,openright,twoside,french]{book}

\usepackage{marvosym}
\input philippe2013
\input philippe2013_activites

\pagestyle{empty}

\begin{document}

\TitreExo{2}{Coordonnées d'un point \\ dans le plan}

\exo Dans un repère orthonormé $\OIJ$, on donne les coordonnées des points suivants :
\[A(-1\pv5) \qq B(1 \pv 2) \qq C(-2 \pv 3) \qetq D(x_D \pv y_D).\]

\begin{enumerate}
    \item Dessiner le repère en prenant deux carreaux comme unité. Placer les points $A$, $B$ et $C$.
    \item Calculer les longueurs $AB$, $BC$ et $AC$.
    \item Le triangle $ABC$ est-il rectangle ? isocèle ?
    \item Calculer les coordonnées du milieu $I$ de $[AB]$.
    \item Placer le point $D$ de telle façon que $ADBC$ soit un parallélogramme.
    \item Lire les coordonnées du point $D$.
    \item Expliquer pourquoi $I$ est le milieu de $[CD]$.
    \item En déduire les coordonnées du point $D$ par le calcul.
\end{enumerate}\[*\]

\exo Dans un repère orthonormé $\OIJ$, on considère les quatre points suivants :
\[K(-4 \pv -1) \qq I(1 \pv 0) \qq L(2 \pv 2) \qetq M(-3 \pv 1).\]
Démontrer de deux façons différentes que le quadrilatère $KILM$ est un parallélogramme.
\[*\]

\exo $ABCD$ est un parallélogramme tel que :
\[AB = \SI{9}{cm} \qq AD = \SI{6}{cm} \qetq \widehat{DAB} = 45\degres.\]
\begin{enumerate}
    \item Faire une figure en grandeur réelle.
    \item Compléter la figure à l'aide des informations suivantes :
    \begin{enumerate}
        \item $K$ est le milieu du segment $[AB]$.
        \item $J$ est le milieu du segment $[AD]$.
        \item $O$ est le centre du parallélogramme.
        \item $E$ est le point d'intersection des droites $(AO)$ et $(DK)$.
    \end{enumerate}
    \item Dans le triangle $ADB$, que représente le point $E$ ? Justifier précisément la réponse.
    \item Les points $B$, $E$ et $J$ sont-il alignés ? Pourquoi ?
    \item On se place dans le repère $(A \pv B, D)$.
    \begin{enumerate}
        \item Donner les coordonnées des points $A$, $B$ et $D$.
        \item Calculer les coordonnées des points $K$, $J$ et $O$.
        \item En utilisant une règle graduée, lire les coordonnées du point $E$.
    \end{enumerate}
\end{enumerate}
\[***\]
\end{document} 