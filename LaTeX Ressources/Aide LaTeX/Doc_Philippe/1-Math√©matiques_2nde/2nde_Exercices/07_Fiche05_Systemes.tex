\documentclass[11pt,openright,twoside,french]{book}

\usepackage{marvosym}
\input philippe2013
\input philippe2013_activites

\pagestyle{empty}

\begin{document}

\TitreExo{5}{Résolution de systèmes \\ de deux équations à deux inconnues}

\exo Parmi les systèmes suivants, quels sont ceux qui ont pour solution le couple $(2 \pv 3)$ ?

\[
(S_1) :\quad \left\{\begin{array}{rcl}
                                3x - 2y & = & 0 \\
                                4x + y & = & 10
                            \end{array}
                    \right. \qq
(S_2) :\quad \left\{\begin{array}{rcl}
                                5x - 4y & = & -2 \\
                                x + y & = & 5
                            \end{array}
                    \right.
\]
\[
(S_3) :\quad \left\{\begin{array}{rcl}
                                x + y & = & 4 \\
                                x - 2y & = & -4
                            \end{array}
                    \right. \qq
(S_4) :\quad \left\{\begin{array}{rcl}
                                -3x + 2y & = & 0 \\
                                3x - 4y & = & -6
                            \end{array}
                    \right.
\]

\[*\]

\exo On considère le système suivant :
\[\left\{\begin{array}{rcl}
                x + 2y & = & -3 \\
                3x - y & = & 5
            \end{array}
    \right.
\]
\begin{enumerate}
    \item Transformer chaque équation en équation réduite de droite.
    \item Résoudre graphiquement le système.
\end{enumerate}\[*\]

\exo On considère le système suivant :
\[\left\{\begin{array}{rcl}
                9x - 3y & = & 3 \\
                -12x + 4y & = & 8
            \end{array}
    \right.
\]
\begin{enumerate}
    \item Transformer chaque équation en équation réduite de droite.
    \item Sans faire de graphique, quelle information a-t-on sur les éventuelles solutions du système. Justifier.
\end{enumerate}\[*\]

\exo Résoudre les systèmes suivants :
\[
(S_5) :\quad \left\{\begin{array}{rcl}
                                5x + y & = & 3 \\
                                2x - 4y & = & 10
                            \end{array}
                    \right. \qq
(S_6) :\quad \left\{\begin{array}{rcl}
                                5x + 3y & = & 2 \\
                                -10x + 4y & = & -14
                            \end{array}
                    \right.
\]
\[*\]

\exo À la boulangerie, Paulette achète deux croissants et quatre pains au chocolat pour \EUR{$6$}. Dans la même boulangerie, Paulo achète deux croissants et un pain au chocolat pour \EUR{$2,70$}.\par
Paulito possède \EUR{$5$} et désire acheter $3$ pains au chocolat et trois croissants dans la même boulangerie. Possède-t-il assez d'argent ?\par
\textbf{Justifier la réponse en utilisant la résolution d'un système.}
                    
\end{document} 