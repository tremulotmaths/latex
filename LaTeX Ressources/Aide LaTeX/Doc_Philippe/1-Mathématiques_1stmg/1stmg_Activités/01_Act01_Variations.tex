\documentclass[10pt,openright,twoside,french]{book}
\input philippe2013
\input philippe2013_activites
\pagestyle{empty}

\begin{document}

\TitreActivite{i.1}{Calculer une évolution}

\begin{enumerate}
    \item Paul a acheté une veste en solde. Il a payé \EUR{$102$} alors que l'ancien prix était de \EUR{$120$}.\par On note $p_1$ le prix avant les soldes et $p_2$ le prix après les soldes.
        \begin{enumerate}
            \item Le prix de la veste a-t-il augmenté ou diminué ? de combien ?
            \item Recopier et compléter la phrase suivante : {\cursive L'évolution de $p_1$ à $p_2$ est égale à : \ldots}
            \item Calculer, en pourcentage, l'évolution du prix de la veste.
            \item Recopier et compléter la phrase suivante : {\cursive L'évolution de $p_1$ à $p_2$ est égale à : \ldots} \%.
        \end{enumerate}\medskip
    \item Paulette a acheté un pantalon en solde. Elle a payé \EUR{$32$} alors que l'ancien prix était de \EUR{$40$}.\par On note $p_3$ le prix avant les soldes et $p_4$ le prix après les soldes.
        \begin{enumerate}
            \item Le prix du pantalon a-t-il augmenté ou diminué ? de combien ?
            \item Recopier et compléter la phrase suivante : {\cursive L'évolution de $p_3$ à $p_4$ est égale à : \ldots}
            \item Calculer, en pourcentage, l'évolution du prix du pantalon.
            \item Recopier et compléter la phrase suivante : {\cursive L'évolution de $p_3$ à $p_4$ est égale à : \ldots} \%.
        \end{enumerate}\medskip
    \item Paul dit à Paulette : \og~J'ai fait une meilleure affaire que toi !~\fg\par Paulette prétend le contraire.
        \begin{enumerate}
            \item Quel est l'argument de Paul ?
            \item Quel est celui de Paulette ?
            \item Qui a raison ?
        \end{enumerate}
    \item Compléter le tableau suivant :

    \begin{center}
    \renewcommand\arraystretch{1.75}
        \begin{tabular}{|c|c|c|c|c|}
            \hline
                \multirow{2}*{$y_1$} & \multirow{2}*{$y_2$} & Hausse ou baisse & \multicolumn{2}{c|}{Variations de $y_1$ à $y_2$}\\
                \cline{4-5}
                & & de $y_1$ à $y_2$ & Absolue & Relative \\
            \hline
                $10$ & $15$ & &  & \\
            \hline
                $0{,}8$ & $0{,}2$ & &  & \\
            \hline
                $50$ & $40$ & &  & \\
            \hline
                $65$ & $65$ & &  & \\
            \hline
                $0$ & $15$ & & & \\
            \hline
                $150$ & & & $12$ &  \\
            \hline
                & $196$ & & $-4$ & \\
            \hline
                $18$ & & & & $6\%$ \\
             \hline
                $150$ & & & & $-12{,}5\%$ \\
            \hline
        \end{tabular}
    \end{center}
\end{enumerate}

\end{document} 