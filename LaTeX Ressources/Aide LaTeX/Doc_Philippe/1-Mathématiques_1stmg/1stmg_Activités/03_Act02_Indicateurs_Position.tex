\documentclass[10pt,openright,twoside,french]{book}
\usepackage{marvosym}
\input philippe2013
\input philippe2013_activites
\pagestyle{empty}


\begin{document}

\TitreActivite{iii.2}{Indicateurs de position\par Savoir interpréter}

\exo Dans un village, on a compté le nombre d'enfants par famille. Voici les résultats obtenus :
\begin{center}
    \begin{tabularx}{0.65\linewidth}{|m{3cm}|*{7}{X|}}
        \hline
            Nombre d'enfants & $0$ & $1$ & $2$ & $3$ & $4$ & $5$ & $6$ \\
        \hline
            Effectifs & $82$ & $124$ & $217$ & $156$ & $52$ & $28$ & $22$ \\
        \hline
    \end{tabularx}
\end{center}

\begin{enumerate}
    \item Calculer le nombre moyen d'enfants par famille. Ce nombre a-t-il une signification réelle ?
    \item Calculer une médiane de cette série et donner une interprétation.\par Pourquoi dit-on \textbf{une} médiane et non \textbf{la} médiane ?
    \item Calculer le premier et le troisième quartile et donner une interprétation.
    \item Sur une page complète, construire le diagramme en bâtons correspondant à cette série.\par En ordonnée, l'unité sera de $1~mm$ pour $1$ enfant.
    \item Construire le polygone des fréquences cumulées croissantes. Comment s'en servir pour trouver une médiane ?
\end{enumerate}\[*\]

\exo
Une étude sur la durée de vie en années de $500$ chauffe-eau fabriqués par une entreprise a donné les résultats suivants :
\begin{center}
\renewcommand\arraystretch{1.5}
    \begin{tabularx}{0.83\linewidth}{|m{2cm}|*{7}{X|}}
        \hline
            Durée de vie & $\intervallefo 0 4$ & $\intervallefo 4 8$ & $\intervallefo{8}{12}$ & $\intervallefo{12}{16}$ & $\intervallefo{16}{20}$ & $\intervallefo{20}{24}$ & $\intervallefo{24}{28}$ \\
        \hline
            Effectifs & $10$ & $36$ & $78$ & $120$ & $154$ & $60$ & $42$ \\
        \hline
    \end{tabularx}
\end{center}

\begin{enumerate}
    \item Donner une interprétation de la troisième colonne.
    \item Calculer la durée de vie moyenne d'un chauffe-eau.
    \item À l'aide d'un graphique dont vous préciserez le nom, déterminer la valeur d'une médiane ainsi que le premier et le troisième quartile.
    \item Quel est le pourcentage de chauffe-eau dont la durée de vie est supérieure à $20$ ans ?
\end{enumerate}

\end{document} 