\documentclass[10pt,openright,twoside,french]{book}

\input philippe2013
\input philippe2013_activites
\pagestyle{empty}


\begin{document}

\TitreActivite{v.3}{Calcul de proportions \\ Inclusion}

Dans une région, la \textbf{population active} correspond aux personnes qui ont un emploi (les actifs) et les chômeurs.\par
Le reste de la population représente la \textbf{population inactive} : les retraités, les étudiants...

Dans une ville, $\np{24000}$ habitants ont $15$ ans ou plus. On appelle $H$ cette population.\par
Parmi eux, $\np{15000}$ personnes font partie de la population active, notée $A$.\par
Il y a $\np{1200}$ chômeurs dans cette ville. On note cette sous-population $C$.

\begin{enumerate}
    \item Compléter le diagramme de Venn suivant modélisant la situation :
    \begin{center}
        \tikz{\draw (0,0) ellipse (5 and 3); \draw (3.25,2) node {$H$};}
    \end{center}
    \item Calculer sous forme de pourcentage :
    \begin{enumerate}
        \item la proportion $p_1$ de la population active parmi l'ensemble des habitants ;
        \item la part (= proportion) $p_2$ de chômeurs parmi l'ensemble de la population active.
    \end{enumerate}
    \item On note $p$, la proportion de chômeur sur l'ensemble des habitants. Calculer $p$.
    \item Calculer $p_1 \times p_2$. Que remarque-t-on ? Pourquoi cela arrive-t-il ?
\end{enumerate}

\end{document} 