\documentclass[12pt,openright,twoside,french]{book}
\input philippe2013
\input philippe2013_activites
\pagestyle{empty}

\begin{document}

\TitreActivite{vi.1}{Schéma de \\ \bsc{Bernoulli}}

\section*{Tirage sans remise}

Dans un jeu télévisé, un candidat joue pour gagner \EUR{$\np{100000}$}. Pour cela, il doit tirer au hasard une boule dans une urne.\par
L'urne est opaque et les boules sont indiscernables au toucher. Il y a $14$ boules bleues et $3$ boules rouges.\par
Le candidat tire dans l'urne une boule et la met de côté. Puis il tire dans la même urne une seconde boule.\par
S'il tire 2 boules rouges, il gagne l'argent.\par
S'il obtient qu'une seule boule rouge, il gagne un voyage.\par
S'il n'obtient pas de boule rouge, il gagne les deux boules bleues.\medskip

\begin{enumerate}
    \item Les deux tirages sont-ils identiques ? Pourquoi ?
    \item Les deux tirages sont-ils indépendants ? Pourquoi ?
    \item On considère un seul tirage. Quel événement peut être considéré comme étant un << succès >> ?
    \item Réaliser l'arbre pondéré des possibles.
    \item Utiliser l'arbre pour donner sous forme d'une fraction irréductible puis sous forme d'un pourcentage à $10^{-2}\%$ près :
    \begin{enumerate}
        \item la probabilité de gagner les \EUR{$\np{100000}$} ;
        \item la probabilité de gagner le voyage.
    \end{enumerate}
\end{enumerate}

\section*{Tirage avec remise}

On garde le même jeu avec les mêmes gains mais la règle change :
après le premier tirage, le candidat remet la boule tirée dans l'urne. Puis les boules sont mélangées et il tire une boule une deuxième fois.\par
Les gains restent identiques à la règle précédente.\medskip

\begin{enumerate}
    \item Les deux tirages sont-ils identiques ? Pourquoi ?
    \item Les deux tirages sont-ils indépendants ? Pourquoi ?
    \item Réaliser l'arbre pondéré des possibles.
    \item Utiliser l'arbre pour donner sous forme d'une fraction irréductible puis sous forme d'un pourcentage à $10^{-2}\%$ près :
    \begin{enumerate}
        \item la probabilité de gagner les \EUR{$\np{100000}$} ;
        \item la probabilité de gagner le voyage.
    \end{enumerate}
    \item Imaginons que l'on réalise l'expérience $n$ fois de suite.\par Comment calculer la probabilité d'obtenir $10$ boules rouges ?
\end{enumerate}
\end{document} 