\documentclass[10pt,openright,twoside,french]{book}

\input philippe2013
\input philippe2013_activites
\pagestyle{empty}


\begin{document}

\TitreActivite{v.1}{Calcul de proportions}

\exo Compléter le tableau suivant :

\begin{center}
\renewcommand\arraystretch{2.5}
    \begin{tabularx}{0.8\linewidth}{|>\centering X|>\centering X|>{\centering\arraybackslash}X|}
        \hline
            \multicolumn{3}{|c|}{Proportion écrite sous forme :}\\
        \hline
            fractionnaire & décimale & de pourcentage \\
        \hline
            $\frac{3}{5}$ & & \\
        \hline
            $\frac{12}{250}$ & & \\
        \hline
            & & $7\%$ \\
        \hline
            & $0,195$ & \\
        \hline
    \end{tabularx}
\renewcommand\arraystretch{1}
\end{center}\[*\]

\exo Dans un lycée, chaque élève étudie comme première langue vivante l'anglais, l'espagnol ou l'anglais.
\begin{enumerate}
    \item Sachant que $32\%$ des élèves étudient l'allemand en première langue, peut-on connaître :
        \begin{enumerate}
            \item le nombre d'élèves qui étudient l'allemand en première langue ? Pourquoi ?
            \item le nombre d'élèves qui étudient l'anglais en première langue ?
            \item le pourcentage d'élèves qui étudient l'espagnol en première langue ?
            \item le pourcentage d'élèves qui étudient l'espagnol ou l'anglais en première langue.
        \end{enumerate}
    \item Sachant qu'il y a $\np{1250}$ élèves dans ce lycée, calculer :
        \begin{enumerate}
            \item le nombre d'élèves qui étudient l'allemand en première langue ;
            \item le nombre d'élèves qui étudient l'espagnol ou l'anglais en première langue.
        \end{enumerate}
    \item Sachant que les élèves qui choisissent d'étudier l'anglais en première langue sont deux fois plus nombreux que ceux qui choisissent d'étudier l'allemand, calculer le nombre puis le pourcentage d'élèves qui étudient :
        \begin{enumerate}
            \item l'anglais en première langue ;
            \item l'espagnol en première langue.
        \end{enumerate}
\end{enumerate}\[*\]

\exo Un établissement scolaire de $800$ élèves regroupe un collège et un lycée. Le principal de ce collège-lycée propose aux lycéens qui le souhaitent de donner des cours de soutien aux classes de troisième dans le but de les aider à préparer le Diplôme National du Brevet.

\begin{enumerate}
    \item Les lycéens représentent $\dfrac 15$ des élèves de l'établissement et $20\%$ d'entre eux se portent volontaires.
    \begin{enumerate}
        \item Calculer le nombre de lycéens de l'établissement. En déduire le nombre de collégiens de l'établissement.
        \item Calculer le nombre de lycéens de l'établissement qui se portent volontaires.
    \end{enumerate}
    \item Au collège, on recense $148$ élèves en troisième. La proportion d'élèves qui souhaite prendre des cours de soutien est égale à $0,25$.
        \begin{enumerate}
            \item \'Ecrire cette proportion sous forme fractionnaire et sous forme d'un pourcentage.
            \item Calculer le nombre d'élèves de troisième désirant prendre des cours de soutien.
            \item En déduire le nombre d'élèves de troisième préférant réviser seuls.
        \end{enumerate}
\end{enumerate}

\end{document} 