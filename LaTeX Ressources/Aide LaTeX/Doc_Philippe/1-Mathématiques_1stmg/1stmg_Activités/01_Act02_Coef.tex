\documentclass[10pt,openright,twoside,french]{book}
\input philippe2013
\input philippe2013_activites
\pagestyle{empty}

\begin{document}

\TitreActivite{i.2}{C{\oe}fficient multiplicateur}

Une petite entreprise emploie deux commerciaux, M. \bsc{Machin} et Mme \bsc{Bidule}.\par
On note $y_1$ le nombre de contrats conclus l'année passée et $y_2$ le nombre de contrats conclus cette année.\medskip

\begin{enumerate}
    \item Le nombre de contrats conclus par M. \bsc{Machin} était l'an dernier de $75$. Il annonce à son patron qu'il a multiplié par $1{,}48$ cette année le nombre de ses contrats signés.
        \begin{enumerate}
            \item Calculer, en utilisant les notations $y_1$ et $y_2$, le nombre de contrats signés cette année. Ce nombre a-t-il augmenté ou diminué ?
            \item Vérifier, en utilisant une formule du cours, que le taux d'évolution $t$ du nombre de contrats conclus de l'année passée à cette année est de $48\%$.
            \item Calculer $1+t$. Que remarque-t-on ?
        \end{enumerate}\smallskip
    \item Le nombre de contrats conclus par Mme \bsc{Bidule} était l'année passée de $120$. Elle annonce à son patron qu'elle a multiplié par $0{,}95$ cette année le nombre de contrats signés.
        \begin{enumerate}
            \item Calculer, en utilisant les notations $y_1$ et $y_2$, le nombre de contrats signés cette année. Ce nombre a-t-il augmenté ou diminué ?
            \item Vérifier, en utilisant une formule du cours, que le taux d'évolution $t$ du nombre de contrats conclus de l'année passée à cette année est de $-5\%$.
            \item Calculer $1+t$. Que remarque-t-on ?
        \end{enumerate}\smallskip
    \item Quel lien peut-on faire entre le \coef multiplicateur et l'évolution ?\smallskip
    \item Compléter le tableau suivant :
    
    \begin{center}
        \renewcommand\arraystretch{1.75}
        \begin{tabularx}{0.95\linewidth}{|c|X|X|}
            \hline
                Lorsqu'une grandeur & \multicolumn{2}{c|}{Cette grandeur est multipliée par :}\\
                \cline{2-3}
                varie de : & s'il s'agit d'une hausse & s'il s'agit d'une baisse \\
            \hline
                $12\%$ & $1 + 12\% = 1 + 0{,}12 = 1{,}12$ & $1 - 12\% = 1 - 0{,}12 = 0{,}88$\\
            \hline
                $1\%$ & & \\
            \hline
                $10{,}4\%$ & & \\
            \hline
                $50\%$ & & \\
            \hline
                $73\%$ & & \\
            \hline
                $115{,}25\%$ & & \\
            \hline
                & $1{,}27 =$ & \\
            \hline
                & $1{,}196 =$ & \\
            \hline
                & & $0{,}85 =$ \\
            \hline
                & & $0{,}713 =$\\
            \hline
        \end{tabularx}
    \end{center}
\end{enumerate}

\end{document} 