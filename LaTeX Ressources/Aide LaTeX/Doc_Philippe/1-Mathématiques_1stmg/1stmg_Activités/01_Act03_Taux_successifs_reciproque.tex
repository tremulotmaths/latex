\documentclass[10pt,openright,twoside,french]{book}
\input philippe2013
\input philippe2013_activites
\pagestyle{empty}

\begin{document}

\TitreActivite{i.3}{Taux d'évolution successifs\par Taux réciproque}

\exo En $\NP{2010}$, un paysan a produit $x$ tonnes de blé.\par
En $\NP{2011}$, sa production a diminué de $8\%$.\par
L'année suivante, il est rassuré car la production a augmenté de $10\%$.

\begin{enumerate}
    \item Le paysan est-il capable de donner rapidement l'évolution globale entre $\NP{2010}$ et $\NP{2012}$ ?
    \item Supposons $x = 500$.
        \begin{enumerate}
            \item À combien de tonnes s'élève sa production en $\NP{2011}$ ? en $\NP{2012}$ ?
            \item Quelle est alors, en pourcentage, l'évolution globale entre $\NP{2010}$ et $\NP{2012}$ ?
        \end{enumerate}
    \item Supposons $x = 635$.
        \begin{enumerate}
            \item À combien de tonnes s'élèvent sa production en $\NP{2011}$ ? en $\NP{2012}$ ?
            \item Quelle est alors, en pourcentage, l'évolution globale entre $\NP{2010}$ et $\NP{2012}$ ?
        \end{enumerate}
    \item Peut-on conclure ?
\end{enumerate}\bigskip

\exo Dans un village au bord de mer, la population est de $125$ habitants.
\begin{enumerate}
    \item Au début des vacances d'été, la population augmente pour atteindre un total de $400$ habitants.\par
        Calculer l'évolution absolue ainsi que le taux d'évolution (en pourcentage) correspondant.
    \item À la fin des vacances, le village retrouve sa population d'origine de $125$ habitants.\par
        Calculer l'évolution absolue ainsi que le taux d'évolution (en pourcentage) correspondant.
    \item Quelles remarques peut-on faire ?
\end{enumerate}

\end{document} 