\documentclass[10pt,french]{article}
\input preambule_2013

\newcounter{exoc}
\newenvironment{exoc}[1]{%
  \refstepcounter{exoc}\textbf{Exercice \theexoc} :\hfill {\textbf{(#1)}}\par
  \medskip}%
{\medskip}

\begin{document}

%--------------------------------------------------
%       SUJET A
%--------------------------------------------------

\pieddepage{}{A}{}

\begin{center}
\begin{tabularx}{\textwidth}{|>\centering m{2.5cm}|>\centering X|>{\centering\arraybackslash} m{2.5cm}|}
	\hline
		1\iere \bsc{s.t.m.g.} &  Mercredi 9 octobre \np{2013} & \textbf{\'Evolution} \\
	\hline
		\multicolumn{3}{|c|}{\bsc{Contrôle de mathématiques}} \\
	\hline
        \multicolumn{1}{|r}{\bsc{Nom}:} & \multicolumn{2}{l|}{} \\
		\multicolumn{1}{|r}{Prénom:} & \multicolumn{2}{l|}{} \\
	\hline
        \multicolumn{3}{|l|}{\bfseries Note et observations :} \\[1cm]
    \hline
\end{tabularx}\bigskip

{\itshape
La qualité et la précision de la rédaction seront prises en compte dans l'appréciation des copies.\par
Le barème est indicatif.\par
\textbf{Tous les calculs doivent être détaillés.}}
\end{center}

\begin{exoc}{1 + 1 + 1 + 2 + 2 = 7 points}
\begin{enumerate}
    \item Paul a eu \EUR{$25$} d'argent de poche le mois dernier. Ce mois-ci, il a gagné \EUR{$34$}.\par Calculer la variation absolue correspondante.
    \item Paulette a eu \EUR{$32$} d'argent de poche le mois dernier. Ce mois-ci, elle a gagné \EUR{$26$}.\par Calculer la variation relative correspondante. Donner le résultat en pourcentage.
    \item Paulo a eu \EUR{$40$} d'argent de poche le mois dernier. Ce mois-ci, il a gagné $12\%$ de plus.\par Calculer le c{\oe}fficient multiplicateur correspondant.
    \item L'argent de poche de Paula a diminué de $3\%$ le mois dernier. Ce mois-ci, son argent de poche a augmenté de $5\%$.\par Calculer le taux d'évolution global correspondant. Donner le résultat en pourcentage.
    \item L'argent de poche de Paulito a diminué de $6\%$ le mois dernier.\par Quel doit être le taux d'évolution de ce mois-ci pour qu'il puisse revenir à son argent de poche de départ ? Donner le résultat en pourcentage avec 2 chiffres après la virgule.
\end{enumerate}
\end{exoc}

\begin{exoc}{1 + 1 + 2 = 4 points}
    $206$ millions de tickets de cinéma ont été vendus en France en $\np{2010}$. Cela correspond à une augmentation de $5,4$ millions par rapport à l'année précédente.
    \begin{enumerate}
        \item Quel était le nombre de tickets vendus en $\np{2009}$ ?
        \item Calculer le taux d'évolution, arrondi à $0,1\%$ près, du nombre de tickets de cinéma vendus de $\np{2009}$ à $\np{2010}$.
        \item En $\np{2011}$, la vente de tickets a augmenté de $3,8\%$. Calculer le taux d'évolution global entre $\np{2009}$ et $\np{2011}$.
    \end{enumerate}
\end{exoc}

\begin{exoc}{1 + 1 + 1 + 3 + 3 = 9 points}
    Le tableau suivant a été fait sur un tableur et représente l'évolution du nombre de chômeurs dans un pays entre $\np{2007}$ et $\np{2010}$.
    \begin{center}
        \begin{tabular}{c*{3}{|c}|}
           \multicolumn{1}{c}{} & \multicolumn{1}{c}{\texttt{A}} & \multicolumn{1}{c}{\texttt{B}} & \multicolumn{1}{c}{\texttt{C}} \\
            \cline{2-4}
            \texttt 1 & Année & Nombre de chômeurs (milliers) & Evolution \\
            \cline{2-4}
            \texttt 2 & {\tt 2007} & {\tt 2226} & \\
            \cline{2-4}
            \texttt 3 & {\tt 2008} & {\tt 2067} & {\tt -7.14\%} \\
            \cline{2-4}
            \texttt 4 & {\tt 2009} & {\tt 2582} & {\tt 24.92\%} \\
            \cline{2-4}
            \texttt 5 & {\tt 2010} & {\tt 2653} & {\tt 2.75\%} \\
            \cline{2-4}
        \end{tabular}
    \end{center}
    
    \begin{enumerate}
        \item Quel est le taux d'évolution du nombre de chômeurs entre $\np{2007}$ et $\np{2008}$ ?
        \item Quelle \textbf{formule} a été inscrite dans la cellule \texttt{C4} ?
        \item Quel est le c{\oe}fficient multiplicateur par lequel il faut multiplier le nombre de chômeur en \np{2009} pour trouver le nombre de chômeur en \np{2010} ?
        \item Calculer le taux d'évolution global de \np{2007} à \np{2010}. Donner le résultat sous forme de pourcentage arrondi au centième près.
        \item Entre \np{2009} et \np{2010}, le taux d'évolution était de $+2,75\%$. Quel doit être le taux d'évolution réciproque en \np{2011} pour que le nombre de chômeurs soit le même que celui de \np{2009} ?
    \end{enumerate}
\end{exoc}

\clearpage\setcounter{exoc}{0}

%--------------------------------------------------
%       SUJET B
%--------------------------------------------------

\pieddepage{}{B}{}

\begin{center}
\begin{tabularx}{\textwidth}{|>\centering m{2.5cm}|>\centering X|>{\centering\arraybackslash} m{2.5cm}|}
	\hline
		1\iere \bsc{s.t.m.g.} &  Mercredi 9 octobre \np{2013} & \textbf{\'Evolution} \\
	\hline
		\multicolumn{3}{|c|}{\bsc{Contrôle de mathématiques}} \\
	\hline
        \multicolumn{1}{|r}{\bsc{Nom}:} & \multicolumn{2}{l|}{} \\
		\multicolumn{1}{|r}{Prénom:} & \multicolumn{2}{l|}{} \\
	\hline
        \multicolumn{3}{|l|}{\bfseries Note et observations :} \\[1cm]
    \hline
\end{tabularx}\bigskip

{\itshape
La qualité et la précision de la rédaction seront prises en compte dans l'appréciation des copies.\par
Le barème est indicatif.\par
\textbf{Tous les calculs doivent être détaillés.}}
\end{center}

\begin{exoc}{1 + 1 + 1 + 2 + 2 = 7 points}
\begin{enumerate}
    \item Paul a eu \EUR{$32$} d'argent de poche le mois dernier. Ce mois-ci, il a gagné \EUR{$26$}.\par Calculer la variation absolue correspondante.
    \item Paulette a eu \EUR{$25$} d'argent de poche le mois dernier. Ce mois-ci, elle a gagné \EUR{$34$}.\par Calculer la variation relative correspondante. Donner le résultat en pourcentage.
    \item Paulo a eu \EUR{$40$} d'argent de poche le mois dernier. Ce mois-ci, il a gagné $16\%$ de plus.\par Calculer le c{\oe}fficient multiplicateur correspondant.
    \item L'argent de poche de Paula a diminué de $5\%$ le mois dernier. Ce mois-ci, son argent de poche a augmenté de $7\%$.\par Calculer le taux d'évolution global correspondant. Donner le résultat en pourcentage.
    \item L'argent de poche de Paulito a diminué de $7\%$ le mois dernier.\par Quel doit être le taux d'évolution de ce mois-ci pour qu'il puisse revenir à son argent de poche de départ ? Donner le résultat en pourcentage avec 2 chiffres après la virgule.
\end{enumerate}
\end{exoc}

\begin{exoc}{1 + 1 + 2 = 4 points}
    $206$ millions de tickets de cinéma ont été vendus en France en $\np{2010}$. Cela correspond à une augmentation de $4,4$ millions par rapport à l'année précédente.
    \begin{enumerate}
        \item Quel était le nombre de tickets vendus en $\np{2009}$ ?
        \item Calculer le taux d'évolution, arrondi à $0,1\%$ près, du nombre de tickets de cinéma vendus de $\np{2009}$ à $\np{2010}$.
        \item En $\np{2011}$, la vente de tickets a augmenté de $4,8\%$. Calculer le taux d'évolution global entre $\np{2009}$ et $\np{2011}$.
    \end{enumerate}
\end{exoc}

\begin{exoc}{1 + 1 + 1 + 3 + 3 = 9 points}
    Le tableau suivant a été fait sur un tableur et représente l'évolution du nombre de chômeurs dans un pays entre $\np{2007}$ et $\np{2010}$.
    \begin{center}
        \begin{tabular}{c*{3}{|c}|}
           \multicolumn{1}{c}{} & \multicolumn{1}{c}{\texttt{A}} & \multicolumn{1}{c}{\texttt{B}} & \multicolumn{1}{c}{\texttt{C}} \\
            \cline{2-4}
            \texttt 1 & Année & Nombre de chômeurs (milliers) & Evolution \\
            \cline{2-4}
            \texttt 2 & {\tt 2007} & {\tt 2226} & \\
            \cline{2-4}
            \texttt 3 & {\tt 2008} & {\tt 2040} & {\tt -8.36\%} \\
            \cline{2-4}
            \texttt 4 & {\tt 2009} & {\tt 2612} & {\tt 28.04\%} \\
            \cline{2-4}
            \texttt 5 & {\tt 2010} & {\tt 2681} & {\tt 2.64\%} \\
            \cline{2-4}
        \end{tabular}
    \end{center}

    \begin{enumerate}
        \item Quel est le taux d'évolution du nombre de chômeurs entre $\np{2007}$ et $\np{2008}$ ?
        \item Quelle \textbf{formule} a été inscrite dans la cellule \texttt{C4} ?
        \item Quel est le c{\oe}fficient multiplicateur par lequel il faut multiplier le nombre de chômeur en \np{2009} pour trouver le nombre de chômeur en \np{2010} ?
        \item Calculer le taux d'évolution global de \np{2007} à \np{2010}. Donner le résultat sous forme de pourcentage arrondi au centième près.
        \item Entre \np{2009} et \np{2010}, le taux d'évolution était de $+2,64\%$. Quel doit être le taux d'évolution réciproque en \np{2011} pour que le nombre de chômeurs soit le même que celui de \np{2009} ?
    \end{enumerate}
\end{exoc}
\end{document} 