\documentclass[12pt,french]{article}
\input preambule_2013
\pagestyle{empty}
\begin{document}

\begin{center}
\psframebox[shadow=true,shadowcolor=gray!75,shadowsize=5pt,framearc=0.5,fillstyle=gradient,gradmidpoint=0.8,gradangle=20,gradbegin=red!80!yellow!40,gradend= white]{
\parbox[c]{0.5\linewidth}{
\begin{center}
\Large\bfseries
Séance informatique\par
\uuline{Calculs de pourcentages}
\end{center}}}
\end{center}\bigskip

\textbf{\'Enoncé de départ :}\par
\textbullet{} L'entreprise $A$ est prête à m'embaucher pour un salaire de \EUR{$\np{1350}$} par mois la première année. Le salaire mensuel sera alors augmenté de $5\%$ à la fin de chaque année.\par
\textbullet{} L'entreprise $B$ est prête à m'embaucher pour un salaire de \EUR{$\np{1500}$} par mois la première année. Le salaire mensuel sera alors augmenté de $2,5\%$ à la fin de chaque année.\par\medskip

$\rightsquigarrow$\ Quelle entreprise me permet de gagner le plus d'argent \textbf{au total} au bout de $5$ ans ? au bout de $10$ ans ?

\section*{Préparation des feuilles de calculs}

\begin{enumerate}
    \item Ouvrir \texttt{Excel}.
    \item Renommer la première feuille : \texttt{Entreprise A}. Pour cela, faire un clic droit sur l'onglet en bas du tableur et choisir \texttt{Renommer}.
    \item Renommer la deuxième feuille : \texttt{Entreprise B}.
    \item Sur la première ligne de \textbf{chaque feuille}, donner les noms suivants aux colonnes :
    \begin{center}
        \begin{tabular}{c*{5}{|c}|}
           \multicolumn{1}{c}{} & \multicolumn{1}{c}{\texttt{A}} & \multicolumn{1}{c}{\texttt{B}} & \multicolumn{1}{c}{\texttt{C}} & \multicolumn{1}{c}{\texttt{D}} & \multicolumn{1}{c}{\texttt{E}} \\
            \cline{2-6}
            \texttt 1 & Année & Salaire mensuel & Augmentation & Salaire annuel & Total \\
            \cline{2-6}
            \texttt 2 & & & & & \\
        \end{tabular}
    \end{center}
    \item \textbf{Sur chaque feuille}, compléter la colonne \texttt{Année} avec les nombres de $1$ à $10$.
\end{enumerate}

\section*{Entreprise A}
\begin{enumerate}
    \item Que faut-il écrire dans la cellule \texttt{B2} ?
    \item Quel calcul faut-il écrire dans la cellule \texttt{C2} pour calculer $5\%$ de la cellule \texttt{B2} ?
    \item Quel calcul faut-il écrire dans la cellule \texttt{D2} pour calculer le salaire annuel (une année = 12 mois) ?
    \item Quel calcul faut-il écrire dans la cellule \texttt{B3} pour calculer le salaire de la deuxième année ?
    \item Calculer alors l'augmentation du salaire à la fin de la deuxième année ainsi que le salaire annuel.
    \item Sur les deux premières années, quel est le salaire total ?
    \item Utiliser le tableur pour compléter automatiquement les $10$ années.
\end{enumerate}

\section*{Entreprise B}
\begin{enumerate}
    \item Suivre les mêmes étapes que précédemment. Attention, le salaire est maintenant de \EUR{$\np{1500}$} et l'augmentation est de $2,5\%$.
    \item Répondre aux questions posées dans l'énoncé.
\end{enumerate}


\end{document}