\documentclass[10pt,openright,twoside]{book}
\usepackage{etex}

\input philippe2011_complet
\usepackage{lscape}

\entete{}{}{\footnotesize\itshape Programme de \premiere \bsc{Stmg}}
\pieddepage{}{\footnotesize - \thepage/\pageref{LastPage} -}{}
\renewcommand\headrulewidth{0pt}

\newcounter{chapp}
\newcommand\chapitre[1]{%
\refstepcounter{chapp}%
\begin{center}
\bfseries
\underbar{\bsc{Chapitre \thechapp}}\par\large\textit{#1}
\end{center}\nopagebreak[4]
}


\newcounter{fct}
\newcommand\SFT{%
\refstepcounter{fct}%
\item[\pfr{SFt.\thefct}]
}

\newcounter{geo}
\newcommand\INF{%
\refstepcounter{geo}%
\item[\pfr{Inf.\thegeo}]
}

\newcounter{stp}
\newcommand\STP{%
\refstepcounter{stp}%
\item[\pfr{StP.\thestp}]
}

\newcommand\EnPlus[1]{\footnotesize\textit{\underbar{Permet de travailler en parallèle :} #1}\par}

\begin{document}

\begin{center}
{\fontfamily{augie}\fontsize{9}{11}\selectfont
{\Large \pfr{\begin{tabular}{cc}
                            Cours de Mathématiques\\
                            Classe de Première\\
                            {\large Sciences et Technologies du Management et de la Gestion}
                        \end{tabular}}}}
\end{center}\bigskip


\noindent\textbf{Légende des trois parties du programme :}
    \begin{enumerate}
        \item[\quad $\checkmark$] \bsc{Inf.} Information chiffrée ;
        \item[\quad $\checkmark$] \bsc{SFt.} Suites et Fonctions ;
        \item[\quad $\checkmark$] \bsc{StP.} Statistiques et Probabilités.
    \end{enumerate}\[*\]


\chapitre{\'Evolution}
    \begin{enumerate}
        \INF Connaître et exploiter les relations $t = \frac{y_2 - y_1}{y_1}$ et $y_2 = (1 + t)y_1$ ;
        \INF Distinguer si un pourcentage exprime une proportion ou une évolution ;
        \INF Connaissant deux taux d'évolution successifs, déterminer le taux d'évolution global ;
        \INF Connaissant un taux d'évolution, déterminer le taux d'évolution réciproque.
    \end{enumerate}\[*\]

\chapitre{Suites}
    \begin{enumerate}
        \SFT Modéliser et étudier une situation simple à l'aide de suites ;
        \SFT $\lozenge$ Mettre en œuvre un algorithme ou utiliser un tableur pour obtenir une liste de termes d'une suite, calculer un terme de rang donné ;
        \SFT Réaliser et exploiter une représentation graphique des termes d'une suite ;
        \SFT Déterminer le sens de variation des suites arithmétiques et des suites géométriques, à l'aide de la raison.
    \end{enumerate}\[*\]

\chapitre{Statistiques descriptives}
    \begin{enumerate}
        \STP Utiliser de façon appropriée les deux couples usuels qui permettent de résumer une série statistique : (moyenne, écart type) et (médiane, écart interquartile) ;
        \STP Rédiger l'interprétation d'un résultat ou l'analyse d'un graphique ;
        \STP \'Etudier une série statistique ou mener une comparaison pertinente de deux séries statistiques à l'aide d'un tableur ou d'une calculatrice.
    \end{enumerate}\[*\]

\chapitre{Polynômes du second degré}
    \begin{enumerate}
        \SFT Résoudre une équation ou une inéquation du second degré ;
        \SFT Mobiliser les résultats sur le second degré dans le cadre de la résolution de problème.
    \end{enumerate}\[*\]

\chapitre{Proportions}
    \begin{enumerate}
        \INF Connaître et exploiter la relation entre effectifs et proportion ;
        \INF Associer proportion et pourcentage ;
        \INF Pour deux sous-populations $A$ et $B$ d'une population $E$, relier les proportions de $A$, de $B$, de $A \cup B$ et de $A \cap B$ ;
        \INF Connaître et exploiter la relation entre proportion de $A$ dans $B$, de $B$ dans $E$ et de $A$ dans $E$ lorsque $A \subset B$ et $B \subset E$ ;
        \INF Représenter des situations par des tableaux ou des arbres pondérés.
    \end{enumerate}\[*\]

\chapitre{Probabilités \bsc{i} : schéma de Bernoulli}
    \begin{enumerate}
        \STP Représenter un schéma de Bernoulli par un arbre pondéré ;
        \STP $\lozenge$ Simuler un schéma de Bernoulli à l'aide d'un tableur ou d'un algorithme ;
        \STP Connaître et utiliser les notations $\{X = k\}$, $\{X < k\}$, $p(X = k)$ et $p(X < k)$.
    \end{enumerate}\[*\]

\chapitre{Dérivation}
    \begin{enumerate}
        \SFT Déterminer l'expression de la fonction dérivée d'une fonction polynôme du second degré ;
        \SFT Utiliser le signe de la fonction dérivée pour retrouver les variations du trinôme et pour déterminer son extremum ;
        \SFT Calculer le nombre dérivé et l'identifier au coefficient directeur de la tangente ;
        \SFT Déterminer une équation de la tangente en un point du graphe d'une fonction trinôme du second degré ;
        \SFT Tracer une tangente.
    \end{enumerate}\[*\]

\chapitre{Probabilités \bsc{ii} : loi binomiale}
    \begin{enumerate}
        \STP Reconnaître des situations relevant de la loi binomiale et en identifier les paramètres ;
        \STP Calculer une probabilité dans le cadre de la loi binomiale à l'aide de la calculatrice ou du tableur ;
        \STP Représenter graphiquement la loi binomiale par un diagramme en bâtons ;
        \STP Déterminer l'espérance de la loi binomiale ;
        \STP Interpréter l'espérance comme valeur moyenne dans le cas d'un grand nombre de répétitions.
    \end{enumerate}\[*\]

\chapitre{Polynômes de degré 3}
    \begin{enumerate}
        \SFT Déterminer l'expression de la fonction dérivée d'une fonction polynôme de degré $3$ ;
        \SFT Dans le cadre d'une résolution de problèmes, utiliser le signe de la fonction dérivée pour déterminer les variations d'une fonction polynôme de degré $3$.
    \end{enumerate}\[*\]\clearpage

\chapitre{\'Echantillonnage}
    \begin{enumerate}
        \STP Déterminer à l'aide de la loi binomiale un intervalle de fluctuation, à environ 95\%, d'une fréquence ;
        \STP Exploiter un tel intervalle pour rejeter ou non une hypothèse sur une proportion.
    \end{enumerate}\[*\]

\begin{center}
{\fontfamily{augie}\fontsize{9}{11}\selectfont
{\Large \pfr{Feuilles automatisées de calcul}}}
\end{center}\bigskip

\noindent\textbf{\'Etude et représentation de séries statistiques, de suites et de fonctions numériques à l'aide d'un tableur ou d'une calculatrice}\par
\begin{itemize}
    \item Choisir la représentation la plus adaptée à une situation donnée : tableau, graphique... ;
    \item Utiliser un adressage absolu ou relatif ;
    \item Mettre en œuvre des fonctions du tableur (mathématiques, logiques, statistiques) en liaison avec les différentes parties du programme ;
    \item Construire un tableau croisé d'effectifs ou de fréquences ; interpréter le tableau obtenu en divisant chaque cellule par la somme de toutes les cellules, ou par la somme des cellules de la même ligne ou colonne.
\end{itemize}\[*\]

\begin{center}
{\fontfamily{augie}\fontsize{9}{11}\selectfont
{\Large \pfr{Algorithmique (objectifs pour le lycée)}}}
\end{center}\bigskip


\noindent\textbf{Instructions élémentaires (affectation, calcul, entrée, sortie)}\par
Les élèves, dans le cadre d'une résolution de problèmes, doivent être capables :
\begin{itemize}
    \item d'écrire une formule permettant un calcul ;
    \item d'écrire un programme calculant et donnant la valeur d'une fonction, ainsi que les instructions d'entrées et sorties nécessaires au traitement.
\end{itemize}

\noindent\textbf{Boucle et itérateur, instruction conditionnelle}\par
Les élèves, dans le cadre d'une résolution de problèmes, doivent être capables :
\begin{itemize}
    \item de programmer un calcul itératif, le nombre d'itérations étant donné ;
    \item de programmer une instruction conditionnelle, un calcul itératif, avec une fin de boucle conditionnelle.
\end{itemize}



\end{document}
