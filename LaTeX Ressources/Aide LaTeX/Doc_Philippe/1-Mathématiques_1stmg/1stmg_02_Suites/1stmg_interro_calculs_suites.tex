\documentclass[10pt,french]{article}

\input preambule_2013
\usepackage{marvosym}
\RegleEntete


\newcommand\competences{
\setcounter{exo}{0}\renewcommand\arraystretch{1.2}
\begin{tabular}{ll} Nom : \\[5pt] Prénom : \end{tabular}
\hfill
\textbf{Note :}
\begin{tabular}{|c|}
\hline
\slashbox{\Huge\bfseries\phantom{10}}{\Huge\bfseries 10}\\
\hline
\end{tabular}\renewcommand\arraystretch{1}
\[***\]
}

\entete{\premiere \stmg}{Calculer les premiers termes \\ d'une suite}{A}
\pieddepage{}{}{}


\begin{document}
\competences

Dans chaque cas, calculer les 4 premiers termes de la suite. \textbf{Détailler précisément les calculs}.

\begin{description}
    \item[Suite 1 :] $\left\{\begin{array}{rcl}
                                        a_0 & = & 1 \\
                                        a_{n +1} & = & -3 a_n + 2
                                \end{array}\right.$
    \item[Suite 2 :] $\left\{\begin{array}{rcl}
                                        b_0 & = & 2 \\
                                        b_{n +1} & = & (e_n)^2  - 1
                                \end{array}\right.$
    \item[Suite 3 :] $c_n = 5n - 6$
    \item[Suite 4 :] $d_n = -n^2 + 2$
    \item[Suite 5 :] $e_n = \dfrac{3}{n + 8}$
\end{description}

\clearpage

%--------------------------------------------------------------------------------------------------------------------------------------------------------------------------
%                           SUJET B
%--------------------------------------------------------------------------------------------------------------------------------------------------------------------------

\entete{\premiere \stmg}{Calculer les premiers termes \\ d'une suite}{B}
\competences

Dans chaque cas, calculer les 4 premiers termes de la suite. \textbf{Détailler précisément les calculs}.

\begin{description}
    \item[Suite 1 :] $a_n = 6n - 5$
    \item[Suite 2 :] $b_n = -n^2 + 3$
    \item[Suite 3 :] $c_n = \dfrac{3}{n + 4}$
    \item[Suite 4 :] $\left\{\begin{array}{rcl}
                                        d_0 & = & 2 \\
                                        d_{n +1} & = & -2 d_n + 3
                                \end{array}\right.$
    \item[Suite 5 :] $\left\{\begin{array}{rcl}
                                        e_0 & = & 0 \\
                                        e_{n +1} & = & (e_n)^2 + 1
                                \end{array}\right.$
\end{description}


\end{document} 