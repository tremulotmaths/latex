\documentclass[10pt,french]{article}

\input preambule_2013
\usepackage{marvosym}
\RegleEntete


\newcommand\competences{
\setcounter{exo}{0}
\begin{tabular}{ll} Nom : \\[5pt] Prénom : \end{tabular}
\hfill
\textbf{Note :}\renewcommand\arraystretch{2.3}
\begin{tabular}{|c|}
\hline
\slashbox{\Huge\bfseries\phantom{10}}{\Huge\bfseries 10}\\
\hline
\end{tabular}\renewcommand\arraystretch{1}
\[***\]
}

\entete{\premiere \stmg}{\Coef multiplicateur \\ \'Evolutions successives et réciproques}{A}
\pieddepage{}{}{}


\begin{document}
\competences

\exo Un particulier achète une maison en janvier \np{2011}. À la fin de l'année \np{2011}, sa valeur a augmenté de $4\%$. À la fin de l'année \np{2012}, sa valeur a diminué de $2\%$ par rapport à la fin de l'année \np{2011}.
\begin{enumerate}
    \item En utilisant le \coef multiplicateur, calculer le prix de la maison à la fin de l'année \np{2011}.
    \item En utilisant les évolutions successives, calculer le taux d'évolution du prix de la maison de janvier \np{2011} à janvier \np{2013}.
\end{enumerate}

À la fin de l'année \np{2013}, le prix de la maison a diminué de $3\%$.

\begin{enumerate}[resume]
    \item Quel doit être le taux d'évolution réciproque à la fin de l'année \np{2014} pour que la maison retrouve le prix précédent.
\end{enumerate}
\[*\]

\exo Compléter le tableau suivant en écrivant les calculs utilisés

\begin{center}
\renewcommand\arraystretch{2.5}
    \begin{tabularx}{\textwidth}{|m{4.25cm}|X|}
    \hline
        Calcul à effectuer & $y_1 = 112$ et $y_2 = 118$ \\
    \hline
        Taux d'évolution de $y_1$ à $y_2$ (arrondi à $0,1\%$ près) & \\
    \hline
        \Coef multiplicateur de $y_1$ à $y_2$ & \\
    \hline
        Taux d'évolution réciproque de $y_2$ à $y_1$ & \\
    \hline
    \end{tabularx}
\renewcommand\arraystretch{1}
\end{center}

\clearpage

%--------------------------------------------------------------------------------------------------------------------------------------------------------------------------
%                           SUJET B
%--------------------------------------------------------------------------------------------------------------------------------------------------------------------------

\entete{\premiere \stmg}{\Coef multiplicateur \\ \'Evolutions successives et réciproques}{B}
\competences

\exo Un particulier achète une maison en janvier \np{2011}. À la fin de l'année \np{2011}, sa valeur a augmenté de $2\%$. À la fin de l'année \np{2012}, sa valeur a diminué de $4\%$ par rapport à la fin de l'année \np{2011}.
\begin{enumerate}
    \item En utilisant le \coef multiplicateur, calculer le prix de la maison à la fin de l'année \np{2011}.
    \item En utilisant les évolutions successives, calculer le taux d'évolution du prix de la maison de janvier \np{2011} à janvier \np{2013}.
\end{enumerate}

À la fin de l'année \np{2013}, le prix de la maison a diminué de $5\%$.

\begin{enumerate}[resume]
    \item Quel doit être le taux d'évolution réciproque à la fin de l'année \np{2014} pour que la maison retrouve le prix précédent.
\end{enumerate}
\[*\]

\exo Compléter le tableau suivant en écrivant les calculs utilisés

\begin{center}
\renewcommand\arraystretch{2.5}
    \begin{tabularx}{\textwidth}{|m{4.25cm}|X|}
    \hline
        Calcul à effectuer & $y_1 = \np{1500}$ et $y_2 = \np{1350}$ \\
    \hline
        Taux d'évolution de $y_1$ à $y_2$ (arrondi à $0,1\%$ près) & \\
    \hline
        \Coef multiplicateur de $y_1$ à $y_2$ & \\
    \hline
        Taux d'évolution réciproque de $y_2$ à $y_1$ & \\
    \hline
    \end{tabularx}
\renewcommand\arraystretch{1}
\end{center}


\end{document} 