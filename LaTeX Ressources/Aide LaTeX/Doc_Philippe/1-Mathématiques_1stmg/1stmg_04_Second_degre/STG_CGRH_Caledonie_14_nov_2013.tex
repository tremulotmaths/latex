\documentclass[10pt]{article}
\usepackage[T1]{fontenc}
\usepackage[utf8]{inputenc}
\usepackage{fourier}
\usepackage[scaled=0.875]{helvet}
\renewcommand{\ttdefault}{lmtt}
\usepackage{amsmath,amssymb,makeidx}
\usepackage{fancybox}
\usepackage[normalem]{ulem}
\usepackage{pifont}
\usepackage{multirow}
\usepackage{tabularx}
\usepackage{textcomp}
\newcommand{\euro}{\eurologo{}}
%Tapuscrit : Denis Vergès
%Sujet aimablement fourni par Clotilde Rouchon et Virginie Picavet
\usepackage{pst-plot,pst-tree}
\newcommand{\R}{\mathbb{R}}
\newcommand{\N}{\mathbb{N}}
\newcommand{\D}{\mathbb{D}}
\newcommand{\Z}{\mathbb{Z}}
\newcommand{\Q}{\mathbb{Q}}
\newcommand{\C}{\mathbb{C}}
\usepackage{geometry,atbegshi}
\geometry{a4paper, height = 25.5cm,hmargin=2.5cm,marginparwidth=2cm,headheight=20pt,headsep=16.5pt,bottom=2cm,footskip=30pt,footnotesep=30pt}
\newcommand{\vect}[1]{\mathchoice%
{\overrightarrow{\displaystyle\mathstrut#1\,\,}}%
{\overrightarrow{\textstyle\mathstrut#1\,\,}}%
{\overrightarrow{\scriptstyle\mathstrut#1\,\,}}%
{\overrightarrow{\scriptscriptstyle\mathstrut#1\,\,}}}
\renewcommand{\theenumi}{\textbf{\arabic{enumi}}}
\renewcommand{\labelenumi}{\textbf{\theenumi.}}
\renewcommand{\theenumii}{\textbf{\alph{enumii}}}
\renewcommand{\labelenumii}{\textbf{\theenumii.}}
\def\Oij{$\left(\text{O},~\vect{\imath},~\vect{\jmath}\right)$}
\def\Oijk{$\left(\text{O},~\vect{\imath},~ \vect{\jmath},~ \vect{k}\right)$}
\def\Ouv{$\left(\text{O},~\vect{u},~\vect{v}\right)$}
\setlength{\voffset}{-1,5cm}
\usepackage{fancyhdr}
\usepackage[frenchb]{babel}
\usepackage[np]{numprint}
\begin{document}
\footnotesize

\setlength\parindent{0mm}
\pagestyle{fancy}
\thispagestyle{empty}
\begin{center}{\large\decofourleft~\textbf{Baccalauréat Nouvelle-Calédonie} (14 novembre 2013)~\decofourright}
\end{center}

\textbf{\textsc{Exercice 3} \hfill 8 points}

\medskip

Dans un lycée un groupe d’élèves participant à un club de presse a réalisé un journal et décidé de l’imprimer pour le vendre.

Les coûts d’impression en euros en fonction du nombre $x$  de journaux sont estimés à l’aide de la fonction $C$ définie par :
\[C(x) =  0,005x^2 - 0,6x +  200\quad  \text{pour $x$ élément de l’intervalle } [0~;~500].\]

La courbe représentative de la fonction $C$ est tracée sur l’annexe.


Pour soutenir l’action des élèves du club de presse, le foyer leur donne une subvention de $150$~\euro. On décide alors de fixer le prix de vente du journal à $1,20$~\euro.

En vendant $x$  journaux, les revenus en euros seront donnés  par la fonction $R$ définie par :

\[R(x) =  150 + 1,2x\quad  \text{pour $x$ élément de l’intervalle } [0~;~500].\]

\begin{enumerate}
\item Calculer les revenus correspondant à la vente de $250$~ journaux.

Tracer sur l'annexe la représentation graphique de la fonction $R$.
\item  À l'aide du graphique déterminer l'intervalle dans lequel doit se trouver le nombre de journaux que le club presse du lycée réalise un bénéfice.
\item On désigne par $B$ la fonction estimant le bénéfice en euros réalisé par le club presse du lycée pour la vente de $x$  journaux. Montrer que la fonction est définie sur [0~;~500] par :

\[B(x) =  - 0,005x^2 + 1,8x -  50.\]

\item \'Etablir le tableau de variation de la fonction $B$ sur l'intervalle [0~;~500]
\item
	\begin{enumerate}
		\item Déterminer le nombre de journaux à vendre pour que le bénéfice soit maximal.
		\item Calculer ce bénéfice.
	\end{enumerate}
\end{enumerate}


\begin{center}
\psset{unit=0.025cm}
\begin{pspicture*}(-35,-25)(425,650)
\psaxes[linewidth=1.5pt,Dx=25,Dy=25]{->}(0,0)(-24,-24)(425,650)
\multido{\n=0.0+12.5}{35}{\psline[linewidth=0.2pt,linecolor=orange](\n,0)(\n,650)}
\multido{\n=0.0+12.5}{53}{\psline[linewidth=0.2pt,linecolor=orange](0,\n)(425,\n)}
\psplot[plotpoints=5000,linewidth=1.25pt,linecolor=blue]{0}{375}{x dup mul 0.005 mul 0.6 x mul sub 200 add}
\uput[d](420,0){$x$}
\uput[l](0,645){$y$}\uput[dl](0,0){O}
\end{pspicture*}
\end{center}
\end{document} 