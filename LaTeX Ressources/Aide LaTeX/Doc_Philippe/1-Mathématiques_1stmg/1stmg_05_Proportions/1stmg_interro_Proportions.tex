\documentclass[10pt,french]{article}

\input preambule_2013
\RegleEntete


\newcommand\competences{
\setcounter{exo}{0}\renewcommand\arraystretch{1.2}
\begin{tabular}{ll} Nom : \\[5pt] Prénom : \end{tabular}
\hfill
\textbf{Note :}
\begin{tabular}{|c|}
\hline
\slashbox{\Huge\bfseries\phantom{10}}{\Huge\bfseries 10}\\
\hline
\end{tabular}\renewcommand\arraystretch{1}
\[***\]
}

\entete{\premiere \stmg}{Calcul de proportion}{A}
\pieddepage{}{}{}


\begin{document}
\competences

\exo Une enquête sur la propreté de leur ville a été menée auprès de $500$ personnes réparties de la manière suivante :
\begin{itemize}
    \item $30\%$ des personnes interrogées ont moins de $35$ ans ;
    \item $40\%$ ont entre $35$ et $50$ ans.
\end{itemize}

À la question << êtes-vous content de la propreté de votre ville  ? >>,
    \begin{itemize}
        \item $300$ des personnes interrogées ont répondu oui ; 
        \item $120$ personnes de moins de $35$ ans ont répondu oui ;
        \item $70\%$ des personnes de plus de $50$ ans ont répondu non.
    \end{itemize}

\begin{enumerate}
    \item À l'aide des informations données, compléter le tableau suivant :
    \begin{center}
    \renewcommand\arraystretch{1.25}
        \begin{tabular}{|c|c|c|c|c|}
            \hline
                & moins de $35$ ans & entre $35$ et $50$ ans & plus de $50$ ans & Total \\
            \hline
                Réponse Oui & & & & \\
            \hline
                Réponse Non & & && \\
            \hline
                Total & & & & \\
            \hline
        \end{tabular}
    \renewcommand\arraystretch{1}
    \end{center}
    
    \item Donner sous forme fractionnaire puis sous forme d'un pourcentage :
        \begin{enumerate}
            \item la proportion $p_1$ de personnes qui ont répondu Non ;
            \item la proportion $p_2$ de personnes qui ont entre $35$ et $50$ ans \textbf{et} qui ont répondu Non ;
            \item  la proportion $p_3$ de personnes qui ont \textbf{moins} de $50$ ans \textbf{et} qui ont répondu Oui.
        \end{enumerate}
    \item Quelle est la proportion $p_4$ de personnes ayant moins de $35$ ans parmi toutes les personnes qui ont répondu Oui ?
    Donner le résultat sous forme fractionnaire puis sous forme d'un pourcentage.
    \item On appelle $N$ la population des personnes qui ont répondu Non et $T$ la population des personnes âgées entre $35$ et $50$ ans.\par
    En détaillant la démarche, calculer $p_{(T \cup N)}$. Donner le résultat en pourcentage.
\end{enumerate}\[*\]

\exo Dans un lycée, la proportion des élèves de première dans l'ensemble des élèves de ce lycée est $36\%$ et la proportion des élèves de première \stmg{} dans l'ensemble des \textbf{premières} est $15,2\%$.\par
Calculer la valeur décimale exacte de la proportion des élèves de première \stmg{} dans l'ensemble du \textbf{lycée}. Donner ensuite la réponse en pourcentage.


\clearpage

%--------------------------------------------------------------------------------------------------------------------------------------------------------------------------
%                           SUJET B
%--------------------------------------------------------------------------------------------------------------------------------------------------------------------------

\entete{\premiere \stmg}{Calcul de proportion}{B}
\competences


\exo Une enquête sur la propreté de leur ville a été menée auprès de $500$ personnes réparties de la manière suivante :
\begin{itemize}
    \item $40\%$ des personnes interrogées ont moins de $35$ ans ;
    \item $30\%$ ont entre $35$ et $50$ ans.
\end{itemize}

À la question << êtes-vous content de la propreté de votre ville  ? >>,
    \begin{itemize}
        \item $210$ des personnes interrogées ont répondu oui ;
        \item $80$ personnes de moins de $35$ ans ont répondu oui ;
        \item $60\%$ des personnes de plus de $50$ ans ont répondu non.
    \end{itemize}

\begin{enumerate}
    \item À l'aide des informations données, compléter le tableau suivant :
    \begin{center}
    \renewcommand\arraystretch{1.25}
        \begin{tabular}{|c|c|c|c|c|}
            \hline
                & moins de $35$ ans & entre $35$ et $50$ ans & plus de $50$ ans & Total\\
            \hline
                Réponse Oui & & & & \\
            \hline
                Réponse Non & & & & \\
            \hline
                Total & & & & \\
            \hline
        \end{tabular}
    \renewcommand\arraystretch{1}
    \end{center}

    \item Donner sous forme fractionnaire puis sous forme d'un pourcentage :
        \begin{enumerate}
            \item la proportion $p_1$ de personnes qui ont répondu Non ;
            \item la proportion $p_2$ de personnes qui ont entre $35$ et $50$ ans \textbf{et} qui ont répondu Non ;
            \item  la proportion $p_3$ de personnes qui ont \textbf{moins} de $50$ ans \textbf{et} qui ont répondu Oui.
        \end{enumerate}
    \item Quelle est la proportion $p_4$ de personnes ayant moins de $35$ ans parmi toutes les personnes qui ont répondu Oui ?
    Donner le résultat sous forme fractionnaire puis sous forme d'un pourcentage.
    \item On appelle $N$ la population des personnes qui ont répondu Non et $T$ la population des personnes âgées entre $35$ et $50$ ans.\par
    En détaillant la démarche, calculer $p_{(T \cup N)}$. Donner le résultat en pourcentage.
\end{enumerate}\[*\]

\exo Dans un lycée, la proportion des élèves de première dans l'ensemble des élèves de ce lycée est $38\%$ et la proportion des élèves de première \stmg{} dans l'ensemble des \textbf{premières} est $13,2\%$.\par
Calculer la valeur décimale exacte de la proportion des élèves de première \stmg{} dans l'ensemble du \textbf{lycée}. Donner ensuite la réponse en pourcentage.


\end{document} 