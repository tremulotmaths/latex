\documentclass[10pt,openright,twoside,french]{book}

\usepackage{marvosym}
\input philippe2013
\input philippe2013_activites

\pagestyle{empty}

\begin{document}

\TitreAlgo{i.1}{Algorithmes de calcul \par de variations}

On considère l'algorithme suivant :

\begin{center}
\small
    \psframebox{
    \parbox{0.5\linewidth}{
        \textbf{Variables}

            \quad $a$ : un nombre réel

            \quad $b$ : un nombre réel

            \quad $c$ : un nombre réel

            \quad $d$ : un nombre réel

        \textbf{Entrée}

            \quad Saisir $a$

            \quad Saisir $b$

        \textbf{Traitement}

            \quad $c$ prend la valeur $b - a$

            \quad $d$ prend la valeur \ldots

        \textbf{Sortie}

            \quad Afficher $c$
            
            \quad Afficher $d$
    }}
\end{center}

\begin{enumerate}
    \item À quoi correspond la valeur est affectée à la variable $c$ ?
    \item On souhaite affectée à la variable $d$ le taux d'évolution entre $a$ et $b$. Compléter alors les pointillés de l'algorithme.
\end{enumerate}

\end{document} 