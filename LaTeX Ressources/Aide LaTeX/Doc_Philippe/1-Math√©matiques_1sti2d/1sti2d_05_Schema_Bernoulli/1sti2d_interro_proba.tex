\documentclass[10pt,french]{book}

\input philippe2013
\RegleEntete


\newcommand\competences{
\setcounter{exo}{0}
\begin{tabular}{ll} Nom : \\[5pt] Prénom : \end{tabular}
\hfill
\textbf{Note :}\renewcommand\arraystretch{2.3}
\begin{tabularx}{0.18\linewidth}{|X|}
\hline
\slashbox{\Huge\bfseries\phantom{10}}{\Huge\bfseries 10}\\
\hline
\end{tabularx}\renewcommand\arraystretch{1}\medskip
}

\entete{\premiere \sti}{Probabilités}{A}
\pieddepage{}{}{}


\begin{document}
\competences

\textbf{Tous les résultats seront arrondis à $10^{-3}$ près.}\[***\]

\exo Une fabrique artisanale de jouets en bois vérifie la qualité de sa production avant sa commercialisation.\par
Chaque jouet produit par l'entreprise est soumis à deux contrôles : d'une part, l'aspect du jouet est examiné afin de vérifier qu'il ne présente pas de défaut de finition, d'autre part sa solidité est testée.\par
Il s'avère, à la suite d'un grand nombre de vérifications, que :
\begin{itemize}
    \item $92\%$ des jouets sont sans défaut de finition ;
    \item parmi les jouets qui sont sans défaut de finition, $95\%$ réussissent le test de solidité ;
    \item parmi les jouets qui ont des défauts de finition, $25\%$ ne réussissent pas le test de solidité.
\end{itemize}\medskip

On considère les événements suivants :
\begin{itemize}
    \item $D$ : << le jouet présente un défaut de finition >> ;
    \item $S$ : << le jouet a réussi le test de solidité >> ;
    \item $A$ : << le jouet ne présente pas de défaut de finition \textbf{et} a réussi le test de solidité >> ;
    \item $B$ : << le jouet a réussi le test de solidité avec ou sans défaut de finition >>.
\end{itemize}\medskip

\begin{enumerate}
    \item Compléter l'arbre pondéré des possibles ci-dessous :
\end{enumerate}
    
\tikzstyle{level 1}=[level distance=2.5cm, sibling distance=-3cm]
\tikzstyle{level 2}=[level distance=3.5cm, sibling distance=-1.5cm]
\tikzstyle{proba} = [rectangle,fill=white,pos=0.5,inner sep=2pt]
\tikzstyle{issue} = [circle,fill=black!20,draw=none,circular drop shadow]

\begin{center}    
    \begin{tikzpicture}[grow=right]
\node at (-3,0){}
    child{
        node[issue]{\color{black!20}$F$}
            child{node[issue](N1){$S$}
                        edge from parent
                        node[proba] {\color{white} $0,0005$}
                    }
            child{node[issue](N2){\color{black!20}$\overline S$}
                        edge from parent
                        node[proba] {\color{white} $0,0005$}
                    }
    	   edge from parent
    node[proba] {\color{white} $0,0005$}
    }
    child{
        node[issue]{$D$}
            child{node[issue](N1){\color{black!20}$S$}
                        edge from parent
                        node[proba] {\color{white} $0,0005$}
                    }
            child{node[issue](N2){\color{black!20}$\overline S$}
                        edge from parent
                        node[proba] {$0,25$}
                    }
    	   edge from parent
    node[proba] {\color{white} $0,0005$}
    };
    \end{tikzpicture}
\end{center}

\begin{enumerate}[resume]
    \item Calculer $p(A)$ et $p(B)$.
\end{enumerate}\[*\]

\exo Pour chacun de ses tirs, un tireur à l'arc débutant atteint la cible avec une probabilité de $0,83$. On s'intéresse de savoir si le tireur va atteindre la cible.\par
Trois tireurs se présentent devant leur cible. Ils ne réalisent qu'un tir chacun. La probabilité qu'un tireur atteigne la cible ne dépend pas des autres tireurs.

\begin{enumerate}
    \item Expliquer précisément pourquoi il s'agit d'un schéma de Bernoulli.
    \item On appelle $A$ l'événement << tous les tireurs atteignent la cible >>. Calculer $p(A)$.
    \item On appelle $B$ l'événement << un seul tireur exactement atteint la cible >>. Calculer $p(B)$.
    \item On appelle $C$ l'événement << au moins un tireur n'atteint pas la cible >>. Calculer $p(C)$.
\end{enumerate}

\clearpage

%--------------------------------------------------------------------------------------------------------------------------------------------------------------------------
%                           SUJET B
%--------------------------------------------------------------------------------------------------------------------------------------------------------------------------

\entete{\premiere \sti}{Probabilités}{B}
\competences

\textbf{Tous les résultats seront arrondis à $10^{-3}$ près.}\[***\]

\exo Une fabrique artisanale de jouets en bois vérifie la qualité de sa production avant sa commercialisation.\par
Chaque jouet produit par l'entreprise est soumis à deux contrôles : d'une part, l'aspect du jouet est examiné afin de vérifier qu'il ne présente pas de défaut de finition, d'autre part sa solidité est testée.\par
Il s'avère, à la suite d'un grand nombre de vérifications, que :
\begin{itemize}
    \item $92\%$ des jouets sont sans défaut de finition ;
    \item parmi les jouets qui sont sans défaut de finition, $95\%$ réussissent le test de solidité ;
    \item parmi les jouets qui ont des défauts de finition, $25\%$ ne réussissent pas le test de solidité.
\end{itemize}\medskip

On considère les événements suivants :
\begin{itemize}
    \item $F$ : << le jouet présente un défaut de finition >> ;
    \item $S$ : << le jouet a réussi le test de solidité >> ;
    \item $A$ : << le jouet ne présente pas de défaut de finition \textbf{et} a réussi le test de solidité >> ;
    \item $B$ : << le jouet a réussi le test de solidité avec ou sans défaut de finition >>.
\end{itemize}\medskip

\begin{enumerate}
    \item Compléter l'arbre pondéré des possibles ci-dessous :
\end{enumerate}

\tikzstyle{level 1}=[level distance=2.5cm, sibling distance=-3cm]
\tikzstyle{level 2}=[level distance=3.5cm, sibling distance=-1.5cm]
\tikzstyle{proba} = [rectangle,fill=white,pos=0.5,inner sep=2pt]
\tikzstyle{issue} = [circle,fill=black!20,draw=none,circular drop shadow]

\begin{center}
    \begin{tikzpicture}[grow=right]
\node at (-3,0){}
    child{
        node[issue]{$F$}
            child{node[issue](N1){\color{black!20}$S$}
                        edge from parent
                        node[proba] {$0,25$}
                    }
            child{node[issue](N2){\color{black!20}$\overline S$}
                        edge from parent
                        node[proba] {\color{white} $0,0005$}
                    }
    	   edge from parent
    node[proba] {\color{white} $0,0005$}
    }
    child{
        node[issue]{\color{black!20} $F$}
            child{node[issue](N1){\color{black!20}$S$}
                        edge from parent
                        node[proba] {\color{white} $0,0005$}
                    }
            child{node[issue](N2){$\overline S$}
                        edge from parent
                        node[proba] {\color{white} $0,0005$}
                    }
    	   edge from parent
    node[proba] {\color{white} $0,0005$}
    };
    \end{tikzpicture}
\end{center}

\begin{enumerate}[resume]
    \item Calculer $p(A)$ et $p(B)$.
\end{enumerate}\[*\]

\exo Pour chacun de ses tirs, un tireur à l'arc débutant atteint la cible avec une probabilité de $0,83$. On s'intéresse de savoir si le tireur va atteindre la cible.\par
Trois tireurs se présentent devant leur cible. Ils ne réalisent qu'un tir chacun. La probabilité qu'un tireur atteigne la cible ne dépend pas des autres tireurs.

\begin{enumerate}
    \item Expliquer précisément pourquoi il s'agit d'un schéma de Bernoulli.
    \item On appelle $A$ l'événement << aucun tireur n'a atteint la cible >>. Calculer $p(A)$.
    \item On appelle $B$ l'événement << un seul tireur exactement a raté la cible >>. Calculer $p(B)$.
    \item On appelle $C$ l'événement << au moins un tireur atteint la cible >>. Calculer $p(C)$.
\end{enumerate}

\end{document} 