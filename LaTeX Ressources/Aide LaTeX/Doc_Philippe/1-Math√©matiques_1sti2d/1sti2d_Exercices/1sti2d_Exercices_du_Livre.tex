\documentclass[10pt,openright,twoside,french]{book}
\input philippe2013
\pagestyle{empty}

\begin{document}
\begin{center}
\psframebox[shadow=true,shadowcolor=gray!75,shadowsize=3pt,%
framearc=0.3,%
fillstyle=gradient,gradmidpoint=0.8,gradangle=20,gradbegin=red!60!yellow!40,gradend= white]{%
\parbox{0.5\linewidth}{%
\begin{center}
\Large\bfseries
Exercices à faire\par avec le livre
\end{center}}}
\end{center}\medskip

\section*{Chapitre 1 : Statistiques descriptives}
\subsubsection*{Page 256}
TP 1 : \textit{utilisation de la calculatrice}

\subsubsection{Page 260}
Exercice 3

\subsubsection{Pages 262-263}
Exercice 12 : \textit{regroupement par classes, utilisation d'une équation de droite pour déterminer une fréquence (interpolation affine)}\par
Exercice 13 : \textit{semblable à l'exercice 12 - à faire en autonomie}\par
Exercices 14 et 15 : \textit{calculs de moyenne et d'écart-type}

\subsubsection{Page 266}
Exercice 23 : \textit{utilisation du couple (moyenne ; écart-type) pour comparer plusieurs séries statistiques}\par
Exercice 25 : \textit{un peu de tout}

\subsubsection{Page 269}
Exercice 35 : \textit{lecture d'un tableur, déterminer des formules}


\end{document} 