\documentclass[12pt,openright,twoside,french]{book}

\input philippe2013
\input philippe2013_activites
\pagestyle{empty}

\begin{document}

\TitreExo{\bsc{xiii}.1}{\'Echantillonnage \\ Prise de décision}\bigskip

\exo Une usine fabrique en grande quantité des boucles de ceintures. L'expérience montre qu'en fabrication normale, $5\%$ des boucles sont défectueuses.\par
On prélève dans la production journalière de l'usine un lot de $80$ boucles ; le prélèvement est assimilé à un tirage successif avec remise.\par
Soit $X$ la variable aléatoire qui, à tout prélèvement ainsi défini, associe le nombre de boucles défectueuses.

\begin{enumerate}
    \item Calculer $p(X = 0)$, $p(X = 10)$, $p(X > 1)$. Interpréter le dernier résultat.
    \item À l'aide de la calculatrice, déterminer l'intervalle de fluctuation à $95\%$ de la fréquence des boucles défectueuses.
\end{enumerate}\[*\]

\exo Madame Paulette, maire d'une ville, est confiante et affirme que $55\%$ des électeurs veulent voter pour elle.\par
Elle commande alors un sondage à un institut ; ce sondage est effectué auprès de $200$ personnes prises au hasard parmi les électeurs. Parmi elles, $95$ personnes déclarent qu'elles voteront pour madame Paulette.
\begin{enumerate}
    \item Déterminer l'intervalle de fluctuation de la fréquence des personnes qui souhaitent voter pour madame Paulette.
    \item Peut-on considérer, au seuil des $95\%$, que l'affirmation du maire est exacte ?
\end{enumerate}\[*\]

\exo L'entreprise $O$ commercialise des bouteilles d'eau minérale. Elle affirme que seulement $5\%$ des bouteilles qu'elle vend ont un taux de nitrate supérieur à $10~mg/L$.\par
On prélève au hasard un échantillon de $50$ bouteilles ; la production est assez importante pour que l'on puisse considérer qu'il s'agit d'un tirage avec remise.\par
On fait l'hypothèse que la fréquence des bouteilles d'eau de la production dont le taux de nitrate est supérieur à $10~mg/L$ est $0,05$ et on considère la variable aléatoire $X$ indiquant le nombre de bouteilles de l'échantillon dont le taux de nitrate est supérieur à $10~mg/L$.

\begin{enumerate}
    \item Déterminer l'intervalle de fluctuation de la fréquence des bouteilles d'eau de la production dont le taux de nitrate est supérieur à $10~mg/L$.
    \item Dans l'échantillon des $50$ bouteilles, $7$ bouteilles ont un taux de nitrate supérieur à $10~mg/L$.\par
    Que peut-on conclure ?
\end{enumerate}
\end{document} 