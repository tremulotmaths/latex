\documentclass[10pt,openright,twoside,french]{book}

\input philippe2013
\input philippe2013_activites

\pagestyle{empty}

\begin{document}

\small\vspace*{-1.75cm}
\TitreExo{\bsc{v}.1}{Calculs de probabilités}

\exo On tire au hasard une carte d'un jeu de 32 cartes. On considère les événements suivants :\par
$A$ : \og la carte obtenue est un carreau~\fg et $B$ : \og la carte obtenue est une figure~\fg.\par
On appelle << figure >> un Roi, une Dame ou un Valet.\smallskip

On donnera les résultats sous forme d'une fraction irréductible.\smallskip

\begin{enumerate}
    \item Calculer les probabilités $p(A)$ et $p(B)$.
    \item Définir, à l'aide d'une phrase en français, l'événement $\overline A$ et calculer $p\left(\overline A\right)$.
    \item Définir, à l'aide d'une phrase en français, l'événement $A \cap B$ et calculer $p\left(A \cap B\right)$.
    \item Les événements $A$ et $B$ sont-ils incompatibles ? Pourquoi ?
    \item Soit $C$ l'événement défini par << la carte obtenue est un carreau ou une figure >>.\par Calculer $p(C)$.
\end{enumerate}

\[*\]

\exo Une enquête a été effectuée auprès de $750$ jeunes titulaires d'un baccalauréat d'enseignement général ou technique 3 ans après l'obtention de leur diplôme :
\begin{itemize}
    \item $24\%$ sont titulaires d'un bac \bsc{s.t.i.} ;
    \item le tiers des $750$ jeunes interrogés ont un emploi ;
    \item $380$ continuent leurs études ; parmi eux, $20\%$ sont titulaires d'un bac \bsc{s.t.i.} ;
    \item $90\%$ de ceux qui sont au chômage sont titulaires d'un bac autre que \bsc{s.t.i.}
\end{itemize}

\begin{enumerate}
    \item Compléter le tableau des effectifs suivants :
        \begin{center}
        \renewcommand\arraystretch{1.5}
            \begin{tabular}{|c|*{4}{>{\centering\arraybackslash}m{2cm}|}}
                \hline
                    \backslashbox{Nature du bac}{Situation} & Ont un emploi & Continuent leurs études & Sont au chômage & Total \\
                \hline
                    Bac \bsc{s.t.i.} & & & & \\
                \hline
                    Autre bac & & & & \\
                \hline
                    Total & & $380$ & & $750$ \\
                \hline
            \end{tabular}
        \end{center}
        
    \item Dans cette questions, les résultats seront donnés sous forme de fractions irréductibles.\par
    On choisit un jeune au hasard parmi les $750$ interrogés.
    \begin{enumerate}
        \item Calculer les probabilités des événements suivants :\par
        $C$ : << le jeune a un bac \bsc{s.t.i.} >> et $D$ : << le jeune continue ses études >>.
        \item Définir par une phrase l'événement $C \cap D$ et calculer $p(C \cap D)$.
        \item Définir par une phrase l'événement $C \cup D$ et calculer $p(C \cup D)$.
        \item Le jeune choisi au hasard est titulaire d'un bac \bsc{s.t.i.}\par
        Quelle est la probabilité $p$ qu'il ait un emploi ?
    \end{enumerate}
\end{enumerate}

\[*\]

\exo Une entreprise fabrique des cahiers qui peuvent présenter deux défauts $D_1$ et $D_2$.\par Après une étude sur ces défauts, il apparaît que $9\%$ des cahiers présentent le défaut $D_1$, $7\%$ des cahiers présentent le défaut $D_2$ et $4\%$ des cahiers présentent les deux défauts simultanément.\par
On choisit au hasard un cahier dans la production de l'entreprise et on considère les événements suivants :\par
$E$ : << le cahier ne présente pas le défaut $D_1$ >> et $F$ : << le cahier ne présente pas le défaut $D_2$ >>.
\begin{enumerate}
    \item Que représente l'événement $E \cup F$ ? Déterminer sa probabilité.
    \item Compléter le tableau suivant :
        \begin{center}
        \renewcommand\arraystretch{2}
            \begin{tabular}{|c|*{2}{>{\centering\arraybackslash}m{3cm}|}c|}
                \cline{2-4}
                     \multicolumn{1}{c|}{} & Cahier présentant le défaut $D_1$ & Cahier ne présentant pas le défaut $D_1$ & Total \\
                \hline
                    Cahier présentant le défaut $D_2$ & & & \\
                \hline
                    Cahier ne présentant pas le défaut $D_2$ & & & \\
                \hline
                    Total & $9\%$ & & $100\%$ \\
                \hline
            \end{tabular}
        \end{center}
    \item Donner les probabilités suivantes sous forme de fractions irréductibles :
    \[p\left(\overline E\right) \qq p\left(\overline E \cap F\right) \qetq p\left(\overline E \cap \overline F\right)\]
\end{enumerate}


\end{document} 