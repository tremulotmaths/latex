\documentclass[12pt,openright,twoside,french]{book}

\input philippe2013
\input philippe2013_activites
\pagestyle{empty}

\begin{document}

\TitreExo{\bsc{xi}.1}{Dérivation \\ \'Etude de fonctions}

\exo Soit la fonction définie sur l'intervalle $I = \intervalleff{-6}{6}$ par :
\[f(x) = \dfrac{2(x+3)^2}{x^2 + 3}.\]
$\calig C_f$ est la courbe représentative de $f$ dans un repère orthonormal \Oij{} d'unité $1~cm$.
\begin{enumerate}
    \item Les points $A$, $B$, $C$, $D$ et $E$ sont cinq points appartenant à la courbe $\calig C_f$ d'abscisses respectives :
        \[x_A = -6 \qq x_B = -3 \qq x_C = -1 \qq x_D = 1 \qq x_E = 6.\]
        Déterminer les ordonnées de ces cinq points.
    \item Déterminer les racines puis le tableau de signes du polynôme : $-12x^2 - 24x + 36$.
    \item Calculer l'expression de $f'$ sur $I$.
    \item Dresser le tableau de signes de $f'$ et en déduire les variations de $f$.
    \item Déterminer l'équation de la tangente $T$ à $\calig C_f$ au point $B$.
    \item Tracer le repère \Oij et y placer les cinq points $A$, $B$, $C$, $D$ et $E$. Y tracer la tangente $T$ et la courbe $\calig C_f$.
\end{enumerate}\[*\]

\exo On désire fabriquer une boîte fermée de la forme d'un parallélépipède rectangle à base carrée.
Le volume $\calig V$ de la boîte doit être égal à $200~cm^3$.\par
Le matériau utilisé pour le fond et le dessus coûte \EUR{$0,16$} le $cm^2$ alors que celui utilisé pour les côtés revient à \EUR{$0,1$} le $cm^2$.\par
On se propose de calculer les dimensions de la boîte afin que son coût de fabrication soit minimal.\par
On désigne par $x$ le côté de la base en $cm$ et par $h$ la hauteur de la boîte en $cm$.

\begin{enumerate}
    \item Faire un schéma représentant la situation.
    \item \'Ecrire une relation existant entre $\calig V$, $x$ et $h$.
    \item Déterminer l'expression $h(x)$ qui donne la hauteur de la boîte en fonction de $x$.
    \item On appelle $P$ la fonction qui donne le prix de fabrication de la boîte en fonction de $x > 0$.\par
    Démontrer que $P(x) = \dfrac{80}{x} + 0,32 x^2$.
    \item Déterminer la dérivée de $P$ pour $x > 0$.
    \item En déduire les dimensions qui permettent d'obtenir une boîte dont le prix de construction est minimal.\par
    Quel est ce prix de construction ?
\end{enumerate}

\end{document} 