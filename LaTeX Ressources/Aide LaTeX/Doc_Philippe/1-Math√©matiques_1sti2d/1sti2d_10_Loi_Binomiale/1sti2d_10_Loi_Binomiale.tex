\documentclass[10pt,openright,twoside,french]{book}

\input philippe2013
\input philippe2013_cours
\input philippe2013_sections
\input philippe2013_chapitre
\renewcommand\PartProgramme{Stats/Probas}
\renewcommand\MaCouleur{Purple}

\pieddepage{}{%
\begin{tikzpicture}[scale=0.65]
\shadedraw [top color=white, bottom color=\MaCouleur, draw=\MaCouleur]
[l-system={Sierpinski triangle, step=1pt, angle=60, axiom=F, order=6.5}]
lindenmayer system -- cycle;
\draw (30:0.65cm) node {\bfseries\textcolor{black}{\thepage}};
\end{tikzpicture}%
}{}


\setcounter{chapter}{9}
\begin{document}
\chapter[Loi binomiale]{Probabilités \bsc{ii}\\ Loi binomiale}\label{loi_binomiale}

\section{Compter le nombre de succès dans un schéma de Bernoulli}
On considère un schéma de Bernoulli de paramètres $n$ et $p$ et de succès $S$.\medskip

\subsection{C{\oe}fficients binomiaux}

\begin{Defi}
    Soit $n$ un entier naturel et $k$ un entier naturel compris entre $0$ et $n$.\par
    On considère une expérience aléatoire constituée de la répétition de $n$ épreuves de Bernoulli identiques et indépendantes, représentée par un arbre.\par
    Le \ipt{c{\oe}fficient binomial} noté $\binom n k$ (lire << $k$ parmi $n$ >> ) est le nombre de chemins de l'arbre réalisant k succès pour les $n$ répétitions de l'épreuve.
\end{Defi}

\begin{Exemple}[s]
    Certains \coef binomiaux sont faciles à calculer :
    \begin{enumerate}
        \item On réalise $4$ épreuves. Combien de chemins réalisent exactement $1$ succès ? Il y en a $4$ donc $\binom 4 1 = 4$.
        \item On réalise $100$ épreuves. Combien de chemins réalisent exactement $1$ succès ? Il y en a $100$ donc $\binom{100}{1} = 100$.
        \item On réalise $18$ épreuves. Combien de chemins réalisent exactement $0$ succès ? Il y en a $1$ donc $\binom{18}{0} = 1$.
        \item Combien de combinaisons à $6$ nombres différents existent-il avec $49$ nombres différents ?
    \end{enumerate}
\end{Exemple}

\begin{Rmq}
    On réalise $10$ épreuves. Combien de chemins réalisent exactement $6$ succès ?\par
    On pourrait faire un arbre de probabilité et compter les chemins un par un mais ce serait trop long : l'arbre possède $2^{10} = \np{1024}$ chemins différents !\par
    On utilise donc la calculatrice pour trouver $210$.
\end{Rmq}

\subsection{Variable aléatoire}

\begin{Defi}
    On note $X$ la fonction qui, à chaque issue du schéma de Bernoulli, associe le nombre de succès obtenus.\par
    On dit que $X$ est la \ipt{variable aléatoire} associé à ce schéma de Bernoulli.
\end{Defi}

\begin{Rmq}
    La variable aléatoire $X$ peut donc prendre toutes les valeurs comprises entre $0$ et $n$.
\end{Rmq}

\begin{Exemple}
    On lance dix fois un dé non pipé à $6$ faces. On cherche à obtenir le $3$.\par
    La variable aléatoire $X$ compte le nombre de $3$ obtenus au bout des $10$ lancers. On peut donc avoir $X = 0$, $X = 1$, $\ldots$, $X = 10$.
\end{Exemple}

\begin{Defi}
    On appelle \ipt{loi de probabilité} de $X$ la donnée de toutes les probabilités de $X$ résumées dans le tableau ci-dessous.
    \begin{center}
        \begin{tabular}{|*{5}{c|}}
            \hline
                $k$ & $0$ & $1$ & $\cdots$ & $n$ \\
            \hline
                $p_k$ & $p(X = 0)$ & $p(X = 1)$ & $\cdots$ & $p(X = n)$ \\
            \hline
        \end{tabular}
    \end{center}
\end{Defi}

\begin{Prop}
    Pour tout entier naturel $k$ tel que $0 \leq k \leq n$, la probabilité $p_k$ que l'événement $S$ soit réalisé exactement $k$ fois à l'issue de $n$ épreuves de Bernoulli indépendantes est donnée par :
    \[p_k = p(X = k) = \binom n k \times p^k \times (1-p)^{n - k}.\]
\end{Prop}

\begin{Exemple}
    On lance un dé à $6$ faces dix fois de suite. Le succès est $S$ : << obtenir le nombre $3$ >> tel que $p(S) = \dfrac 16$. On a donc un schéma de Bernoulli de paramètres $n = 10$ et $p = \frac 1 6$.\par
    On cherche la probabilité d'obtenir exactement $6$ fois le nombre $3$.\par Autrement dit, on cherche $p(X = 6)$ :
    \[p(X = 6) = \binom{10}{6} \times \left(\dfrac 1 6\right)^6 \times \left(\dfrac 5 6\right)^4 \approx \np{0,002170635} \approx 0,22\%.\]
\end{Exemple}

\section{Loi binomiale}

\begin{Defi}
    On considère un schéma de Bernoulli de paramètres $n$ et $p$ et $X$ la variable aléatoire qui compte le nombre de succès.\par
    On appelle \ipt{loi binomiale} de paramètres $n$ et $p$ la loi de probabilité, notée $\calig B(n\pv p)$, définie par :
    \[p(\{X = k\}) = \binom n k \times p^k \times (1-p)^{n - k} \quad \text{pour tout}\quad 0 \leq k \leq n.\]
    On dit que $X$ suit la loi binomiale.
\end{Defi}

\begin{Defi}
    Soit $\Omega$ l'univers associé à une expérience aléatoire.\par
    On suppose $\Omega$ fini ; on note $n$ le nombre d'éléments de $\Omega$ ($n$ entier naturel non nul).\par
    On suppose de plus que les $n$ issues $x_1, x_2, \ldots, x_n$ sont des nombres réels et qu'une loi de probabilité est définie sur $\Omega$ ; pour tout entier naturel $i$ compris entre $1$ et $n$, on note $p_i$ la probabilité de l'événement élémentaire $\{x_i\}$.\par
    L'\ipt{espérance} de la loi de probabilité est le nombre $E$ défini par : \[E = \sum_{i = 1}^n p_i x_i.\]
\end{Defi}

\begin{Rmq}
    $\sum_{i = 1}^n p_i = 1$ donc on peut écrire $E = \dfrac{\sum_{i = 1}^n p_i x_i}{\sum_{i = 1}^n p_i}$ et on retrouve la formule d'une moyenne statistique.\par
    L'espérance d'une loi de probabilité est la valeur que l'on peut espérer obtenir en moyenne \textit{dans le cas d'un grand nombre de répétitions}.
\end{Rmq}

\begin{Prop}
    Soit $\calig B(n \pv p)$ la loi binomiale de paramètres $n$ et $p$.\par
    L'espérance $E$ de $\calig B(n \pv p)$ est $E = np$.
\end{Prop}
\end{document}
