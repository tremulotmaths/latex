\documentclass[10pt,french]{book}

\input philippe2013
\usepackage{marvosym,slashbox}
\RegleEntete


\newcommand\competences{
\setcounter{exo}{0}
\begin{tabular}{ll} Nom : \\[5pt] Prénom : \end{tabular}
\hfill
\textbf{Note :}\renewcommand\arraystretch{2.3}
\begin{tabular}{|c|}
\hline
\slashbox{\Huge\bfseries\phantom{10}}{\Huge\bfseries 10}\\
\hline
\end{tabular}\renewcommand\arraystretch{1}\medskip
}

\entete{\premiere \sti}{Utilisation de la calculatrice}{A}
\pieddepage{}{}{}


\begin{document}
\competences

\textbf{Les résultats sont arrondis au dixième près.}\par\medskip

\exo Voici le relevé de notes d'un élève en mathématiques et en français au cours d'une année.

\begin{center}
\renewcommand\arraystretch{1.5}
	\begin{tabular}{|*{7}{c|}}
	\hline
		{\bf MATHS} & $5$ & $7$ & $10$ & $12$ & $14$ & $15$  \\
    \hline
        Effectif & 1 & 2 & 4 & 3 & 2 & 2 \\
	\hline\hline
		{\bf FRAN\c CAIS} & $8$ & $9$ & $10$ & $11$ & $12$ & $13$ \\
	\hline
        Effectif & 2 & 1 & 2 & 2 & 1 & 1 \\
    \hline
	\end{tabular}
\end{center}

À l'aide de la calculatrice :
\begin{enumerate}
	\item Série de notes en mathématiques :
        \begin{enumerate}
            \item Détermine la moyenne et l'écart-type.
            \item Calculer et \textbf{interpréter} une médiane ainsi que le premier et le troisième quartile.
            \item Déterminer l'intervalle interquartile ainsi que l'écart interquartile. \textbf{Interpréter} l'écart interquartile.
        \end{enumerate}
    \item Série de notes en français :
        \begin{enumerate}
            \item Détermine la moyenne et l'écart-type.
            \item Calculer et \textbf{interpréter} une médiane ainsi que le premier et le troisième quartile.
            \item Déterminer l'intervalle interquartile ainsi que l'écart interquartile. \textbf{Interpréter} l'écart interquartile.
        \end{enumerate}
    \item Comparer les deux séries de notes.
\end{enumerate}\clearpage

%--------------------------------------------------------------------------------------------------------------------------------------------------------------------------
%                           SUJET B
%--------------------------------------------------------------------------------------------------------------------------------------------------------------------------

\entete{\premiere \sti}{Utilisation de la calculatrice}{B}
\competences

\textbf{Les résultats sont arrondis au dixième près.}\par\medskip

\exo Voici le relevé de notes d'un élève en mathématiques et en physique au cours d'une année.

\begin{center}
\renewcommand\arraystretch{1.5}
	\begin{tabular}{|*{7}{c|}}
	\hline
		{\bf PHYSIQUE} & $5$ & $7$ & $10$ & $12$ & $14$ & $15$  \\
    \hline
        Effectif & 2 & 1 & 4 & 3 & 2 & 2 \\
	\hline\hline
		{\bf MATHS} & $8$ & $9$ & $10$ & $11$ & $12$ & $13$ \\
	\hline
        Effectif & 2 & 1 & 2 & 2 & 1 & 2 \\
    \hline
	\end{tabular}
\end{center}

À l'aide de la calculatrice :
\begin{enumerate}
	\item Série de notes en physiques :
        \begin{enumerate}
            \item Détermine la moyenne et l'écart-type.
            \item Calculer et \textbf{interpréter} une médiane ainsi que le premier et le troisième quartile.
            \item Déterminer l'intervalle interquartile ainsi que l'écart interquartile. \textbf{Interpréter} l'écart interquartile.
        \end{enumerate}
    \item Série de notes en maths :
        \begin{enumerate}
            \item Détermine la moyenne et l'écart-type.
            \item Calculer et \textbf{interpréter} une médiane ainsi que le premier et le troisième quartile.
            \item Déterminer l'intervalle interquartile ainsi que l'écart interquartile. \textbf{Interpréter} l'écart interquartile.
        \end{enumerate}
    \item Comparer les deux séries de notes.
\end{enumerate}

\end{document} 