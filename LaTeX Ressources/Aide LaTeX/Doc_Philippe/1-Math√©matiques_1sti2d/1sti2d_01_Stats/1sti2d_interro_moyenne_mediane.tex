\documentclass[10pt,french]{book}

\input philippe2013
\usepackage{marvosym,slashbox}
\RegleEntete


\newcommand\competences{
\setcounter{exo}{0}
\begin{tabular}{ll} Nom : \\[5pt] Prénom : \end{tabular}
\hfill
\textbf{Note :}\renewcommand\arraystretch{2.3}
\begin{tabularx}{0.18\linewidth}{|X|}
\hline
\slashbox{\Huge\bfseries\phantom{10}}{\Huge\bfseries 10}\\
\hline
\end{tabularx}\renewcommand\arraystretch{1}\medskip
}

\entete{\premiere \sti}{Statistiques}{A}
\pieddepage{}{}{}


\begin{document}
\competences

\exo Lors d'une compétition de saut en longueur dans un collège, les professeur d'\bsc{eps} ont relevé les performances suivantes :

\begin{center}
\renewcommand\arraystretch{1.5}
	\begin{tabular}{|*{12}{c|}}
	\hline
		{\bf Longueur en cm} & $90$ & $95$ & $100$ & $105$ & $110$ & $115$ & $120$ & $125$ & $130$ & $135$ & $140$ \\
	\hline
		{\bf Effectifs} & $1$ & $0$ & $4$ & $2$ & $5$ & $8$ & $7$ & $4$ & $3$ & $1$ & $2$ \\
	\hline
	\end{tabular}
\end{center}

\begin{enumerate}
	\item Calculer la longueur moyenne.
	\item Déterminer une médiane par le calcul et interpréter le résultat obtenu à l'aide d'une phrase.
\end{enumerate}\bigskip

\exo Pour être vendues, les pommes doivent être calibrées : elles sont réparties en caisses suivant la valeur de leur diamètre.\par
Dans un lot de pommes, un producteur a évalué le nombre de pommes pour chacun des six calibres rencontrés dans le lot. Il a obtenu le tableau suivant :\medskip

\begin{center}
\renewcommand\arraystretch{1.5}
	\begin{tabular}{|c|c|c|c|c|c|c|}
	\hline
	   {\bf Calibre (en mm)} & $[55;60[$ & $[60;65[$ & $[65;70[$ & $[70;75[$ & $[75;80[$ & $[80;85[$\\
    \hline
        {\bf Effectif (nombre de pommes)} & $13$ & $20$ & $30$ & $24$ & $26$ & $18$\\
    \hline
        {\bf Centre de classe} & & & & & & \\
    \hline
    \end{tabular}
\end{center}
%
\begin{enumerate}
    \item Compléter le tableau.
    \item En utilisant les centres de classe :
        \begin{enumerate}
            \item Calculer le calibre moyen des pommes ramassées.
            \item Calculer une médiane.
        \end{enumerate}
\end{enumerate}\clearpage

%--------------------------------------------------------------------------------------------------------------------------------------------------------------------------
%                           SUJET B
%--------------------------------------------------------------------------------------------------------------------------------------------------------------------------

\entete{\premiere \sti}{Statistiques}{B}
\competences

\exo Lors d'une compétition de saut en longueur dans un collège, les professeur d'\bsc{eps} ont relevé les performances suivantes :

\begin{center}
\renewcommand\arraystretch{1.5}
	\begin{tabular}{|*{12}{c|}}
	\hline
		{\bf Longueur en cm} & $80$ & $85$ & $90$ & $95$ & $100$ & $105$ & $110$ & $115$ & $120$ & $125$ & $130$ \\
	\hline
		{\bf Effectifs} & $1$ & $2$ & $4$ & $2$ & $5$ & $8$ & $7$ & $4$ & $3$ & $1$ & $2$ \\
	\hline
	\end{tabular}
\end{center}

\begin{enumerate}
	\item Calculer la longueur moyenne.
	\item Déterminer une médiane par le calcul et interpréter le résultat obtenu à l'aide d'une phrase.
\end{enumerate}\bigskip

\exo Pour être vendues, les pommes doivent être calibrées : elles sont réparties en caisses suivant la valeur de leur diamètre.\par
Dans un lot de pommes, un producteur a évalué le nombre de pommes pour chacun des six calibres rencontrés dans le lot. Il a obtenu le tableau suivant :\medskip

\begin{center}
\renewcommand\arraystretch{1.5}
	\begin{tabular}{|c|c|c|c|c|c|c|}
	\hline
	   {\bf Calibre (en mm)} & $[50 ; 55[$ & $[55;60[$ & $[60;65[$ & $[65;70[$ & $[70;75[$ & $[75;80[$\\
    \hline
        {\bf Effectif (nombre de pommes)} & $15$ & $20$ & $30$ & $24$ & $26$ & $18$\\
    \hline
        {\bf Centre de classe} & & & & & & \\
    \hline
    \end{tabular}
\end{center}
%
\begin{enumerate}
    \item Compléter le tableau.
    \item En utilisant les centres de classe :
        \begin{enumerate}
            \item Calculer le calibre moyen des pommes ramassées.
            \item Calculer une médiane.
        \end{enumerate}
\end{enumerate}

\end{document}