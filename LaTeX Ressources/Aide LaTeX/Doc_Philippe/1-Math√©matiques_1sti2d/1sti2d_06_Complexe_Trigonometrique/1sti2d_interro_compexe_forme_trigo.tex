\documentclass[10pt,french]{book}

\input philippe2013
\RegleEntete


\newcommand\competences{
\setcounter{exo}{0}
\begin{tabular}{ll} Nom : \\[5pt] Prénom : \end{tabular}
\hfill
\textbf{Note :}\renewcommand\arraystretch{2.3}
\begin{tabularx}{0.18\linewidth}{|X|}
\hline
\slashbox{\Huge\bfseries\phantom{10}}{\Huge\bfseries 10}\\
\hline
\end{tabularx}\renewcommand\arraystretch{1}\medskip
}

\entete{\premiere \sti}{Nombres complexes}{A}
\pieddepage{}{}{}


\begin{document}
\competences

\exo (1 point par bonne réponse détaillée)\par \'Ecrire les nombres suivants sous forme algébrique :
\[z_1 = 2\left(\cos \frac \pi 3 + \I \sin \frac\pi 3\right) \qq z_2 = \intervalleff{4}{\frac\pi 4}\]

\exo (2 points)\par \'Ecrire le nombre $z_3$ sous forme algébrique sachant que \[\arg(z_3) = -\frac{5\pi}{6} \qetq \abs{z_3} = 2\sqrt3.\]

\exo (2 points par bonne réponse détaillée)\par \'Ecrire les nombres suivants sous forme trigonométrique $z = \intervalleff{\rho}{\theta}$ :
\[z_4 = 1 - \I \qq z_5 = -5\sqrt 3 + 5\I \qq z_6 = 7\I.\]

\clearpage

%--------------------------------------------------------------------------------------------------------------------------------------------------------------------------
%                           SUJET B
%--------------------------------------------------------------------------------------------------------------------------------------------------------------------------

\entete{\premiere \sti}{Nombres complexes}{B}
\competences

\exo (1 point par bonne réponse détaillée)\par \'Ecrire les nombres suivants sous forme algébrique :
\[z_1 = 6\left(\cos \frac \pi 4 + \I \sin \frac\pi 4\right) \qq z_2 = \intervalleff{4}{\frac\pi 3}\]

\exo (2 points)\par \'Ecrire le nombre $z_3$ sous forme algébrique sachant que \[\arg(z_3) = \frac{5\pi}{6} \qetq \abs{z_3} = 8\sqrt3.\]

\exo (2 points par bonne réponse détaillée)\par \'Ecrire les nombres suivants sous forme trigonométrique $z = \intervalleff{\rho}{\theta}$ :
\[z_4 = -1 + \I \qq z_5 = 4\sqrt 3 - 4\I \qq z_6 = 7.\]

\end{document} 