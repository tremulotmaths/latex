\documentclass[10pt,french]{book}

\input philippe2013
\RegleEntete


\newcommand\competences{
\setcounter{exo}{0}
\begin{tabular}{ll} Nom : \\[5pt] Prénom : \end{tabular}
\hfill
\textbf{Note :}\renewcommand\arraystretch{2.3}
\begin{tabularx}{0.18\linewidth}{|X|}
\hline
\slashbox{\Huge\bfseries\phantom{10}}{\Huge\bfseries 10}\\
\hline
\end{tabularx}\renewcommand\arraystretch{1}\medskip
}

\entete{\premiere \sti}{Cercle trigonométrique}{A}
\pieddepage{}{}{}

\newcommand{\Droite}[5]{
  \coordinate (ATemp) at ($(#2)!{-#4}!(#3)$);
  \coordinate (BTemp) at ($(#2)!{1+#5}!(#3)$);
  \draw[#1] (ATemp)--(BTemp);
}

\begin{document}
\competences

\exo Compléter le tableau suivant en donnant des valeurs exactes :
\begin{center}
\renewcommand\arraystretch{2.5}
    \begin{tabular}{|c|*{6}{>{\centering\arraybackslash} p{2em}|}}
        \hline
            $\alpha$ en radian & $0$ & $\frac \pi 2$ & $\pi$ & $\frac\pi4$ & $\frac \pi6$ & $\frac \pi3$ \\
        \hline
            $\cos(\alpha)$ &&&&&& \\
        \hline
    \end{tabular}
\renewcommand\arraystretch{1}
\end{center}

\exo Le cercle trigonométrique ci-dessous a été partagé régulièrement. Placer les points suivants sur le cercle en respectant l'angle écrit entre parenthèse.\par\medskip
Par exemple, $D\left(\rad{\frac\pi2}\right)$ signifie : placer le point $D$ tel que $\widehat{IOD} = \rad{\frac\pi2}$.

\[A(\ang{45}) \qq B(\ang{225}) \qq C(\ang{-60})\]

\[D\left(\rad{\frac\pi2}\right) \qq E\left(\rad{-\frac\pi6}\right) \qq F\left(\rad{\frac{2\pi}{3}}\right)\]

\begin{center}
    \begin{tikzpicture}[>=latex,scale=4]
        \coordinate (O) at (0,0); \draw (O) node[below left] {$O$};
        \coordinate (I) at (1,0); \draw (I) node[below right=-2pt] {$I$};
        \coordinate (J) at (0,1);
        \Droite{->}{O}{I}{1.1}{0.1};
        \Droite{->}{O}{J}{1.1}{0.1};
        \draw (O) circle (1);
        \foreach \x in {0,15,...,345} \draw (\x:1) node {+};
    \end{tikzpicture}
\end{center}

\exo La mesure principale d'un angle est la seule mesure de l'angle appartenant à l'intervalle $\intervalleof{-\pi}{\pi}$.
\begin{enumerate}
    \item Répondre aux questions suivantes par OUI ou NON :
    \begin{enumerate}
        \item $\frac{8\pi}{5}$ peut être considérée comme la mesure principale d'un angle ? \ldots\smallskip
        \item $\frac{2\pi}{7}$ peut être considérée comme la mesure principale d'un angle ? \ldots\smallskip
        \item $\frac{-3\pi}{2}$ peut être considérée comme la mesure principale d'un angle ? \ldots\smallskip
        \item $\frac{-13\pi}{12}$ peut être considérée comme la mesure principale d'un angle ? \ldots
    \end{enumerate}
    \item Donner la mesure principale des angles suivants dont une mesure est donnée :
    \[\alpha = \dfrac{33\pi}{2} \qq \beta = \dfrac{76\pi}{6}\]
\end{enumerate}


\clearpage

%--------------------------------------------------------------------------------------------------------------------------------------------------------------------------
%                           SUJET B
%--------------------------------------------------------------------------------------------------------------------------------------------------------------------------

\entete{\premiere \sti}{Cercle trigonométrique}{B}
\competences

\exo Compléter le tableau suivant en donnant des valeurs exactes :
\begin{center}
\renewcommand\arraystretch{2.5}
    \begin{tabular}{|c|*{6}{>{\centering\arraybackslash} p{2em}|}}
        \hline
            $\alpha$ en radian & $0$ & $\pi$ & $\frac \pi 2$ & $\frac\pi3$ & $\frac \pi6$ & $\frac \pi4$ \\
        \hline
            $\sin(\alpha)$ &&&&&& \\
        \hline
    \end{tabular}
\renewcommand\arraystretch{1}
\end{center}

\exo Le cercle trigonométrique ci-dessous a été partagé régulièrement. Placer les points suivants sur le cercle en respectant l'angle écrit entre parenthèse.\par\medskip
Par exemple, $D\left(\rad{\pi}\right)$ signifie : placer le point $D$ tel que $\widehat{IOD} = \rad{\pi}$.

\[A(\ang{60}) \qq B(\ang{270}) \qq C(\ang{-45})\]

\[D\left(\rad{\pi}\right) \qq E\left(\rad{-\frac\pi3}\right) \qq F\left(\rad{\frac{5\pi}{6}}\right)\]

\begin{center}
    \begin{tikzpicture}[>=latex,scale=4]
        \coordinate (O) at (0,0); \draw (O) node[below left] {$O$};
        \coordinate (I) at (1,0); \draw (I) node[below right=-2pt] {$I$};
        \coordinate (J) at (0,1);
        \Droite{->}{O}{I}{1.1}{0.1};
        \Droite{->}{O}{J}{1.1}{0.1};
        \draw (O) circle (1);
        \foreach \x in {0,15,...,345} \draw (\x:1) node {+};
    \end{tikzpicture}
\end{center}

\exo La mesure principale d'un angle est la seule mesure de l'angle appartenant à l'intervalle $\intervalleof{-\pi}{\pi}$.
\begin{enumerate}
    \item Répondre aux questions suivantes par OUI ou NON :
    \begin{enumerate}
        \item $\frac{4\pi}{5}$ peut être considérée comme la mesure principale d'un angle ? \ldots\smallskip
        \item $\frac{8\pi}{7}$ peut être considérée comme la mesure principale d'un angle ? \ldots\smallskip
        \item $\frac{-3\pi}{4}$ peut être considérée comme la mesure principale d'un angle ? \ldots\smallskip
        \item $\frac{-21\pi}{22}$ peut être considérée comme la mesure principale d'un angle ? \ldots
    \end{enumerate}
    \item Donner la mesure principale des angles suivants dont une mesure est donnée :
    \[\alpha = \dfrac{37\pi}{2} \qq \beta = \dfrac{80\pi}{6}\]
\end{enumerate}

\end{document} 