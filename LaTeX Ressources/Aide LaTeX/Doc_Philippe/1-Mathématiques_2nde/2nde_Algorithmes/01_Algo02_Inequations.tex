\documentclass[10pt,openright,twoside,french]{book}

\usepackage{marvosym}
\input philippe2013
\input philippe2013_activites

\pagestyle{empty}

\begin{document}

\TitreAlgo{i.2}{Inéquations}

\exo
Une société veut imprimer des livres. La location de la machine revient à \EUR{$750$} par jour et les frais de fabrication s'élèvent à \EUR{$3{,}75$} par livre.\par
La société souhaite savoir le nombre de livre à imprimer pour que le prix de revient d'un livre soit inférieur ou égal à \EUR{$6$}. Pour cela, elle utilise la formule suivante :
\[\text{Prix de revient} = \frac{\text{Prix total}}{\text{Nombre total de livres}}\]
Elle utilise également l'algorithme suivant :

\begin{center}
\small
    \psframebox{
    \parbox{0.5\linewidth}{
        \textbf{Variables}

            \quad $Livre$ : un entier naturel

            \quad $Total$ : un nombre réel
            
            \quad $Revient$ : un nombre réel

        \textbf{Entrée}

            \quad Saisir $Livre$

        \textbf{Traitement}

            \quad Affecter à $Total$ la valeur $750 + 3{,}50 \times Livre$

            \quad Affecter à $Revient$ la valeur $Total/Livre$

        \textbf{Sortie}

            \quad Si $Revient > 6$

                \qquad Alors Afficher "Le prix de revient est trop élevé !"
            
                \qquad Sinon Afficher "Le prix de revient est correct !"
                
            \quad FinSi
    }}
\end{center}

\begin{enumerate}
    \item Comment la société doit-elle utiliser cet algorithme ?
    \item Déterminer par le calcul le nombre minimum de livres à imprimer pour répondre aux contraintes énoncées.
\end{enumerate}

\[*\]

\exo
Deux chauffeurs de taxi proposent à leurs clients des tarifs différents :
\begin{description}
    \item[Chauffeur A :] Une prise en charge de \EUR{$4{,}80$} et un coût supplémentaire de \EUR{$1{,}15$} par kilomètre parcouru.
    \item[Chauffeur B :] Une prise en charge de \EUR{$3{,}20$} et un coût supplémentaire de \EUR{$1{,}20$} par kilomètre parcouru.
\end{description}

\begin{enumerate}
    \item Paulette a besoin d'effectuer un parcours de $15~km$. Quel chauffeur a-t-elle intérêt à choisir ?
    \item Déterminer les nombres de kilomètres pour lesquels Paulette a intérêt à choisir le chauffeur A. \'Ecrire le résultat sous forme d'un intervalle.
    \item \'Ecrire un algorithme qui affiche le prix payé à chaque chauffeur en fonction du nombre de kilomètres parcourus.
    \item Améliorer cet algorithme pour qu'il affiche une phrase déterminant le chauffeur qu'il faut choisir en fonction des kilomètres à parcourir ; par exemple :
    \begin{center}
        \texttt{Pour parcourir ... kilomètres, le chauffeur ... coûte moins cher.}
    \end{center}
\end{enumerate}

\end{document}