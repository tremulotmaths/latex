\documentclass[10pt,twoside,openright]{book}

\usepackage{monpackage}

\entete{\seconde{} 1}{Fonctions affines par morceaux}{27/03/2006}

\pieddepage{}{}{}

\begin{document}

\exo

\begin{enumerate}
\item Tracer les droites $d_1$ et $d_2$ repr�sentant les fonctions
affines d�finies sur $\R$ par $$g(x) = -x + 1 \qq h(x) = 2x + 1.$$

\item Soit la fonction $f$ d�finie sur $\R$ par
$$\left\{\begin{array}{rclcl}
f(x) & = & -x + 1 & \mathrm{si} & x \leq 0\\
f(x) & = & 2x + 1 & \mathrm{si} & x > 0
\end{array}\right.$$

Repasser en rouge :
\begin{itemize}
\puce la partie de $d_1$ correspondant � $x \leq 0$

\puce La partie de $d_2$ correspondant � $x > 0$.
\end{itemize}
\end{enumerate}

{\itshape La courbe rouge ainsi obtenue est la courbe repr�sentative
de la fonction $f$. On dit que $f$ est une fonction affine par
morceaux.}\vsd

\exo

En utilisant la m�me m�thode que dans l'exercice pr�c�dent,
repr�senter la fonction $f$ telle que
$$\left\{\begin{array}{rclcl}
f(x) & = & 3x - 2 & \mathrm{si} & x \leq 1\\[6pt]
f(x) & = & \dfrac13 x + \dfrac23 & \mathrm{si} & x > 1
\end{array}\right.$$

\exo

Paul va au lyc�e.

\begin{itemize}
\puce Il se rend � pied d'un pas r�gulier jusqu'� la station de
m�tro la plus proche (� 500 m de chez lui) en 8 minutes.

\puce Il prend alors le m�tro et parcourt 3 km en 4 minutes.

\puce Sur le quai du m�tro, il rencontre une amie avec qui il
discute pendant 10 minutes sans se d�placer.

\puce Finalement, il reprend sa marche et arrive au lyc�e, situ� �
250 m du m�tro, en 4 minutes.
\end{itemize}\vsc

\begin{enumerate}
\item Repr�senter sur un graphique la distance (en m�tres) parcourue par
Paul en fonction du temps (en minutes).

\item En combien de temps Paul a-t-il parcouru les 325 premiers
m�tres en sortant de chez lui ? Le d�terminer par le calcul.

\item Quelle est la vitesse moyenne du m�tro ? Le d�terminer par le
calcul.
\end{enumerate}\vsd

\exo

On consid�re la fonction $f : x \mapsto |x|$, appel�e << fonction
valeur absolue >>.

\begin{enumerate}
\item \begin{enumerate}
\item Quel est l'ensemble de d�finition de cette fonction ?

\item Peut-on dire qu'il s'agit d'une fonction affine par morceaux ?
Justifier.
\end{enumerate}
\item \'Etudier la parit� de la fonction $f$.

\item Dresser le tableau de variations de $f$ sur $\R$.

\item Construire la repr�sentation graphique de cette fonction sur
l'intervalle $[-5 \pv 5]$.
\end{enumerate}


\end{document}
