\documentclass[10pt,openright,twoside,french]{book}

\input philippe2013


\pagestyle{empty}

\begin{document}

Sur un site internet, on peut acheter des blu-ray à \EUR{$15$} l'unité. Les frais de port sont de \EUR{$5$}, quel que soit le nombre de blu-ray acheté.
\begin{enumerate}
    \item Combien doit-on payer pour $4$ blu-ray achetés ?
    \item Quelle est l'expression de la fonction $P$ donnant le prix total en fonction du nombre $n$ de blu-ray acheté ?
    \item Le prix total est-il proportionnel au nombre de blu-ray acheté ?
    \item La fonction $P$ est-elle une fonction affine ? Pourquoi ?
    \item Paulo a un budget de \EUR{$100$} pour Noël. Combien de blu-ray peut-il acheter ?
\end{enumerate}\bigskip

\hrulefill\bigskip

Sur un site internet, on peut acheter des blu-ray à \EUR{$15$} l'unité. Les frais de port sont de \EUR{$5$}, quel que soit le nombre de blu-ray acheté.
\begin{enumerate}
    \item Combien doit-on payer pour $4$ blu-ray achetés ?
    \item Quelle est l'expression de la fonction $P$ donnant le prix total en fonction du nombre $n$ de blu-ray acheté ?
    \item Le prix total est-il proportionnel au nombre de blu-ray acheté ?
    \item La fonction $P$ est-elle une fonction affine ? Pourquoi ?
    \item Paulo a un budget de \EUR{$100$} pour Noël. Combien de blu-ray peut-il acheter ?
\end{enumerate}\bigskip

\hrulefill\bigskip

Sur un site internet, on peut acheter des blu-ray à \EUR{$15$} l'unité. Les frais de port sont de \EUR{$5$}, quel que soit le nombre de blu-ray acheté.
\begin{enumerate}
    \item Combien doit-on payer pour $4$ blu-ray achetés ?
    \item Quelle est l'expression de la fonction $P$ donnant le prix total en fonction du nombre $n$ de blu-ray acheté ?
    \item Le prix total est-il proportionnel au nombre de blu-ray acheté ?
    \item La fonction $P$ est-elle une fonction affine ? Pourquoi ?
    \item Paulo a un budget de \EUR{$100$} pour Noël. Combien de blu-ray peut-il acheter ?
\end{enumerate}\bigskip

\hrulefill\bigskip

Sur un site internet, on peut acheter des blu-ray à \EUR{$15$} l'unité. Les frais de port sont de \EUR{$5$}, quel que soit le nombre de blu-ray acheté.
\begin{enumerate}
    \item Combien doit-on payer pour $4$ blu-ray achetés ?
    \item Quelle est l'expression de la fonction $P$ donnant le prix total en fonction du nombre $n$ de blu-ray acheté ?
    \item Le prix total est-il proportionnel au nombre de blu-ray acheté ?
    \item La fonction $P$ est-elle une fonction affine ? Pourquoi ?
    \item Paulo a un budget de \EUR{$100$} pour Noël. Combien de blu-ray peut-il acheter ?
\end{enumerate}\bigskip

\hrulefill\bigskip

Sur un site internet, on peut acheter des blu-ray à \EUR{$15$} l'unité. Les frais de port sont de \EUR{$5$}, quel que soit le nombre de blu-ray acheté.
\begin{enumerate}
    \item Combien doit-on payer pour $4$ blu-ray achetés ?
    \item Quelle est l'expression de la fonction $P$ donnant le prix total en fonction du nombre $n$ de blu-ray acheté ?
    \item Le prix total est-il proportionnel au nombre de blu-ray acheté ?
    \item La fonction $P$ est-elle une fonction affine ? Pourquoi ?
    \item Paulo a un budget de \EUR{$100$} pour Noël. Combien de blu-ray peut-il acheter ?
\end{enumerate}
\end{document} 