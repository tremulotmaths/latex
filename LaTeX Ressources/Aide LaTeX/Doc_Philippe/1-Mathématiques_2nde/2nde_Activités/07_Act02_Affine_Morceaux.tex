\documentclass[12pt,french]{book}
\input preambule_2013
\input philippe2013_activites
\pagestyle{empty}


\begin{document}

\TitreActivite{vii.2}{Fonctions affines \\ par morceaux}

\exo

\begin{enumerate}
    \item Tracer les droites $d_1$ et $d_2$ représentant les fonctions affines définies sur $\R$ par \[g(x) = -x + 1 \qetq h(x) = 2x + 1.\]
    \item Soit la fonction $f$ définie sur $\R$ par
    \[\left\{\begin{array}{rclcl}
    f(x) & = & -x + 1 & \text{si} & x \leqslant 0\\
    f(x) & = & 2x + 1 & \text{si} & x > 0
    \end{array}\right.\]
    Repasser en rouge :
        \begin{itemize}
            \item la partie de $d_1$ correspondant à $x \leq 0$
            \item La partie de $d_2$ correspondant à $x > 0$.
        \end{itemize}
\end{enumerate}

{\itshape La courbe rouge ainsi obtenue est la courbe représentative de la fonction $f$. On dit que $f$ est une \textbf{fonction affine par morceaux.}}\[*\]

\exo

En utilisant la même méthode que dans l'exercice précédent, représenter la fonction $f$ telle que
    \[\left\{\begin{array}{rclcl}
        f(x) & = & 3x - 2 & \text{si} & x \leqslant 1\\[6pt]
        f(x) & = & \dfrac13 x + \dfrac23 & \text{si} & x > 1
    \end{array}\right.\]\[*\]

\exo

Paulito va au lycée.

\begin{itemize}
    \item Il se rend à pieds d'un pas régulier jusqu'à la station de métro la plus proche (à $500~m$ de chez lui) en $8$ minutes.
    \item Il prend alors le métro et parcourt $3~km$ en $4$ minutes.
    \item Sur le quai du métro, il rencontre une amie avec qui il discute pendant $10$ minutes sans se déplacer.
    \item Finalement, il reprend sa marche et arrive au lycée, situé à $250~m$ du métro, en $4$ minutes.
\end{itemize}

\begin{enumerate}
    \item Représenter sur un graphique la distance (en mètres) parcourue par Paulito en fonction du temps (en minutes).
    \item En combien de temps Paul a-t-il parcouru les $325$ premiers mètres en sortant de chez lui ? Le déterminer par le calcul.
    \item Quelle est la vitesse moyenne du métro ? Le déterminer par le calcul.
\end{enumerate}\[*\]


\end{document}
