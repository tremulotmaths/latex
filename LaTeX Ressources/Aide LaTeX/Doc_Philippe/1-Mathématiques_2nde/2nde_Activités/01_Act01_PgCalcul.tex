\documentclass[10pt,openright,twoside,french]{book}
\input philippe2013
\input philippe2013_activites
\pagestyle{empty}

\begin{document}

\TitreActivite{i.1}{Programme de calcul}

On considère le programme de calcul suivant :\medskip

\psframebox{
\parbox{0.3\linewidth}{
\begin{itemize}
    \item[\textbullet] Choisir un nombre ;
    \item[\textbullet] Retrancher $1$ ;
    \item[\textbullet] \'Elever au carré ;
    \item[\textbullet] Retrancher $12$ ;
    \item[\textbullet] Diviser par $6$.
\end{itemize}
}}\medskip

\begin{enumerate}
    \item Tester ce programme de calcul avec deux nombres différents.
    \item Quelle est la \textit{nature} du résultat obtenu avec $-5$ ?
    \item Quelle est la \textit{nature} du résultat obtenu avec $1$ ?
    \item Quelle est la \textit{nature} du résultat obtenu avec $3$ ?
    \item Quels nombres peut-on choisir pour obtenir le nombre $0$ ?
    \item Recopier la phrase suivante en devinant le dernier mot manquant :
    \begin{center}
    \cursive
    Un programme de calcul est une suite finie d'opérations effectuées dans un ordre précis et appliquées à une donnée de départ (le nombre choisi).\par
    Cela correspond à ce que l'on appelle un a\ldots
    \end{center}
\end{enumerate}


\end{document} 