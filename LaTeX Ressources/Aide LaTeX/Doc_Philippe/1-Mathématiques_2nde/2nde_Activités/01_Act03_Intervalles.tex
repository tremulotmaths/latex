\documentclass[10pt,openright,twoside,french]{book}
\input philippe2013
\input philippe2013_activites
\pagestyle{empty}

\begin{document}

\TitreActivite{i.3}{Intervalles de nombres}

Paul a choisi un nombre. Voici quelques indications pour nous aider à le retrouver :
\begin{itemize}
    \item C'est un entier relatif non nul ;
    \item Il est supérieur ou égal à $-5$ ;
    \item Il est strictement inférieur à $4$ ;
    \item C'est un multiple de $3$.
\end{itemize}\bigskip

\begin{enumerate}
    \item Sur une seule droite graduée, représenter les différents indices.
    \item Est-il possible de déterminer le nombre choisi par Paul ?
    \item Parmi tous les indices, modifier un seul mot pour permettre d'avoir une seule solution.
\end{enumerate}
\end{document} 