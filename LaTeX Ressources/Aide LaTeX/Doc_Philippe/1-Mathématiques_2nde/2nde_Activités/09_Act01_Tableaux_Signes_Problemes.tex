\documentclass[11pt,openright,twoside,french]{book}
\usepackage{marvosym}
\input philippe2013
\input philippe2013_activites
\pagestyle{empty}


\begin{document}

\TitreActivite{ix.1}{Résoudre un problème \\ à l'aide d'un tableau de signes}\medskip

\exo
\begin{enumerate}
    \item On considère l'expression $A$ définie pour tout $x \in \R$ par $A(x) = -3x^2 + 17x - 20$.
    \begin{enumerate}
        \item Démontrer que $A(x) = (3x - 5)(-x + 4)$.
        \item Résoudre $A(x) = 0$.
    \end{enumerate}
    \item On considère l'expression $B$ définie pour tout $x \in \R$ par $B(x) = 4x^2 - 9$.
    \begin{enumerate}
        \item À l'aide d'une identité remarquable, factoriser $B$.
        \item Résoudre $B(x) = 0$.
    \end{enumerate}
    \item À l'aide d'un tableau de signe, résoudre l'inéquation $\dfrac{-3x^2 + 17x - 20}{4x^2 - 9} \geq 0$.
\end{enumerate}\medskip

\[*\]\medskip

\exo Résoudre l'inéquation $\dfrac{2x + 3}{x - 1} \leq 4$.\medskip

\[*\]\medskip

\exo Une entreprise fabrique un produit. Pour une période donnée, le coût total de production, en euros, est donné en fonction du nombre $p$ d'articles fabriqués par :
\[C(p) = 2p^2 + 10p + 900 \quad \text{pour}\quad 0 < p < 80.\]

\begin{enumerate}
    \item Combien coûte la production de $30$ articles ?
    \item Combien coûte la production de $0$ article ? Comment peut-on l'expliquer ?
\end{enumerate}\medskip

Tous les articles fabriqués sont vendus. La recette totale en euros est donnée par \[R(p) = 120p.\]

\begin{enumerate}[resume]
    \item Quelle est la recette gagnée lorsque $30$ articles sont produits et vendus ?
    \item Quel est le bénéfice obtenu pour $30$ articles produits et vendus ?
    \item Quel est le bénéfice réalisé pour $60$ articles produits et vendus ? Comment interpréter le résultat ?
\end{enumerate}\medskip

On note $B(p)$ le bénéfice total réalisé pour $p$ articles produits et vendus.
\begin{enumerate}[resume]
    \item Vérifier que $B(p) = -2(p^2 - 55p + 450)$.
    \item Démontrer que $B(p) = -2(p - 10)(p - 45)$.
    \item Pour quels nombres d'articles produits et vendus la production est-elle rentable ?
    \item À l'aide de la calculatrice, représenter les fonctions $C$ et $R$. Adaptez la fenêtre pour visualiser correctement les données du problème. Que remarque-t-on ?
\end{enumerate}
\end{document} 