\documentclass[10pt,openright,twoside,french]{book}
\input philippe2013
\input philippe2013_activites
\pagestyle{empty}

\begin{document}

\TitreActivite{vi.1}{Indicateurs de position}

\exo
On s'intéresse à la distance entre des établissements scolaires publics et la piscine utilisée par chacun d'entre eux.\par
Une étude du ministère de l'\'Education Nationale a déterminé que cette distance était, au moment de l'étude :
\begin{itemize}
    \item comprise entre $0{,}2~km$ et $1{,}5~km$ dans huit régions ;
    \item supérieure à $1{,}5~km$ et au plus égale à $2{,}5~km$ dans onze régions ;
    \item supérieure à $2{,}5~km$ dans trois régions.
\end{itemize}\medskip

\begin{enumerate}
    \item Considérons neuf lycées notées $A$, $B$,\ldots, $I$ dont la distance à la piscine correspondante est donnée dans le tableau suivant :
    \begin{center}
    \renewcommand\arraystretch{1.5}
        \begin{tabular}{|c|>\centering p{1.75em}|>\centering p{1.75em}|>\centering p{1.75em}|>\centering p{1.75em}|>\centering p{1.75em}|%
        >\centering p{1.75em}|>\centering p{1.75em}|>\centering p{1.75em}| p{1.75em}|}
            \hline
            Lycée & $A$ & $B$ & $C$ & $D$ & $E$ & $F$ & $G$ & $H$ & \multicolumn{1}{c|}{$I$} \\
            \hline
            Distance en km & $1{,}8$ & $1{,}0$ & $20{,}2$ & $0$ & $0{,}6$ & $0$ & $0{,}8$ & $2{,}6$ & $0$\\
            \hline
        \end{tabular}
    \end{center}

    Pour cet ensemble de neuf lycées, calculer la distance moyenne à la piscine fréquentée. Dans laquelle des trois catégories définies ci-dessus doit-on classer cet ensemble de neuf lycées ?\medskip

    \item Les neuf lycées ont les effectifs suivants :
    \begin{center}
    \renewcommand\arraystretch{1.5}
        \begin{tabular}{|c|>\centering p{2.5em}|>\centering p{2.5em}|>\centering p{2.5em}|>\centering p{2.5em}|>\centering p{2.5em}|%
        >\centering p{2.5em}|>\centering p{2.5em}|>\centering p{2.5em}| p{2.5em}|}
            \hline
            Lycée & $A$ & $B$ & $C$ & $D$ & $E$ & $F$ & $G$ & $H$ & \multicolumn{1}{c|}{$I$} \\
            \hline
            Effectifs & $930$ & $\NP{1130}$ & $\NP{420}$ & $\NP{1710}$ & $\NP{1450}$ & $\NP{1430}$ & $\NP{1920}$ & $\NP{530}$ & $\NP{1250}$\\
            \hline
        \end{tabular}
    \end{center}

    Calculer la distance moyenne par élève parcourue pour se rendre à la piscine (les informations du premier tableau doivent être utilisées).\medskip

    \item Afin de calculer les frais de déplacements entre les lycées et les piscines, laquelle des deux distances moyennes paraît la plus appropriée ?
\end{enumerate}\[*\]

\exo Dans un village, on a compté le nombre d'enfants par famille. Voici les résultats obtenus :
\begin{center}
    \begin{tabularx}{0.65\linewidth}{|m{3cm}|*{7}{X|}}
        \hline
            Nombre d'enfants & $0$ & $1$ & $2$ & $3$ & $4$ & $5$ & $6$ \\
        \hline
            Effectifs & $82$ & $124$ & $217$ & $156$ & $52$ & $28$ & $22$ \\
        \hline
    \end{tabularx}
\end{center}

\begin{enumerate}
    \item Calculer le nombre moyen d'enfants par famille. Ce nombre a-t-il une signification réelle ?
    \item Calculer une médiane de cette série et donner une interprétation.\par Pourquoi dit-on \textbf{une} médiane et non \textbf{la} médiane ?
    \item Calculer le premier et le troisième quartile et donner une interprétation.
    \item Sur une page complète, construire le diagramme en bâtons correspondant à cette série.\par En ordonnée, l'unité sera de $1~mm$ pour $1$ enfant.
\end{enumerate}


\end{document} 