\documentclass[10pt,openright,twoside,french]{book}

\input philippe2013
\input philippe2013_cours
\input philippe2013_sections
\input philippe2013_chapitre


\begin{document}
%\small
\pagestyle{empty}

\begin{Defi}
    Une \iptb{équation}\index{equation@équation} est une égalité où figure une inconnue.\par
    \iptb{Résoudre} une équation revient à trouver la (ou les) valeur(s) de l'inconnue pour laquelle (ou lesquelles) l'égalité est vérifiée.
\end{Defi}

\begin{Prop}
On peut ajouter un même nombre à chaque membre d'une égalité pour obtenir ainsi une égalité équivalente :
    \[a = b \qLRq a \rouge{\ +\ c } = b \rouge{\ +\ c}\]
\end{Prop}

\begin{Demo}
    Puisque $a = b$ alors $a - b = 0$ et puisque $c - c = 0$, on a :
    \[0 = a - b + c - c = a + c - b - c = a + c - (b + c) = 0 \quad \text{donc} \quad a+c = b+c.\]
\end{Demo}

\begin{Prop}
    On peut multiplier par un même nombre non nul les deux membres d'une égalité pour obtenir ainsi une égalité équivalente :
    \[\text{Avec } c\neq 0,\quad a = b \qLRq a \rouge{\,\times\,c} = b \rouge{\,\times\,c}\]
\end{Prop}

\begin{Demo}
    Puisque $a = b$ alors $a - b = 0$ et puisque $c \neq 0$, on a :
    \[0 = c(a - b) = ca - cb = 0 \quad \text{donc} \quad a \times c = b \times c.\]
\end{Demo}

\subsubsection{\'Equation du premier degré}
\begin{Defi}
    Une \iptb{équation à une inconnue du premier degré}\index{equation@équation!équation du premier degré} est une équation de la forme $ax + b = 0$ où $x$ est l'inconnue et $a$ et $b$ sont des paramètres donnés tels que $a \neq 0$.\par
\end{Defi}

\subsubsection{\'Equation produit}\index{equation@équation!équation produit}
\begin{Prop}[(admise)]
    Un produit de facteur est nul si et seulement si l'un des facteurs est nul :
    \[A(x) \times B(x) = 0 \qLRq A(x) = 0 \text{ ou } B(x) = 0\]
\end{Prop}


\subsubsection{Résolution d'un problème}
\begin{Exemple}
    Lors d'une séance de cinéma, on a accueilli $56$ spectateurs. Certains ont payé le tarif réduit (\EUR{$5$}), les autres le tarif normal (\EUR{$8$}). La recette de cette séance se monte à \EUR{$376$}.\par Combien de spectateurs ont payé le tarif réduit ? (réponse : $24$)
\end{Exemple}

Voici les étapes de la résolution d'un problème en utilisant les équations :
\begin{enumerate}
    \item choix de l'inconnue ;
    \item mise en équation du problème ;
    \item résolution de l'équation ;
    \item réponse au problème.
\end{enumerate}

\end{document}