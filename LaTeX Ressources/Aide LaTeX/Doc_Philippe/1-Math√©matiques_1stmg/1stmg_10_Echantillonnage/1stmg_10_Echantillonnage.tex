\documentclass[10pt,openright,twoside,french]{book}

\input philippe2013
\input philippe2013_cours
\input philippe2013_sections
\input philippe2013_chapitre
\renewcommand\PartProgramme{Stats/Probas}
\renewcommand\MaCouleur{Melon!150}

\pieddepage{}{%
\begin{tikzpicture}[scale=0.65]
\shadedraw [top color=white, bottom color=\MaCouleur, draw=\MaCouleur]
[l-system={Sierpinski triangle, step=1pt, angle=60, axiom=F, order=6.5}]
lindenmayer system -- cycle;
\draw (30:0.65cm) node {\bfseries\textcolor{black}{\thepage}};
\end{tikzpicture}%
}{}


\setcounter{chapter}{9}
\begin{document}
\chapter[\'Echantillonnage et prise de décision]{\'Echantillonnage\\ Prise de décision}\label{echantillonnage}

On considère une population dont une proportion $p$ d'individus possèdent une certaine particularité $\calig P$. On note $f$ la fréquence des individus d'un échantillon de taille $n$, prélevé au hasard et avec remise, qui possèdent la particularité $\calig P$.\medskip

\begin{Exemple}
    Dans la population active d'un pays, il y a $21\%$ de chômeurs. Ici, $p = 0,21$.\par
    On prélève un échantillon de $10$ personnes et on s'intéresse à celles qui sont au chômage. Il y en a $2$. Donc $n = 10$ et $f = \dfrac{2}{10}$.
\end{Exemple}

\section{Intervalle de fluctuation d'une fréquence}

 On suppose que la proportion a pour valeur $p$ et on note $X$ la variable aléatoire qui compte le nombre d'individus de l'échantillon qui possèdent la particularité $\calig P$.\par
 Alors $X$ suit la loi binomiale de paramètres $n$ et $p$.
 
 \begin{Defi}
    Soit $p$ la valeur de la proportion. On note $a$ le plus petit nombre entier tel que $p(X \leq a) > 0,025$ et $b$ le plus petit nombre entier tel que $p(X \leq b) \geq 0,975$.\par
    On appelle \ipt{intervalle de fluctuation} à $95\%$ de la fréquence $f$ pour un échantillon de taille $n$, l'intervalle $\intervalleff{\frac a n}{\frac b n}$.
 \end{Defi}
 
 Autrement dit, la fréquence $f$ appartient à l'intervalle de fluctuation dans au moins $95\%$ des cas.
 
 
 \section{Prise de décision}

Dans certaines situations, il y a un doute sur la valeur $p$ donnée de la proportion et s'il s'agit de décider si l'hypothèse << la proportion a pour valeur $p$ >> est acceptable ou non. Pour cela, on prélève au hasard et avec remise un échantillon de $n$ individus de la population.

 \begin{Prop}
    On note $f$ la fréquence des individus de l'échantillon qui possède la particularité $\calig P$.\par
    Lorsque $f$ appartient à l'intervalle de fluctuation, on accepte l'hypothèse << la proportion a pour valeur $p$ >>, avec un risque de $5\%$.\par
    Sinon, on rejette l'hypothèse.
 \end{Prop}
 
 \begin{Exemple}
    Une publicité affirme que $80\%$ des personnes utilisant un certain produit ont des cheveux qui repoussent au cours du premier mois d'utilisation.\par
    On sélectionne au hasard $50$ personnes qui ont testé le produit. Parmi elles, $35$ déclarent que des cheveux ont repoussé au cours du premier mois d'utilisation.\par
    À partir de cet échantillon, peut-on accepter, au risque de $5\%$, l'affirmation de la publicité ?
 \end{Exemple}

\end{document}
