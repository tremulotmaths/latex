\documentclass[10pt,openright,twoside,french]{book}

\input philippe2013
\input philippe2013_activites

\pagestyle{empty}

\begin{document}

\TitreExo{\bsc{vi}.1}{Schéma de Bernoulli}

\exo Dans un magasin, une étude a montré que $5\%$ des clients qui entrent achètent un certain produit. On considère le choix de chaque client indépendant du choix des autres clients.\par
À l'ouverture, quatre clients entrent dans le magasin. On se demande combien d'entre eux vont acheter ce produit.
\begin{enumerate}
    \item Montrer qu'il s'agit ici d'un schéma de Bernoulli dont on précisera le << succès >> et les paramètres.
    \item Traduire cette situation par un arbre pondéré.
    \item Quelle est la probabilité qu'un et un seul de ces clients achète le produit ? On donnera le résultat arrondi à $10^{-2}$ près.
    \item Quelle est la probabilité qu'au moins un de ces clients achète le produit ? On donnera le résultat arrondi à $10^{-2}$ près.
\end{enumerate}\[*\]

\exo Durant une saison, l'un des meilleurs buteurs du Top 14 de rugby réussit la transformation d'un essai avec la probabilité de $0,83$. On s'intéresse au nombre d'essais transformés par ce joueur lors d'un match au cours duquel l'équipe marque $3$ essais. On considère que la réussite d'une transformation ne dépend pas de celle de la transformation précédente.
\begin{enumerate}
    \item Montrer qu'il s'agit ici d'un schéma de Bernoulli dont on précisera le << succès >> et les paramètres.
    \item Traduire cette situation par un arbre pondéré.
    \item Quelle est la probabilité que le joueur transforme exactement $2$ essais ? On donnera le résultat arrondi à $10^{-3}$ près.
    \item Quelle est la probabilité que le joueur transforme au moins $1$ essai ? On donnera le résultat arrondi à $10^{-3}$ près.
\end{enumerate}\[*\]

\exo Un jeune athlète pratique le saut en hauteur. Il se prépare à une série de trois compétitions. Pour chacune d'entre elles, il devra décider de la hauteur de son premier saut. Au vu de ses performances habituelles, il estimes à $0,95$ la probabilité de réussir un saut à $1,50$ mètre au premier essai. On considère que le résultat de ce premier essaie à une compétition est indépendant du résultat à l'autre compétition, le temps de repos entre deux compétitions étant suffisant.
\begin{enumerate}
    \item Quelle est la probabilité que l'athlète réussisse son premier essai lors des trois compétitions ? On donnera le résultat arrondi à $10^{-3}$ près.
    \item Quelle est la probabilité que l'athlète réussisse son premier essai dans au moins une des trois compétitions ? On donnera le résultat arrondi à $10^{-3}$ près.
\end{enumerate}\[*\]


\end{document} 