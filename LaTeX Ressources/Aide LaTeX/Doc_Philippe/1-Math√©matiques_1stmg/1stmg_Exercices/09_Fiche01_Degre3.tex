\documentclass[11pt,openright,twoside,french]{book}

\input philippe2013
\input philippe2013_activites
\pagestyle{empty}

\begin{document}

\TitreExo{\bsc{ix}.1}{Polynôme de degré 3}\bigskip

\exo \textbf{\'Etude de la propagation d'une maladie.}\par\medskip
Après l'apparition d'une maladie virale, les responsables de la santé publique ont estimé que le nombre de personnes frappées par la maladie au jour $t$ à partir du jour d'apparition du premier cas est :
\[M(t) = 45t^2 - t^3, \quad \text{pour}\quad t \in \intervalleff{0}{25}.\]

\begin{enumerate}
    \item Calculer $M(0)$ et $M(10)$ et interpréter les résultats.
\end{enumerate}

La vitesse de propagation de la maladie est assimilée à la dérivée du nombre de personnes malades en fonction de $t$.

\begin{enumerate}[resume]
    \item \begin{enumerate}
                    \item Calculer $M'(t)$.
                    \item En déduire la vitesse de propagation le cinquième jour et le dixième jour.
                    \item Déterminer le jour où la vitesse de propagation est maximale et calculer cette vitesse.
                    \item Déterminer les jours où la vitesse de propagation est supérieure à $600$ personnes par jour.
            \end{enumerate}
    \item \begin{enumerate}
                    \item \'Etudier le sens de variation de la fonction $M$ sur l'intervalle $\intervalleff{0}{25}$.
                    \item Interpréter les variations de $M$.
            \end{enumerate}
\end{enumerate}\[*\]

\exo \textbf{Rythme de production.}\par\medskip
Le nombre d'objets produits par une entreprise dépend de la quantité de travail fournie : c'est l'une des théorie de l'économiste \textbf{Taylor} (1856 - 1915).\par
On se propose d'étudier un modèle théorique, élaboré à partir de l'étude d'un grand nombre de chaînes de montage, pour connaître le temps de travail maximum que l'on peut demander aux ouvriers.\par
Soit $t$ le temps de travail fourni en heure, $t$ appartenant à l'intervalle $\intervalleff 0 5$.\par
La production, exprimée en nombre d'objets, est donnée en fonction de $t$ par :
\[P(t) = 100(-t^3 + 6t^2).\]

\begin{enumerate}
    \item
        \begin{enumerate}
            \item Calculer le nombre d'objets produits pour une heure de travail.
            \item Calculer $P(2)$ et interpréter le résultat.
            \item La quantité d'objets fabriqués est-elle proportionnelle au temps de travail ? Justifier la réponse.
        \end{enumerate}
    \item La dérivée de la production est le \textbf{rythme de production}, ou aussi \textbf{vitesse de production} par heure.
        \begin{enumerate}
            \item Déterminer $P'$.
            \item Calculer le rythme de production pour un temps de travail de $2$ heures.
            \item Dresser le tableau de variations de $P$ sur l'intervalle $\intervalleff{0}{5}$.
            \item Quel est le temps de travail qui engendre une production maximale ? Dans ces conditions, combien d'objets sont fabriqués ?
        \end{enumerate}
\end{enumerate}

\end{document} 