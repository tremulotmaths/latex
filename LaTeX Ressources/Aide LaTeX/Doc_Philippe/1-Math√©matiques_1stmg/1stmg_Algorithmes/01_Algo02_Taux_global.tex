\documentclass[10pt,openright,twoside,french]{book}

\usepackage{marvosym}
\input philippe2013
\input philippe2013_activites

\pagestyle{empty}

\begin{document}

\TitreAlgo{i.2}{Taux global}

Dans l'algorithme suivant, les évolutions sont données sous leur forme décimale et non pas sous forme de pourcentage.\par
Le but de l'algorithme est de calculer le taux d'évolution global $T$ connaissant les deux taux $T1$ et $T2$ successifs.\par\medskip

Compléter alors l'algorithme à l'aide du cours :

\begin{center}
\small
    \psframebox{
    \parbox{0.5\linewidth}{
        \textbf{Variables}

            \quad \ldots : un nombre réel

            \quad \ldots : un nombre réel
            
            \quad \ldots : un nombre réel

        \textbf{Entrée}

            \quad Saisir \ldots

            \quad \ldots

        \textbf{Traitement}

            \quad \ldots

        \textbf{Sortie}

            \quad \ldots
    }}
\end{center}


\end{document} 