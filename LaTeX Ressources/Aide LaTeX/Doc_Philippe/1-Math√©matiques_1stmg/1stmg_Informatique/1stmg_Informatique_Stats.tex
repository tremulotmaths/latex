\documentclass[10pt,french]{article}
\input preambule_2013
\pagestyle{empty}

\setlength{\textheight}{27cm}% Hauteur de la zone de texte
\setlength{\topmargin}{-2.5cm}% Marge en haut

\begin{document}
\small
\begin{center}
\psframebox[shadow=true,shadowcolor=gray!75,shadowsize=5pt,framearc=0.5,fillstyle=gradient,gradmidpoint=0.8,gradangle=20,gradbegin=red!80!yellow!40,gradend= white]{
\parbox[c]{0.5\linewidth}{
\begin{center}
\Large\bfseries
Séance informatique\par
\uuline{Statistiques}
\end{center}}}
\end{center}

\subsection*{\'Enoncé de départ}

On relève dans deux région de France le prix mensuel de location en euros et au \SI{}{m^2} de $30$ studios.\par\medskip

\begin{tabular}{*{10}{|c}|}
    \multicolumn{10}{c}{\textbf{Région 1}} \\
\hline
    12 & 12 & 16 &15 & 16 & 15 & 11 & 15 & 13 & 14 \\
\hline
    13 & 15 & 13 & 12 & 10 & 11 & 13 & 9 & 16 & 15 \\
\hline
    13 & 14 & 14 & 13 & 12 & 16 & 15 & 16 & 13 & 15 \\
\hline
\end{tabular}
\hfill
\begin{tabular}{*{10}{|c}|}
    \multicolumn{10}{c}{\textbf{Région 2}} \\
\hline
    11 & 12 & 14 & 12 & 13 & 12 & 11 & 14 & 13 & 14 \\
\hline
    13 & 14 & 13 & 13 & 11 & 11 & 13 & 13 & 12 & 14 \\
\hline
    12 & 11 & 14 & 13 & 14 & 11 & 14 & 13 & 14 & 12 \\
\hline
\end{tabular}\bigskip

Le ministre du logement nous demande une étude permettant de comparer les prix des studios dans ces deux régions.\bigskip

\begin{enumerate}
    \item Que signifie le $12$ en haut à gauche du premier tableau ?\par
    \tikz{\draw (0,0) rectangle (16.1,1);}
\end{enumerate}

\subsection*{Feuille de calcul 1 : indicateurs de position}

\begin{enumerate}
    \item Ouvrir Excel.
    \item Sélectionner la première feuille de calcul dans les onglets en bas à gauche.
    \item Recopier et présenter proprement les deux tableaux des deux régions.\par Utiliser la mise en forme du tableur pour votre présentation (changement des largeurs de colonnes, des hauteurs de lignes...).
    \item Sur le tableur toujours, quelques lignes en dessous, réaliser le tableau ci-dessous :\par
        \begin{tabular}{c|r|*{6}{>{\centering\arraybackslash}m{1.5cm}|}}
           \multicolumn{1}{c}{} & \multicolumn{1}{c}{\texttt{A}} & \multicolumn{1}{c}{\texttt{B}} & \multicolumn{1}{c}{\texttt{C}}  & \multicolumn{1}{c}{\texttt{D}} & \multicolumn{1}{c}{\texttt{E}} & \multicolumn{1}{c}{\texttt{F}} & \multicolumn{1}{c}{\texttt{G}} \\
            \cline{2-8}
            \texttt 10 &  & Moyenne & Minimum & Q1 & Médiane & Q3 & Maximum \\
            \cline{2-8}
            \texttt 11 & Région $1$ & &  &&&&\\
            \cline{2-8}
            \texttt 12 & Région $2$ & &  &&&&\\
        \end{tabular}
    \item Utiliser les commandes ci-dessous pour compléter le tableau à l'aide du tableur (remplacer \texttt{...} en sélectionnant les cellules qui vous intéressent) :
    \begin{description}
        \item[\texttt{=MOYENNE(...)}] permet de calculer la moyenne d'une série statistique.
        \item[\texttt{=MEDIANE(...)}] permet de calculer la médiane d'une série statistique.
        \item[\texttt{=QUARTILE(...;1)}] permet de calculer le premier quartile d'une série statistique. Pour le 3\ieme quartile, remplacer {\tt 1} par {\tt 3}.
        \item[\texttt{=MIN(...)} et \texttt{MAX(...)}] permet de calculer le minimum et le maximum d'une série statistique.
    \end{description}
    \item En utilisant la moyenne, comparer les deux régions.\par \tikz{\draw (0,0) rectangle (16.1,1);}
    \item Afin d'informer le ministre plus précisément, interpréter convenablement $Q_1$, la médiane et $Q_3$ pour chaque région.\par
    \tikz{\draw (0,0) rectangle (16.1,3);}
\end{enumerate}

\subsection*{Feuille de calcul 2 : comparaison graphique}
\begin{enumerate}
    \item Ouvrir la feuille de calcul 2 en cliquant sur le deuxième onglet en bas à gauche.
    \item Compléter à la main le tableau suivant :
    \begin{center}
        \begin{tabular}{*{10}{|c}|}
        \hline
            \textbf{Prix} & 09 & 10 & 11 & 12 & 13 & 14 & 15 & 16 & Total \\
        \hline
            \textbf{Effectif Région 1} & &&&&&&&& \\
        \hline
            \textbf{Effectig Région 2} & &&&&&&&& \\
        \hline
        \end{tabular}
    \end{center}
    \item Reproduire le tableau complété dans le tableur.
    \item À l'aide du tableur, effectuer un diagramme en bâtons représentant les deux séries statistiques.
\end{enumerate}

\end{document} 