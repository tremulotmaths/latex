\documentclass[10pt,openright,twoside,french]{book}

\input philippe2013
\input philippe2013_cours
\input philippe2013_sections
\input philippe2013_chapitre
\renewcommand\PartProgramme{Stats/Probas}
\renewcommand\MaCouleur{Melon!150}

\pieddepage{}{%
\begin{tikzpicture}[scale=0.65]
\shadedraw [top color=white, bottom color=\MaCouleur, draw=\MaCouleur]
[l-system={Sierpinski triangle, step=1pt, angle=60, axiom=F, order=6.5}]
lindenmayer system -- cycle;
\draw (30:0.65cm) node {\bfseries\textcolor{black}{\thepage}};
\end{tikzpicture}%
}{}

\setcounter{chapter}{6}

\begin{document}
\chapter[Dérivation]{Introduction \\ à la dérivation}\label{ch_derivation}

\section{Fonction dérivée d'une fonction polynôme du second degré}
\subsection{Définition}

\begin{Defi}
    Soit $f$ la fonction définie pour tout $x\in \R$ par $f(x) = ax^2 + bx + c$ avec $a \neq 0$.\par
    La \ipt{dérivée} de $f$ (ou fonction dérivée de $f$) est notée $f'$ et est définie pour tout $x \in \R$ par :
    \[f'(x) = 2ax + b.\]
\end{Defi}

\begin{Rmq}
    La fonction dérivée d'une fonction polynôme du second degré est une fonction affine.
\end{Rmq}

\begin{Exemple}
    Si $f(x) = 3x^2 - 4x + 7$ alors $f'(x) = 2 \times 3 x -4 = 6x - 4$.
\end{Exemple}

\subsection{Lien entre une fonction et sa dérivée}

\begin{Thm}[(admis)]
    Soit $f$ une fonction polynôme du second degré dont on note $f'$ la dérivée. $I$ est un intervalle.
    \begin{itemize}
        \item Si, pour tout $x \in I$, $f' (x)> 0$, alors $f$ est strictement croissante sur $I$.
        \item Si, pour tout $x \in I$, $f' (x)< 0$, alors $f$ est strictement décroissante sur $I$.
        \item Si $f'(x) = 0$, alors $f$ admet un maximum ou un minimum en $x$.
    \end{itemize}
\end{Thm}

\begin{Rmq}
    Puisque $f'(x) = 2ax + b$ alors $f'(x) = 0 \qLRq 2ax + b = 0 \qLRq x = \dfrac{-b}{2a}.$\par
    De plus, le signe de la fonction affine $f'$ dépend du signe du c{\oe}fficient directeur $2a$ et donc uniquement du signe de $a$.
\end{Rmq}\medskip

\begin{minipage}{0.45\linewidth}
\begin{center}
    $a > 0$\medskip

\begin{tikzpicture}[scale=0.8]
\small
    \tkzTab[nocadre]{$x$/1.5,Signe de \\ $2ax + b$/1.5,Variations de $f$/1.5}{$-\infty$,$\dfrac{-b}{2a}$,$+\infty$}{,-,z,+}{+/,-/,+/}
    \draw(5.5,-4.75) node {minimum};
\end{tikzpicture}
\end{center}
\end{minipage}\hfill
\begin{minipage}{0.45\linewidth}
\begin{center}
    $a < 0$\medskip

\begin{tikzpicture}[scale=0.8]
\small
    \tkzTab[nocadre]{$x$/1.5,Signe de \\ $2ax + b$/1.5,Variations de $f$/1.5}{$-\infty$,$\dfrac{-b}{2a}$,$+\infty$}{,+,z,-}{-/,+/,-/}
    \draw(5.5,-4.75) node {maximum};
\end{tikzpicture}
\end{center}
\end{minipage}

\section{Nombre dérivé et tangente}
\begin{Defi}
    Soit $f$ une fonction définie sur $\R$, $f'$ sa dérivée et $a \in \R$.\par
    On appelle \ipt{nombre dérivé}\index{dérivée!nombre} en $a$ le nombre égal à $f'(a)$.
\end{Defi}

\begin{Exemple}
    La fonction $f$ définie par $f(x) = -4x^2 + 5x -1$ admet pour dérivée $f'$ définie par $f'(x) = -8x + 5$.\par
    Le nombre dérivé en $4$ est alors égal à $f'(4) = -8 \times 4 + 5 = -27$.
\end{Exemple}

\begin{Defi}
    Soit $f$ une fonction. On note $\calig C_f$ sa courbe représentative dans un repère. On appelle $A(x_A\pv y_A)$ un point de $\calig C_f$.\par
    On appelle \ipt{tangente} au point $A$ la droite qui passe par $A$ et qui n'a aucun autre point commun avec $\calig C_f$ dans un voisinage très proche de $A$.
    \begin{center}
        \begin{tikzpicture}[scale=.5,xscale=2]
            \tkzInit[xmin=-3,xmax=3,ymin=-1,ymax=9,xstep=1,ystep=1]
            \tkzClip
            \tkzGrid[sub,subxstep=0.5,subystep=0.25,color=brown](-4,-1)(4,9)
            \tkzGrid
            \tkzDrawX\tkzDrawY
            \tkzSetUpPoint[shape=circle, size = 3, color=black, fill=lightgray]
            \tkzFct[line width=1pt,color=blue]{x**2}
            \tkzDrawTangentLine[color=red,line width=0.75pt](1)
            \tkzDefPointByFct[draw,ref=A](1)
            \tkzLabelPoint[above](A){$A$}
        \end{tikzpicture}
    \end{center}
\end{Defi}

\begin{Prop}[(admise)]
    Soit $f$ une fonction. On note $\calig C_f$ sa courbe représentative dans un repère. On note $A(x_A\pv y_A)$ un point de $\calig C_f$ et $(T)$ la tangente en $A$.\par
    Le c{oe}fficient directeur de $(T)$ est alors égale à $f'(x_A)$.
\end{Prop}

\begin{Exemple}
    Soit $f$ la fonction définie sur $\R$ par $f(x) = -2x^2 + x + 1$.\par
    On veut connaître l'équation réduite de la tangente $(T)$ au point $A$ d'abscisse $-1$.\par\smallskip
    L'ordonnée de $A$ est égale à $f(-1) = -2 \times (-1)^2 + (-1) + 1 = -2$.\par\smallskip
    L'équation réduite est de la forme $y = mx + p$ avec $m = f'(-1)$. On calcule donc l'expression de $f'$ :
        $f'(x) = 2 \times (-2x) + 1 = -4x + 1$ donc $m = -4 \times (-1) + 1 = 5$.\par\smallskip
    Pour finir, on sait que $A\in (T)$ donc ses coordonnées vérifient l'équation réduite de $(T)$ :
    \[y_A = 5x_A + p \qLRq -2 = 5 \times (-1) + p \qLRq -2 + 5 = p \qLRq p = 3.\]
    L'équation réduite de $T$ est donc : $y = 5x + 3$.
\end{Exemple}
    \begin{center}
        \begin{tikzpicture}[scale=.5,xscale=2]
            \tikzset{tan style/.style={-}}
            \tkzInit[xmin=-2,xmax=2,ymin=-10,ymax=1.5,xstep=1,ystep=1]
            \tkzClip
            \tkzGrid[sub,subxstep=0.5,subystep=0.25,color=brown](-2,-10)(2,1.5)
            \tkzGrid
            \tkzDrawX\tkzDrawY
            \tkzSetUpPoint[shape=circle, size = 3, color=black, fill=lightgray]
            \tkzFct[line width=1pt,color=blue]{(-1)*2*x**2 + x + 1}
            \tkzDrawTangentLine[line width=0.75pt,color=red,kl=0.6,kr=0.6](-1)
            \tkzDefPointByFct[draw,ref=A](-1)
            \tkzLabelPoint[above left](A){\footnotesize $A$}
        \end{tikzpicture}
    \end{center}


\end{document}
