\documentclass[10pt,french]{article}

\input preambule_2013
%\usepackage{marvosym,slashbox}
\RegleEntete


\newcommand\competences{
\setcounter{exo}{0}
\begin{tabular}{ll} Nom : \\[5pt] Prénom : \end{tabular}
\hfill
\textbf{Note :}\renewcommand\arraystretch{2.3}
\begin{tabularx}{0.17\linewidth}{|X|}
\hline
\slashbox{\Huge\bfseries\phantom{10}}{\Huge\bfseries 10}\\
\hline
\end{tabularx}\renewcommand\arraystretch{1}
\[***\]
}

\entete{\premiere \stmg}{Calculs d'évolutions}{A}
\pieddepage{}{}{}


\begin{document}
\competences

\textbf{\bsc{Si nécessaire, les résultats seront arrondis au centième près} (2 chiffres après la virgule).}\bigskip

Le tableau ci-dessous donne la consommation d'eau minérale dans un pays : il s'agit de la consommation moyenne en litres par personne sur une année.

\begin{center}
\renewcommand\arraystretch{1.5}
    \begin{tabular}{*{5}{|c}|}
    \hline
        Année & $\np{1980}$ & $\np{1990}$ & $\np{2000}$ & $\np{2010}$ \\
    \hline
        Consommation (en L) & $40$ & $55$ & $90$ & $149$ \\
    \hline
    \end{tabular}
\renewcommand\arraystretch{1}
\end{center}\medskip

\begin{enumerate}
    \item Calculer la variation absolue entre l'année $\np{1980}$ et l'année $\np{1990}$.\vspace{2cm}
    \item Calculer la variation relative entre l'année $\np{1980}$ et l'année $\np{2000}$.\vspace{2cm}
    \item Calculer la variation absolue et le taux d'évolution entre l'année $\np{1980}$ et $\np{2010}$.\vspace{2cm}
\end{enumerate}

En $\np{2020}$, le pays prévoit une moyenne égale à $130~L$ par personne.

\begin{enumerate}[resume]
    \item Compléter les phrases suivantes en écrivant les calculs nécessaires (attention aux signes) :\par
    {\cursive Entre 2\,010 et 2\,020, la variation absolue est égale à \ldots} \par\medskip
    \textbf{Calculs :}\vspace{2cm}
        
    {\cursive Entre 2\,010 et 2\,020, la variation relative est égale à \ldots \par\medskip}
    \textbf{Calculs :}\vspace{2cm}
\end{enumerate}

\clearpage

%--------------------------------------------------------------------------------------------------------------------------------------------------------------------------
%                           SUJET B
%--------------------------------------------------------------------------------------------------------------------------------------------------------------------------

\entete{\premiere \stmg}{Calculs d'évolutions}{B}
\competences

\textbf{\bsc{Si nécessaire, les résultats seront arrondis au centième près} (2 chiffres après la virgule).}\bigskip

Le tableau ci-dessous donne la consommation d'eau minérale dans un pays : il s'agit de la consommation moyenne en litres par personne sur une année.

\begin{center}
\renewcommand\arraystretch{1.5}
    \begin{tabular}{*{5}{|c}|}
    \hline
        Année & $\np{1980}$ & $\np{1990}$ & $\np{2000}$ & $\np{2010}$ \\
    \hline
        Consommation (en L) & $40$ & $55$ & $90$ & $149$ \\
    \hline
    \end{tabular}
\renewcommand\arraystretch{1}
\end{center}\medskip

\begin{enumerate}
    \item Calculer la variation absolue entre l'année $\np{1990}$ et l'année $\np{2000}$.\vspace{2cm}
    \item Calculer la variation relative entre l'année $\np{1990}$ et l'année $\np{2010}$.\vspace{2cm}
    \item Calculer la variation absolue et le taux d'évolution entre l'année $\np{1980}$ et $\np{2000}$.\vspace{2cm}
\end{enumerate}

En $\np{2020}$, le pays prévoit une moyenne égale à $120~L$ par personne.

\begin{enumerate}[resume]
    \item Compléter les phrases suivantes en écrivant les calculs nécessaires (attention aux signes) :\par
    {\cursive Entre 2\,010 et 2\,020, la variation absolue est égale à \ldots} \par\medskip
    \textbf{Calculs :}\vspace{2cm}

    {\cursive Entre 2\,010 et 2\,020, la variation relative est égale à \ldots \par\medskip}
    \textbf{Calculs :}\vspace{2cm}
\end{enumerate}

\end{document} 