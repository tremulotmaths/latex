%___________________________
%===    Configurations perso 08.07.2013
%------------------------------------------------------
\usepackage{etex}
%___________________________
%===    Pour le français
%------------------------------------------------------
\usepackage[utf8]{inputenc}
\usepackage[T1]{fontenc}
\usepackage{babel}
\frenchbsetup{CompactItemize=false}
\FrenchFootnotes

%___________________________
%===    Polices d'écriture
%------------------------------------------------------
%\usepackage[noDcommand]{kpfonts}
\usepackage{mathpazo}
%\usepackage{ccfonts}
%\usepackage[sloped,poorman]{fourier}
%\usepackage{fourier-orns}
\usepackage{frcursive} % Pour l'écriture cursive

%___________________________
%===    Les couleurs
%------------------------------------------------------
\usepackage[dvipsnames,table]{xcolor}
%
\newcommand{\Cyan}[1]{{\color{cyan} #1}}
\newcommand{\Yellow}[1]{{\color{yellow} #1}}
\newcommand{\Magenta}[1]{{\color{magenta} #1}}
\newcommand{\bleu}[1]{{\color{blue} #1}}
\newcommand{\rouge}[1]{{\color{red} #1}}
\newcommand{\verte}[1]{{\color{green} #1}}
\newcommand{\Purple}[1]{{\color{purple} #1}}
\newcommand{\Orange}[1]{{\color{orange} #1}}
\newcommand{\violet}[1]{{\color{violet} #1}}

\definecolor{midblue}{rgb}{0.145,0.490,0.882}
\newcommand\MaCouleur{midblue}
%___________________________
%===    Réglages mise en page
%------------------------------------------------------
\usepackage{geometry,atbegshi}
\geometry{a4paper, height = 25.5cm,hmargin=2.5cm,marginparwidth=2cm,headheight=20pt,headsep=16.5pt,bottom=2cm,footskip=30pt,footnotesep=30pt}
\usepackage{lscape}
\usepackage{xspace}
\setlength{\parindent}{0pt}

\usepackage{fancyhdr}
\pagestyle{fancy}
%
\newcommand\RegleEntete[1][0.4pt]{\renewcommand{\headrulewidth}{#1}}
\renewcommand{\headrulewidth}{0pt}
\newcommand{\entete}[3]{\lhead{#1} \chead{\sffamily\textbf{#2}} \rhead{#3}}
\newcommand{\pieddepage}[3]{\lfoot{#1} \cfoot{#2} \rfoot{#3}}
%
\renewcommand{\chaptermark}[1]{\markboth{#1}{}}
    %en-tete droite page [paire] et {impaire}
%\rhead[]{\textbf{\leftmark.}}
    %en-tete gauche page [paire] et {impaire}
%\lhead[\textbf{\chaptername~\thechapter.}]{}
\entete{}{\color{\MaCouleur} \textbullet~\leftmark~\textbullet}{}
%
\pieddepage{}{\color{\MaCouleur}$\stackrel{***}{\thepage}$}{}
%
\fancypagestyle{plain}{ \fancyhead{} \renewcommand{\headrulewidth}{0pt}}

\usepackage{enumerate}
\usepackage{enumitem}
\setenumerate[1]{font=\bfseries,label=\arabic*\degres)}
\setenumerate[2]{font=\itshape,label=(\alph*)}

%___________________________
%===    Raccourcis classe
%------------------------------------------------------
\newcommand\seconde{2\up{nde}\xspace}
\newcommand\premiere{1\up{ère}\xspace}
\newcommand\stmg{\bsc{Stmg}}
\newcommand\sti{\bsc{Sti2d}}


%___________________________
%===    Réglages Maths
%------------------------------------------------------
\everymath{\displaystyle}
\usepackage{yhmath}
\usepackage[euler-digits]{eulervm} %-> police maths
%
\usepackage{amssymb,amsmath,mathtools}
%\usepackage{stmaryrd} %\llbracket et \rrbracket % crochets doubles pour intervalles d'entier
\newcommand{\crochets}[2]{\ensuremath{\llbracket #1 ; #2 \rrbracket}}

\newcommand{\intervalle}[4]{\mathopen{#1}#2\mathclose{}\mathpunct{};#3\mathclose{#4}}
\newcommand{\intervalleff}[2]{\intervalle{[}{#1}{#2}{]}}
\newcommand{\intervalleof}[2]{\intervalle{]}{#1}{#2}{]}}
\newcommand{\intervallefo}[2]{\intervalle{[}{#1}{#2}{[}}
\newcommand{\intervalleoo}[2]{\intervalle{]}{#1}{#2}{[}}

\usepackage{bm} % pour l'écriture en gras
\newcommand{\bs}[1]{\ensuremath{\bm{#1}}}
\usepackage{cancel} % pour les simplifications de fractions
\renewcommand\CancelColor{\red}
%\usepackage{siunitx} % écriture de nombres et d'unités
%\sisetup{output-decimal-marker={,},detect-all}
\usepackage[autolanguage,np]{numprint}
\newcommand\NP[1]{\ensuremath\numprint{#1}}
%
\usepackage{dsfont} %écriture des ensemble N, R, C ...
\newcommand{\C}{\mathds C}
\renewcommand{\Re}{\mathfrak{Re}}
\renewcommand{\Im}{\mathfrak{Im}}
\newcommand{\R}{\mathds R}
\newcommand{\Q}{\mathds Q}
\newcommand{\D}{\mathds D}
\newcommand{\Z}{\mathds Z}
\newcommand{\N}{\mathds N}
\newcommand\Ind{\mathds 1} %= fonction indicatrice
\newcommand\p{\mathds P} %= probabilité
\newcommand\E{\mathds E} % Espérance
\newcommand\V{\mathds V} % Variance
%
\usepackage{mathrsfs}   % Police de maths jolie caligraphie
\newcommand{\calig}[1]{\ensuremath{\mathscr{#1}}}
\newcommand\mtc[1]{\ensuremath{\mathcal{#1}}}
%
\renewcommand\leq{\ensuremath{\leqslant}}
\renewcommand\geq{\ensuremath{\geqslant}}

%___________________________
%===    Pour les tableaux
%------------------------------------------------------
\usepackage{array}
\usepackage{longtable}
\usepackage{tabularx,tabulary}
\usepackage{multirow}
\setlength\columnseprule{0.4pt}

%___________________________
%===    Divers packages
%------------------------------------------------------
\usepackage{bclogo}
%\usepackage[tikz]{bclogo}
\usepackage{textcomp}
\usepackage{multicol}
\usepackage{eurosym}
\usepackage{soul} % Pour souligner : \ul
\usepackage{ulem} % Pour souligner double : \uuline
                      % Pour souligner ondulé : \uwave
                      % Pour barrer horizontal : \sout
                      % Pour barrer diagonal : \xout
\usepackage{tikz,tkz-tab}
\usetikzlibrary{calc,shapes,arrows,plotmarks,lindenmayersystems,decorations,decorations.pathreplacing,intersections}
\pgfdeclarelindenmayersystem{Sierpinski triangle}{
  \rule{F -> G-F-G}
  \rule{G -> F+G+F}}

\usepackage{marvosym}
\usepackage{slashbox}

%___________________________
%===    Quelques raccourcis perso
%------------------------------------------------------
\newcommand\pfr[1]{\psframebox[linecolor=red]{#1}}
\newcommand\coef[1][]{c{\oe}fficient#1\xspace}
\newcommand{\vect}[1]{\ensuremath{\overrightarrow{#1}}}
\newcommand\abs[1]{\ensuremath{\left\vert #1 \right\vert}}
\newcommand\Arc[1]{\ensuremath{\wideparen{#1}}}

\def\OIJ{\ensuremath{\left(O \pv I,~J\right)}}
\def\Oij{\ensuremath{\left(O~;~\vect{\imath},~\vect{\jmath}\right)}}
\def\Ouv{\ensuremath{\left(\text{O}~;~\vect{u},~\vect{v}\right)}}

\newcounter{exo}
\newcommand\exo{
\refstepcounter{exo}
\Writinghand\ \textbf{Exercice \theexo.}\par
}

%___________________________
%===    Gestion des espaces
%------------------------------------------------------
\newcommand{\pv}{\ensuremath{\: ; \,}}
\newlength{\EspacePV}
\setlength{\EspacePV}{1em plus 0.5em minus 0.5em}
\newcommand{\qq}{\hspace{\EspacePV} ; \hspace{\EspacePV}}
\newcommand{\qetq}{\hspace{\EspacePV} \text{et} \hspace{\EspacePV}}
\newcommand{\qLq}{\hspace{\EspacePV} \Leftrightarrow \hspace{\EspacePV}}
\newcommand{\qRq}{\hspace{\EspacePV} \Rightarrow \hspace{\EspacePV}}
\newcommand{\qLRq}{\hspace{\EspacePV} \Leftrightarrow \hspace{\EspacePV}}

\usepackage{siunitx}
\sisetup{output-decimal-marker={,}}
\newcommand\I{\ensuremath{\num{i}}}

%___________________________
%===    Mise en forme du sommaire
%------------------------------------------------------
\addto\captionsfrench{%
\renewcommand\contentsname{
\color{\MaCouleur}\hfill\Huge\bfseries \uuline{Sommaire}
}}
\setcounter{tocdepth}{0}

\usepackage[dotinlabels]{titletoc}
\titlecontents{chapter}[6pc]
    {\addvspace{0.8pc}\bfseries}
    {\contentslabel
        [\thecontentslabel]{1.5pc}}
     {}{\hfill\contentspage}
     [\addvspace{0.5pt}]

\usepackage[dotinlabels]{titletoc}
\titlecontents{part}[0pt]
    {\addvspace{1pc}\Large\bfseries}
    {\Huge\contentslabel
        [\thecontentslabel]{0pt}}
     {}{}
     [\addvspace{0.5pt}]

\titlecontents{section}[3.5pc]
    {\addvspace{0pt}}%
    {\contentslabel[\thecontentslabel]{2.5pc}}%\contentspush{\thecontentslabel\ }}
    {}{\titlerule*[15pt]{.}\contentspage}
     []

%___________________________
%===    Mise en forme de l'index
%------------------------------------------------------
\usepackage{makeidx}
\renewenvironment{theindex}
{\clearpage \chapter*{\sffamily\fontsize{36}{40}\selectfont\hfill Index\\ \hfill des notions définies}
\renewcommand{\item}{\par\hangindent 40pt}
\begin{multicols}{2}
}
{\end{multicols}}
\makeindex
%
\newcommand\ipt[1]{\textcolor{red}{\textbf{#1}}\protect\index{#1@\lowercase{#1}}}
\newcommand\iptb[1]{\textcolor{red}{\textbf{#1}}}
%


\usepackage[pdfborder={0 0 0},bookmarksnumbered]{hyperref} 