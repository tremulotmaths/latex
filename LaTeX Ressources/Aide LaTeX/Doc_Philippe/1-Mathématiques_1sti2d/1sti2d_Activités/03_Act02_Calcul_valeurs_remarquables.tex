\documentclass[10pt,openright,twoside,french]{book}
\input preambule_2013

\newcommand\TitreActivite[2]{%
    \setcounter{exo}{0}
        \begin{center}
            \psframebox[shadow=true,shadowcolor=gray!75,shadowsize=3pt,%
            framearc=0.3,%
            fillstyle=gradient,gradmidpoint=0.8,gradangle=20,gradbegin=red!60!yellow!40,gradend= white]{%
                \parbox{0.5\linewidth}{%
                    \begin{center}
                        \Large\bfseries
                        \uuline{Activité \bsc{#1}}\par
                        #2
                    \end{center}}}
        \end{center}\bigskip
}

\pagestyle{empty}


\begin{document}

\TitreActivite{iii.2}{Valeurs remarquables \par \'Equations}

\exo On considère un repère $\OIJ$ et le cercle trigonométrique $\calig U$. On place le point $M$ sur $\calig U$ tel que $\widehat{IOM} = \rad{\dfrac{\pi}{3}}$.\par
On veut démontrer que $\cos\left(\dfrac\pi3\right) = \dfrac12$ et $\sin\left(\dfrac\pi3\right) = \dfrac{\sqrt 3}{2}$.

\begin{enumerate}
    \item Faire une figure.
    \item Démontrer que le triangle $MIO$ est un triangle équilatéral.
    \item On appelle $H$ le projeté orthogonal de $M$ sur l'axe des abscisse.
        \begin{enumerate}
            \item Quel lien existe-t-il entre le point $H$ et $\cos\left(\dfrac\pi3\right)$ ?
            \item Que représente la droite $(MH)$ pour le triangle $OMI$ (plusieurs réponses attendues) ? Justifier.
            \item En déduire la valeur de $\cos\left(\dfrac\pi3\right)$.
        \end{enumerate}
    \item On appelle $K$ le projeté orthogonal de $M$ sur l'axe des ordonnées.
        \begin{enumerate}
            \item Quel lien existe-t-il entre le point $K$ et $\sin\left(\dfrac\pi3\right)$ ?
            \item Expliquer pourquoi $OK = MH$.
            \item Dans le triangle $OHM$ rectangle en $H$, calculer la valeur exacte de la longueur $MH$.
            \item En déduire la valeur de $\sin\left(\dfrac\pi3\right)$.
        \end{enumerate}
\end{enumerate}\[*\]

\exo En s'appuyant sur une représentation dans le cercle trigonométrique, résoudre les équations suivantes \textbf{dans l'intervalle donné} :
\begin{enumerate}
    \item $\cos(x) = \dfrac{\sqrt 3}{2}$ avec $x \in \intervalleof{-\pi}{\pi}$.
    \item $\cos(x) = \cos\left(\dfrac{\pi}{8}\right)$ avec $x \in \R$.
    \item $\sin(x) = -\dfrac{\sqrt 2}{2}$ avec $x \in \intervalleof{-\pi}{\pi}$.
    \item $\sin(x) = \sin\left(-\dfrac{2\pi}{5}\right)$ avec $x \in \R$.
\end{enumerate}\[*\]

\exo
\begin{enumerate}
    \item Résoudre dans $\intervalleof{-\pi}{\pi}$ l'équation $\cos(x) =\dfrac 12$.
    \item Dans un repère orthonormal $\Oij$, tracer le cercle trigonométrique $\calig U$. Placer les points $M$ et $M'$ de $\calig U$ repérés par les solutions de l'équation de la question précédente.
    \item En rouge, repasser la partie de $\calig U$ dont les points ont une abscisse supérieure à $\dfrac 12$.
    \item En déduire les solutions de l'inéquation $\cos(x) > \dfrac 12$ dans $\intervalleof{-\pi}{\pi}$.
\end{enumerate}
\[***\]

\end{document} 