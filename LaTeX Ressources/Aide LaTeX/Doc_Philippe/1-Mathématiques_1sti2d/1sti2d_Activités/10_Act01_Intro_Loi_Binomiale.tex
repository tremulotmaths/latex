\documentclass[12pt,openright,twoside,french]{book}

\input philippe2013
\input philippe2013_activites
\pagestyle{empty}


\begin{document}

\TitreActivite{x.1}{La loi binomiale}

Un automobiliste passe tous les jours à la même intersection en partant de chez lui. Cette intersection est équipée d’un feu tricolore.\par On appelle succès l’événement $S$ : << l’automobiliste arrive devant le feu alors qu’il est au vert >> . On suppose que la probabilité de $S$ est $p(S) = \frac 3 5$.\bigskip

\begin{enumerate}
    \item On étudie la situation deux jours de suite.
        \begin{enumerate}
            \item Quels sont les paramètres du schéma de Bernoulli décrit dans la situation ?
            \item Déterminer à l’aide d’un arbre tous les résultats possibles. Donner la probabilité de chacun de ces résultats.
            \item On note $X$ la fonction qui, à chacun des résultats, associe le nombre de fois où le feu est vert (c'est-à-dire le nombre de succès). On dit que $X$ est une \textbf{variable aléatoire}.\par
                Donner la liste des valeurs prises par $X$.\par
            \item Calculer la probabilité de chacune de ces valeurs et regrouper ces probabilités dans un tableau.\par
                On dit que l’on a établi la \textbf{loi de probabilité} de la variable aléatoire $X$.
        \end{enumerate}
    
    \item On étudie la situation trois jours de suite. On appelle $Y$ la variable aléatoire qui associe à chaque issue le nombre de succès.\par
    \'Etablir la loi de probabilité de $Y$.
    
    \item On étudie la situation quatre jours de suite. On appelle $Z$ la variable aléatoire qui associe à chaque issue le nombre de succès.\par
    \'Etablir la loi de probabilité de $Z$.

\end{enumerate}
\end{document} 