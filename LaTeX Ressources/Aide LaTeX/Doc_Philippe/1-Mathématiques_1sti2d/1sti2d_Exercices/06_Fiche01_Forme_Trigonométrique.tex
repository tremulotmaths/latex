\documentclass[10pt,openright,twoside,french]{book}

\input philippe2013
\input philippe2013_activites

\pagestyle{empty}

\begin{document}

\TitreExo{\bsc{vi}.1}{Nombres complexes \\ Module et argument}

\exo Le plan est muni d'un repère orthonormé direct \Ouv.\par
Les points $A$ et $B$ ont pour affixes respectives $z_A = \sqrt 6 + \I\sqrt2$ et $z_B = \sqrt 5 + \I\sqrt{15}$.
\begin{enumerate}
    \item Démontrer que $\sqrt{8} = 2\sqrt 2$ puis déterminer un argument de $z_A$.
    \item Démontrer que $\sqrt{20} = 2\sqrt 5$ puis déterminer un argument de $z_B$.
    \item En déduire une mesure de l'angle $\left(\vect{OA} \pv \vect{OB}\right)$.
    \item Calculer $\arg\left(\frac{z_B}{z_A}\right)$.
\end{enumerate}\[*\]

\exo Le plan est muni d'un repère orthonormé direct \Ouv.
\begin{enumerate}
    \item Déterminer et représenter l'ensemble $(E_1)$ des points $M$ d'affixe $z$ tels que :\[\abs{z - 2\I} = 1.\]
    \item Déterminer et représenter l'ensemble $(E_2)$ des points $M$ d'affixe $z$ tels que :\[\abs{z - 2\I} = \abs{z + 4 - \I}.\]
    \item Déterminer et représenter l'ensemble $(E_3)$ des points $M$ d'affixe $z$ tels que :\[\abs{2z - 8 + 2\I} = 8.\]
    \item Déterminer et représenter l'ensemble $(E_4)$ des points $M$ d'affixe $z$ tels que :\[\abs{z - 3 + \I} = \abs{z + 4 - 2\I}.\]
\end{enumerate}\[*\]

\exo Le plan est muni d'un repère orthonormé direct \Ouv.\par
On considère les points $A$, $B$ et $C$ définis par leurs affixes respectives :
\[z_A = -2 \qq z_B = 1 + \I\sqrt 3 \qetq z_C = 1 - \I\sqrt 3.\]
\begin{enumerate}
    \item Montrer que le triangle $ABC$ est équilatéral.
    \item Calculer les longueurs $OA$, $OB$ et $OC$.
    \item Justifier précisément ce que représente le point $O$ pour le triangle $ABC$ ?
    \item $D$ est le point d'affixe $z_D$ tel que $ABCD$ est un parallélogramme. Calculer $z_D$.
    \item Donner la nature précise du quadrilatère $ABCD$ en justifiant la réponse.
\end{enumerate}\[*\]

\exo Le plan est muni d'un repère orthonormé direct \Ouv.\par
On considère les points $A$, $B$, $E$  et $F$ définis par leurs affixes respectives :
\[z_A = 5 - 5\I \qq z_B = 3 + 3\I \qq z_E = 7 - 4\I \qetq z_F = 5 + 3\I.\]
\begin{enumerate}
    \item Faire une figure.
    \item Déterminer l'écriture trigonométrique des nombres complexes $z_A$ et $z_B$.
    \item En déduire la nature du triangle $OAB$.
    \item Soit $\calig C$ le cercle circonscrit au triangle $OAB$ de centre $I$. Déterminer $z_I$, affixe de $I$.
    \item Les points $E$ et $F$ appartiennent-ils à $\calig C$ ?
\end{enumerate}

\end{document} 