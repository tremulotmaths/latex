\documentclass[12pt,openright,twoside,french]{book}

\input philippe2013
\input philippe2013_activites

\geometry{a4paper, height = 26.5cm,hmargin=1.5cm,marginparwidth=2cm,headheight=20pt,headsep=16.5pt,bottom=2cm,footskip=30pt,footnotesep=30pt}

\pagestyle{empty}

\begin{document}
\small

\TitreExo{\bsc{xiii}.1}{\'Echantillonage \\ Prise de décision}

\exo
Un rayon lumineux qui traverse une plaque de verre teinté perd $15\%$ de son intensité lumineuse, mesurée en candelas.
\begin{enumerate}
	\item On note $I_0$ l'intensité lumineuse d'un rayon avant la traversée d'une plaque et $I_1$ son intensité à la sortie. Calculer $I_1$ lorsque $I_ 0 = 18$.
	\item On superpose que $n$ plaques de verre identiques et on note $I_n$ l'intensité du rayon lumineux après avoir traveré la $n\ieme$ plaque.\par
	Montrer que $\left(I_n\right)$ est une suite géométrique et préciser sa raison.
	\item Exprimer alors $I_n$ en fonction de $n$.
	\item Trouver l'intensité lumineuse d'un rayon dont l'intensité après avoir traversé $5$ plaques est égale à $20$.
	\item Déterminer, à l'aide de la calculatrice, le nombre minimum de plaques à utiliser pour que l'intensité d'un rayon sortant soit inférieure au dixième de l'intensité d'un rayon entrant.
\end{enumerate}\[*\]

\exo En $\np{2010}$, une voiture neuve vaut \EUR{$\np{22000}$}. On désigne par $P_n$ sa valeur en l'année $\np{2010} + n$ : on a donc $P_0 = \np{22000}$.\par
La valeur de ce véhicule une année donnée est égale à la valeur de l'année précédente diminuée de $20\%$ à laquelle on rajoute \EUR{$500$}.

\begin{enumerate}
	\item Calculer $P_1$, $P_2$ et $P_3$.
	\item \begin{enumerate}
	      	\item Exprimer $P_{n+1}$ en fonction de $P_n$.
	      	\item Pour tout $n \in \N$, on pose $Q_n = P_n - 2500$. Montrer que $(Q_n)$ est une suite géométrique.
	      	\item Donner le terme général de $(Q_n)$ en fonction de $n$, puis en déduire l'expression de $P_n$ en fonction de $n$.
	      \end{enumerate}
	\item À l'aide de la calculatrice, déterminer à partir de quelle année la valeur de cette voiture sera inférieure à \EUR{$5000$}.
	\item À l'aide de la calculatrice, lire les valeurs de $P_n$ pour $n$ allant de $0$ à $30$. Que constate-t-on ?
\end{enumerate}\[*\]

\exo$P$ représente une population de coccinelles de $\np{3000}$ insectes qui augmente de $4\%$ tous les ans.\par
Quel est la but de l'algorithme suivant ?\medskip

{\obeylines
\textbf{Variables}
	\quad $N$ : nombre entier positif
	\quad $P$ : nombre entier positif \smallskip
\textbf{Initialisations}
	\quad $N$ prend la valeur $0$\smallskip
\textbf{Entrées}
	\quad Saisir $P$\smallskip
\textbf{Traitement}
	\quad Tant que $P\leq 6000$
		\qquad $P$ prend la valeur $1,04 \times P$
		\qquad $N$ prend la valeur $N+1$
	\quad FIN Tant que\smallskip
\textbf{Sorties}
	\quad Afficher $P$
	\quad Afficher $N$
}


\end{document} 