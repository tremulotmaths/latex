\documentclass[10pt,openright,twoside,french]{book}

\usepackage{marvosym}
\input philippe2013
\input philippe2013_activites

\pagestyle{empty}

\begin{document}

\TitreAlgo{i.1}{Calculer des indicateurs\par sur une liste de nombres}

On considère une liste de nombre, appelée par exemple \texttt{Valeurs}. On la "remplit" avec les nombres que l'on souhaite :
\begin{center}
    \texttt{Valeurs = [10 ; 15 ; 12,5 ; 7 ; 9,5 ; 11]}
\end{center}
Dans un algorithme, le premier terme de la liste s'écrit \texttt{Valeurs[1]}, le deuxième \texttt{Valeurs[2]} et, en règle générale, le i\ieme s'écrit \texttt{Valeurs[i]}. Ici, on a par exemple \texttt{Valeurs[3] = 12,5}.\medskip

L'algorithme ci-dessous calcule la moyenne des nombres de la liste. La première boucle \texttt{Pour} sert à enregistrer les nombres de la liste les uns à la suite des autres.

\begin{center}
\small
    \psframebox{
    \parbox{0.5\linewidth}{
        \textbf{Variables}

            \quad $Valeurs$ : une liste de nombre

            \quad $i$ : un entier naturel

            \quad $s$ : un nombre réel

            \quad $m$ : un nombre réel

            \quad $N$ : un entier naturel

        \textbf{Initialisation}

            \quad $s$ prend la valeur $0$

        \textbf{Entrée}

            \quad Afficher "Combien de nombres dans la liste ?"

            \quad Saisir $N$

        \textbf{Traitement}

            \quad Pour $i$ allant de $1$ à $N$

            \qquad Saisir $Valeurs[i]$

            \quad FinPour

            \quad Pour $i$ allant de $1$ à $N$

            \qquad Affecter à $s$ la valeur $s + Valeurs[i]$

            \quad FinPour

            \quad Affecter à $m$ la valeur $s/N$

        \textbf{Sortie}

            \quad Afficher $m$
    }}
\end{center}

\begin{enumerate}
    \item Quelle variable est affectée à la moyenne ?
    \item À la fin de la deuxième boucle \texttt{Pour}, quelle valeur est finalement affectée à la variable $s$ ?
    \item Modifier légèrement l'algorithme ci-dessus pour qu'il calcule la variance de la liste de nombre.
    \item Après l'affichage de la variance, faire afficher l'écart-type.
\end{enumerate}

\end{document} 