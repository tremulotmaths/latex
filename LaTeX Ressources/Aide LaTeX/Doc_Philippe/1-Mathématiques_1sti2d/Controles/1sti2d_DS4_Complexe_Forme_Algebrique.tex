\documentclass[10pt,french]{book}
\input philippe2013

\newcounter{exoc}
\newenvironment{exoc}[1]{%
  \refstepcounter{exoc}\textbf{Exercice \theexoc} :\hfill {\textbf{#1}}\par
  \medskip}%
{\medskip}

\begin{document}

\pagestyle{empty}

\begin{center}
\renewcommand\arraystretch{1.5}
\begin{tabularx}{\textwidth}{|>\centering m{2.5cm}|>\centering X|>{\centering\arraybackslash} m{2.5cm}|}
	\hline
		1\iere \bsc{e.e.a.c.} & Mardi 17 décembre \np{2013} & \textbf{Forme algébrique} \\
	\hline
		\multicolumn{3}{|c|}{\bsc{Contrôle de mathématiques}} \\
	\hline
        \multicolumn{1}{|r}{\bsc{Nom}:} & \multicolumn{2}{l|}{} \\
		\multicolumn{1}{|r}{Prénom:} & \multicolumn{2}{l|}{} \\
	\hline
        \multicolumn{3}{|l|}{\bfseries Note et observations :} \\[1cm]
    \hline
\end{tabularx}\bigskip
\renewcommand\arraystretch{1}

{\itshape
La qualité et la précision de la rédaction seront prises en compte dans l'appréciation des copies.\par
Le barème est indicatif.}
\end{center}

\begin{exoc}{1 + 1 + 1 + 1 + 1 = 5 pts}
On considère les nombres complexes suivants :
\[z_1 = 1 + 3\I \qetq z_2 = 2 - 2\I.\]
\'Ecrire les nombres complexes suivants sous forme algébrique :
\[a = z_1 + z_2 \qq b = z_2 - 2\overline{z_1} \qq c = z_1 \times z_2 \qq d = (1 - z_1)(1 + z_2) \qq e = \frac{z_1}{z_2}\]
\end{exoc}

\begin{exoc}{2 + 2 = 4 pts}
    Résoudre dans $\C$ les deux équations suivantes :
        \[(E_1) : 1 + 2\I z = 3 - 5\I \qetq (E_2) : 2z - 4\I = -\I\times\overline z + 5\]
\end{exoc}

\begin{exoc}{2 + 2 = 4 pts}
    Le plan est rapporté à un repère orthonormal \Ouv.
    On considère les points $F$, $G$ et $H$ d'affixes respectives :
    \[z_F = 3 + \I\sqrt 3 \qq z_G = \left(\frac 12 + \I \frac{\sqrt 3}{2}\right)\times z_F  \qetq z_H = z_F - 2\I\sqrt 3.\]
    \begin{enumerate}
        \item En détaillant les calculs, déterminer la forme algébrique de $z_G$ et $z_H$.
        \item Démontrer que le quadrilatère $OHFG$ est un parallélogramme.
    \end{enumerate}
\end{exoc}

\begin{exoc}{1 + 1 + 2 + 2 + 1 = 7}
    Le plan est rapporté à un repère orthonormal \Ouv d'unité graphique $\SI{2}{carreaux}$.\par
    On considère les points $A$, $B$ et $C$ d'affixes respectives :
    \[z_A = \I \qq z_B = 4 + 3\I \qetq z_C = \frac{4+3\I}{1+2\I}.\]
    \begin{enumerate}
        \item Démontrer par un calcul détaillé que $z_C = 2 - i$.
        \item $D(z_D)$ est le milieu de $[BC]$. Déterminer par un calcul détaillé l'affixe de $D$.
        \item On considère le point $E(z_E)$ tel que $ABEC$ est un parallélogramme.\par Déterminer par un calcul détaillé l'affixe du point $E$.
        \item Le point $F(z_F)$ est tel que $\vect{AB} + \vect{BC} + \vect{CF} = \vect{FO}$.\par Déterminer par un calcul détaillé l'affixe du point $F$.
        \item Sur la copie, dessiner le repère \Ouv et placer les points $A$, $B$, $C$, $D$, $E$ et $F$.
    \end{enumerate}
\end{exoc}{}



\end{document} 