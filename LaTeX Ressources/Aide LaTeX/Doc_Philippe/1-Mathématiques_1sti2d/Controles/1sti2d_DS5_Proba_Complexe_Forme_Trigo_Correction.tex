\documentclass[10pt,french]{book}
\input philippe2013

\newcounter{exoc}
\newenvironment{exoc}[1]{%
  \refstepcounter{exoc}\textbf{Exercice \theexoc} :\hfill {\textbf{#1}}\par
  \medskip}%
{\medskip}

\begin{document}

\pagestyle{empty}

\begin{center}
\psframebox[shadow=true,shadowcolor=gray!75,shadowsize=3pt,%
framearc=0.3,%
fillstyle=gradient,gradmidpoint=0.8,gradangle=20,gradbegin=red!60!yellow!40,gradend= white]{%
\parbox{0.75\linewidth}{%
\begin{center}
\Large\bfseries
\uuline{Contrôle Nombres complexes - Probabilités}\par
Correction
\end{center}}}
\end{center}\bigskip

\hrulefill\bigskip

Sur 22 copies : \pfr{Moyenne : 08/20}. \textbf{17 élèves ont une note strictement inférieure à 10/20 !!}.\par
Merci de lire très attentivement la correction et de \textbf{refaire plusieurs fois les exercices} afin de comprendre et de retenir toutes les méthodes.\bigskip

\hrulefill\bigskip

\exo
\begin{enumerate}
\item $z = \intervalleff{12}{\frac\pi 6}$.

\[\begin{array}{rcl}
    z & = & \intervalleff{12}{\frac\pi 6} \\[8pt]
      & = & 12\left(\cos \frac\pi 6 + \I \sin \frac \pi 6\right) \\[8pt]
      & = & 12\left(\frac{\sqrt 3}{6} + \I\frac 12\right)\\[8pt]
    \multicolumn{3}{l}{\pfr{z = 6\sqrt 3 + 6\I}}
\end{array}\]\bigskip

\item $z' = 2 - 2\I\sqrt 3$\bigskip

On calcule d'abord le module :

\[\begin{array}{rcl}
\abs{z' }& = & \sqrt{x^2 + y^2} \\
            & = & \sqrt{2^2 + (-2\sqrt 3)^2}\\
            & = & \sqrt{4 + 4\times 3} \\
            & = & \sqrt{16} \\
    \multicolumn{3}{l}{\pfr{\abs{z'} = 4}}
\end{array}\]

Ensuite, on met $\abs{z'}$ en facteur dans l'expression de départ.

\[\begin{array}{rcl}
z'& = & 2 - 2\I\sqrt 3 \\[8pt]
            & = & 4\left(\frac{2}{4} - \I\frac{2\sqrt3}{4}\right)\\[8pt]
            & = & 4\left(\frac{1}{2} - \I\frac{\sqrt3}{2}\right)
\end{array}\]

On trouve alors : $\cos \theta = \frac 1 2$ et $\sin \theta = -\frac{\sqrt 3}{2}$ ce qui correspond à \pfr{$\theta = -\frac\pi 3$}.

\[\pfr{z' = \intervalleff{4}{-\frac{\pi}{3}}}\]
\end{enumerate}\clearpage

{\small
\exo \[z_A = \sqrt 3 + \I \qq z_B = z_A - 4\I \qetq z_C = 3 \overline{z_A} + 2\I.\]

\begin{enumerate}
\item \begin{enumerate}
    \item\strut
    
    \begin{center}
        \begin{tabularx}{0.75\linewidth}{>\centering X|>{\centering\arraybackslash} X}
            $\begin{array}{rcl}
                z_B & = & z_A - 4\I \\
                &=& \sqrt 3 + \I - 4 \I \\
                \multicolumn{3}{l}{\pfr{z_B = \sqrt 3 - 3\I}}
            \end{array}$ &
            $\begin{array}{rcl}
                z_C & = & 3\overline{z_A} + 2\I \\
                &=& 3(\sqrt 3 - \I) + 2\I \\
                &=&3\sqrt 3 - 3\I + 2\I \\
                \multicolumn{3}{l}{\pfr{z_C = 3\sqrt 3 - \I}}
            \end{array}$ 
        \end{tabularx}
    \end{center}
    
    \item $ABC$ est un triangle équilatéral lorsque $AB = BC = AC$. On calcule donc chaque longueur.\medskip

\hspace*{-2cm}
        \begin{tabularx}{1.2\linewidth}{>\centering X|>\centering X|>{\centering\arraybackslash} X}
            $\begin{array}{rcl}
                AB & = & \abs{z_B - z_A} \\
                &=& \abs{\sqrt3 - 3\I - (\sqrt 3 + \I)} \\[5pt]
                &=&  \abs{\sqrt 3 - 3\I -\sqrt 3 - \I} \\[5pt]
                &=&\abs{-4\I}\\
                \multicolumn{3}{l}{\pfr{AB = 4}}
            \end{array}$ &
            $\begin{array}{rcl}
                AC & = & \abs{z_C - z_A} \\
                &=& \abs{3\sqrt3 - \I - (\sqrt 3 + \I)} \\[5pt]
                &=&  \abs{3\sqrt 3 - \I -\sqrt 3 - \I} \\[5pt]
                &=&\abs{2\sqrt 3 - 2\I}\\[5pt]
                &=&\sqrt{(2\sqrt 3)^2 + (-2)^2}\\
                &=&\sqrt{16}\\
                \multicolumn{3}{l}{\pfr{AC = 4}}
            \end{array}$ &
            $\begin{array}{rcl}
                BC & = & \abs{z_C - z_B} \\
                &=& \abs{3\sqrt3 - \I - (\sqrt 3 - 3\I)} \\[5pt]
                &=&  \abs{3\sqrt 3 - \I -\sqrt 3 + 3\I} \\[5pt]
                &=&\abs{2\sqrt 3 + 2\I}\\[5pt]
                &=&\sqrt{(2\sqrt 3)^2 + 2^2}\\
                &=&\sqrt{16}\\
                \multicolumn{3}{l}{\pfr{BC = 4}}
            \end{array}$
        \end{tabularx}\medskip
    
    $AB = BC = AC$ donc le triangle $ABC$ est bien un triangle équilatéral.

\item $z_A= \sqrt 3 + \I$

\[\begin{array}{rcl}
\abs{z_A }& = & \sqrt{(\sqrt 3)^2 + 1^2}\\
            & = & \sqrt{3 + 1} \\
            & = & \sqrt{4} \\
    \multicolumn{3}{l}{\pfr{\abs{z_A} = 2}}
\end{array} \qq
\begin{array}{rcl}
z_A& = & \sqrt 3 + \I \\[8pt]
            & = & 2\left(\frac{\sqrt 3}{2} + \I\frac{1}{2}\right)
\end{array}\]

On trouve alors : $\cos \theta =\frac{\sqrt 3}{2}$ et $\sin \theta = \frac 1 2$ ce qui correspond à \pfr{$\theta = \frac\pi 6$}.

\[\pfr{z_A = \intervalleff{2}{\frac{\pi}{6}}}\]

$z_B= \sqrt 3 - 3\I$

\[\begin{array}{rcl}
\abs{z_B }& = & \sqrt{(\sqrt 3)^2 + (-3)^2}\\
            & = & \sqrt{3 + 9} \\
            & = & \sqrt{12} \\
            & = & \sqrt{4 \times 3} \\
    \multicolumn{3}{l}{\pfr{\abs{z_B} = 2\sqrt3}}
\end{array} \qq
\begin{array}{rcl}
z_B& = & \sqrt 3 - 3\I \\[8pt]
            & = & 2\sqrt 3\left(\frac{\sqrt 3}{2\sqrt3} + \I\frac{-3}{2\sqrt3}\right) \\[8pt]
            & = & 2\sqrt3\left(\frac{1}{2} + \I\frac{-\sqrt 3}{2}\right) \\[8pt]
\end{array}\]

On trouve alors : $\cos \theta =\frac{1}{2}$ et $\sin \theta = -\frac{\sqrt3}{2}$ ce qui correspond à \pfr{$\theta = -\frac\pi 3$}.

\[\pfr{z_B = \intervalleff{2\sqrt3}{\frac{-\pi}{3}}}\]

\item $\left(\vect{OB} \pv \vect{OA}\right) = \left(\vect{OB} \pv \vect u\right) + \left(\vect u \pv \vect{OA}\right) = -\left(\vect u \pv \vect{OB}\right)+ \left(\vect u \pv \vect{OA}\right)$ donc\par
$\left(\vect{OB} \pv \vect{OA}\right) =  -\arg(z_B) + \arg z_A = -\left(-\frac \pi 3\right) + \frac\pi 6 =\frac \pi 3 + \frac\pi 6 = \pfr{\dfrac \pi 2}.$
\item Le triangle $OAB$ possède un angle droit donc c'est un \pfr{triangle rectangle}.
    \end{enumerate}
\end{enumerate}
}

 
\end{document}