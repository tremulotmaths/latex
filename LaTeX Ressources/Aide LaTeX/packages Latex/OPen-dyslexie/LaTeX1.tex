%Time-stamp: <31/10/12>
\documentclass{scrartcl}
\KOMAoptions{paper=a4, fontsize=11pt}

%%%%%%%%%%%%%%%%%%%%%%%%%%%%%%%%%%%%%%%%%%%%%%
%%%%%%%%%%%%%% Dyslexie %%%%%%%%%%%%%%%%%%%%%%
%%%%%%%%%%%%%%%%%%%%%%%%%%%%%%%%%%%%%%%%%%%%%%


%%%   Les trois options suivantes sont    %%% 
%%%   spécifiques aux classes Komascript  %%%

% Augmentation de la taille de la police des sections et sous-sections
\setkomafont{section}{\Huge}
\setkomafont{subsection}{\Large}

% Augmentation de l'espace entre les lignes
\linespread{2}

% Appel �  la police Opendyslexic et augmentation
% des espaces entre les lettres et entre les mots
\usepackage{fontspec}
\setmainfont[LetterSpace=30,  WordSpace=2, Mapping=tex-text]{OpenDyslexic}

% Appel �  la police mathématique Asana
% et augmentation de la taille de la police
\usepackage{unicode-math}
\setmathfont[Scale=1.2]{Asana Math}

% Pour les figures (tikz) : épaisseurs des traits
\usepackage{tikz}
\tikzset{every picture/.style={line width=1.5pt}}

%%%%%%%%%%%%%%%%%%%%%%% babel et autres  %%%%%%%%%%%%%%%%%%%%%%%

\usepackage[francais]{babel}

% test
\usepackage{blindtext}
\blindmathtrue

%%%%%%%%%%%%%%%%%%%%%%%%%%%%%%%%%%%%%%%%%%%%%%%%%%%%%%%%%%%%%%%%
%%%%%%%%%%%%%%%%%%%%%%%%%%%%%%%%%%%%%%%%%%%%%%%%%%%%%%%%%%%%%%%%
%%%%%%%%%%%%%%%%%%%%%%%%%%%%%%%%%%%%%%%%%%%%%%%%%%%%%%%%%%%%%%%%

\begin{document}

\section{Exercice}

  On considère un cube $ABCDEFGH$ d'arête 2~cm et l'on désigne par $I$
  le milieu de $[FG]$ et par $M$ un point quelconque du segment
  $[BF]$.\par On note $x=BM$.\par On s'intéresse �  la fonction $f$
  qui, �  $x$, associe la longueur $AM+MI$.

  \begin{center}
  \begin{tikzpicture}[scale=1.5]
    \coordinate (A) at (0,0);
    \coordinate (B) at (5,0);
    \coordinate (F) at  (5,5);
    \coordinate (E) at  (0,5);
    
    \coordinate (D) at  (2,2);
    \coordinate (C) at  (7,2);
    \coordinate (G) at  (7,7);
    \coordinate (H) at  (2,7);

    \coordinate (M) at (5,1);
    \coordinate (I) at (6,6);

    \draw (A) node[below left]{$A$};
    \draw (B) node[below right]{$B$};
    \draw (F) node[above]{$F$};
    \draw (E) node[above left]{$E$};
    \draw (D) node[below]{$D$};
    \draw (C) node[right]{$C$};
    \draw (G) node[above right]{$G$};    
    \draw (H) node[above left]{$H$};
    \draw (M) node[right]{$M$};
    \draw (I) node[above]{$I$};
    \draw (A) -- (B) -- (F) -- (E) -- cycle;
    \draw (E) -- (H) -- (G) -- (F)  (G) -- (C) -- (B);
    \draw[dotted] (A)--(D)--(C) (D)--(H);
    \draw (A)--(M)--(I);
  \end{tikzpicture}
\end{center}

\begin{enumerate}
\item En mettant �  plat les 2 faces $EFBA$ et $BFGC$ du patron du
  cube, faire une figure et la coder.
\item Quelles sont les valeurs prises par $x$?
\end{enumerate}

\section{blindtext}

\blindtext

\end{document}

%%% Local Variables: ***
%%% mode: latex ***
%%% TeX-engine: xetex ***
%%% TeX-PDF-mode: t ***
%%% End: ***