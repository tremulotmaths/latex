%\documentclass[10pt]{article}
%\usepackage[latin1]{inputenc}
%\usepackage[frenchb]{babel}
%\usepackage[T1]{fontenc}
%\usepackage{listings}
%\usepackage[colorlinks=true,urlcolor=blue]{hyperref}
%\usepackage[vmargin=2cm,hmargin=2cm]{geometry}
%\usepackage[svgnames,table]{xcolor}
%\usepackage{kpfonts}
%\usepackage{cellspace}
%\usepackage{aurical}
%\usepackage{graphicx}
%\RequirePackage{titlesec}
%\usepackage[electrum]{pas-keyboard}
% <---------- Param�tres globaux -------------------->
%\author{St�phane PASQUET \\ \href{http://www.mathweb.fr}{http://www.mathweb.fr} \\ \href{mailto:contact@mathweb.fr}{contact@mathweb.fr}}
%\date{\today}
%\setlength{\parindent}{0cm}
%\definecolor{touchColor}{cmyk}{0,0,.05,0}
%\pagecolor{touchColor}
%\title{Documentation\\ \texttt{pas-keyboard.sty V1.02}}

\documentclass[a4paper,french]{article}
\usepackage[latin1]{inputenc}
\usepackage[T1]{fontenc}
\usepackage{babel} 
\usepackage{etex}
\usepackage{fourier}
\usepackage[table]{xcolor}
\usepackage[colorlinks=true,urlcolor=blue]{hyperref}
\usepackage{pas-keyboard}
\usepackage{titlesec}
\usepackage{tcolorbox}
	\tcbuselibrary{skins}
	\tcbuselibrary{theorems}
	\tcbuselibrary{breakable}

% --- Propre � cette doc

\graphicspath{{images/}}
\usepackage{longtable}
\usepackage{tabularx}
\usepackage{cellspace}
\setlength{\cellspacetoplimit}{4pt}
\setlength{\cellspacebottomlimit}{4pt}

% ----------------------

\setlength{\parindent}{0pt}

\input{doc.codes.tex}
\input{doc.styles.tex}

\begin{document}

\begin{center}
\begin{tcolorbox}[head]
{\bfseries\LARGE Documentation \texttt{pas-keyboard} }\\[3mm]
{\large Version 1.02 -- \today}
\end{tcolorbox}

{\large 
\href{http://www.mathweb.fr/contact.html}{St�phane Pasquet}}
\end{center}

\begin{tcolorbox}[toc]
\makeatletter
\@starttoc{toc}
\makeatother
\end{tcolorbox}

\section{Introduction et installation}

L'extension \texttt{pas-keyboard} est con�ue pour afficher des touches d'ordinateur.\\
Bien s�r, il y a l'extension \texttt{keystroke}, mais le rendu me ne satisfaisait pas. C'est la raison pour laquelle j'ai eu envie de cr�er cette extension.

\medskip

Cette extension charge automatiquement les extensions suivantes :

\medskip

\begin{quote}
tikz (avec les librairies : calc, arrows)
\end{quote}

\medskip

Sous Ubuntu, on pourra d�compresser \texttt{pas-crosswords.zip} dans le r�pertoire :

\begin{verbatim}
./texlive/texmf-local/tex/latex/
\end{verbatim}

de sorte � avoir :


\begin{verbatim}
./texlive/texmf-local/tex/latex/pas-keyboard/latex/pas-keyboard.sty
./texlive/texmf-local/tex/latex/pas-keyboard/doc/pas-keyboard.tex
./texlive/texmf-local/tex/latex/pas-keyboard/doc/pas-keyboard.pdf
./texlive/texmf-local/tex/latex/pas-keyboard/doc/doc.codes.tex
./texlive/texmf-local/tex/latex/pas-keyboard/doc/doc.styles.tex
\end{verbatim}

\medskip

Apr�s installation, n'oubliez pas de taper la commande \texttt{texhash} dans le terminal pour mettre � jour la base de donn�es des extensions.

\bigskip 

Avec Miktex (sous Windows) ou macTex (sous Mac OS), j'imagine que l'arborescence ressemble � ce qui est �crit pr�c�demment.

\section{Appel de l'extension}

Afin d'imiter le mieux les touches d'un ordinateur, j'ai fait appel � la fonte \texttt{electrum} disponible sur la page :
\begin{center}
\url{http://www.ctan.org/tex-archive/fonts/electrumadf/}. 
\end{center}

Cependant, conscient des difficult�s que pose la mise en place d'une telle fonte, j'ai d�cid� de faire appel � celle-ci uniquement sur l'appel d'une option. Ainsi, pour faire appel � cette police de caract�res, l'extension devra �tre appel�e avec l'option suivante :

\medskip

\begin{tcblisting}{codeTEX}
\usepackage[electrum]{pas-keyboard}
\end{tcblisting}

\medskip

Sans cette option, les caract�res des touches seront mis en helvetica.

\section{Les commandes}

Voici une liste des diff�rentes commandes :

\medskip\arrayrulecolor{brown}

\begin{center}
\begin{longtable}{|>{\centering\arraybackslash}Sc|>{\centering\arraybackslash}Sc|}
\hline\rowcolor{brown}\color{white}
Commandes & \color{white}R�sultats\\
\hline
\texttt{\textbackslash key$\{$M$\}$}& \key{M}\\
\hline
\texttt{\textbackslash keyCTRL}& \keyCTRL\\
\hline
\texttt{\textbackslash keyUpArrow}& \keyUpArrow\\
\hline
\texttt{\textbackslash keyDownArrow}& \keyDownArrow\\
\hline
\texttt{\textbackslash keyLeftArrow}& \keyLeftArrow\\
\hline
\texttt{\textbackslash keyRightArrow}& \keyRightArrow\\
\hline
\texttt{\textbackslash keyShift}& \keyShift\\
\hline
\texttt{\textbackslash keySHIFT}& \keySHIFT\\
\hline
\texttt{\textbackslash keyBack}& \keyBack\\
\hline
\texttt{\textbackslash keyBACK}& \keyBACK\\
\hline
\texttt{\textbackslash keyEnter}& \keyEnter\\
\hline
\texttt{\textbackslash keyTab}& \keyTab\\
\hline
\texttt{\textbackslash keyCapsOff}& \keyCapsOff\\
\hline
\texttt{\textbackslash keyCapsOn}& \keyCapsOn\\
\hline
\texttt{\textbackslash keySupInf}& \keySupInf\\
\hline
\texttt{\textbackslash keySquare}& \keySquare\\
\hline
\texttt{\textbackslash keyEsc}& \keyEsc\\
\hline
\texttt{\textbackslash keyHome}& \keyHome\\
\hline
\texttt{\textbackslash keyHOME}& \keyHOME\\
\hline
\texttt{\textbackslash keyPgUp}& \keyPgUp\\
\hline
\texttt{\textbackslash keyPgDown}& \keyPgDown\\
\hline
\end{longtable}
\end{center}

\bigskip

Bien entendu, vous pouvez mettre n'importe quelle lettre dans la commande \texttt{\textbackslash key} :

\medskip

\begin{tcblisting}{listing}
\key{A}\key{B}\key{C}
\end{tcblisting}

\section{Les param�tres}

Vous avez la possibilit� de changer les couleurs :

\begin{center}
\begin{tabularx}{\linewidth}{|>{\centering\arraybackslash}p{2cm}|>{\centering\arraybackslash}X|p{6.5cm}|}
\hline\rowcolor{brown}
\color{white}Nom & \color{white}Correspondance & \color{white}Valeur par d�faut\\
\hline
\texttt{touchColor} & Fond des touches & \textbackslash\texttt{definecolor$\{$touchColor$\}\{$cmyk$\}\{$0,0,0,0$\}$}\\
\hline
\texttt{textColor} & Texte des touches & \textbackslash\texttt{definecolor$\{$textColor$\}\{$cmyk$\}\{$1,1,1,1$\}$}\\
\hline
\texttt{drawColor} & Contours des touches & \textbackslash\texttt{definecolor$\{$drawColor$\}\{$cmyk$\}\{$1,1,1,1$\}$}\\
\hline
\texttt{shadowColor} & Ombre des touches & \textbackslash\texttt{definecolor$\{$shadowColor$\}\{$cmyk$\}$ $\{$0.30,0.20,0.20,0$\}$}\\
\hline
\texttt{lightON} & Couleur de la \og diode \fg{} de la touche CAPS LOCK quand elle est active & \textbackslash\texttt{definecolor$\{$lightON$\}\{$cmyk$\}$ $\{$0.81,0.11,1,0.02$\}$}\\
\hline
\texttt{lightOFF} & Couleur de la \og diode \fg{} de la touche CAPS LOCK quand elle est inactive & \textbackslash\texttt{definecolor$\{$lightOFF$\}$ $\{$cmyk$\}\{$0.55,0.42,0.42,0.09$\}$}\\
\hline
\end{tabularx}
\end{center}

\medskip

Vous pouvez donc changer la couleur de votre choix en la red�finissant avec la commande \textbackslash\texttt{definecolor}.

\section{Suppl�ments}

Vous pouvez ins�rer ce que vous souhaitez dans les touches, y compris des dessins comme le montre le script suivant :

\medskip

\begin{tcblisting}{listing}
\newcommand{\dessin}{\tikz{\clip (0,-0.1) 
rectangle +(.4,.2);\draw (0,0) .. controls 
(.2,.3) and (.3,-.3) .. (.4,.2);}}
\key{\dessin}
\end{tcblisting}

\medskip

Mais vous n'�tes pas oblig�s d'ins�rer du code TiKZ ; un code PST ou m�me une image est possible comme le montre le script suivant :

\medskip

\begin{tcblisting}{listing}
\key{\includegraphics[scale=0.5]{icone.png}}
\end{tcblisting}

\medskip

Je vous rappelle que la commande \textbackslash\texttt{includegraphics} s'emploie en ayant appel� l'extension \texttt{graphicx}.

\section{Mises � jour}

\begin{tcolorbox}[histo]
2013/10/03
~
\tcblower
Modification de la commande \texttt{\textbackslash key} afin de centrer verticalement les touches sur la ligne courante.
\end{tcolorbox}
\end{document}