\documentclass[a4paper,french]{article}
\usepackage[latin1]{inputenc}
\usepackage[T1]{fontenc}
\usepackage{babel} 
\usepackage{etex}
\usepackage{fourier}
\usepackage[table]{xcolor}
\usepackage[colorlinks=true,urlcolor=blue]{hyperref}
\usepackage{titlesec}
\usepackage{tcolorbox}
	\tcbuselibrary{skins}
	\tcbuselibrary{theorems}
	\tcbuselibrary{breakable}

\setlength{\parindent}{0pt}

\input{doc.codes.tex}
\input{doc.styles.tex}

\usepackage[table]{xcolor}
\usepackage{amsmath,amssymb}
\usepackage{fancyvrb}
\usepackage{cellspace}
\setlength{\cellspacetoplimit}{3pt}
\setlength{\cellspacebottomlimit}{3pt}
\usepackage{longtable}
% Les couleurs
\def\bkcolor{gray!10}
\def\fstlinecolor{gray}
\def\txtfstlinecolor{white}
\def\bordercolor{black}

\begin{VerbatimOut}{commandes.cxx}
tableau(L):={
 local n,p,k,s,compt;
 n:=L[0];
 p:=L[1];
 compt:=0;
 T:="\\arrayrulecolor{\\bordercolor}\\begin{longtable}{|S{>{\\centering\\arraybackslash}p{2cm}}|Sc|} \\hline\\rowcolor{\\fstlinecolor}\\textcolor{\\txtfstlinecolor}{$\\pmb{k}$} & \\textcolor{\\txtfstlinecolor}{$\\pmb{P(X\\leqslant k)}$} \\\\ \\hline";
 for (k:=0;k<=n;k++)
 {
         s:=binomial_cdf(n,p,k);
         if (s>0.025 and compt==0) { T:=T+"\\rowcolor{\\bkcolor}"; compt:=1;}
         if (s>0.975 and compt==1) { T:=T+"\\rowcolor{\\bkcolor}"; compt:=2;}                T:=T+k+"&"+latex(s)+"\\\\ \\hline ";
 };
 T:=T+"\\end{longtable}";
 return(T);
}:;

tableaumin(L):={
 local n,p,k,min,max,s,compt;
 n:=L[0];
 p:=L[1];
 min:=L[2];
 max:=L[3]
 compt:=0;
T:="\\arrayrulecolor{\\bordercolor}\\begin{longtable}{|S{>{\\centering\\arraybackslash}p{2cm}}|Sc|} \\hline\\rowcolor{\\fstlinecolor}\\textcolor{\\txtfstlinecolor}{$\\pmb{k}$} & \\textcolor{\\txtfstlinecolor}{$\\pmb{P(X\\leqslant k)}$} \\\\ \\hline";
 for (k:=min;k<=max;k++)
 {
         s:=binomial_cdf(n,p,k);
         if (s>0.025 and compt==0) { T:=T+"\\rowcolor{\\bkcolor}"; compt:=1;}
         if (s>0.975 and compt==1) { T:=T+"\\rowcolor{\\bkcolor}"; compt:=2;}                T:=T+k+"&"+latex(s)+"\\\\ \\hline ";
 };
 T:=T+"\\end{longtable}";
 return(T);
}:;

tableauminbreak(L):={
 local n,p,k,min,max,break1,break2,s,compt;
 n:=L[0];
 p:=L[1];
 min:=L[2];
 max:=L[3];
 break1:=L[4];
 break2:=L[5];
 compt:=0;
 T:="\\arrayrulecolor{\\bordercolor}\\begin{longtable}{|S{>{\\centering\\arraybackslash}p{2cm}}|Sc|} \\hline\\rowcolor{\\fstlinecolor}\\textcolor{\\txtfstlinecolor}{$\\pmb{k}$} & \\textcolor{\\txtfstlinecolor}{$\\pmb{P(X\\leqslant k)}$} \\\\ \\hline";
 for (k:=min;k<=break1;k++)
 {
         s:=binomial_cdf(n,p,k);
         if (s>0.025 and compt==0) { T:=T+"\\rowcolor{\\bkcolor}"; compt:=1;}
         if (s>0.975 and compt==1) { T:=T+"\\rowcolor{\\bkcolor}"; compt:=2;}                T:=T+k+"&"+latex(s)+"\\\\ \\hline ";
 };
 T:=T+"$\\vdots$ & $\\vdots$\\\\ \\hline";
 for (k:=break2;k<=max;k++)
 {
         s:=binomial_cdf(n,p,k);
         if (s>0.025 and compt==0) { T:=T+"\\rowcolor{\\bkcolor}"; compt:=1;}
         if (s>0.975 and compt==1) { T:=T+"\\rowcolor{\\bkcolor}"; compt:=2;}                T:=T+k+"&"+latex(s)+"\\\\ \\hline ";
 };
 T:=T+"\\end{longtable}";
 return(T);
}:;

build(fichier,type,L):={
 name:=fichier+".tex";
 if (type=="complet")
 {
 	donnees:=tableau(L);
 }
 if (type=="min")
 {
 	donnees:=tableaumin(L);
 }
 if (type="minbreak")
 {
 	donnees:=tableauminbreak(L);
 }
 Sortie:=fopen(name);
 Resultat:=cat(donnees);
 fprint(Sortie,Unquoted,Resultat);
 fclose(Sortie);
}:;
maple_mode(0);
read("commande.txt");
\end{VerbatimOut}

\newenvironment{echantillon}
{\VerbatimEnvironment\begin{VerbatimOut}{commande.txt}}
{\end{VerbatimOut}
\immediate\write18{giac < commandes.cxx}
\immediate\write18{giac  commandes.cxx}
}

\begin{document}

\begin{center}
\begin{tcolorbox}[head]
{\bfseries\LARGE Documentation \texttt{pas-echant.mod.tex} }\\[3mm]
{\large Version 1.00 -- \today}
\end{tcolorbox}

{\large 
\href{http://www.mathweb.fr/contact.html}{St�phane Pasquet}}
\end{center}

\begin{tcolorbox}[toc]
\makeatletter
\@starttoc{toc}
\makeatother
\end{tcolorbox}

\section{Introduction et installation}

Le module (et non l'extension) \texttt{pas-echant.mod.tex} a �t� con�u dans le but de cr�er facilement une table de valeurs de $P(X\leqslant k)$, pour $0\leqslant k\leqslant n$, o� $X\hookrightarrow\mathcal{B}(n~;p)$.

\medskip


Ce module charge les extensions suivantes :

\begin{quote}
xcolor (avec l'option : table)\\
amsmath\\
amssymb\\
fancyvrb\\
cellspace\\
longtable
\end{quote}

et d�finit :

\begin{tcblisting}{codeTEX}
\setlength{\cellspacetoplimit}{3pt}
\setlength{\cellspacebottomlimit}{3pt}
\end{tcblisting}

\medskip

Ce module utilise Giac/Xcas. Il est donc n�cessaire de faire quelques r�glages avec votre �diteur de sorte qu'il reconnaisse l'utilisation de Giac/Xcas.

\medskip

Sous Ubuntu, on pourra d�compresser \texttt{pas-echant.zip} dans le r�pertoire :

\begin{verbatim}
./texlive/texmf-local/tex/latex/
\end{verbatim}

de sorte � avoir :


\begin{verbatim}
./texlive/texmf-local/tex/latex/pas-echant/latex/pas-echant.mod.tex
./texlive/texmf-local/tex/latex/pas-echant/doc/pas-echant.tex
./texlive/texmf-local/tex/latex/pas-echant/doc/doc.codes.tex
./texlive/texmf-local/tex/latex/pas-crosswords/doc/doc.styles.tex
./texlive/texmf-local/tex/latex/pas-echant/doc/pas-echant.pdf
\end{verbatim}

\medskip

Apr�s installation, n'oubliez pas de taper la commande \texttt{texhash} dans le terminal pour mettre � jour la base de donn�es des extensions.

\bigskip 

Avec Miktex (sous Windows) ou macTex (sous Mac OS), j'imagine que l'arborescence ressemble � ce qui est �crit pr�c�demment.

\medskip

Pour faire appel � ce module, on l'appellera en pr�ambule :

\begin{tcblisting}{codeTEX}
\documentclass{article}
...
\usepackage[table]{xcolor}
\usepackage{amsmath,amssymb}
\usepackage{fancyvrb}
\usepackage{cellspace}
\setlength{\cellspacetoplimit}{3pt}
\setlength{\cellspacebottomlimit}{3pt}
\usepackage{longtable}
% Les couleurs
\def\bkcolor{gray!10}
\def\fstlinecolor{gray}
\def\txtfstlinecolor{white}
\def\bordercolor{black}

\begin{VerbatimOut}{commandes.cxx}
tableau(L):={
 local n,p,k,s,compt;
 n:=L[0];
 p:=L[1];
 compt:=0;
 T:="\\arrayrulecolor{\\bordercolor}\\begin{longtable}{|S{>{\\centering\\arraybackslash}p{2cm}}|Sc|} \\hline\\rowcolor{\\fstlinecolor}\\textcolor{\\txtfstlinecolor}{$\\pmb{k}$} & \\textcolor{\\txtfstlinecolor}{$\\pmb{P(X\\leqslant k)}$} \\\\ \\hline";
 for (k:=0;k<=n;k++)
 {
         s:=binomial_cdf(n,p,k);
         if (s>0.025 and compt==0) { T:=T+"\\rowcolor{\\bkcolor}"; compt:=1;}
         if (s>0.975 and compt==1) { T:=T+"\\rowcolor{\\bkcolor}"; compt:=2;}                T:=T+k+"&"+latex(s)+"\\\\ \\hline ";
 };
 T:=T+"\\end{longtable}";
 return(T);
}:;

tableaumin(L):={
 local n,p,k,min,max,s,compt;
 n:=L[0];
 p:=L[1];
 min:=L[2];
 max:=L[3]
 compt:=0;
T:="\\arrayrulecolor{\\bordercolor}\\begin{longtable}{|S{>{\\centering\\arraybackslash}p{2cm}}|Sc|} \\hline\\rowcolor{\\fstlinecolor}\\textcolor{\\txtfstlinecolor}{$\\pmb{k}$} & \\textcolor{\\txtfstlinecolor}{$\\pmb{P(X\\leqslant k)}$} \\\\ \\hline";
 for (k:=min;k<=max;k++)
 {
         s:=binomial_cdf(n,p,k);
         if (s>0.025 and compt==0) { T:=T+"\\rowcolor{\\bkcolor}"; compt:=1;}
         if (s>0.975 and compt==1) { T:=T+"\\rowcolor{\\bkcolor}"; compt:=2;}                T:=T+k+"&"+latex(s)+"\\\\ \\hline ";
 };
 T:=T+"\\end{longtable}";
 return(T);
}:;

tableauminbreak(L):={
 local n,p,k,min,max,break1,break2,s,compt;
 n:=L[0];
 p:=L[1];
 min:=L[2];
 max:=L[3];
 break1:=L[4];
 break2:=L[5];
 compt:=0;
 T:="\\arrayrulecolor{\\bordercolor}\\begin{longtable}{|S{>{\\centering\\arraybackslash}p{2cm}}|Sc|} \\hline\\rowcolor{\\fstlinecolor}\\textcolor{\\txtfstlinecolor}{$\\pmb{k}$} & \\textcolor{\\txtfstlinecolor}{$\\pmb{P(X\\leqslant k)}$} \\\\ \\hline";
 for (k:=min;k<=break1;k++)
 {
         s:=binomial_cdf(n,p,k);
         if (s>0.025 and compt==0) { T:=T+"\\rowcolor{\\bkcolor}"; compt:=1;}
         if (s>0.975 and compt==1) { T:=T+"\\rowcolor{\\bkcolor}"; compt:=2;}                T:=T+k+"&"+latex(s)+"\\\\ \\hline ";
 };
 T:=T+"$\\vdots$ & $\\vdots$\\\\ \\hline";
 for (k:=break2;k<=max;k++)
 {
         s:=binomial_cdf(n,p,k);
         if (s>0.025 and compt==0) { T:=T+"\\rowcolor{\\bkcolor}"; compt:=1;}
         if (s>0.975 and compt==1) { T:=T+"\\rowcolor{\\bkcolor}"; compt:=2;}                T:=T+k+"&"+latex(s)+"\\\\ \\hline ";
 };
 T:=T+"\\end{longtable}";
 return(T);
}:;

build(fichier,type,L):={
 name:=fichier+".tex";
 if (type=="complet")
 {
 	donnees:=tableau(L);
 }
 if (type=="min")
 {
 	donnees:=tableaumin(L);
 }
 if (type="minbreak")
 {
 	donnees:=tableauminbreak(L);
 }
 Sortie:=fopen(name);
 Resultat:=cat(donnees);
 fprint(Sortie,Unquoted,Resultat);
 fclose(Sortie);
}:;
maple_mode(0);
read("commande.txt");
\end{VerbatimOut}

\newenvironment{echantillon}
{\VerbatimEnvironment\begin{VerbatimOut}{commande.txt}}
{\end{VerbatimOut}
\immediate\write18{giac < commandes.cxx}
\immediate\write18{giac  commandes.cxx}
}
\begin{document}
...
\end{document}
\end{tcblisting}


\section{Installation de Giac/Xcas}

Le module \texttt{pas-echant.tex} fait appel � Giac/Xcas ; par cons�quent, vous devez installer Xcas � partir, par exemple, de la page : \begin{center}
\url{http://www-fourier.ujf-grenoble.fr/~parisse/giac_fr.html}
\end{center}

\section{Param�trage de l'�diteur LaTeX}

N'oubliez pas de mettre en option de compilation  :

\begin{verbatim}
--shell-escape
\end{verbatim}

afin que les commande Xcas s'ex�cutent. Par exemple, en compilant avec Pdflatex, on aura (avec TexMaker, mais pour les autres �diteurs, �a ressemble � cela aussi) :

\begin{verbatim}
pdflatex -synctex=1 -interaction=nonstopmode --shell-escape %.tex
\end{verbatim}

\section{Les trois fonctions principales}

\begin{tcblisting}{codeTEX}
\begin{echantillon}
build("<nom du tableau>","<option>",[n,p <,options>]);
\end{echantillon}
\input{<nom du tableau>}
\end{tcblisting}

\medskip

Ici, la commande \texttt{build} va construire un tableau de valeurs selon l'option indiqu�e et va le sauvegarder dans un fichier externe \texttt{.tex} dont le nom est celui inform� pour la \texttt{<nom du tableau>}. Ensuite, on affiche le tableau de valeurs � l'aide de la commande \texttt{\textbackslash input\{<nom du tableau>\}}.

\medskip

Les 3 options possibles sont :

\begin{itemize}
\item[$\bullet$] \og complet \fg{} 
\item[$\bullet$] \og min \fg{} 
\item[$\bullet$] \og minbreak \fg{} 
\end{itemize}

\subsection{Construire un tableau complet}

J'ai ici construit le tableau des valeurs avec $n=50$ et $p=0,487$ :

\medskip

\begin{tcblisting}{listing}
\begin{center}
\begin{echantillon}
build("tableau1","complet",[50,0.487])
\end{echantillon}
\input{tableau1}
\end{center}
\end{tcblisting}


Cette fonction surligne les valeurs de $a$ et $b$ telles que :
\begin{itemize}
\item[$\bullet$] $P(X\leqslant a)>2,5\%$
\item[$\bullet$] $P(X\leqslant b)\geqslant 97,5\%$
\end{itemize}

L'intervalle de fluctuation au seuil de 95\% est alors $\left[\dfrac{a}{n};\dfrac{b}{n}\right]$.

\subsection{Construire un tableau minimal}

Quand $n$ est tr�s grand, pour �viter que la construction se mette sur plusieurs pages, on peut utiliser la fonction :

\begin{tcblisting}{listing}
\begin{echantillon}
build("tableau2","min",[50,0.487,15,33]);
\end{echantillon}
\input{tableau2}
\end{tcblisting}


\subsection{Construire un tableau minimal tronqu�}

Dans l'�ventualit� o� il y aurait beaucoup de valeurs entre $a$ et $b$, on peut tronquer le tableau avec la 3\ieme{} option :

\begin{tcblisting}{listing}
\begin{echantillon}
build("tableau3","minbreak",[50,0.487,16,32,18,30]);
\end{echantillon}
\input{tableau3}
\end{tcblisting}

\newpage

\section{Changer les couleurs}

\subsection{Couleur des lignes pour $a$ et $b$}

\begin{tcblisting}{codeTEX}
\def\bkcolor{<couleur souhait�e>}
% Par d�faut, la couleur est : gray!10
\end{tcblisting}

\subsection{Couleur de la premi�re ligne}

\begin{tcblisting}{codeTEX}
\def\fstlinecolor{<couleur souhait�e>} % couleur de fond
% Par d�faut, la couleur est : gray
\def\txtfstlinecolor{<couleur souhait�e>} % couleur du texte
% Par d�faut, la couleur est : white
\end{tcblisting}

\subsection{Couleur des traits}

\begin{tcblisting}{codeTEX}
\def\bordercolor{<couleur souhait�e>}
% Par d�faut, la couleur est : black
\end{tcblisting}

\begin{tcblisting}{listing}
\def\fstlinecolor{blue!40}
\def\txtfstlinecolor{blue!50!black}
\def\bkcolor{blue!20}
\def\bordercolor{blue!50!black}
\begin{center}
\input{tableau3}
\end{center}
\end{tcblisting}

\end{document}