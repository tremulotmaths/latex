%%
%% Exemple pour l'utilisation du package cours.sty
%% 
%% Libre d'utiliser, modifier, distribuer
%% Germain Vallverdu <germain.vallverdu@univ-pau.fr>
%% 20 Octobre 2010
%%
\documentclass[12pt]{article}

\usepackage[utf8]{inputenc}
\usepackage[T1]{fontenc}
\usepackage[francais]{babel}
\usepackage{lipsum}
\usepackage{tabularx}
\renewcommand{\arraystretch}{1.5}
\usepackage[left=2.5cm,right=2.5cm,top=3cm,bottom=3cm]{geometry}

\usepackage{cours}

\title{Package cours.sty}
\author{Germain Vallverdu - \texttt{germain.vallverdu@univ-pau.fr}}
\date{\today}

% * * * * * * * * * * * * * * * * * * * * * * * * * * * * * * * * * * * * * * *
% * 
% * begin document
% * 
% * * * * * * * * * * * * * * * * * * * * * * * * * * * * * * * * * * * * * * *

\begin{document}

\maketitle

Le paquet \verb!cours.sty! apporte quelques commandes permettant de dessiner des icônes avec tikz et
des environnements utilisant ces icônes. Les icônes ou les environnement peuvent être utilisés dans
le cadre de cours pour poser une question, faire une remarque ou encore insister sur l'importance
d'un commentaire.

\section{Icônes}

Le package définit des commandes permettant de dessiner avec tikz des icônes pour une question une
remarque ou insister sur l'importance d'un commentaire. 

\begin{center}
\begin{tabularx}{0.8\textwidth}{clX}
    \hline
    \textbf{icône} & \textbf{description} & \textbf{commande} \\
    \hline
    \iconequestion & Une icône pour les questions & \verb!\iconequestion! \\
    \iconeattention & Une icône circulaire pour dire attention & \verb!\iconeattention! \\
    \iconetriangleattention & Une icône triangulaire pour dire attention &
    \verb!\iconetriangleattention! \\
    \iconeremarque & Une icône pour une remarque & \verb!\iconeremarque! \\
    \hline
\end{tabularx}
\end{center}

\section{Environnments}

\lipsum[1]

\begin{question}
    Mais au fait, comprennez vous le latin ?
\end{question}

\lipsum[2]

\begin{remarque}
    Vous auriez du faire du latin au collège.
\end{remarque}

\lipsum[3]

\begin{attention}
    La suite est importante, tant pis pour ceux qui ne comprennent pas le latin.
\end{attention}

\lipsum[4]

\section{Modifications des couleurs}

Il est possible de modifier les
couleurs des icônes ou du texte dans les environnements. Quatre commandes sont
disponibles. Les trois premières, du type \verb!\SetCouleurXXX! ou \verb!XXX! est
\verb!Question!, \verb!Remarque! ou \verb!Attention! permettent de définir la couleur de
l'environnement associé.

\begin{minipage}{0.49\textwidth}
    ~
\end{minipage}
\begin{minipage}{0.49\textwidth}
    \footnotesize\verb!\SetCouleurQuestion{40,40,40}}!
\end{minipage}

\begin{minipage}{0.49\textwidth}
    \begin{question}
        Couleur par defaut
    \end{question}
\end{minipage}
\begin{minipage}{0.49\textwidth}
    \SetCouleurQuestion{40,40,40}
    \begin{question}
        par defaut on donne les valeurs RGB entre 0 et 255
    \end{question}
\end{minipage}

\begin{minipage}{0.49\textwidth}
    \footnotesize\verb!\SetCouleurQuestion[cmyk]{0,1,0,1}}!
\end{minipage}
\begin{minipage}{0.49\textwidth}
    \footnotesize\verb!\SetCouleurQuestion[rgb]{0.9,0.3,0.1}}!
\end{minipage}

\begin{minipage}{0.49\textwidth}
    \SetCouleurQuestion[cmyk]{1,0,0,1}
    \begin{question}
        On peut donner les valeurs cmyk
    \end{question}
\end{minipage}
\begin{minipage}{0.49\textwidth}
    \SetCouleurQuestion[rgb]{0.9,0.3,0.1}
    \begin{question}
        ou les valeurs rgb entre 0 et 1
    \end{question}
\end{minipage}

De la même manière on a les commandes \verb!\SetCouleurAttention! et
\verb!\SetCouleurRemarque! pour les deux autres cas qui s'utilisent de la même manière.

La commande \verb!\SetCouleurDefaut! permet de rétablir les couleurs par défaut. \\

Commandes \verb!\SetCouleurAttention! ou \verb!\SetCouleurRemarque! ou
\verb!\SetCouleurQuestion! :
\begin{itemize}
    \item[\hspace{2ex}\bfseries\#1] (Optionnel) système de couleur (RGB, rgb, cmyk
        \ldots). Par défaut RGB.
    \item[\hspace{2ex}\bfseries\#2] code couleur qui dépend du système de couleur choisi. \\
\end{itemize}


\end{document}
