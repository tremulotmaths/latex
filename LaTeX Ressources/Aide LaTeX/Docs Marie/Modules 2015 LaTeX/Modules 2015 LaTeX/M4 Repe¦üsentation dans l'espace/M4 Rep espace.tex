\documentclass[11pt,dvips]{article} 
\usepackage[cp1252]{inputenc}
\usepackage[T1]{fontenc}
\usepackage{lmodern}
\usepackage[scaled=0.875]{helvet} 
  \renewcommand{\ttdefault}{lmtt}
\usepackage[colorlinks=true,urlcolor=blue, linkcolor=blue]{hyperref}
\usepackage{geometry}
\geometry{a4paper,hmargin=2.5cm,vmargin=2.5cm}
\usepackage{framed}
\usepackage{fancyhdr}
\usepackage{fancybox}
\usepackage{multicol}
\usepackage[usenames,dvipsnames,table]{xcolor}
\usepackage{epsfig}
\usepackage[framed]{ntheorem}
\usepackage[frenchb]{babel}

%-----------------------------------------------------------------------------------------
% Pr�sentation et compl�ments
%-----------------------------------------------------------------------------------------
\usepackage{setspace}
\usepackage{soul}
\usepackage[normalem]{ulem}

%Changer la numerotation des listes_________________________ 
\usepackage{enumerate}
 \renewcommand{\labelenumi}{\theenumi.}
 \renewcommand{\labelenumii}{\theenumii.}

% Num�rotation des sections_______________________________
\renewcommand{\thesection}{\Roman{section}}  
%\renewcommand{\thesubsection}{\qquad\arabic{subsection} } % chiffres arabes

% Autres symboles et caract�res_____________________________
\usepackage{textcomp}
    \newcommand{\euro}{\eurologo{}} 
\usepackage{siunitx} 
\usepackage{pifont,dingbat}

% Images et figures________________________________________________
\usepackage{graphicx}
\usepackage{pstricks,pst-plot,pst-text,pstricks-add,pst-eucl,pst-tree} 
\usepackage{pst-solides3d}                                               % pour cr�er diff�rents solides de l'espace
\usepackage{caption}

% Tableaux_______________________________________________
\usepackage{tabularx}
\usepackage{colortbl} 
\usepackage{array}

%--------------------------------------------------------------------------------------------
% Ecrire des math�matiques
%--------------------------------------------------------------------------------------------
\usepackage[fleqn]{amsmath}
\usepackage{bm}
\usepackage{amssymb}
\usepackage{mathrsfs}  
\usepackage[np]{numprint} 
\usepackage[upright]{fourier}  %!!!! change la police lorsqu'il est actif !!!  ???
   \renewcommand{\rmdefault}{lmr}

 
 %Figures tikz____________________________________________
 \usepackage{tkz-base, tkz-euclide, tkz-fct, tkz-tab}  % tkz-tukey ?
\usetikzlibrary{arrows, plotmarks}
\usetkzobj{all}
  
% QCM___________________________________________________
\usepackage{alterqcm}
    %Exemple: %\begin{alterqcm}
                       %\AQuestion{Question}{{Proposition 1},{Proposition 2},{Proposition 3}}
                       %\end{alterqcm}

%--------------------------------------------------------------------------------------------
% Raccourcis math�matiques 
%--------------------------------------------------------------------------------------------

%Ensembles de nombres____________________________________
\newcommand{\R}{\mathbb{R}}
\newcommand{\N}{\mathbb{N}}
\newcommand{\D}{\mathbb{D}}
\newcommand{\Z}{\mathbb{Z}}
\newcommand{\Q}{\mathbb{Q}}
\newcommand{\C}{\mathbb{C}}
%\newcommand{\P}{\mathbb{P}}

%Rep�res________________________________________________
\def\Oij{$\left(\text{O}\,;~\vect{\imath},~\vect{\jmath}\right)$}
\def\Oijk{$\left(\text{O}\,;~\vect{\imath},~ \vect{\jmath},~ \vect{k}\right)$}
\def\Ouv{$\left(\text{O}\,;~\vect{u},~\vect{v}\right)\ $}

%Arc de cercles____________________________________________
\newlength{\longarc}
\newcommand{\arc}[1]{\settowidth{%
\longarc}{$#1$}
\addtolength{\longarc}{-0.5em}%
\unitlength \longarc \ensuremath{%
\stackrel{\begin{picture}(1,0.2)
\qbezier(0,0)(0.5,0.2)(1,0)   % commande \arc{AB}
\end{picture}}{#1}}}

%Syst�me de deux �quations________________________________
\newcommand{\Syst}[2]{\left\{\begin{array}{lcr} #1\\ #2 \end{array}\right.}
\newcommand{\Systt}[3]{\left\{\begin{array}{lcr} #1\\ #2\\ #3 \end{array}\right.}

%Vecteurs________________________________________________
\newcommand{\vect}[1]{\mathchoice%
{\overrightarrow{\displaystyle\mathstrut#1\,\,}}%
{\overrightarrow{\textstyle\mathstrut#1\,\,}}%
{\overrightarrow{\scriptstyle\mathstrut#1\,\,}}%
{\overrightarrow{\scriptscriptstyle\mathstrut#1\,\,}}}

%Coordonn�es____________________________________________
\newcommand{\cp}[2]{\left( \,#1\, ;\,#2\,\right)}
\newcommand{\ce}[3]{\left( \,#1\, ;\,#2\,;\, #3\,\right)}
\newcommand{\cec}[3]{\left( \begin{array}{c}#1\\#2\\#3\end{array} \right)}

%Op�rations vecteurs_______________________________________
\newcommand{\ps}[2]{\vect{#1}\cdot\vect{#2}}
\newcommand{\pv}[2]{\vect{#1}\wedge\,\vect{#2}}
\newcommand{\pmi}[3]{\vect{#1}\cdot \left(\pv{#2}{#3}\right)}
\newcommand{\detv}[4]{\left| \begin{array}{c c}#1&#3\\#2&#4\end{array} \right|}

%Intervalles_______________________________________________
\newcommand{\Iff}[2]{\left[ \,#1\, ;\,#2\,\right]}
\newcommand{\Ifo}[2]{\left[ \,#1\, ;\,#2\,\right[}
\newcommand{\Ioo}[2]{\left] \,#1\, ;\,#2\,\right[}
\newcommand{\Iof}[2]{\left] \,#1\, ;\,#2\,\right]}

%Valeurs absolues, normes___________________________________
\newcommand{\abs}[1]{\left\lvert#1\right\rvert}
%\newcommand{\norme}[1]{\left\lVert#1\right\rVert}
\newcommand{\norme}[1]{\|#1\|}

%Calcul int�gral______________________________________
%\newcommand{\Int}[3]{\displaystyle{\int_#1^#2\,#3}}
\newcommand{\Int}{\displaystyle\int_}
\newcommand{\dx}{\;\text{d}x}
\newcommand{\dt}{\;\text{d}t}

%Limites__________________________________________________
\newcommand{\limx}[2]{\displaystyle{\lim_{x\to#1}\, #2}}      % pour �crire une limite � l'int�rieur d'un texte sans  \[   \]
\newcommand{\limt}[2]{\displaystyle{\lim_{t\to#1}\, #2}}
\newcommand{\limh}[2]{\displaystyle{\lim_{h\to#1}\, #2}}
%Nombres complexes_______________________________________
\def\i{\text{i}}

%Probabilit�s
\newcommand{\ev}[2]{$#1$\,:\,\og #2 \fg}  

%Statistiques
\newcommand{\cov}[2]{\text{cov}\,\left(#1\,;\,#2\right)}

%Suites
\newcommand{\un}{\left(u_n\right)}
\newcommand{\vn}{\left(v_n\right)}

%Trigonom�trie sph�rique____________________
%\def\psPointSphere(#1,#2,#3)#4                                        % placer un point d�fini par ses coordonn�es sph�riques
%{% #1 rayon  #2 longitude  #3 latitude
%\pstVerb{ /xP #1 #2 cos #3 cos mul mul def  /yP #1 #2 sin #3 cos mul mul def /zP #1 #3 sin mul def}
%\psPoint(xP,yP,zP){#4}}
\newcommand{\psPSphere}[4]                                        % placer un point d�fini par ses coordonn�es sph�riques
{% #1 rayon  #2 longitude  #3 latitude #4 nom du point
\pstVerb{ /xP #1 #2 cos #3 cos mul mul def  /yP #1 #2 sin #3 cos mul mul def /zP #1 #3 sin mul def}
\psPoint(xP,yP,zP){#4}}

% Expressions______________________________________________
\newcommand{\ssi}{si, et seulement si, }

%----------------------------------------------------------------------------------------------
%Ins�rer un algorithme avec Algobox
%----------------------------------------------------------------------------------------------
\usepackage{ucs}
\usepackage{algorithm}
\usepackage{algpseudocode}
\usepackage{alltt}
%%ALGO                                                  ____ Mis dans une macro_____
%\definecolor{fond}{RGB}{136,136,136}
%\definecolor{sicolor}{RGB}{128,0,128}
%\definecolor{tantquecolor}{RGB}{221,111,6}
%\definecolor{pourcolor}{RGB}{187,136,0}
%\definecolor{bloccolor}{RGB}{128,0,0}
%\newenvironment{cadrecode}{%
%\def\FrameCommand{{\color{fond}\vrule width 5pt}\fcolorbox{fond}{white}}%
%\MakeFramed {\hsize \linewidth \advance\hsize-\width \FrameRestore}\begin{footnotesize}}%
%{\end{footnotesize}\endMakeFramed}
%\makeatletter
%\def\therule{\makebox[\algorithmicindent][l]{\hspace*{.5em}\color{fond} \vrule width 1pt height .75\baselineskip depth .25\baselineskip}}%
%\newtoks\therules
%\therules={}
%\def\appendto#1#2{\expandafter#1\expandafter{\the#1#2}}
%\def\gobblefirst#1{#1\expandafter\expandafter\expandafter{\expandafter\@gobble\the#1}}%
%\def\Ligne{\State\unskip\the\therules}% 
%\def\pushindent{\appendto\therules\therule}%
%\def\popindent{\gobblefirst\therules}%
%\def\printindent{\unskip\the\therules}%
%\def\printandpush{\printindent\pushindent}%
%\def\popandprint{\popindent\printindent}%
%\def\Variables{\Ligne \textcolor{bloccolor}{\textbf{VARIABLES}}}
%\def\Si#1{\Ligne \textcolor{sicolor}{\textbf{SI}} #1 \textcolor{sicolor}{\textbf{ALORS}}}%
%\def\Sinon{\Ligne \textcolor{sicolor}{\textbf{SINON}}}%
%\def\Pour#1#2#3{\Ligne \textcolor{pourcolor}{\textbf{POUR}} #1 \textcolor{pourcolor}{\textbf{ALLANT\_DE}} #2 \textcolor{pourcolor}{\textbf{A}} #3}%
%\def\Tantque#1{\Ligne \textcolor{tantquecolor}{\textbf{TANT\_QUE}} #1 \textcolor{tantquecolor}{\textbf{FAIRE}}}%
%\algdef{SE}[WHILE]{DebutTantQue}{FinTantQue}
%  {\pushindent \printindent  \textcolor{tantquecolor}{\textbf{DEBUT\_TANT\_QUE}}}
%  {\printindent \popindent  \textcolor{tantquecolor}{\textbf{FIN\_TANT\_QUE}}}%
%\algdef{SE}[FOR]{DebutPour}{FinPour}
%  {\pushindent \printindent \textcolor{pourcolor}{\textbf{DEBUT\_POUR}}}
%  {\printindent \popindent  \textcolor{pourcolor}{\textbf{FIN\_POUR}}}%
%\algdef{SE}[IF]{DebutSi}{FinSi}%
%  {\pushindent \printindent \textcolor{sicolor}{\textbf{DEBUT\_SI}}}
%  {\printindent \popindent \textcolor{sicolor}{\textbf{FIN\_SI}}}%
%\algdef{SE}[IF]{DebutSinon}{FinSinon}
%  {\pushindent \printindent \textcolor{sicolor}{\textbf{DEBUT\_SINON}}}
%  {\printindent \popindent \textcolor{sicolor}{\textbf{FIN\_SINON}}}%
%\algdef{SE}[PROCEDURE]{DebutAlgo}{FinAlgo}
%   {\printandpush \textcolor{bloccolor}{\textbf{DEBUT\_ALGORITHME}}}%
%   {\popandprint \textcolor{bloccolor}{\textbf{FIN\_ALGORITHME}}}%
%\makeatother
%\newenvironment{algobox}%
%{%
%\begin{ttfamily}
%\begin{algorithmic}[1]
%\begin{cadrecode}
%}
%{%
%\end{cadrecode}
%\end{algorithmic}
%\end{ttfamily}
%}

%_____________________________________________________
%Ins�rer des calculs r�alis�s avec maxima
%__________________________________________________
\setlength{\parskip}{\medskipamount}
\setlength{\parindent}{0pt}
\DeclareUnicodeCharacter{00B5}{\ensuremath{\mu}}
\usepackage{color}
\usepackage{ifthen}
%\DeclareMathOperator{\abs}{abs}
\usepackage{animate} % This package is required because the wxMaxima configuration option
                      % "Export animations to TeX" was enabled when this file was generated.

\definecolor{labelcolor}{RGB}{100,0,0}

%----------------------------------------------------------------------------------------------
%Mise en forme des th�or�mes
%----------------------------------------------------------------------------------------------
%\makeatletter
%\def\ntheorem@breakstrut{\rule{0pt}{1\baselineskip}}
%\renewtheoremstyle{break}%
%  {\item[\rlap{\vbox{\hbox{\hskip\labelsep \theorem@headerfont
%          ##1\ ##2\theorem@separator}\hbox{\ntheorem@breakstrut}}}]}            % augmente l'espace entre le titre et le texte
%  {\item[\rlap{\vbox{\hbox{\hskip\labelsep \theorem@headerfont                      % th num�rot�s
%          ##1\ ##2\ (##3)\theorem@separator}\hbox{\ntheorem@breakstrut}}}]}
%
%\def\ntheorem@nonumberbreakstrut{\rule{0pt}{0.9\baselineskip}}
%\renewtheoremstyle{nonumberbreak}%
%  {\item[\rlap{\vbox{\hbox{\hskip\labelsep \theorem@headerfont
%          ##1\theorem@separator}\hbox{\ntheorem@breakstrut}}}]}            % augmente l'espace entre le titre et le texte
%  {\item[\rlap{\vbox{\hbox{\hskip\labelsep \theorem@headerfont              % th non num�rot�s
%          ##1\ (##3)\theorem@separator}\hbox{\ntheorem@breakstrut}}}]}
%\makeatother
%
%\theoremstyle{break}
%\theorembodyfont{\upshape}
%\newframedtheorem{Th}{Th�or�me}
%\newframedtheorem{Prop}{Propri�t�}
%\theorembodyfont{\slshape}
%\definecolor{azure}{rgb}{.88,.932,.932}
%\shadecolor{azure}
%\newshadedtheorem{Def}{D�finition}
%\theoremstyle{nonumberbreak}
%\theorembodyfont{\rmfamily}
%\newtheorem{Ex}{Exemple}
%\newtheorem{Exo}{Exercice}
%\theoremheaderfont{\scshape}
%\newtheorem{Rem}{Remarque}


\usepackage{boiboites}
%_________________________________________________________

\renewcommand{\arraystretch}{1.5} 

\input{th_ex.tex}

\fancyhf{}
\rhead{ \textcolor{orange!90}{\textsc{Visualisation\\Construction}}}
\lhead{}
\rfoot{ \textcolor{blue}{Module}}
\lfoot{ \textcolor{blue}{Seconde}}
\cfoot{}
\renewcommand \headrulewidth{0pt}
\renewcommand {\footrule}{{\color{orange!70}\rule{\textwidth}{0.9pt}}\\}
\pagestyle{fancy}

\newboxedtheoremFiEx[boxcolor=orange!70, background=orange!8, titlebackground=white,
titleboxcolor = orange]{Chapitre}{Module n� 4}{}

\newboxedtheorembl[boxcolor=white, background=white, titlebackground=white,
titleboxcolor = blue]{Exo}{Exercice}{compteurEX}


\begin{document}
\vspace{-30pt}
%\textcolor{blue}{  %_____________texte en bleu
\begin{Chapitre}
   \begin{center}
   \textcolor{blue}{\bsc{Repr�sentation dans l'espace}} 
   \end{center}  
   \vspace{-10pt} 
\end{Chapitre}
\begin{flushright}
\end{flushright}
\setlength{\parskip}{3ex plus 1ex minus 1ex}
\setlength{\parindent}{0cm} 
\vspace{-30pt}
%_______________________________________________________________
\textcolor{blue}{\textbf{I. Figures et r�alit�}}\\[8pt]
\begin{minipage}{0.6\linewidth}
\og Toutes les images sont des mensonges, l'absence d'image est aussi mensonge\fg.
\end{minipage}
\begin{minipage}{0.4\linewidth}
\begin{flushright} \textsc{Bouddha} \end{flushright}
\end{minipage}

\begin{minipage}{0.6\linewidth}
\begin{center}
\definecolor{ffwwqq}{rgb}{1,.648,.31}
\definecolor{ffzztt}{rgb}{.932,.796,.68}
\definecolor{ffccww}{rgb}{1,.855,.725}
\definecolor{ffwwqz}{rgb}{.804,.52,0}
\begin{tikzpicture}[scale=0.3, line cap=round,line join=round,>=triangle 45,x=1.0cm,y=1.0cm]
\clip(-3.65567500235,-2.07934723942) rectangle (9.0762559089,8.20151675843);
\fill[color=ffccww,fill=ffccww!80,fill opacity=0.1] (2.64709645348,7.00484932585) -- (3.74832919025,7.00484932585) -- (-1.,1.) -- (5.17993174805,1.00313091045) -- (5.74890199539,0.0120214473524) -- (-3.,0.) -- cycle;
\fill[color=ffwwqq,fill=ffwwqq!60,fill opacity=0.25] (-1.,1.) -- (0.150942311962,1.00058309662) -- (3.56479040079,5.5548928891) -- (7.41910497949,-0.997441894688) -- (7.91465971104,-0.153163463163) -- (3.74832919025,7.00484932585) -- cycle;
\fill[color=ffwwqq,fill=ffwwqq!80,fill opacity=0.4] (2.98794300574,4.78533851185) -- (5.74890199539,0.0120214473524) -- (-3.,0.) -- (-2.,-1.) -- (7.41910497949,-0.997441894688) -- (3.56479040079,5.5548928891) -- cycle;
%\draw (-3.,0.)-- (-2.,-1.);
%\draw (-2.,-1.)-- (7.41910497949,-0.997441894688);
%\draw (-3.,0.)-- (5.74890199539,0.0120214473524);
%\draw (-3.,0.)-- (2.64709645348,7.00484932585);
%\draw (-1.,1.)-- (3.72997531131,6.96814156796);
%\draw (-1.,1.)-- (5.17993174805,1.00313091045);
%\draw (7.41910497949,-0.997441894688)-- (7.91465971104,-0.153163463163);
%\draw (7.91465971104,-0.153163463163)-- (3.74832919025,7.00484932585);
%\draw (3.74832919025,7.00484932585)-- (2.64709645348,7.00484932585);
%\draw (0.150942311962,1.00058309662)-- (3.56479040079,5.5548928891);
%\draw (3.56479040079,5.5548928891)-- (7.41910497949,-0.997441894688);
%\draw (5.74890199539,0.0120214473524)-- (2.98794300574,4.78533851185);
\draw [color=ffwwqz] (2.64709645348,7.00484932585)-- (3.74832919025,7.00484932585);
\draw [color=ffwwqz] (3.74832919025,7.00484932585)-- (-1.,1.);
\draw [color=ffwwqz] (-1.,1.)-- (5.17993174805,1.00313091045);
\draw [color=ffwwqz] (5.17993174805,1.00313091045)-- (5.74890199539,0.0120214473524);
\draw [color=ffwwqz] (5.74890199539,0.0120214473524)-- (-3.,0.);
\draw [color=ffwwqz] (-3.,0.)-- (2.64709645348,7.00484932585);
\draw [color=ffwwqz] (-1.,1.)-- (0.150942311962,1.00058309662);
\draw [color=ffwwqz] (0.150942311962,1.00058309662)-- (3.56479040079,5.5548928891);
\draw [color=ffwwqz] (3.56479040079,5.5548928891)-- (7.41910497949,-0.997441894688);
\draw [color=ffwwqz] (7.41910497949,-0.997441894688)-- (7.91465971104,-0.153163463163);
\draw [color=ffwwqz] (7.91465971104,-0.153163463163)-- (3.74832919025,7.00484932585);
\draw [color=ffwwqz] (3.74832919025,7.00484932585)-- (-1.,1.);
\draw [color=ffwwqz] (2.98794300574,4.78533851185)-- (5.74890199539,0.0120214473524);
\draw [color=ffwwqz] (5.74890199539,0.0120214473524)-- (-3.,0.);
\draw [color=ffwwqz] (-3.,0.)-- (-2.,-1.);
\draw [color=ffwwqz] (-2.,-1.)-- (7.41910497949,-0.997441894688);
\draw [color=ffwwqz] (7.41910497949,-0.997441894688)-- (3.56479040079,5.5548928891);
\draw [color=ffwwqz] (3.56479040079,5.5548928891)-- (2.98794300574,4.78533851185);
\end{tikzpicture}
   \end{center}
\end{minipage}
\begin{minipage}{0.4\linewidth}
\begin{flushright}Le Tripoutre de  \textsc{Penrose} (1958) \end{flushright}
\end{minipage}

Certains objets n'existent pas et pourtant, on peut les dessiner. \\
Le n�erlandais \textsc{Maurits Cornelis Escher} est celui qui cr�a le plus grand nombre de ces curieuses constructions. \\[10pt]
\emph{Recherche � r�aliser}\\
Se documenter sur  \textsc{M. C. Escher} et imprimer une de ses constructions.
\vspace{15pt}

%_______________
\textcolor{blue}{\textbf{II. Puzzle en 3D}}\\[8pt]
$ABCDEFGH$ est un cube dont l'ar�te $a$ mesure 4 cm.
\begin{center}
   \definecolor{ttttff}{rgb}{0.2,0.2,1.}
\definecolor{qqqqff}{rgb}{0.,0.,1.}
\begin{tikzpicture}[scale=0.5, line cap=round,line join=round,>=triangle 45,x=1.0cm,y=1.0cm]
\clip(-2.52483226207,-3.0109435999) rectangle (7.16407517611,5.04175761527);
\draw  (-1.,-2.)-- (4.,-2.);
\draw  (4.,-2.)-- (6.,-1.);
\draw [dash pattern=on 2pt off 2pt] (-1.,-2.)-- (1.,-1.);
\draw [dash pattern=on 2pt off 2pt] (1.,-1.)-- (6.,-1.);
\draw  (6.,-1.)-- (6.,4.);
\draw  (6.,4.)-- (4.,3.);
\draw  (4.,3.)-- (4.,-2.);
\draw  (-1.,-2.)-- (-1.,3.);
\draw  (-1.,3.)-- (1.,4.);
\draw  (1.,4.)-- (6.,4.);
\draw  (4.,3.)-- (-1.,3.);
\draw [dash pattern=on 2pt off 2pt] (1.,4.)-- (1.,-1.);
\begin{scriptsize}
\draw [fill=qqqqff] (4.,-2.) circle (0.5pt);
\draw[color=qqqqff] (4.05207510491,-2.37) node {$A$};
\draw [fill=qqqqff] (-1.,-2.) circle (0.5pt);
\draw[color=qqqqff] (-1.3,-2.37) node {$B$};
\draw [fill=qqqqff] (1.,-1.) circle (0.5pt);
\draw[color=qqqqff] (1.03632245858,-1.28) node {$C$};
\draw [fill=qqqqff] (6.,-1.) circle (0.5pt);
\draw[color=qqqqff] (6.3,-1.28) node {$D$};
\draw [fill=qqqqff] (4.,3.) circle (0.5pt);
\draw[color=qqqqff] (4.05207510491,3.5) node {$E$};
\draw [fill=qqqqff] (-1.,3.) circle (0.5pt);
\draw[color=qqqqff] (-1.3,3.5) node {$F$};
\draw [fill=qqqqff] (1.,4.) circle (0.5pt);
\draw[color=qqqqff] (0.715497708975,4.4) node {$G$};
\draw [fill=qqqqff] (6.,4.) circle (0.5pt);
\draw[color=qqqqff] (6.3,4.4) node {$H$};
\end{scriptsize}
\end{tikzpicture}
\end{center}
\begin{enumerate}
\item Reproduire ce cube, puis tracer la pyramide $ABCGF$ de base $BCGF$.
\item Justifier que chaque base lat�rale de cette pyramide est un triangle rectangle.
\item  Dessiner, en vraie grandeur, la base carr�e $BCGF$ avec $BC=4$ cm, puis chacune des faces triangulaires $ABC$, $ABF$, $AFG$, $ACG$ � partir des c�t�s du carr�.
\item Fabriquer le patron de cette pyramide.
\item \begin{enumerate}
	\item V�rifier que la r�union correctement r�alis�e de trois pyramides, ainsi obtenues, permet de reconstituer le cube initial.
	\item Retrouver alors la formule qui donne le volume d'une pyramide.
	\end{enumerate}

\end{enumerate}
%\vspace{15pt}
\newpage
%_______________
\textcolor{blue}{\textbf{III. Calculs en vraie grandeur}}\\[8pt]
La G�ode est une salle de cin�ma, � la cit� des sciences et de l'industrie de Paris. Cette g�ode a la forme d'une calotte sph�rique pos�e sur le sol. Elle est issue d'une sph�re de diam�tre 36 m�tres, et son point culminant se trouve � 29 m�tres du sol.\\
On se propose de d�terminer la surface au sol disponible dans une telle structure.
\begin{center}
   \definecolor{zzttqq}{rgb}{0.6,0.2,0.}
\definecolor{qqqqcc}{rgb}{0.,0.,0.8}
\definecolor{qqzzcc}{rgb}{0.,0.6,0.8}
\definecolor{qqqqff}{rgb}{0.,0.,1.}
\begin{tikzpicture}[scale=0.8, line cap=round,line join=round,>=triangle 45,x=1.0cm,y=1.0cm]
\clip(0.582499189153,-4.2732020381) rectangle (16.808577103,3.99852716738);
\fill[color=zzttqq,fill=zzttqq!20,fill opacity=0.1] (1.5,-3.5) -- (5.,-1.) -- (16.,-1.) -- (13.,-3.5) -- cycle;
\draw [shift={(9.,0.)},color=qqzzcc,fill=qqzzcc!15,fill opacity=0.15]  plot[domain=-0.588002603548:3.72959525714,variable=\t]({1.*3.60555127546*cos(\t r)+0.*3.60555127546*sin(\t r)},{0.*3.60555127546*cos(\t r)+1.*3.60555127546*sin(\t r)});
\draw [rotate around={0.:(9.,-2.)},color=qqzzcc,fill=qqzzcc!25,fill opacity=0.3] (9.,-2.) ellipse (3.02745041323cm and 0.40676283581cm);
\draw [dash pattern=on 3pt off 2pt,color=qqqqcc] (9.,0.)-- (9.,-2.);
\draw [dash pattern=on 3pt off 2pt,color=qqqqcc] (6.,-2.)-- (12.,-2.);
\draw [dash pattern=on 3pt off 2pt,color=qqqqcc] (9.,0.)-- (6.,-2.);
\draw [color=zzttqq] (1.5,-3.5)-- (5.,-1.);
\draw [color=zzttqq] (5.,-1.)-- (16.,-1.);
\draw [color=zzttqq] (16.,-1.)-- (13.,-3.5);
\draw [color=zzttqq] (13.,-3.5)-- (1.5,-3.5);

\begin{scriptsize}
\draw [fill=qqqqff] (9.,0.) circle (1.0pt);
\draw[color=qqqqff] (9.2,0.269306326539) node {$O$};
\draw [fill=qqqqff] (6.,-2.) circle (1.0pt);
\draw[color=qqqqff] (5.66058714264,-2.2) node {$P$};
\draw [fill=qqqqff] (9.,-2.) circle (1.0pt);
\draw[color=qqqqff] (9.2,-2.2) node {$H$};
\end{scriptsize}
\end{tikzpicture}
\end{center}
\begin{enumerate}
\item Reproduire la figure.
\item Calculer la longueur $OH$.
\item  Calculer une valeur approch�e de $PH$, arrondie au cm.
\item Calculer une valeur approch�e de l'aire de la surface au sol disponible dans la salle.
\end{enumerate}

\end{document}

