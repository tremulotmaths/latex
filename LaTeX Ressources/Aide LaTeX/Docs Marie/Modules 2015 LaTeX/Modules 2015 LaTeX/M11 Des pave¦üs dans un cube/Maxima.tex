
% Ins�rer une page de calculs Maxima


%% Created with wxMaxima 15.04.0
%\documentclass{article}
%\setlength{\parskip}{\medskipamount}
%\setlength{\parindent}{0pt}
%\usepackage[utf8]{inputenc}
%\DeclareUnicodeCharacter{00B5}{\ensuremath{\mu}}
%\usepackage{graphicx}
%\usepackage{color}
%\usepackage{amsmath}
%\usepackage{ifthen}
\newsavebox{\picturebox}
\newlength{\pictureboxwidth}
\newlength{\pictureboxheight}
\newcommand{\includeimage}[1]{
    \savebox{\picturebox}{\includegraphics{#1}}
    \settoheight{\pictureboxheight}{\usebox{\picturebox}}
    \settowidth{\pictureboxwidth}{\usebox{\picturebox}}
    \ifthenelse{\lengthtest{\pictureboxwidth > .95\linewidth}}
    {
        \includegraphics[width=.95\linewidth,height=.80\textheight,keepaspectratio]{#1}
    }
    {
        \ifthenelse{\lengthtest{\pictureboxheight>.80\textheight}}
        {
            \includegraphics[width=.95\linewidth,height=.80\textheight,keepaspectratio]{#1}
            
        }
        {
            \includegraphics{#1}
        }
    }
}
%\DeclareMathOperator{\abs}{abs}
%\usepackage{animate} % This package is required because the wxMaxima configuration option
%                      % "Export animations to TeX" was enabled when this file was generated.
%
%\definecolor{labelcolor}{RGB}{100,0,0}

%\begin{document}
%
%\noindent
%%%%%%%%%%%%%%%%
%%%% INPUT:
%\begin{minipage}[t]{8ex}{\color{red}\bf
%\begin{verbatim}
%(%i1) 
%\end{verbatim}}
%\end{minipage}
%\begin{minipage}[t]{\textwidth}{\color{blue}
%\begin{verbatim}
%V(x):=x^2*(6-x);
%\end{verbatim}}
%\end{minipage}
%%%% OUTPUT:
%
%
%\begin{math}\displaystyle
%\parbox{8ex}{\color{labelcolor}(\%o1) }
%\mathrm{V}\left( x\right) :={{x}^{2}}\cdot \left( 6-x\right) \mbox{}
%\end{math}
%%%%%%%%%%%%%%%%
%
%
%\noindent
%%%%%%%%%%%%%%%%
%%%% INPUT:
%\begin{minipage}[t]{8ex}{\color{red}\bf
%\begin{verbatim}
%(%i2) 
%\end{verbatim}}
%\end{minipage}
%\begin{minipage}[t]{\textwidth}{\color{blue}
%\begin{verbatim}
%V(x)-V(4);
%\end{verbatim}}
%\end{minipage}
%%%% OUTPUT:
%
%
%\begin{math}\displaystyle
%\parbox{8ex}{\color{labelcolor}(\%o2) }
%\left( 6-x\right) \cdot {{x}^{2}}-32\mbox{}
%\end{math}
%%%%%%%%%%%%%%%%
%
%
%\noindent
%%%%%%%%%%%%%%%%
%%%% INPUT:
%\begin{minipage}[t]{8ex}{\color{red}\bf
%\begin{verbatim}
%-->  
%\end{verbatim}}
%\end{minipage}
%\begin{minipage}[t]{\textwidth}{\color{blue}
%\begin{verbatim}
%V(x)-V(4)
%\end{verbatim}}
%\end{minipage}
%
%
%\noindent
%%%%%%%%%%%%%%%%
%%%% INPUT:
%\begin{minipage}[t]{8ex}{\color{red}\bf
%\begin{verbatim}
%(%i3) 
%\end{verbatim}}
%\end{minipage}
%\begin{minipage}[t]{\textwidth}{\color{blue}
%\begin{verbatim}
%expand(V(x)-V(4));
%\end{verbatim}}
%\end{minipage}
%%%% OUTPUT:
%
%
%\begin{math}\displaystyle
%\parbox{8ex}{\color{labelcolor}(\%o3) }
%-{{x}^{3}}+6\cdot {{x}^{2}}-32\mbox{}
%\end{math}
%%%%%%%%%%%%%%%%
%
%
%\noindent
%%%%%%%%%%%%%%%%
%%%% INPUT:
%\begin{minipage}[t]{8ex}{\color{red}\bf
%\begin{verbatim}
%(%i4) 
%\end{verbatim}}
%\end{minipage}
%\begin{minipage}[t]{\textwidth}{\color{blue}
%\begin{verbatim}
%factor(%);
%\end{verbatim}}
%\end{minipage}
%%%% OUTPUT:
%
%
%\begin{math}\displaystyle
%\parbox{8ex}{\color{labelcolor}(\%o4) }
%-{{\left( x-4\right) }^{2}}\cdot \left( x+2\right) \mbox{}
%\end{math}
%%%%%%%%%%%%%%%%
%
%\end{document}
