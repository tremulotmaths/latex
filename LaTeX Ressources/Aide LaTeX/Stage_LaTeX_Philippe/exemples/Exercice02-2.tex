\documentclass[12pt,french]{article}
\usepackage[utf8]{inputenc}
\usepackage[T1]{fontenc}
\usepackage{kpfonts}
\usepackage[a4paper,margin=2cm]{geometry}
\usepackage{mathtools,amssymb}
\usepackage[dvipsname]{xcolor}
\usepackage{babel}

\begin{document}
\begin{center}
    \scshape\bfseries Contrôle de Mathématiques
\end{center}\bigskip

Les consignes suivantes sont importantes :
\begin{itemize}
    \item il faut répondre aux questions par des phrases complètes ;
    \item le soin de la copie et la rédaction seront pris en compte dans l'appréciation de la copie.
\end{itemize}\bigskip

\textit{Exercice 1 :}\hfill (5 points)

\begin{enumerate}
    \item Donner la définition d'un quadrilatère.
    \item Donner la définition d'un parallélogramme.
        \begin{enumerate}
            \item Quelle propriété permet de dire qu'un parallélogramme est un rectangle ?\label{rect}
            \item Quelle propriété permet de dire qu'un parallélogramme est un losange ?\label{los}
        \end{enumerate}
    \item À partir des questions \ref{rect} et \ref{los}, déterminer une propriété permettant de dire qu'un parallélogramme est un carré.
\end{enumerate}
\end{document} 