\documentclass[12pt,french]{article}
\usepackage[utf8]{inputenc}
\usepackage[T1]{fontenc}
\usepackage{kpfonts}
\usepackage[a4paper,margin=2cm]{geometry}
\usepackage{mathtools,amssymb}
\usepackage{babel}

\begin{document}
\textbf{Définition 1.} Pour tous réels $a$, $b$ et $c \neq 0$, on a :\par
$\frac{a}{c} + \frac b c = \frac{a+b}{c}$. % Facile !

\textbf{Définition 2.} Soit $x \geqslant 0$ et $A \geqslant 0$ :
$\sqrt x = A \Leftrightarrow A^2 = x$. % Evident !


$\sqrt[3]{x}    =   A \Leftrightarrow A^3 = x$. % marche aussi avec x < 0 !

\'Evidemment,
ce
n'est
pas                bien                  compliqué.
\end{document}
C'est vraiment trop bien !! 