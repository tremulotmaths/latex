\documentclass[10pt,openright,oneside,french]{book}
\input preambule_2014
\usepackage{xargs}
\usepackage{environ}

\renewcommand\MaCouleur{cyan}

\renewcommand*{\hrulefill}[1][0.3mm]{\leavevmode \leaders \hrule height #1 \hfill \kern 0pt} % filet horizontal à épaisseur modifiable

%___________________________
%===    Définition des dimensions de la page
%------------------------------------------------------

\setlength\paperheight{297mm}
\setlength\paperwidth{210mm}
\setlength\parindent{0mm} % longueur de l'alinea
\setlength{\evensidemargin}{1cm}% Marge gauche sur pages paires
\setlength{\oddsidemargin}{1cm}% Marge gauche sur pages impaires
\setlength{\marginparwidth}{2.65cm} % taille réservé pour les notes de marges
\setlength{\marginparsep}{0.5cm} % espace entre les notes de marge et le texte
\setlength{\headsep}{5pt}% Entre le haut de page et le texte
\setlength{\headheight}{20pt}% Haut de page
\setlength{\textheight}{27cm}% Hauteur de la zone de texte
\setlength{\textwidth}{14.5cm}% Largeur de la zone de texte

%___________________________
%===    Marges colorées
%------------------------------------------------------

%------------------------------------------------------------------------------

% Paramétrage rectangles et texte (zone modifiable)
\newcommand{\RWidth}{4cm} % largeur rectangle
\newcommand{\RHOffset}{-5mm} % Position par rapport à la marge
                            % (Si négatif, on est en pleine page)
\newcommand{\RColor}{\MaCouleur!25} % couleur rectangle
\newcommand{\TColor}{black} % couleur texte
\newcommand{\vcorrection}{2.8cm} % décalage verticale texte
\newcommand{\hcorrection}{0cm} % décalage horizontale texte

% Zone à ne pas modifier (normalement)
\newcommand\RectangleDroite[2]{%
 % Rectangle à droite pour les pages impaires
 \leavevmode%
 \hbox to0pt{\hss#1}%
 \setbox0=\hbox{%
   \hsize=0pt
   \vbox to0pt{%
     \vss
     \hbox to0pt{%
       \color{\RColor}%
       % Pour le rectangle de droite, le calcul du décalage est un
       % poil lourd !
       \hspace*{%
         \dimexpr
           \paperwidth-\textwidth-\oddsidemargin-1in-\hoffset
           -\RWidth-\RHOffset
       }%
       % Rectangle
       \vrule width\RWidth
              height\dimexpr\headheight+\topmargin+1in+\voffset+5mm\relax
              depth\dimexpr \paperheight +1cm\relax
       % On replace le point de référence horizontalement
       \dimen0=\RWidth
       \dimen0=-0.7\dimen0
       \advance\dimen0 \hcorrection
       \vrule width\dimen0
              height0pt
       % Texte
       \color{\TColor}%
       \hbox to0pt{%
         \hss
         \raisebox{\dimexpr-0.5\paperheight+\vcorrection}{%
           \vtop to0pt{%
             \vss
             \rotatebox{-90}{\Large\bfseries #2}%
             \vss
           }%
         }%
       }%
       \hss
     }%
   }%
 }%
 \dp0=0pt
 \ht0=0pt
 \box0
}
\newcommand\RectangleGauche[2]{%
 % Rectangle à gauche pour les pages paires
 \leavevmode%
 \hbox to0pt{#1\hss}%
 \setbox0=\hbox{%
   \hsize=0pt
   \vbox to0pt{%
     \vss
     \hbox to0pt{%
       \hss
       \color{\RColor}%
       \vrule width\RWidth
              height\dimexpr\headheight+\topmargin+1in+\voffset+5mm\relax
              depth\dimexpr \paperheight +1cm\relax
       % Texte
       \color{\TColor}%
       \hbox to0pt{%
         % On replace le point de référence horizontalement
         \dimen0=\RWidth
         \dimen0=-0.5\dimen0
         \advance\dimen0 -\hcorrection
         \kern\dimen0
         \raisebox{\dimexpr-0.5\paperheight+\vcorrection}{%
           \vtop to0pt{%
             \vss
             \rotatebox{90}{\Large\bfseries #2}%
             \vss
           }%
         }%
         \hss
       }%
       % Pour le rectangle de gauche, le calcul du décalage est un
       % peu plus court !
       \hspace*{%
         \dimexpr
           \evensidemargin+1in+\hoffset-\RWidth-\RHOffset
       }%
     }%
   }%
 }%
 \dp0=0pt
 \ht0=0pt
 \box0
}

\fancypagestyle{plain}{%
 \fancyhf{}
 \fancyfoot[C]{}
 %\fancyhead[RO]{\RectangleDroite{}{}}
 \fancyhead[LO]{\RectangleGauche{}{}} % Normalement inutile mais bon...
 \renewcommand\headrulewidth{0pt}
}
\fancypagestyle{normal}{%
 \fancyhf{}
 \fancyfoot[C]{}
 %\fancyhead[RO]{\RectangleDroite{}{}}
 \fancyhead[LE]{}
 \fancyhead[RE]{}
 \fancyhead[LO]{\RectangleGauche{}{}}
 \renewcommand\headrulewidth{0pt}
}
\pagestyle{normal} % style par défaut

%------------------------------------------------------------------------------

%___________________________
%===    Mise en forme des sections
%------------------------------------------------------


\makeatletter
\renewcommand{\thesection}{\color{red}\Roman{section}.\hspace{-0.5em}}
\renewcommand{\section}{%
    \@startsection{section}%
    {1}%
    {0pt}%
    {-15pt \@plus -1ex \@minus -.2ex}%
    {10pt \@plus .2ex \@minus .2ex}%
    {\normalfont\Large\bfseries\color{red}}%
}
\makeatother


\makeatletter
\renewcommand{\thesubsection}{\color{OliveGreen}\Alph{subsection}.\hspace{-0.5em}}
\renewcommand{\subsection}{%
    \@startsection{subsection}%
    {2}%
    {10pt}%
    {-12pt \@plus -1ex \@minus -.2ex}%
    {10pt \@plus .2ex \@minus.2ex}%
    {\normalfont\large\bfseries\color{OliveGreen}}%
}
\makeatother

\makeatletter
\newcounter{sss}[subsection]
\renewcommand{\thesss}{\arabic{sss}.\hspace{-0.5em}}
\renewcommand{\subsubsection}%
            {\@startsection{subsubsection}%
            {3}%
            {1em}%
             {-1ex\@plus -1ex \@minus -.2ex}%
             {0.7ex \@plus .2ex}%
             {\hspace*{20pt}\refstepcounter{sss}\normalfont\bfseries\thesss}}
\makeatother

%___________________________
%===    Mise en forme des chapitres
%------------------------------------------------------
\newcommand\PartProgramme{Fonctions}
\newlength\LongueurPartProgramme
\newlength{\LongueurCompteur}
\newcommand\Taille{\Large}
\newcommand\taille{0.9\linewidth}

\newcommand{\pbs}[1]{\let\tmp\\#1\let\\\tmp}
\makeatletter
\renewcommand{\@makechapterhead}[1]{%
\settowidth{\LongueurPartProgramme}{\Taille\textbf{\PartProgramme}}
    {\color{\MaCouleur}
        \settowidth{\LongueurCompteur}{\bfseries\itshape\fontsize{70}{0}\selectfont\thechapter}
        \addtolength{\LongueurCompteur}{10pt}
        \hspace{-\LongueurCompteur}
        \begin{tabular}{@{}m{\LongueurCompteur}@{}m{\taille}@{}}%
        {\bfseries\itshape\fontsize{70}{0}\selectfont\thechapter} &
        \parbox{\taille}{\raggedright\Huge \bfseries #1\par\nobreak}
        \end{tabular}
        \par\nobreak
        \vspace*{10pt}
        \textcolor{\MaCouleur!25}{\hspace{-1em}\hrulefill[0.5ex]} \parbox{\LongueurPartProgramme}{\centering\Taille\textbf{\PartProgramme}}
    }
    \vspace*{20pt}
}
\makeatother

%_________________________
%===    Environnements de cours
%------------------------------------------------------
\newlength{\Longueur}
%___________________________
%===    Définitions
%------------------------------------------------------

%\newcounter{defi}[chapter]
\NewEnviron{Defi}[2][]%
{
    \settowidth{\Longueur}{\textbf{Définition#1 \textbullet{}} }
    \hspace{-\Longueur}
    \hspace{-7pt}%
    \begin{tikzpicture}
        \node[fill=\MaCouleur!25,rounded corners=2.5pt,anchor=north west] at (0,0)%
            {\begin{minipage}[t]{\Longueur}
                \textbf{Définition#1 \textbullet{}}\par
                \raggedright{\small\textit{#2}}
            \end{minipage}%
            \begin{minipage}[t]{0.995\textwidth}
                \BODY
            \end{minipage}};
    \end{tikzpicture}\par\medskip
}

%___________________________
%===    Exemples du cours
%------------------------------------------------------
\newenvironment{Exemple}[1][]
{\sffamily\small\color{gray!125}%
\begin{minipage}[t]{0.1\textwidth}
    \textbf{Exemple#1} \textbullet
\end{minipage}\quad
\begin{minipage}[t]{0.85\textwidth}
}
{\end{minipage}\normalsize\par\bigskip}

%___________________________
%===    Propriété
%------------------------------------------------------

\NewEnviron{Prop}[2][]%
{
    \settowidth{\Longueur}{\textbf{Propriété#1 \textbullet{}} }
    \hspace{-\Longueur}
    \hspace{-7pt}%
    \begin{tikzpicture}
        \node[fill=\MaCouleur!25,rounded corners=2.5pt,anchor=north west] at (0,0)%
            {\begin{minipage}[t]{\Longueur}
                \textbf{Propriété#1 \textbullet{}}\par
                \raggedright{\small\textit{#2}}
            \end{minipage}%
            \begin{minipage}[t]{0.995\textwidth}
                \BODY
            \end{minipage}};
    \end{tikzpicture}\par\medskip
}

%___________________________
%===    Théorème
%------------------------------------------------------

\NewEnviron{Thm}[2][]%
{
    \settowidth{\Longueur}{\textbf{Théorème#1 \textbullet{}} }
    \hspace{-\Longueur}
    \hspace{-7pt}%
    \begin{tikzpicture}
        \node[fill=\MaCouleur!25,rounded corners=2.5pt,anchor=north west] at (0,0)%
            {\begin{minipage}[t]{\Longueur}
                \textbf{Théorème#1 \textbullet{}}\par
                \raggedright{\small\textit{#2}}
            \end{minipage}%
            \begin{minipage}[t]{0.995\textwidth}
                \BODY
            \end{minipage}};
    \end{tikzpicture}\par\medskip
}

%___________________________
%===    Démonstration
%------------------------------------------------------

\NewEnviron{Demo}[1][]%
{
    \settowidth{\Longueur}{\textbf{Démonstration#1 \textbullet{}} }
    \hspace{-\Longueur}\textbf{Démonstration#1 \textbullet{}} \BODY\par
    \hfill$\square$
    \textcolor{\MaCouleur!25}{\hspace{-0.1\linewidth}\rule[1ex]{1.1\linewidth}{2pt}}
}

%___________________________
%===    Démonstration
%------------------------------------------------------

\NewEnviron{Rmq}[1][]%
{
    \settowidth{\Longueur}{\textbf{Remarque#1 \textbullet{}} }
    \hspace{-\Longueur}\textbf{Remarque#1 \textbullet{}} \BODY\par
    \textcolor{\MaCouleur!25}{\hspace{-0.1\linewidth}\rule[1ex]{1.1\linewidth}{2pt}}
}





%___________________________
%===    Pour écrire des notes dans la marge
%===    À manipuler avec précaution car bricolage
%------------------------------------------------------

\newcommand\Note[2][0cm]{
\hspace{-3.3cm}
    \raisebox{#1}[0cm][0cm]{
        \makebox[0cm][l]{
            \parbox{2.65cm}{
                \raggedright
                \scriptsize #2
                }
            }
        }
\hspace{2.85cm}
}

\pieddepage{}{\thepage}{}

\dominitoc

\begin{document}
\setcounter{chapter}{2}
\chapter{Les environnements}

\section{Environnements encadrés}

\subsection{Définitions}

\subsubsection{Sans texte}
On en déduit alors la définition suivante :\medskip

\begin{Defi}{}
1\ier argument facultatif s donc on écrit rien ou rien entre les crochets\par
2\ieme argument obligatoire donc \{ \}
\end{Defi}

\begin{Exemple}
    Développer l'expression $A(x) = 5(x - 1) - (3x - 2)(-4 + x)$.\par\smallskip
    $\begin{array}{r@{\ =\ }ll}
        A(x) & 5(x - 1) - (3x - 2)(-4 + x) & \\[7pt]
        A(x) & 5x - 5 - (3x - 2)(-4 + x) & \rightsquigarrow\ \text{on commence par développer} \\[7pt]
        A(x) & 5x - 5 -(-12x + 3x^2 + 8 - 2x) & \rightsquigarrow\ \text{attention aux signes !}\\[7pt]
        A(x) & 5x - 5 + 12x - 3x^2 - 8 + 2x & \rightsquigarrow\ \text{suppression de la parenthèse}\\[7pt]
        A(x) & -3x^2 + 5x + 12x + 2x - 8 & \rightsquigarrow\ \text{puis réduction de l'expression} \\[7pt]
        \multicolumn{3}{l}{\pfr{A(x) = -3x^2 + 19x - 8}}
    \end{array}$
\end{Exemple}

\begin{Defi}[s]{}
On définit les ensembles de nombres :
    \begin{enumerate}
        \item L'ensemble des entiers natures $\N$ est constitué des nombres entiers positifs.\par
        \item L'ensemble des entiers relatifs $\Z$ est constitué des entiers naturels ainsi que de nombres entiers négatifs.\par
        \item L'ensemble des nombres rationnels $\Q$ est constitué de tous les nombres qui peuvent s'écrire sous forme de fractions, ce qui inclut les ensembles $\N$ et $\Z$ mais aussi les nombres décimaux.\par
        \item L'ensemble des nombres irrationnels est constitué des nombres qui ne peuvent pas s'écrire sous forme de fractions.
        \item L'ensemble des nombres réels $\R$ est constitué de tous les nombres rationnels et des nombres irrationnels.
    \end{enumerate}
\end{Defi}

\begin{Exemple}[s]
    $3$ est un entier naturel alors que $-6$ est un entier relatif.\par
    $18,5$ et $\dfrac 13$ sont des rationnels alors que $\pi$ est un irrationnel.
\end{Exemple}

\subsubsection{Avec texte}

\begin{Defi}{à connaître par c{\oe}ur}
1\ier argument facultatif s donc on écrit rien ou rien entre les crochets\par
2\ieme argument obligatoire donc \{  à connaître ...\}
\end{Defi}

\begin{Exemple}
    Blablabla
\end{Exemple}

\begin{Defi}[s]{à connaître...}
1\ier argument facultatif s donc on écrit rien ou rien entre les crochets\par
2\ieme argument obligatoire donc \{  à connaître ...\}
\end{Defi}

\begin{Exemple}
    Blablabla
\end{Exemple}


\subsection{Propriétés}

On en déduit les propriétés suivantes : \medskip

\begin{Prop}[s] {}
    Pour tous nombres $k$, $a$, $b$, $c$ et $d$, on a :
    \[k(a + b) = ka + kb \qetq (a + b)(c + d) = ac + ad + bc+ bd.\]
    \[ka + kb = k(a + b) \qetq a(c + d) + b(c + d) = (a + b)(c + d).\]
\end{Prop}

\begin{Exemple}
    Factoriser l'expression $B(x) = 2(x+4)+2(x-1) - (2x + 3)(4x + 6)$.\par\smallskip
    $\begin{array}{r@{\ =\ }ll}
        B(x) & \textcolor{red}{2}(x+4)+\textcolor{red}{2}(x-1) - (2x + 3)(4x + 6) & \rightsquigarrow\ \text{on repère les facteurs communs} \\[7pt]
        B(x) & 2(x + 4 + x - 1) - (2x + 3)(4x + 6) & \rightsquigarrow\ \text{pas de problème de signes avec un $+$} \\[7pt]
        B(x) & 2\textcolor{red}{(2x + 3)} - \textcolor{red}{(2x + 3)}(4x + 6) & \rightsquigarrow\ \text{on continue tant qu'il y a un facteur commun}\\[7pt]
        B(x) & (2x + 3)(2 - (4x + 6)) & \rightsquigarrow\ \text{attention au signe $-$} \\[7pt]
        B(x) & (2x + 3)(2 - 4x - 6) & \rightsquigarrow\ \text{suppression de la parenthèse} \\[7pt]
        B(x) & (2x + 3)(\textcolor{red}{-4} + \textcolor{red}{-4}x) & \rightsquigarrow\ \text{il y a encore un facteur commun} \\[7pt]
        \multicolumn{3}{l}{\pfr{B(x) = -4(2x + 3)(1 + x)}}
    \end{array}$
\end{Exemple}

\begin{Prop}[]{(admise)}
    En voilà une propriété qu'elle est trop dure à démontrer à cause de vos petits cerveaux de moineaux~!
\end{Prop}

\subsection{Théorèmes}
Toujours le même principe pour s et texte\medskip

\begin{Thm}{TVI}
    Ce théorème là est tellement important qu'on n'a rien compris.
\end{Thm}

\begin{Thm}[s]{}
    C'était le frère de \bsc{Thalès} et le neveu d'\bsc{Euclide}... Ou peut-être le contraire...
\end{Thm}

\subsection{Démonstrations}
\begin{Demo}
    C'est à cause du frère de la tante de la s{oe}ur de la belle-mère de leur cousine germaine mariée volontairement de force contre son gré après avoir dit que \[1 + 1 = 2\] sans démontrer proprement que $1 + 0 = 1$ alors depuis, il y a des problèmes de famille non résolus d'autant plus que \bsc{Freud} a lâchement décidé de les laisser se débrouiller sous prétexte qu'il n'est pas né à leur époque.
\end{Demo}

\begin{Exemple}
    Blablabla
\end{Exemple}

\begin{Demo}[s]
Nous avons les preuves suivantes :
\begin{itemize}
\item $1+1 = 2$
\item $2+1 = 3$
\end{itemize}
\bsc{c.q.f.d.}
\end{Demo}

\begin{Exemple}[s]
    Blablabla\par
    Blebleble\par
    Bliblibli\par
    Blobloblo
\end{Exemple}

\section{Remarque}

\begin{Rmq}
    C'est trop bien les vacances !
\end{Rmq}

\begin{Rmq}[s]
Remarquons alors :

    \begin{enumerate}
        \item c'est la fin de ce chapitre ;
        \item c'est trop bien \LaTeX ;
        \item je dois aller faire mon repassage.
    \end{enumerate}
\end{Rmq}\clearpage


\renewcommand\PartProgramme{Probabilités / Statistiques}
\renewcommand\MaCouleur{Orange}

\chapter{Quelques remarques dans la marge}


\section{Introduction}

\Note[-5em]{car $\int_0^1x^2dx=\frac13$ et c'est donc facile même si c'est un long texte}
Le Portugal (la République portugaise, en portugais : República Portuguesa), est un pays du sud de l'Europe, membre de l'Union européenne, situé à l'ouest de la péninsule Ibérique, et de forme longue orientée nord-est sud-ouest. Ce pays, le plus occidental de l'Europe continentale, est délimité au nord et à l'est par l'Espagne et au sud et à l'ouest par l'océan Atlantique. Il comprend également les archipels des Açores et de Madère, situés dans l'hémisphère Nord de l'océan Atlantique.

Fondé au XIIe siècle, le royaume de Portugal devient au XVe siècle siècle l'une des principales puissances d'Europe occidentale, jouant un rôle majeur dans les Grandes Découvertes et se taillant un vaste empire colonial en Afrique, en Asie et en Amérique du Sud. La puissance du pays décline à partir du XVIIe siècle. La monarchie portugaise est renversée en 1910, à l'issue d'un soulèvement militaire qui contraint le roi Manuel II à l'exil. La Première République portugaise (Portugais : Primeira República) est le régime politique en vigueur au Portugal entre la fin de la monarchie constitutionnelle marquée par la révolution du 5 octobre 1910 et le coup militaire du 28 mai 1926. Puis, pendant plus de quarante ans, le pays est soumis au régime autoritaire d'António de Oliveira Salazar, jusqu'à la révolution des Œillets de 1974 qui met fin à la dictature salazariste et ré-installe la démocratie dans le pays.\medskip

\begin{Defi}{à connaître par c{\oe}ur}
    Mais que vient faire cette définition ici ??
\end{Defi}

Le Portugal devient à la fin du XXe siècle un pays développé, économiquement prospère, socialement et politiquement stable. Membre fondateur de l'OTAN en 1949 et de l’OCDE en 1948, il est également membre de l'ONU depuis 1955, du conseil de l'Europe depuis 1976, de l'Union européenne depuis 1986, et de l’espace Schengen. Enfin, il est l'un des pays fondateurs de la zone euro en 1999. En 2011, la dégradation économique mondiale conduit le Portugal à la récession et provoque une crise socio-économique et politique.

Allié des États-Unis, le Portugal entretient également d'importantes relations bilatérales avec le Maroc, le Brésil, l'Espagne et la France qui sont ses quatre partenaires privilégiés.

Dans ce pays qui a connu la dictature de 1926 à 1974, l'économie n'a pris son essor qu'après 1975, amenant près d'un million et demi de Portugais à aller travailler en dehors du pays pour fuir la misère et les guerres coloniales. Les « fortes zones d'immigration » sont le Brésil, la France, le Luxembourg (14,1 \% de la population totale du pays), la Suisse, les États-Unis, l'Argentine, le Venezuela, le Canada, ainsi que la principauté d'Andorre (où 15,75 \% de la population est portugaise) et divers autres pays. À l'heure actuelle, la diaspora portugaise est l'une des principales diasporas européennes et mondiales.

Le tourisme, principalement balnéaire, est une ressource très importante, notamment en Algarve et dans la région de Lisbonne qui font du Portugal un des pays le plus visité au monde avec plus de 25 millions de touristes chaque année. Le Portugal est également un grand pays viticole, notamment réputé pour le vin de Porto. Le Portugal est par ailleurs le premier producteur mondial de liège.

\section{Histoire}
\subsection{Préhistoire et Antiquité pré-romaine}

\Note{Partie très importante}
Les plus anciennes traces de civilisation retrouvées au Portugal datent du Paléolithique : peintures et gravures rupestres des grottes d'Escoural (Alentejo), de Mazouco (Tras-os-Montes) et surtout de Vale de Côa, datées entre 22000 et 10000 av. J.-C. La majorité de ces traces se trouvent au nord du Tage et témoignent de l'existence de peuples de chasseurs-cueilleurs. Vers 10000 av. J.-C., les Ibères peuplent l'intérieur des terres de la péninsule. Ce territoire prend, dans l'Histoire, le nom de << péninsule Ibérique >>.

Entre 4000 et 2000 av. J.-C., le Portugal et la Galice voient se développer une culture mégalithique originale par rapport au reste de la péninsule, caractérisée par son architecture funéraire et rituelle particulière, et par la pratique de l'inhumation collective. On peut encore trouver dans le pays de nombreuses traces monumentales, la plupart dans l'Alentejo : le cromlech d'Almendres près d'Évora, ceux de Vale Maria do Meio ou de Portela de Mogos, ainsi que le dolmen de Zambujeiro.

[...]

\subsection{Conquête germanique}

Au début du Ve siècle, les Germaniques envahirent la péninsule Ibérique, à savoir les Suèves et les Vandales (Silinges et Hasdingi) avec leurs alliés, les Sarmates et Alains où ils formeraient leur royaume.

\Note[-1cm]{Copie honteuse de Wikipédia}
Le royaume des Suèves était le royaume post-romaine germanique, établi dans les anciennes provinces romaines de Gallaecia et le Nord de la Lusitanie. Vers 410 et au VIe siècle, il devint un royaume officiellement déclaré, où le roi Hermeric fait un traité de paix avec les Gallaecians avant de passer sa mort pour Rechila, son fils. En 448 Rechila mourut, laissant l'état de l'expansion à Rechiar. Il a maintenu son indépendance jusqu'en 585, quand il a été annexé par les Wisigoths, et est devenu la sixième province du royaume wisigoth de Hispania.

En l'an 500, le royaume wisigoth a été installé en Hispanie, fourré à Tolède. Les Wisigoths finalement conquis les Suèves et sa capitale Bracara (aujourd'hui Braga) en 584-585. En l'an 700, l'ensemble de la péninsule ibérique a été gouvernée par les Wisigoths, ayant survécu jusqu'à 711, lorsque le roi Roderic.

\end{document}