\documentclass[french,10pt]{book}
\input preambule_2013

\renewcommand\thesection{\Roman{section})}

%___________________________
%===    Environnements de cours
%------------------------------------------------------

%___________________________
%===    Définitions
%------------------------------------------------------
\newcounter{defi}[chapter]
\newenvironment{Defi}[1][]%
{
\refstepcounter{defi}
\normalsize
\textit{\textbf{Définition \thedefi.}} #1\par\quad
}
{
\medskip
}%

%___________________________
%===    Propriétés
%------------------------------------------------------
\newcounter{propri}[chapter]
\newenvironment{Prop}[1][]%
{
\refstepcounter{propri}
\normalsize
\quad \textit{\textbf{Propriété \thepropri.}} #1\par
}
{
\medskip
}%

%___________________________
%===    Théorèmes
%------------------------------------------------------
\newcounter{thm}[chapter]
\newenvironment{Thm}[1][]%
{
\refstepcounter{thm}
\normalsize
\centering{\textit{\textbf{Théorème \thethm.}} #1}\par
}
{
\medskip
}%

%___________________________
%===    Démonstration
%------------------------------------------------------
\newenvironment{Demo}[1][]%
{
\normalsize
{\flushright{\textit{\textbf{Démonstration.}} #1\par}}
}
{
\strut\hfill$\square$\medskip
}

%___________________________
%===    Remarques
%------------------------------------------------------
\newenvironment{Rmq}[1][]%
{
\normalsize
\textit{\textbf{Remarque#1.}}\par
}
{
\medskip
}%



\begin{document}

\chapter{Quelques environnements}

\section{Définition}

\begin{Defi}
    Ceci est une définition importante à connaître par c{\oe}ur !
\end{Defi}

\begin{Defi}[(encore plus importante)]
    Ceci est une définition hyper importante à connaître par c{\oe}ur !
\end{Defi}

\section{Propriété}

\begin{Prop}
    En voilà une propriété qu'elle est bien !
\end{Prop}

\begin{Prop}[(admise)]
    En voilà une propriété qu'elle est trop dure à démontrer à cause de vos petits cerveaux de moineaux !
\end{Prop}

\section{Théorème et démonstration}

\begin{Thm}
    Ce théorème là est tellement important qu'on n'a rien compris.
\end{Thm}

\begin{Thm}[(de \bsc{Pythagore})]
    C'était le frère de \bsc{Thalès} et le neveu d'\bsc{Euclide}... Ou peut-être le contraire...
\end{Thm}

\begin{Demo}
    C'est à cause du frère de la tante de la s{oe}ur de la belle-mère de leur cousine germaine mariée volontairement de force contre son gré après avoir dit que \[1 + 1 = 2\] sans démontrer proprement que $1 + 0 = 1$ alors depuis, il y a des problèmes de famille non résolus d'autant plus que \bsc{Freud} a lâchement décidé de les laisser se débrouiller sous prétexte qu'il n'est pas né à leur époque.
\end{Demo}

\section{Remarque}

\begin{Rmq}
    C'est trop bien les vacances !
\end{Rmq}

\begin{Rmq}[s]
    \begin{enumerate}
        \item c'est la fin de ce chapitre ;
        \item c'est trop bien \LaTeX ;
        \item je dois aller faire mon repassage.
    \end{enumerate}
\end{Rmq}


%\begin{center}
%        \begin{tikzpicture}[scale=.5,xscale=2]
%            \tikzset{tan style/.style={-}}
%            \tkzInit[xmin=-2,xmax=2,ymin=-10,ymax=1.5,xstep=1,ystep=1]
%            \tkzClip
%            \tkzGrid[sub,subxstep=0.5,subystep=0.25,color=brown](-2,-10)(2,1.5)
%            \tkzGrid
%            \tkzDrawX\tkzDrawY
%            \tkzSetUpPoint[shape=circle, size = 3, color=black, fill=lightgray]
%            \tkzFct[line width=1pt,color=blue]{(-1)*2*x**2 + x + 1}
%            \tkzDrawTangentLine[line width=0.75pt,color=red,kl=0.6,kr=0.6](-1)
%            \tkzDefPointByFct[draw,ref=A](-1)
%            \tkzLabelPoint[above left](A){\footnotesize $A$}
%        \end{tikzpicture}
%    \end{center}
\end{document}