\documentclass{article}
\usepackage[frenchstyle]{kpfonts}%  f\"{u}r Mathezeichen
%\usepackage{libertine}% f\"{u}r rm und sf
\usepackage{libertine-type1}% f\"{u}r rm und sf
\usepackage[T1]{fontenc}

\usepackage{amsmath}
\usepackage[%
   web={usetemplates,pro,german,latextoc,centertitlepage,designv,tight},
    attachsource={tex},
    uselayers,eforms,ocganime,
    graphicxsp={showembeds},
    aebxmp
]{aeb_pro}

\embedEPS[transparencyGroup]{p0}{mathevii_screen_ng}
\usepackage[nomessages]{fp}
%
\usepackage[distiller]{pstricks}
\usepackage{pst-grad}
%\usepackage{pst-func}
\usepackage{pst-tools}
\usepackage{pstricks-add}
\usepackage[absolute,overlay]{textpos}
\usepackage{fancyvrb}

\definecolor{randnotiz}{rgb}{0.545,0.271,0.075}

\DeclareDocInfo
{
    title=Animation zu Scheitel- und Wechselwinkeln,
    author=J\"{u}rgen Gilg und Thomas S\"{o}ll,
    university=Rh\"{o}n-Gymnasium Bad Neustadt,
    email=Thomas.Soell@lehrer.uka.de,
    subject=Animation,
    keywords={Adobe Acrobat, JavaScript, OCG, Layers, animation},
    talksite=,
    prepared={\today},
%    talkdate={Mai 29, 2012},
    copyrightStatus=True,
    copyrightNotice={Copyright (C) \the\year, T. S\"{o}ll},
    copyrightInfoURL=
}



\begin{docassembly}
\insertPreDocAssembly
%\executeSave()
aebTrustedFunctions(this, aebSaveAs, "Save");
\end{docassembly}

\newcommand{\Geodreieck}{%
\psset{unit=1.6,dimen=middle}
%\begin{pspicture}(-5,0)(5,0)
%\SpecialCoor
%
\scriptsize
\pspolygon[linewidth=0pt,fillstyle=solid,fillcolor=lightgray!30,linestyle=none,opacity=0.3](-4.675,0.125)(4.675,0.125)(0,4.8)%
%
\begin{psclip}%
{\pspolygon[linestyle=none](5,0)(4.85,0)(-0.15,5)(0,5)}%
\multido{\n=1+1}{89}{%
\rput[l]{\n}(!\n\space dup sin exch cos div dup 1 add 5 exch div exch 1 index mul)%
{\psline[linewidth=.25\pslinewidth](0,0)(-.2,0)}%
}%
\end{psclip}%
\begin{psclip}%
{\pspolygon[linestyle=none](-5,0)(-4.85,0)(0.15,5)(0,5)}%
\multido{\n=1+1}{89}{%
\rput[l]{-\n}(!\n\space dup sin exch cos div dup 1 add 5 exch div neg exch 1 index mul neg)%
{\psline[linewidth=.25\pslinewidth](0,0)(.2,0)}%
}%
\end{psclip}%
\multido{\n=5+10}{9}{%
\rput[l]{\n}(!\n\space dup sin exch cos div dup 1 add 5 exch div exch 1 index mul)%
{\psline[linewidth=.5\pslinewidth](0,0)(-.4,0)}%
\rput[l]{-\n}(!\n\space dup sin exch cos div dup 1 add 5 exch div neg exch 1 index mul neg)%
{\psline[linewidth=.5\pslinewidth](0,0)(.4,0)}%
}%
\multido{\n=10+10}{8}{%
\psline(3.15;\n)(!\n\space dup sin exch cos div dup 1 add 5 exch div exch 1 index mul)%
\psline(-3.15;-\n)(!\n\space dup sin exch cos div dup 1 add 5 exch div neg exch 1 index mul neg)%
}%
\multido{\n=6+1}{84}{%
\psline[linewidth=.5\pslinewidth](-2.77;-\n)(-2.7;-\n)%
\psline[linewidth=.5\pslinewidth](2.77;\n)(2.7;\n)%
}%
\multido{\n=5+5}{17}{%
\psline(2.81;\n)(2.7;\n)%
\psline(-2.81;-\n)(-2.7;-\n)%
}
\psline[linestyle=dashed,dash=7pt 4pt](-2.6;-45)(-.5;-45)%
\psline[linestyle=dashed,dash=7pt 4pt](2.6;45)(.5;45)%
%
\pscustom[fillstyle=solid,fillcolor=yellow!70,linestyle=none,opacity=0.3]{%
\psarc(0,0){2.85}{7}{173}%
\psarcn(0,0){3.15}{173}{7}%
}%
{\multido{\n=10+10}{17}{\rput{0}(3;\n){\n}}}%
\multido{\n=0+1}{8}{\rput{180}(!\n\space 0.625 mul 0.25){\n}}%
\multido{\n=1+1}{7}{\rput{180}(!\n\space 0.625 mul neg 0.25){\n}}%
\multido{\n=-70+1}{141}{\rput{180}(!\n\space 0.0625 mul neg 0){\psline[linewidth=0.5\pslinewidth](0,0)(0,-0.08)}}%
\multido{\n=-14+1}{29}{\rput{180}(!\n\space 0.3125 mul neg 0){\psline(0,0)(0,-0.13)}}%
%
\multido{\n=3+1}{30}{\rput{180}(!0 \n\space 0.0625 mul 0.03 add){%
\psline[linewidth=0.5\pslinewidth](-1.625,0)(-1.5,0)%
\psline[linewidth=0.5\pslinewidth](1.625,0)(1.5,0)%
}}%
%
\begin{psclip}%
{\pscircle[linestyle=none](0,0){2.6}}%
\multido{\n=1+1}{7}{\rput{180}(!0 \n\space 0.3125 mul 0.03 add){%
\psline(-2.6,0)(-1.69,0)%
\psline(2.6,0)(1.69,0)%
\psline(1.625,0)(1.4375,0)%
\psline(-1.625,0)(-1.4375,0)%
\psline(1.1875,0)(0.15,0)%
\psline(-1.1875,0)(-0.15,0)%
}}%
\multido{\n=1+1}{3}{%
\rput{180}(!1.3 \n\space 0.625 mul 0.03 add){\n}%
\rput{180}(!-1.3 \n\space 0.625 mul 0.03 add){\n}%
}%
\end{psclip}%
%
\psline(0.4;90)(2.85;90)%
\psline(3.15;90)(5;90)%
\pspolygon(5,0)(0,5)(-5,0)%
%\end{pspicture}
}

\newcommand{\Stift}[4]{%
\definecolor{Holzfarbe}{rgb}{1,0.937,0.835}
{\psset{unit=#1}
\begin{pspicture}[showgrid=false](-1,0)(1,20)
\def\farbA{#2}%
\def\farbB{#3}%
\def\farbC{#4}%
\def\r{0.5}
\def\lang{20}
%
\pspolygon[fillstyle=solid,fillcolor=Holzfarbe](0,0)(0.6,2.4)(-0.6,2.4)%
\pspolygon[fillstyle=solid,fillcolor=\farbA](0,0)(0.2,0.8)(-0.2,0.8)%
\pscustom[fillstyle=solid,fillcolor=\farbB,linecolor=\farbC,opacity=1]{%
\psarc(!0 0.2 \r\space dup mul 0.2 dup mul sub sqrt atan cos \r\space mul 2.4 add){\r}{!0.2 \r\space dup mul 0.2 dup mul sub sqrt atan neg 90 sub}{!0.2 \r\space dup mul 0.2 dup mul sub sqrt atan 90 sub}%
\psline(0.2,\lang)(-0.2,\lang)(!-0.2 2.4)
}%
\pscustom[fillstyle=solid,fillcolor=\farbB,linecolor=\farbC,opacity=1]{%
\psarc(!-0.4 0.2 \r\space dup mul 0.2 dup mul sub sqrt atan cos \r\space mul 2.4 add){\r}{!0.2 \r\space dup mul 0.2 dup mul sub sqrt atan neg 90 sub}{!0.2 \r\space dup mul 0.2 dup mul sub sqrt atan 90 sub}%
\psline(-0.2,\lang)(-0.6,\lang)(!-0.6 2.4)
}%
\pscustom[fillstyle=solid,fillcolor=\farbB,linecolor=\farbC,opacity=1]{%
\psarc(!0.4 0.2 \r\space dup mul 0.2 dup mul sub sqrt atan cos \r\space mul 2.4 add){\r}{!0.2 \r\space dup mul 0.2 dup mul sub sqrt atan neg 90 sub}{!0.2 \r\space dup mul 0.2 dup mul sub sqrt atan 90 sub}%
\psline(0.6,\lang)(0.2,\lang)(!0.2 2.4)
}%
\end{pspicture}
}}


\universityLayout{fontsize=LARGE,fontseries=bfseries,color=randnotiz}
\titleLayout{fontsize=Huge,fontseries=bfseries,color=orange}
\authorLayout{fontsize=large,fontseries=bfseries,color=randnotiz}

\sectionLayout{
fontfamily=sffamily,
fontseries=bfseries,
fontsize=Large,
color=randnotiz
}
\subsectionLayout{
fontfamily=sffamily,
fontseries=bfseries,
fontsize=Large,
color=randnotiz,
numdingcolor=randnotiz
}
\subsubsectionLayout{%
fontfamily=sffamily,
fontseries=bfseries,
fontsize=Large,
color=randnotiz,
numdingcolor=randnotiz
}

\input{aeb_pro_icon.def}

\placeAnimeCtrlBtnFaces{\myIconPath}{Scheitelw1}
\begin{document}
\template[name=p0]{mathevii_screen_ng}
\maketitle

\section*{Scheitel- und Nebenwinkel}


\btnAnimeCtrlPresets{%
\BG{randnotiz}% Buttonfarbe
\S{B} %Linien-Stil: B=beveled (abgefast), S=solid, I=inset, U=underlined, D=dashed
\H{O} %Highlighting: I=invert, O=outline, P=push, N=none
}%
\animeSetup{%
controls=skin1,
%usetworows,
%nospeedcontrol,
ctrlbdrywidth=thin,%thin, medium, thick
ctrlbdrycolor=orange,
ctrlwidth=11bp+3bp,
ctrlheight=7bp+4bp,
ocgAnimeName=Scheitelw1,
nFrames=180,
speed=1
}%
\begin{minipage}[b]{0.4\linewidth}
\ \fcolorbox{randnotiz}{webyellow}{\parbox{\linewidth}{\centering \textcolor{orange}{\emph{\textbf{Animationsbuttons}}}\\[4pt]\insertCtrlButtons}} \\[8pt]
\end{minipage}
\begin{minipage}[b]{0.4\linewidth}
\begin{ocgAnime}{ocgAnimeName=Scheitelw1,controls=none}%,type=loop}
\FPdiv{\myDeltaA}{180}{180}%
%--------------------------------------------------------------------------------------------------------------

%---------------------------------------------------------------------------------------------------------------
\def\thisframe{%
\animeBld%
\rput{!\xi\space 90 add}(-3,5){\Stift{0.42}{red}{red}{black}}%
%---------------------------------------------------------------------------
\pswedge*[linecolor=cyan,opacity=0.5](-3,5){3}{0}{!\xi}
\psarc[linewidth=1.2pt,arrowsize=0.2,arrowinset=0.1]{->}(-3,5){3}{0}{!\xi}
\pswedge*[linecolor=cyan,opacity=0.5](-3,5){3}{180}{!\xi\space 180 add}
\psarc[linewidth=1.2pt,arrowsize=0.2,arrowinset=0.1]{->}(-3,5){3}{180}{!\xi\space 180 add}
%---------------------------------------------------------------------------
\pswedge*[linecolor=magenta!40,opacity=0.5](-3,5){2.5}{!\xi}{180}
\psarc[linewidth=1.2pt,arrowsize=0.2,arrowinset=0.1]{->}(-3,5){2.5}{!\xi}{180}
\pswedge*[linecolor=magenta!40,opacity=0.5](-3,5){2.5}{!\xi\space 180 add}{0}
\psarc[linewidth=1.2pt,arrowsize=0.2,arrowinset=0.1]{->}(-3,5){2.5}{!\xi\space 180 add}{0}
%---------------------------------------------------------------------------
\rput{!\xi}(-3,5){\psline[linecolor=black,linewidth=1.0pt](-3.5,0)(3.2,0)}
\psline[linecolor=black,linewidth=1.0pt](-6.5,5)(0.2,5)
\rput(-3,5){\psdot}
%---------------------------------------------------------------------------------------------------------------------------------------
\uput{1.5}[!\xi\space 2 div 0 add](-3,5){\psframebox[fillstyle=solid,fillcolor=yellow!20,opacity=0.8,linestyle=none]{\makebox[2em][l]{\strut \psPrintValue[fontscale=12,dot,PSfont=Symbol,postString=\string\260]{\xi\space cvi}}}}
%---------------------------------------------------------------------------------------------------------------------------------------
\uput{1.5}[!\xi\space 2 div 180 add](-3,5){\psframebox[fillstyle=solid,fillcolor=yellow!20,opacity=0.8,linestyle=none]{\makebox[2em][l]{\strut \psPrintValue[fontscale=12,dot,PSfont=Symbol,postString=\string\260]{\xi\space cvi}}}}
%---------------------------------------------------------------------------------------------------------------------------------------
\uput{1.3}[!\xi\space 2 div 90 add](-3,5){\psframebox[fillstyle=solid,fillcolor=yellow!20,opacity=0.8,linestyle=none]{\makebox[2em][l]{\strut \psPrintValue[fontscale=12,dot,PSfont=Symbol,postString=\string\260]{180 \xi\space sub cvi}}}}
%---------------------------------------------------------------------------------------------------------------------------------------
\uput{1.3}[!\xi\space 2 div 270 add](-3,5){\psframebox[fillstyle=solid,fillcolor=yellow!20,opacity=0.8,linestyle=none]{\makebox[2em][l]{\strut \psPrintValue[fontscale=12,dot,PSfont=Symbol,postString=\string\260]{180 \xi\space sub cvi}}}}
\eBld%
}
\def\xi{0}
%----------------------------------------------------------------------------------------------------------------
\psscalebox{0.65}{%
\begin{pspicture}[showgrid=false](-6,-3)(2,12)
\rput{0}(-3,5){\Geodreieck}
\rput{-180}(-3,5){\Geodreieck}
\rput{90}(-3,5){\Stift{0.42}{black!80}{blue!50}{black}}\rput{0}(-3,5){\psdot}
%----------------------------------------------------------------------------------------------------------------
\multido{\i=0+1}{180}{\FPadd{\xi}{\xi}{\myDeltaA}\thisframe}%
\end{pspicture}
}
%
\end{ocgAnime}
\end{minipage}


\end{document}
