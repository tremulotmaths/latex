\documentclass[border=10pt]{standalone}
\usepackage{pst-plot,pst-node,pst-tools}

\makeatletter
\pst@addfams{pst-tools}
%
\define@boolkey[psset]{pst-tools}[Pst@]{markZeros}[false]{}
\define@boolkey[psset]{pst-tools}[Pst@]{onlyNode}[false]{}
\define@boolkey[psset]{pst-tools}[Pst@]{commaV}[true]{}
\define@boolkey[psset]{pst-tools}[Pst@]{originV}[false]{}
%----------------------------------------------------------------------
\SpecialCoor%
%   Default Parameter
\psset[pst-tools]{%
commaV=true,originV=false,onlyNode=false,markZeros=false,
decimals=2,xShift=0,yShift=0,fontscale=10,PSfont=Times-Roman,
}
%--------------------------------------------------------------------
%------------- calculate the value of an intersectionpoint -----------
%---------------------------------------------------------------------
\def\psZero{\def\pst@par{}\pst@object{psZero}}
\def\psZero@i(#1,#2)#3#4{{% (#1,#2) Intervall f\"{u}r die Nullstelle, #3 Funktion, #4 Knotenname
\begin@SpecialObj %
\pst@Verb{
/Function {#3} def
\ifPst@algebraic /Function (#3) tx@AlgToPs begin AlgToPs end cvx def \fi
  /Xinf2 #1 def
  /Xsup2 #2 def
{ /xM2 Xinf2 Xsup2 add 2 div def
   /x Xinf2 def
   /F_1 Function def
   /x xM2 def
   /F_M Function def
   F_M 0 eq {exit} if
   F_1 F_M mul 0 ge {/Xinf2 xM2 def} {/Xsup2 xM2 def} ifelse
 Xinf2 Xsup2 sub abs 1e-6 le {exit} if } loop
}%
\pnode(! xM2 0){#4}%
\addto@pscode{
\ifPst@onlyNode \else
  /dec \psk@decimals\space def
  /schrift { \psk@PSfont findfont \psk@fontscale scalefont setfont } bind def
  /Wert { 10 dec exp mul round 10 dec exp div dec 0 eq {cvi 15 string cvs} {15 string cvs } ifelse
  \ifPst@commaV dot2comma \fi show } def
  /Function {#3} def
  \ifPst@algebraic /Function (#3) tx@AlgToPs begin AlgToPs end cvx def \fi
      /Xinf #1 def
      /Xsup #2 def
{
 /xM Xinf Xsup add 2 div def
   /x Xinf def
   /F_1 Function def
   /x xM def
   /F_M Function def
   F_M 0 eq {exit} if
   F_1 F_M mul 0 ge {/Xinf xM def} {/Xsup xM def} ifelse
 Xinf Xsup sub abs 1e-6 le {exit} if } loop
\ifPst@originV 0 0 \else
xM \psk@xShift\space add \pst@number\psxunit mul 0 \psk@yShift\space add \pst@number\psyunit mul \fi moveto schrift xM Wert \fi}%
\ifPst@markZeros%
\psdot(#4)\else\fi%
\end@SpecialObj%
}%
\ignorespaces
}%
\makeatother

\begin{document}


\def\funkf{2*x^2-4.2*x-5.4+ln(x+10)+2^(2/(1+x^2))+sqrt(x+3)}

\begin{pspicture}[showgrid=false](-1.9,-2.5)(3,4.1)
\begin{psclip}%
{\psframe[linestyle=none](-1.75,-2.45)(2.75,4.15)}
\psgrid[subgriddiv=2,gridlabels=0,gridwidth=0.7pt,gridcolor=black!50,subgridwidth=0.6pt,subgridcolor=black!30](-3,-9)(6,6)
\end{psclip}
\psaxes[xDecimals=0, yDecimals=0,labelFontSize=\scriptstyle,arrowscale=1.3,arrowinset=0.05,arrowlength=1.9, Dy=1,dy=1,dx=1,Dx=1,subticks=0,comma,tickwidth=0.9pt]{->}(0,0)(-1.75,-2.25)(2.85,4.25)[$x$,-90][$y$,180]% Achsen
\psplot[algebraic=true,plotpoints=500,linecolor=magenta,yMaxValue=5.8,yMinValue=-8.5]{-0.55}{2.75}{\funkf}%
\psZero[originV=false,algebraic=true,xShift=-0.1,yShift=0.2,linecolor=green,onlyNode=true](0,1){\funkf}{N1}
\psZero[originV=false,algebraic=true,xShift=-0.25,yShift=0.1,fontscale=5,linecolor=blue!70,markZeros=true,dotstyle=x](1,4){\funkf}{N2}
\uput{0.25}[0](0,-1.75){$x_{1}^{} \approx \,\psframe[fillstyle=solid,fillcolor=yellow!20,opacity=0.6,linestyle=none](0,-0.05)(0.8,0.3)$\,\psZero[algebraic=true,originV=true](0,1){\funkf}{NS1}}
\uput{0.25}[0](0,-2.25){$x_{2}^{} \approx \,\psframe[fillstyle=solid,fillcolor=cyan!20,opacity=0.6,linestyle=none](0,-0.05)(0.9,0.25)%
$\,\psZero[decimals=5,fontscale=7,algebraic=true,originV=true,linecolor=red](1,4){\funkf}{NS2}}
\pscircle[fillstyle=solid,linecolor=cyan,linewidth=0.5pt,opacity=0.8,fillcolor=orange!50](N1){4pt}
\pcline[linecolor=red,nodesepB=0.1]{->}(-1,-1)(N1)
\rput[l](-2.4,3.5){\psframebox[fillstyle=solid,fillcolor=brown!20,opacity=0.7,linestyle=none]{$\scriptscriptstyle f(x)=2x^2-4,2x-5,4+\ln(x+10)+2^{\frac{2}{1+x^2}}+\sqrt{x+3}$}}
\end{pspicture}


\end{document} 