\documentclass[12pt,fleqn,dvipsnames]{article}
\usepackage{amsmath,amssymb}
\usepackage{colortbl}
\usepackage[ngerman]{babel}
\usepackage[T1]{fontenc}
\usepackage[sfmath,frenchstyle]{kpfonts}
\usepackage[lining]{libertine}% f\"{u}r rm und sf
\renewcommand*{\familydefault}{\sfdefault}

\usepackage{ragged2e,booktabs}
%\usepackage{booktabs}

\usepackage[%
    driver=dvips,
    web={latextoc,usetemplates,pro,german,nobullets},
    exerquiz={german,execJS,showgrayletters,proofing},
    uselayers,
    eforms={useui},
    aebxmp,
    attachsource={tex}
]{aeb_pro}

\DeclareInitView{
windowoptions={showtitle}
}

\definecolor{SVCrot}{HTML}{DF0101}


\sectionLayout{indent=0pt,fontsize=LARGE,color=SVCrot}

\DeclarePageLayout{%
screensize={29.7cm}{21cm},
margins={0.7in}{0.55in}{0.6in}{0.75in},
topmargin=30pt,
webfootskip=40pt,
additionalheadsep=15pt
}
\parindent0pt
\parskip4pt

%\usepackage[dvipsnames]{xcolor} %% Farben sind im Dokument xcolor.pdf definiert
\usepackage[distiller,rgb]{pstricks}
\usepackage{pst-spectra,pst-grad,pst-slpe,pst-blur,pst-node,pst-diffraction}
\usepackage{pst-coil,pst-circ,pst-eucl,pst-solides3d,pstricks-add}






\newcommand{\Zylinder}[9]{%  #1 L\"{a}nge, #2 Radius, #3 gradbegin, #4 gradend, #5 gradmidpoint
\pscustom[dimen=#9,fillstyle=gradient,gradbegin=#3,gradend=#4,gradmidpoint=#5,gradangle=90,linecolor=#6,linewidth=#7,linestyle=#8]{%
\psellipticarc(0,0)(!#2 #2 0.3 mul){180}{360}
\psellipticarcn(0,#1)(!#2 #2 0.3 mul){0}{-180}
\closepath
}
\psellipse[fillstyle=solid,fillcolor=#6,linestyle=none](0,#1)(!#2 #2 0.3 mul)
}

\definecolor{TextA}{rgb}{0.2,0.402,0.333}% Pfeile Medium Aquamarine

%% AutoSave
\begin{docassembly}
\executeSave()
\end{docassembly}

\begin{document}


\newpage
\section*{Elektronenbeugungsr\"{o}hre}

Der experimentelle Nachweis der Elektronenbeugung an Kristallen gelang 1927 C. J. Davisson (1881 -- 1958) und L. H. Germer (1886 -- 1971)

Louis De Broglie (1892 -- 1987) hatte 1923 als Hypothese formuliert, dass sich Materie mit Elementen der Wellentheorie beschreiben lassen sollte und aus theoretischen \"{U}berlegungen f\"{u}r die Wellenl\"{a}nge (nach ihm als de-Broglie-Wellenl\"{a}nge bezeichnet) $\lambda = \frac{h}{m\cdot v} = \frac{h}{p}$ hergeleitet. 1961 wurde das Doppelspaltexperiment mit Elektronen durch Claus J\"{o}nsson (1931 -- ) an k\"{u}nstlich hergestellten Feinspalten durchgef\"{u}hrt.

\begin{pspicture}[showgrid=false](-1,-1.5)(13,6.5)
\definecolor{Kristallblau}{rgb}{0.1176,0.5647,1}% Dodger blau
\definecolor{RandFarbe}{rgb}{0.1176,0.5647,1}%{0,0.75,1}% DeepSkyBlue

%%R\"{o}hre
%\psellipse[fillstyle=solid,fillcolor=Kristallblau!50,opacity=0.3,linestyle=dashed](1.5,2.5)(0.35,1.5)

\pscustom[fillstyle=gradient,gradbegin=cyan!20,gradend=white,gradmidpoint=0.45,gradangle=0]{%
\psellipticarcn(1.5,2.5)(0.35,1.5){270}{90}
\psellipticarcn(10,2.5)(3,3.75){151.7}{-151.7}
%\psarcn(11,2.5){3.5}{154.623}{-154.623}
\closepath%
}
\psellipse[fillstyle=solid,fillcolor=Kristallblau!50,opacity=0.3,linewidth=0.4pt,linestyle=none](1.5,2.5)(0.35,1.5)
\psellipticarcn[linewidth=1pt](1.5,2.5)(0.35,1.5){270}{90}
\psellipse[fillstyle=solid,fillcolor=Kristallblau!50,opacity=0.3,linewidth=0.4pt,linestyle=none](7.3,2.5)(0.35,1.5)


\psellipticarcn[linewidth=1.8pt,linecolor=SeaGreen](12,2.5)(0.5,2.75){270}{90}
\psellipticarc[linewidth=1.5pt,linecolor=SeaGreen!50](12,2.5)(0.5,2.75){270}{90}

\pscustom[fillstyle=solid,linestyle=none,fillcolor=SeaGreen!30]{%
\psellipticarcn(12.36,2.5)(0.4,2.2){270}{90}
\psellipticarcn(10,2.5)(3,3.75){30}{-30}
\closepath%
}


%%Heiz
\pscoil[coilarm=0.075cm,coilwidth=3mm,coilheight=0.3,linewidth=0.5pt,linecolor=orange](2.5,2.2)(2.5,2.8)
\psline[arrowscale=1](2.5,2.8)(2.5,3)(0.5,3)
\psline[arrowscale=1](2.5,2.2)(2.5,2)(0.5,2)
\psellipse[fillstyle=solid,fillcolor=black](1.5,3)(0.03,0.04)
\psellipse[fillstyle=solid,fillcolor=black](1.5,2)(0.03,0.04)
\battery[labeloffset=0.85](0.5,2)(0.5,3){$U_{\text{H}}$}
\psline[arrowscale=1]{*-}(0.5,2)(0.5,0)
\battery[labeloffset=.85](4.5,0)(0.5,0){$U_{\text{a}}$}
\psline(4.5,0)(4.5,2)
\psline[arrowscale=1]{*-}(2,2)(2,1.5)(3.2,1.5)(3.2,2)


\rput{-90}(2.85,2.5){\Zylinder{0.75}{0.5}{black!90!cyan!80}{gray!10}{0.3}{black!90!cyan!80}{1pt}{none}{middle}}
\rput{-90}(2.85,2.5){%
\psellipse[fillstyle=gradient,gradbegin=black!90!cyan!80,gradend=gray!10,gradmidpoint=0.65,gradangle=90,linewidth=0.3pt,linecolor=black!85](0,0.75)(!0.48 dup 0.3 mul)}
\psframe[fillstyle=gradient,gradbegin=red!10,gradend=red!70,gradmidpoint=0.5,gradangle=0,linestyle=none,framearc=0.3](3.47,2.45)(4.5,2.55)
\rput{-90}(4.45,2.5){\Zylinder{0.1}{0.5}{black!90!cyan!80}{gray!10}{0.3}{black!90!cyan!80}{1pt}{none}{middle}}
\rput{-90}(4.45,2.5){%
\psellipse[fillstyle=gradient,gradbegin=black!90!cyan!80,gradend=gray!10,gradmidpoint=0.65,gradangle=90,linewidth=0.3pt,linecolor=black!85](0,0.1)(!0.2 dup 0.3 mul)}
\psframe[fillstyle=gradient,gradbegin=red!10,gradend=red!70,gradmidpoint=0.5,gradangle=0,linestyle=none,framearc=0.3](4.51,2.45)(7.4,2.55)
\psellipse[fillstyle=solid,fillcolor=Kristallblau!50,opacity=0.3,linewidth=0.4pt,linestyle=none](7.3,2.5)(0.35,1.5)
\psellipticarcn[linewidth=0.5pt](7.27,2.5)(0.35,1.5){270}{90}

\rput{-90}(7.3,2.5){\Zylinder{0.07}{0.6}{black!90!cyan!80}{gray!10}{0.3}{black!90!cyan!80}{1pt}{none}{middle}}
\psline[linecolor=red,linewidth=1.2pt](7.4,2.5)(12.7,4)
\psline[linewidth=0.6pt,linestyle=dashed](7.4,2.5)(13,2.5)
\psarc[arrows=<->,arrowscale=1.3,arrowinset=0.05](7.4,2.5){2.5}{0}{16.44}
\uput{1.7}[8.22](7.4,2.5){\small $2\alpha$}

\psellipticarcn[linewidth=1.8pt,linecolor=SeaGreen](12.36,2.5)(0.4,2.2){270}{90}
\psellipticarc[linewidth=1.5pt,linecolor=SeaGreen!50](12.36,2.5)(0.4,2.2){270}{90}
\psellipticarcn[linewidth=1.8pt,linecolor=SeaGreen](12.7,2.5)(0.3,1.5){270}{90}
\psellipticarc[linewidth=1.5pt,linecolor=SeaGreen!50](12.7,2.5)(0.3,1.5){270}{90}
\psellipticarcn[fillstyle=solid,fillcolor=SeaGreen,linestyle=none](12.98,2.5)(0.12,0.28){270}{90}
\psellipticarcn[fillstyle=solid,fillcolor=SeaGreen,linecolor=SeaGreen,linewidth=1.7pt](12.92,2.5)(0.1,0.28){80}{-80}

\pscustom[linewidth=1pt]{%
\psellipticarcn(1.5,2.5)(0.35,1.5){270}{90}
\psellipticarcn(10,2.5)(3,3.75){151.7}{-151.7}
%\psarcn(11,2.5){3.5}{154.623}{-154.623}
\closepath%
}

\psline(2.6,2.72)(2.6,5)
\uput[90](2.6,5){Kathode}
\psline(4.5,3)(4.5,5)
\uput[90](4.5,5){Anode}
\psline(7.335,1.9)(6.5,1.2)(6.5,0)
\uput[-90](6.5,0){Graphitfolie}

\end{pspicture}

\section*{Netzebenen im Graphit}

Der Winkel des einfallenden Strahles wird nicht, wie in der Optik \"{u}blich, gegen das Lot, sondern gegen die Netzebene gemessen. Der einfallende Strahl wird durch die Reflexion an den Netzebenen um den doppelten Winkel abgelenkt.

\def\FM{\psBall[linecolor=blue!50,slopebegin=blue!20,sloperadius=0.25,linewidth=0.1pt,slopecenter=0.65 0.6,linestyle=solid](0,0){blue}{8pt}}
\def\LiP{\psBall[linecolor=black!60,slopebegin=gray!20,sloperadius=0.09,linewidth=0.1pt,slopecenter=0.65 0.6,linestyle=solid](0,0){black!80}{3.5pt}}

\definecolor{Myblau}{rgb}{0.1176,0.5647,1}% Dodger blau
\definecolor{Mygreen}{rgb}{0.196,0.804,0.196}% LimeGreen
\definecolor{Mypurple}{rgb}{0.627,0.125,0.941}% Purple
\definecolor{Mynavajo}{rgb}{0.545,0.475,0.369}% 139 121 94


{\psset{unit=1.75cm}
\begin{pspicture}[showgrid=false](1,0.5)(11,5)
\def\wi1{30}
\def\laenge{4}
\pnode(1,3){E1}
\pnode(11,3){E2}
\pnode(1,2){E3}
\pnode(11,2){E4}

\pnode(6,3){S1}
\pnode(6,2){S2}

\rput(S1){\pnode(!\laenge\space 180 \wi1\space sub PtoC){T1}}
\rput(S1){\pnode(\laenge;\wi1){T2}}
\rput(S2){\pnode(!\laenge\space 180 \wi1\space sub PtoC){T3}}
\rput(S2){\pnode(\laenge;\wi1){T4}}

\psIntersectionPoint(T3)(S2)(E1)(E2){S3}
\psIntersectionPoint(T4)(S2)(E1)(E2){S4}
\psIntersectionPoint(T4)(S2)(T1)(S1){S5}
\psIntersectionPoint(T2)(S1)(T3)(S2){S6}

\pstProjection[PointName=none,PointSymbol=none]{T3}{S2}{S1}[W1]
\pstProjection[PointName=none,PointSymbol=none]{T4}{S2}{S1}[W2]

\psRelNode(T1)(S1){1.75}{R1}
\psRelNode(T3)(S2){1.64}{R2}
\pscustom[fillstyle=solid,opacity=0.3,fillcolor=yellow,linestyle=none]{%
\pslineByHand[VarStepEpsilon=1.5,varsteptol=1](T1)(T3)\pslineByHand[VarStepEpsilon=1.5,varsteptol=0.6](T1)(T3)
\psline(T3)(R2)
\pslineByHand[VarStepEpsilon=1.5,varsteptol=1](R2)(R1)
}

\pscustom[fillstyle=solid,opacity=0.3,fillcolor=orange!40,linestyle=none]{%
\pslineByHand[VarStepEpsilon=1.5,varsteptol=1](T1)(T3)\pslineByHand[VarStepEpsilon=1.5,varsteptol=0.6](T2)(T4)
\psline(T4)(S2)(S1)(T2)
}

\multido{\rA=1+1}{3}{%
\pcline[linecolor=cyan,linewidth=0.5pt,nodesep=-0.4](2,\rA)(10,\rA)
\multido{\rB=2+1}{9}{%
\rput(\rB,\rA){\LiP}
}
}
\pcline[arrowlength=1.5,arrowscale=1.2,arrowinset=0.02,tbarsize=8pt,offset=-.4]{<->}(10,1)(10,2)
\ncput*{\small $d$}
\pcline[arrowlength=1.5,arrowscale=1.2,arrowinset=0.02,tbarsize=8pt,offset=-.4]{<->}(10,2)(10,3)
\ncput*{\small $d$}

{\psset{linecolor=red,ArrowInside=->,ArrowInsidePos=0.25,arrowlength=1.6,arrowscale=1.2,arrowinset=0.04}%
\pcline(T1)(S1)
\pcline(T3)(S2)
\pcline[ArrowInsidePos=0.75](S1)(T2)
\pcline[ArrowInsidePos=0.75](S2)(T4)
}
\pcline[linecolor=red,linestyle=dashed,nodesepB=-3](T1)(S1)
\pcline[linecolor=red,linestyle=dashed,nodesepB=-2.5](T3)(S2)

\psarc[linecolor=Myblau,linewidth=0.6pt,arrowinset=0.04,arrowscale=1.1,arrowlength=1.6]{<->}(S3){0.8}{!180 \wi1\space sub}{180}
\uput{0.42}[!180 \wi1\space 2 div sub]{0}(S3){\small\textcolor{Myblau}{$\alpha_{1}^{}$}}
%\psarc[linecolor=Myblau,linewidth=0.6pt,arrowinset=0.04,arrowscale=1.1,arrowlength=1.6]{<->}(S2){1.5}{!180 \wi1\space sub}{180}
%\uput{0.9}[!180 \wi1\space 2 div sub]{0}(S2){\small\textcolor{Myblau}{$\alpha_{1} $}}

\psarc[linecolor=Myblau,linewidth=0.6pt,arrowinset=0.04,arrowscale=1.1,arrowlength=1.6]{<->}(S4){0.8}{0}{!\wi1\space}
\uput{0.42}[!\wi1\space 2 div]{0}(S4){\small\textcolor{Myblau}{$\alpha_{1}^{}$}}
\psarc[linecolor=Myblau,linewidth=0.6pt,arrowinset=0.04,arrowscale=1.1,arrowlength=1.6]{<->}(S5){0.8}{!\wi1\space neg}{!\wi1\space}
\uput{0.24}[0]{0}(S5){\small\textcolor{Myblau}{$2\cdot \alpha_{1}^{}$}}


\psline[linecolor=Mypurple](S1)(S2)
\psline[linecolor=Mypurple](S1)(W1)
\psline[linecolor=Mypurple](S1)(W2)

\pcline[linecolor=Mygreen](W1)(S2)
\pcline[linecolor=Mygreen,offset=-0.15,arrowlength=1.4,arrowscale=1.1,arrowinset=0.02,tbarsize=8pt]{|<->|}(W1)(S2)
\nbput[labelsep=1pt,nrot=:U]{\scriptsize \textcolor{Mygreen}{$\Delta s_{\scriptscriptstyle 1}^{}$}}
\pcline[linecolor=Mynavajo](S2)(W2)
\pcline[linecolor=Mynavajo,offset=-0.15,arrowlength=1.4,arrowscale=1.1,arrowinset=0.02,tbarsize=8pt]{|<->|}(S2)(W2)
\nbput[labelsep=1pt,nrot=:U]{\scriptsize \textcolor{Mynavajo}{$\Delta s_{\scriptscriptstyle 2}^{}$}}

\psset{RightAngleType=german,RightAngleSize=0.18}
\pstRightAngle{S1}{W2}{S2}
\pstRightAngle{S2}{W1}{S1}
\pstMarkAngle[MarkAngleRadius=0.5,LabelSep=0.41]{W1}{S1}{S2}{\scriptsize $\alpha_{\scriptscriptstyle 1}^{}$}
\pstMarkAngle[MarkAngleRadius=0.55,LabelSep=0.43]{S2}{S1}{W2}{\scriptsize $\alpha_{\scriptscriptstyle 1}^{}$}

\rput[r]{\wi1}(9.25,4.4){zum 1. Maximum}
%1er maximum
\end{pspicture}
}

\end{document} 

�cran de graphite mince
anneaux de diffraction
