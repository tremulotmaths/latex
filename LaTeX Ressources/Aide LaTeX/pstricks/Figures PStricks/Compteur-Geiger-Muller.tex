\documentclass[dvipsnames]{article}
\usepackage{amsmath,amssymb}
\usepackage{pst-grad,pstricks}
\usepackage[a4paper,margin=2cm]{geometry}
% Exp�rience de Franck et Hertz
\begin{document}
%Geiger-M\"{u}ller-Z\"{a}hlrohr
\definecolor{khaki}{HTML}{F0E68C}
\psset{unit=0.75}
\begin{center}
\begin{pspicture}(-6.5,-4.5)(8,2)
\psframe(-6.5,-4.5)(8,2)
\psframe[fillstyle=gradient,gradbegin=Goldenrod,gradend=Yellow!35,gradmidpoint=0.75,linestyle=none](-4.5,-1)(3,1)
%% Stromzuleitung
\psline(0,-2.4)(0,-4)
\psline{-o}(0,-4)(2.3,-4)
\uput[90](2.3,-4){$-$}
\psline{o-}(2.7,-4)(5,-4)
\uput[90](2.7,-4){$+$}
\psline(5,-4)(5,0)(4,0)
\psframe[fillstyle=solid,fillcolor=white](4.8,-1.6)(5.2,-2.4)
\psdot[linecolor=blue](5,-1.4)
\psdot[linecolor=blue](5,-2.6)
\psline[linecolor=blue]{->}(5,-1.4)(7.5,-1.4)
\psline[linecolor=blue]{->}(5,-2.6)(7.5,-2.6)
\uput[90](6.6,-2.6){\parbox{2cm}{\centering\small \textcolor{blue}{vers l'amplificateur}}}
%% Isolator
\psellipticwedge[fillstyle=solid,fillcolor=khaki](3.5,0)(1,1.7){30}{330}
\psellipticwedge[fillstyle=solid,fillcolor=khaki](3,0)(1,1.7){30}{330}
\psline(3,0)(3.5,0)
\psline(3.85,0.85)(4.35,0.85)
\psline(3.85,-0.85)(4.35,-0.85)
%% Geh\"{a}use
%% hinten oben
\pscustom[fillstyle=gradient,gradbegin=OliveGreen,gradend=OliveGreen!20,gradmidpoint=0.8]{%
\psellipticarcn(-3.8,0)(1,1.7){90}{52}
\psline(-2.9,1)(4,1)
\psellipticarc(3.6,0)(1,1.7){52}{90}
\psline(3.6,1.7)(-3.4,1.7)
}
%% hinten unten
\pscustom[fillstyle=gradient,gradbegin=OliveGreen,gradend=OliveGreen!20,gradmidpoint=.2]{%
\psellipticarc(-3.8,0)(1,1.7){270}{308}
\psline(-2.9,-1)(4,-1)
\psellipticarcn(3.6,0)(1,1.7){308}{270}
\psline(3.6,-1.7)(-3.4,-1.7)
}
%% vorne oben
\pscustom[fillstyle=gradient,gradbegin=OliveGreen,gradend=OliveGreen!20,gradmidpoint=0.8]{%
\psellipticarcn(-5,0)(1,1.8){90}{45}
\psline(-4.1,0.8)(-2.8,0.8)
\psellipticarc(-3.7,0)(1,1.8){45}{90}
\psline(-3.7,1.8)(-5,1.8)
}
%% vorne unten
\pscustom[fillstyle=gradient,gradbegin=OliveGreen,gradend=OliveGreen!20,gradmidpoint=0.2]{%
\psellipticarc(-5,0)(1,1.8){270}{315}
\psline(-4.1,-0.8)(-2.8,-0.8)
\psellipticarcn(-3.7,0)(1,1.8){315}{270}
\psline(-3.7,-1.8)(-5,-1.8)
}
%% Draht
\psframe[fillstyle=gradient,gradbegin=gray,gradend=white,gradmidpoint=0.5,linestyle=none](2.8,-0.163)(4,0.163)
\psellipse[fillstyle=solid,fillcolor=gray,linecolor=gray](2.8,0)(0.075,0.15)
\psellipticarc[linewidth=1pt,linecolor=gray](3.95,0)(0.075,0.15){270}{90}
\psframe[linecolor=BurntOrange,fillstyle=gradient,gradbegin=BurntOrange,gradend=BurntOrange!30,gradmidpoint=0.5,linewidth=0.3pt](-2.5,-0.1)(2.8,0.1)
\psellipse[fillstyle=solid,fillcolor=BurntOrange,linestyle=none](-2.5,0)(0.06,0.1)
\psellipticarc[linewidth=1pt,linecolor=BurntOrange](2.78,0)(0.05,0.1){270}{90}
%% Anschluss unten
\pscustom[fillstyle=gradient,gradbegin=OliveGreen,gradend=OliveGreen!20,gradmidpoint=0.25,gradangle=90]{
\pscurve(-0.5,-1.4)(0,-1.25)(0.5,-1.4)
\psline(0.5,-1.4)(0.5,-2.5)
\pscurve(0.5,-2.5)(0,-2.54)(-0.5,-2.5)
\psline(-0.5,-2.5)(-0.5,-1.4)
}
\psellipse[fillstyle=gradient,gradbegin=cyan!10,gradend=cyan!50,gradmidpoint=0.35,gradangle=-60,linewidth=3pt](-5,0)(1,1.85)
\end{pspicture}
\end{center}
\end{document}