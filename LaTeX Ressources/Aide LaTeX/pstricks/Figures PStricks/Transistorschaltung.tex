\documentclass[a4paper]{article}
\usepackage[T1]{fontenc}
\usepackage[latin1]{inputenc}
\usepackage{pst-circ}


\begin{document}

\def\runter{^{\phantom{x}}}
\newcommand{\qrq}{\quad \Rightarrow \quad}
\def\x{\hbox{$\,$}}





\vspace{1cm}

\begin{pspicture}(-1,-.5)(7.5,8)\psset{subgriddiv=0,griddots=5,gridlabels=7pt,dotscale=0.75, arrowscale=2}%
%\psgrid
%
%   Definition der Punkte
 \pnode(4.5,3){A} % Basis des Transistors
 \pnode(5,2){B} % Emitter des Transistors
 \pnode(5,4){C} % Kollektor des Transistors
 \pnode(3,3){D}
 \pnode(5,7){E}
 \pnode(5,0){F}
 \pnode(7,7){G}
 \pnode(7,0){H}
 \pnode(0,3){I}
 \pnode(0,7){J}
 \pnode(5,0){K}
 \pnode(0,0){O}
%
%   Transistor
\transistor[transistortype=NPN,arrowscale=1%,transistoribaselabel=$i_B$, transistoricollectorlabel
%=$i_C$, transistoriemitterlabel =$i_E$
](A)(B)(C) %%
%%   Widerstand R1
\resistor[labeloffset=1.4cm](I)(J){$\sf R_{1}=100\x \Omega$} %%
%%   Widerstand R2
\resistor[labeloffset=1.4cm](O)(I){$\sf R_{2}=80\x \Omega$} %%
%%Lampe
\psscalebox{0.75}{ \lamp(E)(C){$\sf L$}} %%
%%   Verbindungen
\wire[intensity=true,intensitywidth=2\pslinewidth, intensitycolor=red](G)(J) \rput(3,7.5){$\sf I=0,02\x A$}
%%
\psline(I)(A) \psline(O)(H) \psline(B)(K)
%%   Punkte markieren
\qdisk(I){2pt} \qdisk(E){2pt} \qdisk(F){2pt} \psdots[dotscale=1.5,dotstyle=o](G)
\psdots[dotscale=1.5,dotstyle=o](H) %
%%   Spannung
\tension[labeloffset=-1cm,tensionwidth =2\pslinewidth](H)(G){$\sf U=2,5\x V$}
\rput(7.5,7){$+$}
\rput(7.5,0){$-$}
\end{pspicture}%

\end{document}
