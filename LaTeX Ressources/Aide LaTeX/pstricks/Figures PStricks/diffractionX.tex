\documentclass{article}
\usepackage{amsmath,amssymb}
\usepackage{pst-slpe,pst-eucl,pstricks-add,pst-solides3d}
\usepackage[a4paper,margin=2cm]{geometry}
\begin{document}
% Principe de la diffraction des rayons X par un cristal de LiF 
\begin{pspicture}[showgrid=false](-3,-4)(13,8)
\def\FM{\psBall[linecolor=blue!50,slopebegin=blue!20,sloperadius=0.25,linewidth=0.1pt,slopecenter=0.65 0.6,linestyle=solid](0,0){blue}{8pt}}
\def\LiP{\psBall[linecolor=black!60,slopebegin=gray!10,sloperadius=0.13,linewidth=0.1pt,slopecenter=0.65 0.6,linestyle=solid](0,0){black!80}{4.57pt}}
\psset{viewpoint=100 32 32,Decran=100,lightsrc=viewpoint}
\psPoint(0,0,0){A00}
\psPoint(0,4,0){A01}
\psPoint(0,8,0){A02}
\psPoint(4,0,0){A10}
\psPoint(8,0,0){A20}
\psPoint(4,4,0){A11}
\psPoint(4,8,0){A12}
\psPoint(8,4,0){A21}
\psPoint(8,8,0){A22}
\psPoint(0,0,4){B00}
\psPoint(0,4,4){B01}
\psPoint(0,8,4){B02}
\psPoint(4,0,4){B10}
\psPoint(8,0,4){B20}
\psPoint(4,4,4){B11}
\psPoint(4,8,4){B12}
\psPoint(8,4,4){B21}
\psPoint(8,8,4){B22}
\psPoint(0,0,8){C00}
\psPoint(0,4,8){C01}
\psPoint(0,8,8){C02}
\psPoint(4,0,8){C10}
\psPoint(8,0,8){C20}
\psPoint(4,4,8){C11}
\psPoint(4,8,8){C12}
\psPoint(8,4,8){C21}
\psPoint(8,8,8){C22}
{\psset{linewidth=0.5pt}
\psline(A00)(A02)
\psline(A00)(A20)
\psline(A00)(C00)
\psline(A10)(A12)
\psline(A10)(C10)
\psline(A20)(A22)
\psline(A20)(C20)
\psline(A02)(C02)
\psline(A10)(C10)
\psline(A11)(C11)
\psline(A01)(C01)
\psline(A01)(A21)
\psline(A02)(A22)
\psline(B00)(B02)
\psline(B00)(B20)
\psline(B10)(B12)
\psline(B01)(B21)
\psline(B02)(B22)
}
\rput(A00){\LiP}
\rput(B01){\LiP}
\rput(B10){\LiP}
\rput(B00){\FM}
\pspolygon[fillstyle=solid,fillcolor=green!30,opacity=0.5,linestyle=none](A01)(A21)(C20)(C00)
\psline(A21)(C20)(C00)
\rput(B11){\FM}
\rput(A01){\FM}
\psPoint(4,12,10){R1}
\psLNode(C10)(A11){0.75}{N1}
\rput(N1){\psdot}
\pstInterLL[PointSymbol=none,PointName=none]{R1}{N1}{C11}{A12}{S1}
\pstRotation[RotAngle=-22,PointSymbol=none,PointName={\text{normale}}]{N1}{R1}[M]
\pstInterLL[PointSymbol=none,PointName=none]{M}{N1}{C11}{A12}{S2}
\pstRotation[RotAngle=-44,PointSymbol=none,PointName=none]{N1}{R1}[R2]
\pstInterLL[PointSymbol=none,PointName=none]{R2}{N1}{C11}{A12}{S3}
\pstTranslation[PointSymbol=none,PointName=none]{N1}{R1}{S2}[R3]
\pstTranslation[PointSymbol=none,PointName=none]{N1}{R2}{S2}[R4]
\psline[linecolor=cyan,linestyle=dashed,dash=4pt 2pt](S2)(N1)
\psline[linecolor=red,ArrowInside=->,ArrowInsidePos=0.5,arrowinset=0.05,arrowscale=1.5,arrowlength=1.6](S1)(N1)
\psline[linecolor=red,ArrowInside=->,ArrowInsidePos=0.5,arrowinset=0.05,arrowscale=1.5,arrowlength=1.6](N1)(S3)
\pstMarkAngle[MarkAngleRadius=1.6,arrows=<->,arrowinset=0.05,arrowscale=1.3,arrowlength=1.4]{S2}{N1}{S1}{}
\uput{1.1}[30](N1){$\alpha$}
\pstMarkAngle[MarkAngleRadius=1.7,arrows=<->,arrowinset=0.05,arrowscale=1.3,arrowlength=1.4]{S3}{N1}{S2}{}
\uput{1.1}[5](N1){$\alpha$}
\pspolygon[fillstyle=solid,fillcolor=green!30,opacity=0.5,linestyle=none](A02)(A22)(C21)(C01)
\psline(A22)(C21)(C01)
\pstInterLL[PointSymbol=none,PointName=none]{B10}{B12}{C11}{A12}{S8}
{\psset{linewidth=0.5pt}
\pcline[nodesepB=8pt](A11)(B11)
\psline(B12)(S8)
\psline(B20)(B22)
\psline(A21)(C21)
\psline(A12)(C12)
\psline(A22)(C22)
\psline(C02)(C22)
\psline(C20)(C22)
\psline(C00)(C02)
\psline(C00)(C20)
\psline(C10)(C12)
\psline(C01)(C21)
}
\rput(A20){\LiP}
\rput(A11){\LiP}
\rput(A02){\LiP}
\rput(A22){\LiP}
\rput(C00){\LiP}
\rput(C20){\LiP}
\rput(C11){\LiP}
\rput(C02){\LiP}
\rput(C22){\LiP}
\rput(B12){\LiP}
\rput(B21){\LiP}
\rput(A12){\FM}
\rput(A21){\FM}
\rput(A10){\FM}
\rput(B20){\FM}
\rput(B02){\FM}
\rput(B22){\FM}
\rput(C12){\FM}
\rput(C21){\FM}
\rput(C10){\FM}
\rput(C01){\FM}
\uput{8pt}[-90]{0}(A20){$\text{Li}^{+}$}
\uput{11pt}[-90]{0}(A21){$\text{F}^{-}$}
\psLNode(C21)(A22){0.2}{D1}
\pstProjection[PointSymbol=none]{C20}{A21}{D1}[D2]
\pcline[arrowinset=0.05,arrowscale=1.8,arrowlength=1.8]{<->}(D2)(D1)
\naput[nrot=:U,labelsep=2pt]{$d$}
\pstRightAngle[RightAngleType=german,RightAngleSize=0.5]{D1}{D2}{C20}
\pstRightAngle[RightAngleType=german,RightAngleSize=0.5]{C21}{D1}{D2}
\rput(S1){\psdot}
\rput(S2){\psdot}
\rput(S3){\psdot}
\psline[linecolor=red,ArrowInside=->,ArrowInsidePos=0.45,arrowinset=0.05,arrowscale=1.5,arrowlength=1.6](R1)(S1)
\psline[linecolor=cyan,linestyle=dashed,dash=4pt 2pt](M)(S2)
\pcline[linecolor=red,nodesepA=3,ArrowInside=->,ArrowInsidePos=0.45,arrowinset=0.05,arrowscale=1.5,arrowlength=1.6](R3)(S2)
\pcline[linecolor=red,nodesepB=4,ArrowInside=->,ArrowInsidePos=0.55,arrowinset=0.05,arrowscale=1.5,arrowlength=1.6](S2)(R4)
\pcline[linecolor=red,nodesepB=1,ArrowInside=->,ArrowInsidePos=0.5,arrowinset=0.05,arrowscale=1.5,arrowlength=1.6](S3)(R2)
\pstMarkAngle[MarkAngleRadius=2,arrows=<->,arrowinset=0.05,arrowscale=1.3,arrowlength=1.4]{M}{S2}{R3}{}
\uput{1.4}[30](S2){$\alpha$}
\pstMarkAngle[MarkAngleRadius=2.0,arrows=<->,arrowinset=0.05,arrowscale=1.3,arrowlength=1.4]{R4}{S2}{M}{}
\uput{1.4}[5](S2){$\alpha$}
\uput{0.2}[-35](R1){\shortstack[l]{rayons\strut\\[-6pt] incidents\strut}}
\uput{0.1}[120](R2){\shortstack[l]{rayons\strut\\[-6pt] diffract\'{e}s\strut}}
\end{pspicture}

\end{document}