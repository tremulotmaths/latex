\documentclass{article}
\usepackage{pst-grad,multido}
\SpecialCoor
\begin{document}
\definecolor{LinsenF}{rgb}{0.529,0.808,0.98}
\psset{unit=0.75}
\begin{center}
\begin{pspicture}(-11.5,-6)(6,5.5)
\psframe(-11.5,-6)(7,5.5)
\rput(-6.5,0){%
\begin{psclip}{\pscircle[linewidth=2pt,fillstyle=solid,fillcolor=LinsenF!50](0,0){4.75}}\psline[linecolor=gray](-2.3,-4)(-2.3,4)
\end{psclip}
\rput(0,1.25){\Large $\mu$Sv}
\uput[0](-2.3,-3){Image du fil de quartz}
\multido{\rA=-4+0.4}{21}{%
\psline(\rA,-0.15)(\rA,0.15)
}
\multido{\rB=-4+2}{5}{%
\psline[linewidth=1pt](\rB,-0.25)(\rB,0.25)
}
\uput[90](-4,0.25){0}
\uput[90](-2,0.25){50}
\uput[90](0,0.25){100}
\uput[90](2,0.25){150}
\uput[90](4,0.25){200}
}
%% Geh\"{a}use
\pspolygon*(-0.5,5.2)(-0.7,5.2)(-0.7,4.75)(-0.8,4.75)(-0.8,3.4)(-0.7,3.4)(-0.7,-5.5)(-0.3,-5.5)(-0.3,-5.2)(-0.5,-5.2)
\pspolygon*(0.5,5.2)(0.7,5.2)(0.7,4.75)(0.8,4.75)(0.8,3.4)(0.7,3.4)(0.7,-5.5)(0.3,-5.5)(0.3,-5.2)(0.5,-5.2)
\pscurve[linewidth=3pt](0.75,4.7)(1.1,5.3)(1.3,5)(0.85,2)
\psline[linewidth=1pt](-0.7,5.2)(0.7,5.2)
\psline[linewidth=1pt](-0.7,-5.5)(0.7,-5.5)
%% Optik
\pspolygon[linejoin=1,fillstyle=solid,fillcolor=gray](-0.5,5.2)(-0.5,2.8)(-0.35,2.8)(-0.35,1.2)(-0.2,1.2)(-0.2,3)(-0.3,3)(-0.3,4.75)
\pspolygon[linejoin=1,fillstyle=solid,fillcolor=gray](0.5,5.2)(0.5,2.8)(0.35,2.8)(0.35,1.2)(0.2,1.2)(0.2,3)(0.3,3)(0.3,4.75)
\psellipse[fillstyle=gradient,gradbegin=white,gradend=LinsenF,gradmidpoint=0.5,gradangle=2](0,4.65)(0.3,0.1)
\psellipse[fillstyle=gradient,gradbegin=white,gradend=LinsenF,gradmidpoint=0.5,gradangle=2](0,1.2)(0.2,0.1)
\psline[linewidth=1pt](-0.3,3.15)(0.3,3.15)
\psline(-0.2,3)(0.2,3)
%%Federbalg
\psline(-0.2,-5)(-0.3,-4.8)(-0.2,-4.7)(-0.3,-4.6)(-0.2,-4.5)(-0.3,-4.4)(-0.2,-4.3)(-0.25,-4.2)
\psframe(-0.25,-4.2)(0.25,-3.9)
\psline(0.2,-5)(0.3,-4.8)(0.2,-4.7)(0.3,-4.6)(0.2,-4.5)(0.3,-4.4)(0.2,-4.3)(0.25,-4.2)
\psline(-0.2,-5)(0.2,-5)
\psframe[fillstyle=solid,fillcolor=gray](-0.5,-5.2)(0.5,-5)
\psframe[fillstyle=solid,fillcolor=white](-0.25,-5.2)(0.25,-5)
\psline[linewidth=2pt](0,-5)(0,-3.75)
%%Ionisationskammer
\psline(-0.5,-3.8)(0.5,-3.8)
\pscustom[fillstyle=solid,fillcolor=gray]{%
\psline(-0.1,0.8)(-0.1,0.5)
\psarc(-0.1,0.25){0.25}{90}{180}
\psline(-0.35,0.25)(-0.35,-3.8)
\psline(-0.5,-3.8)(-0.5,0.8)
\psline(-0.5,0.8)(-0.1,0.8)
}
\pscustom[fillstyle=solid,fillcolor=gray]{%
\psline(0.1,0.8)(0.1,0.5)
\psarcn(0.1,0.25){0.25}{90}{0}
\psline(0.35,0.25)(0.35,-3.8)
\psline(0.5,-3.8)(0.5,0.8)
\psline(0.5,0.8)(0.1,0.8)
}
\pscurve[linewidth=1.5pt](0.2,-0.2)(0.15,-1.2)(0,-3.6)
\pscurve[linewidth=1.5pt](0.2,-0.2)(0.3,0.2)(0.1,0.3)(-0.2,-0.8)
\psline[linewidth=1.5pt](0,-3.6)(-0.25,-3.6)
\psframe[fillstyle=solid,fillcolor=white](-0.45,-0.8)(0.45,-1.75)

\rput[l](2,4.7){Oculaire}
\rput[l](2,3.1){R\'{e}ticule gradu\'{e}}
\rput[l](2,2.3){Optique}
\rput[l](2,1.2){Objectif}
\rput[l](2,0.4){Fil de Quartz}
\rput[l](2,-0.3){Chambre d'ionisation}
\rput[l](2,-1.3){Isolant}
\rput[l](2,-2.5){Chambre d'ionisation}
\rput[l](2,-4.5){Soufflet de chargement}
%\psgrid
\end{pspicture}
\end{center}
\end{document} 