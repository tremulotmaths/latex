\documentclass[12pt]{article}

\usepackage[T1]{fontenc}
\usepackage[latin1]{inputenc}
\usepackage{amsmath,amssymb}
\renewcommand{\familydefault}{\sfdefault}
\usepackage{mathpazo}
\usepackage[scaled=.95]{helvet}


%\usepackage{geometry}

%%% Setzen der R\"{a}nder und Dimensionen
%\paperheight=21cm
%\paperwidth=38cm



\usepackage{graphics}
\usepackage[rgb]{pstricks}

\usepackage{pst-grad,pst-text}
\usepackage{pst-slpe,pst-blur,pst-node}
\usepackage{pst-plot,pst-eucl}
\usepackage{pst-thick}
\usepackage{pst-node}
\usepackage{pstricks-add}

\parindent0pt
\pagestyle{empty}




\begin{document}

\psset{dimen=medium}
%----------------------------------------------------------------------------------------------
%------------------- Definition der verschiedenen Farben --------------------------------------
%----------------------------------------------------------------------------------------------
\definecolor{Speisewasser}{rgb}{0.11765,0.5647,1}
\definecolor{Kuehlturmwasser}{rgb}{0.153,0.251,0.545}
\definecolor{DampfND}{rgb}{1,0.2706,0}%   Niederdruck-Farbe
\definecolor{DampfMD}{rgb}{1,0.549,0}%   Mitteldruck-Farbe
\definecolor{DampfHD}{rgb}{0.698,0.1333,0.1333}%   Hochdruck-Farbe
%----------------------------------------------------------------------------------------------
%------------------- Definition der Wasserpumpe mit F\"{u}llfarbenoption --------------------------
%----------------------------------------------------------------------------------------------
\newcommand{\pumpeA}[1]{%
%\psset{linewidth=0.0116}
\pnode(0,0){O0}
\pnode(-0.51,0.225){O1}\pnode(0,0.225){O3}
\pnode(-0.51,0.125){O4}\pnode(0,0.125){O6}
\pstInterLC[Radius=\pstDistVal{0.25},PointSymbol=none,PointName=none]{O1}{O3}{O0}{}{O2}{O10}
\pstInterLC[Radius=\pstDistVal{0.25},PointSymbol=none,PointName=none]{O4}{O6}{O0}{}{O5}{O11}
\pnode(0.55,-0.225){U1}\pnode(0,-0.225){U3}
\pnode(0.55,-0.125){U4}\pnode(0,-0.125){U6}
\pstInterLC[Radius=\pstDistVal{0.25},PointSymbol=none,PointName=none]{U1}{U3}{O0}{}{U2}{U10}
\pstInterLC[Radius=\pstDistVal{0.25},PointSymbol=none,PointName=none]{U4}{U6}{O0}{}{U5}{U11}
%----------------------------------------------------------------------------------------------
\pscustom[fillstyle=solid,fillcolor=#1,linestyle=none]{%
\psline(O1)(O2)
\pstArcnOAB{O0}{O2}{U5}
\psline(U5)(U4)
\psline(U1)(U2)
\pstArcnOAB{O0}{U2}{O5}
\psline(O5)(O4)
\closepath%
}
\pscustom[linewidth=0.0135]{
\psline(O1)(O2)
\pstArcnOAB{O0}{O2}{U5}
\psline(U5)(U4)
\psline(U1)(U2)
\pstArcnOAB{O0}{U2}{O5}
\psline(O5)(O4)
}
%\pscustom[fillstyle=solid,fillcolor=#1,linestyle=none]{
%    \psline(O2)(O1)(O4)(O5)
%    \pstArcnOAB{O0}{O5}{O2}
%}
%\psline(O2)(O1)
%\psline(O4)(O5)
%\psline[linewidth=0.1,linecolor=#1](-0.5,0.175)(-0.55,0.175)
%\psline(O1)(-0.5,0.225)
%\psline(O4)(-0.5,0.125)
%
\multido{\na=0+45}{8}{%
\rput{\na}(0,0){\psarc[linewidth=1pt]{c-c}(-0.13,0.2){0.2}{-50}{-5}}
}
\pscircle[fillstyle=solid,fillcolor=black](O0){0.07}
\pscircle[fillstyle=solid,fillcolor=gray!50](O0){0.0325}
\psarc[linewidth=0.5pt,arrowlength=1.2,arrowinset=0.05]{->}(O0){0.325}{60}{120}
%\psline[doubleline=true,border=0.4pt,linewidth=0.75pt,doublecolor=#1,bordercolor=black,linecolor=#1,linearc=0.1](-0.5,0.175)(-1,0.175)
%\psline[doubleline=true,border=0.4pt,linewidth=0.75pt,doublecolor=#1,bordercolor=black,linecolor=#1,linearc=0.1](0.5,-0.175)(1,-0.175)
}
%----------------------------------------------------------------------------------------------
%----------------------------------------------------------------------------------------------
%------------------- Definition der Wasserleitung mit Speisewasser ----------------------------
%----------------------------------------------------------------------------------------------
\newpsstyle{WasserleitungSP}{doubleline=true,border=0.4pt,linewidth=0.75pt, doublecolor=Speisewasser,bordercolor=black,linecolor=Speisewasser,linearc=0.2}
%----------------------------------------------------------------------------------------------
%------------- Definition der Wasserleitung mit Ableitung K\"{u}hlturm ----------------------------
%----------------------------------------------------------------------------------------------
\newpsstyle{WasserleitungKT}{doubleline=true,border=0.4pt,linewidth=0.75pt, doublecolor=Kuehlturmwasser,bordercolor=black,linecolor=Kuehlturmwasser,linearc=0.2}
%----------------------------------------------------------------------------------------------
%------------- Definition der Dampleitung Hochdruck ----------------------------
%----------------------------------------------------------------------------------------------
\newpsstyle{LeitungND}{doubleline=true,border=0.4pt,linewidth=0.75pt, doublecolor=DampfND,bordercolor=black,linecolor=DampfND,linearc=0.2}
%----------------------------------------------------------------------------------------------
%------------- Definition der Dampleitung Mitteldruck ----------------------------
%----------------------------------------------------------------------------------------------
\newpsstyle{LeitungMD}{doubleline=true,border=0.4pt,linewidth=0.75pt, doublecolor=DampfMD,bordercolor=black,linecolor=DampfMD,linearc=0.2}
%----------------------------------------------------------------------------------------------
%------------- Definition der Dampleitung Niederdruck ----------------------------
%----------------------------------------------------------------------------------------------
\newpsstyle{LeitungHD}{doubleline=true,border=0.4pt,linewidth=0.75pt, doublecolor=DampfHD,bordercolor=black,linecolor=DampfHD,linearc=0.2}
%----------------------------------------------------------------------------------------------
%-------------------------------- Definition des K\"{u}hlturm -------------------------------------
%----------------------------------------------------------------------------------------------
\def\Kuehlturm{%
%----------------------------------------------------------------------------------------------
%-------------------------------- Wolken mit Schatten aus pst-blur ----------------------------
%----------------------------------------------------------------------------------------------
\psellipse[linestyle=none,opacity=0.1,fillstyle=solid,fillcolor=gray!30,shadowcolor=gray!40,
shadowangle=70,shadow=true,blur=true,shadowsize=30pt,blurradius=20pt,blursteps=150,blurbg=white](5.1,10.9)(1.5,0.6)
\psellipse[linestyle=none,opacity=0.1,fillstyle=solid,fillcolor=gray!30,shadowcolor=gray!50,
shadowangle=60,shadow=true,blur=true,shadowsize=30pt,blurradius=20pt,blursteps=150,blurbg=gray!10](4.5,10.3)(1.7,0.6)
\psellipse[linestyle=none,opacity=0.1,fillstyle=solid,fillcolor=gray!30,shadowcolor=gray!50,
shadowangle=90,shadow=true,blur=true,shadowsize=30pt,blurradius=20pt,blursteps=150,blurbg=gray!20](4.7,9.5)(1.9,0.6)
\psellipse[linestyle=none,opacity=0.1,fillstyle=solid,fillcolor=gray!30,shadowcolor=gray!60,
shadowangle=75,shadow=true,blur=true,shadowsize=30pt,blurradius=20pt,blursteps=150,blurbg=gray!30](4,9)(2.4,0.5)
\rput(3.2,12.0){%
\pscustom[fillstyle=solid,fillcolor=gray!30,opacity=0.45,linestyle=none,linearc=0.3]{%
\pscurve(-1.3,0)(0,0.6)(1.3,0.2)
\pscurve(2,0.4)(3.4,0.5)(3.9,0)
\pscurve(4.5,0.2)(5.5,-1)(4,-2)
\pscurve(4,-2)(0,-3)(-2,-2)(-1.3,0)
\closepath%
}}
%----------------------------------------------------------------------------------------------
%-------------------- Definition der K\"{u}hlturmrandlinien mit F\"{u}llung ----------------------------
%----------------------------------------------------------------------------------------------
\rput{-90}(4,0){%
\pscustom[linecolor=gray!40,linewidth=2pt,fillcolor=gray!10,fillstyle=solid]{%
      \psplot{-9.5}{-0.5}{x 7.5 add dup mul 0.03 mul 2.5 add}
      \psline(!-0.5 -0.5 7.5 add dup mul 0.03 mul 2.5 add)(!0 -0.5 7.5 add dup mul 0.03 mul 2.5 add)
      \psline(!0 -0.5 7.5 add dup mul 0.03 mul 2.5 add neg)(!-0.5 -0.5 7.5 add dup mul 0.03 mul 2.5 add neg)
      \psplot{-0.5}{-9.5}{x 7.5 add dup mul 0.03 mul 2.5 add neg}
\closepath%
}
}
%----------------------------------------------------------------------------------------------
%------------------------- Dicke Pfeile im K\"{u}hlturm nach oben ---------------------------------
%----------------------------------------------------------------------------------------------
{\psset{ArrowFill=true,arrowinset=0,arrowscale=0.7,arrowlength=0.5,arrowsize=1.5,framearc=0.05,dimen=outer}%
\psline[linewidth=0.6cm,linecolor=gray!40]{->}(4,6)(4,7.25)%
\psline[linewidth=0.6cm,linecolor=gray!40]{->}(2.5,6)(2.5,7.25)%
\psline[linewidth=0.6cm,linecolor=gray!40]{->}(5.5,6)(5.5,7.25)%
}
%----------------------------------------------------------------------------------------------
%---------------------------- Wasserbecken im K\"{u}hlturm unten ----------------------------------
%----------------------------------------------------------------------------------------------
\psframe[linecolor=black,linewidth=1pt,fillstyle=solid,fillcolor=Kuehlturmwasser](0.08,0.05)(7.92,0.4)
\psline[linewidth=1pt,linecolor=Kuehlturmwasser](0.098,0.4)(7.902,0.4)
%----------------------------------------------------------------------------------------------
%----------------------- Wei{\ss}e Dreiecke f\"{u}r Luftzufuhr und Luftpfeile -------------------------
%----------------------------------------------------------------------------------------------
\multips(0.6,0.63)(1.13,0){7}{\pspolygon[linecolor=gray!60,linewidth=1.3pt,fillstyle=solid,fillcolor=white](-0.3,0)(0.3,0)(0,0.7)}
\multips(1.17,0.63)(1.13,0){6}{\pspolygon[linecolor=gray!60,linewidth=1.3pt,fillstyle=solid,fillcolor=white](0,0)(0.3,0.7)(-0.3,0.7)}
%--------------------------------------------------------------------------------------------------------------
\rput{-20}(0,1){%
\pscurve[linewidth=0.5pt,arrowscale=1,arrowinset=0.04,arrowsize=0.08,arrowlength=1.4]{->}(0,0)(0.3,-0.1)(0.6,0)
}
\rput{-20}(2.2,1){%
\pscurve[linewidth=0.5pt,arrowscale=1,arrowinset=0.04,arrowsize=0.08,arrowlength=1.4]{->}(0,0)(0.3,-0.1)(0.6,0)
}
\rput{20}(7.4,0.8){%
\pscurve[linewidth=0.5pt,arrowscale=1,arrowinset=0.04,arrowsize=0.08,arrowlength=1.4]{<-}(0,0)(0.3,-0.1)(0.6,0)
}
\rput{20}(5.2,0.8){%
\pscurve[linewidth=0.5pt,arrowscale=1,arrowinset=0.04,arrowsize=0.08,arrowlength=1.4]{<-}(0,0)(0.3,-0.1)(0.6,0)
}
%----------------------------------------------------------------------------------------------
%------------------------------------ Spr\"{u}hwasser im K\"{u}hlturm ---------------------------------
%----------------------------------------------------------------------------------------------
\psline[style=WasserleitungSP](-4.5,3.5)(6.5,3.5)
\pcline[linewidth=0.5pt,arrowscale=1,arrowinset=0.04,arrowsize=0.08,arrowlength=1.4,linecolor=white]{->}(-0.5,3.5)(0,3.5)
\multirput{0}(1.8,3.5)(1.1,0){5}{%
\psline[linecolor=Speisewasser,linewidth=1.0pt]{c-c}(0,0)(0,-0.15)
\psline[linecolor=Speisewasser,linewidth=1.0pt]{c-c}(0,-0.3)(0,-0.45)
\psline[linecolor=Speisewasser,linewidth=1.0pt]{c-c}(0,-0.7)(0,-0.85)
\psline[linecolor=Speisewasser,linewidth=1.0pt]{c-c}(0,-1.2)(0,-1.35)
\psline[linecolor=Speisewasser,linewidth=1.0pt]{c-c}(0,-1.8)(0,-1.95)
\psline[linecolor=Speisewasser,linewidth=1.0pt]{c-c}(0,-2.5)(0,-2.65)
}
\multirput{5}(1.8,3.5)(1.1,0){5}{%
\psline[linecolor=Speisewasser,linewidth=1.0pt]{c-c}(0,0)(0,-0.15)
\psline[linecolor=Speisewasser,linewidth=1.0pt]{c-c}(0,-0.3)(0,-0.45)
\psline[linecolor=Speisewasser,linewidth=1.0pt]{c-c}(0,-0.7)(0,-0.85)
\psline[linecolor=Speisewasser,linewidth=1.0pt]{c-c}(0,-1.2)(0,-1.35)
\psline[linecolor=Speisewasser,linewidth=1.0pt]{c-c}(0,-1.8)(0,-1.95)
\psline[linecolor=Speisewasser,linewidth=1.0pt]{c-c}(0,-2.5)(0,-2.65)
}
\multirput{10}(1.8,3.5)(1.1,0){5}{%
\psline[linecolor=Speisewasser,linewidth=1.0pt]{c-c}(0,0)(0,-0.15)
\psline[linecolor=Speisewasser,linewidth=1.0pt]{c-c}(0,-0.3)(0,-0.45)
\psline[linecolor=Speisewasser,linewidth=1.0pt]{c-c}(0,-0.7)(0,-0.85)
\psline[linecolor=Speisewasser,linewidth=1.0pt]{c-c}(0,-1.2)(0,-1.35)
\psline[linecolor=Speisewasser,linewidth=1.0pt]{c-c}(0,-1.8)(0,-1.95)
\psline[linecolor=Speisewasser,linewidth=1.0pt]{c-c}(0,-2.5)(0,-2.65)
}
\multirput{-5}(1.8,3.5)(1.1,0){5}{%
\psline[linecolor=Speisewasser,linewidth=1.0pt]{c-c}(0,0)(0,-0.15)
\psline[linecolor=Speisewasser,linewidth=1.0pt]{c-c}(0,-0.3)(0,-0.45)
\psline[linecolor=Speisewasser,linewidth=1.0pt]{c-c}(0,-0.7)(0,-0.85)
\psline[linecolor=Speisewasser,linewidth=1.0pt]{c-c}(0,-1.2)(0,-1.35)
\psline[linecolor=Speisewasser,linewidth=1.0pt]{c-c}(0,-1.8)(0,-1.95)
\psline[linecolor=Speisewasser,linewidth=1.0pt]{c-c}(0,-2.5)(0,-2.65)
}
\multirput{-10}(1.8,3.5)(1.1,0){5}{%
\psline[linecolor=Speisewasser,linewidth=1.0pt]{c-c}(0,0)(0,-0.15)
\psline[linecolor=Speisewasser,linewidth=1.0pt]{c-c}(0,-0.3)(0,-0.45)
\psline[linecolor=Speisewasser,linewidth=1.0pt]{c-c}(0,-0.7)(0,-0.85)
\psline[linecolor=Speisewasser,linewidth=1.0pt]{c-c}(0,-1.2)(0,-1.35)
\psline[linecolor=Speisewasser,linewidth=1.0pt]{c-c}(0,-1.8)(0,-1.95)
\psline[linecolor=Speisewasser,linewidth=1.0pt]{c-c}(0,-2.5)(0,-2.65)
}
%----------------------------------------------------------------------------------------------
%--------------- Pumpe +  Wasserleitungen vom K\"{u}hlturm zum Fluss und Kondensator---------------
%----------------------------------------------------------------------------------------------
\psline[style=WasserleitungKT](0.098,0.25)(-1.5,0.25)(-1.5,-1)
\rput(-3.5,2.075){\pumpeA{Kuehlturmwasser}}
%\psline[style=WasserleitungKT](-4,2.25)(-4.5,2.25)
\psline[style=WasserleitungKT](-0.8,0.292)(-0.8,1.9)(-3,1.9)
\pcline[linewidth=0.5pt,arrowscale=1,arrowinset=0.04,arrowsize=0.08,arrowlength=1.4,linecolor=white]{->}(-0.8,0.6)(-0.8,1.1)
\psline[style=WasserleitungKT](-2.5,1.856)(-2.5,-1)
\psline[style=WasserleitungSP](-2,3.456)(-2,-1)
%----------------------------------------------------------------------------------------------
%--------------------------- Fluss ------------------------------------------------------------
%----------------------------------------------------------------------------------------------
\pscustom[fillstyle=solid,fillcolor=blue!30,opacity=0.35,linestyle=none,linearc=0.3]{%
\psline(-4,-0.7)(-3.75,-1.4)(-3.5,-1.8)(0,-1.8)(0.25,-1.4)(0.5,-0.7)
\closepath%
}
%----------------------------------------------------------------------------------------------
\rput(-3.75,-0.8){\psplot[plotpoints=1000,linewidth=1.2pt,linecolor=Kuehlturmwasser]{0}{4}{x 800 mul sin 0.05 mul}}%
\rput(-3.75,-1.2){\psplot[plotpoints=1000,linewidth=1.2pt,linecolor=Kuehlturmwasser]{0}{4}{x 800 mul sin 0.05 mul}}%
\rput(-3.8,-1.0){\psplot[plotpoints=1000,linewidth=1.2pt,linecolor=Kuehlturmwasser]{0}{1.5}{x 800 mul sin 0.05 mul}}%
\rput(-1,-1.0){\psplot[plotpoints=1000,linewidth=1.2pt,linecolor=Kuehlturmwasser]{0}{1.2}{x 800 mul sin 0.05 mul}}%
\rput(-2,-1.5){\scriptsize Fluss}
}
%----------------------------------------------------------------------------------------------
%----------------------------------------------------------------------------------------------
\def\Maschinenhaus{%
%----------------------------------------------------------------------------------------------
%---------------------------- Maschinenhaus-Geb\"{a}ude ----------------------------------
%----------------------------------------------------------------------------------------------
\psframe[linecolor=gray!40,linewidth=2pt,fillstyle=solid,fillcolor=gray!10](-1.5,1)(-12.5,7.8)

%-------------------------------------------------------------------------------------------
%--------------- Kondensator ---------------------------------------------------------------
%-------------------------------------------------------------------------------------------
\pnode(-5,3){CML}% Mittelpunkt des linken Kreisbogens
\pnode(-9,3){CMR}% Mittelpunkt des rechten Kreisbogens
\rput(CML){\pnode(5;167){C0}}% Oberer linker Punkt des Kondensators
\rput(CMR){\pnode(5;13){C1}}% Oberer rechter Punkt des Kondensators
\pscustom[dimen=inner,linewidth=1.2pt,fillstyle=gradient,gradmidpoint=0,gradangle=0,gradbegin=Speisewasser,gradend=DampfND]{%
\psarcn[liftpen=0](CML){5}{193}{167}%
\psarcn[liftpen=0](CMR){5}{13}{-13}%
\closepath%
}
\psline[style=LeitungND](-8.5,7)(-8.5,7.3)(-7.5,7.3)(-7.5,7)%  Dampfleitung vom Hochdruckteil der Turbine zum Niederdruckteil
\pcline[linewidth=0.5pt,arrowscale=1,arrowinset=0.04,arrowsize=0.08,arrowlength=1.4,linecolor=white]{->}(-8.25,7.3)(-7.75,7.3)
\psline[style=LeitungHD](-9.5,7)(-9.5,7.5)(-16.5,7.5)(-16.5,6.99)%  Dampfleitung vom Dampferzeuger zum Hochdruckteil der Turbine
\pcline[linewidth=0.5pt,arrowscale=1,arrowinset=0.04,arrowsize=0.08,arrowlength=1.4,linecolor=white]{->}(-12.5,7.5)(-12,7.5)
%----------------------Leitung durch Kondensator ----------------------------------------------------
\psline[style=WasserleitungKT](-3.9,2.25)(-9,2.25)(-9,2.67)(-5,2.67)(-5,3.083)(-9,3.083)(-9,3.5)(-3.9,3.5)
%
\psframe[linestyle=none,fillstyle=slope,slopeend=Speisewasser,slopebegin=Kuehlturmwasser](-8.5,3.542)(-4,3.458)
%\psframe[linestyle=none,fillstyle=gradient,gradmidpoint=0,gradangle=90,gradbegin=Speisewasser,gradend=Kuehlturmwasser](-4,3.458)(-8.5,3.542)
%
\pcline[linewidth=0.5pt,arrowscale=1,arrowinset=0.04,arrowsize=0.08,arrowlength=1.4,linecolor=white]{->}(-7.25,3.5)(-6.75,3.5)
\pcline[linewidth=0.5pt,arrowscale=1,arrowinset=0.04,arrowsize=0.08,arrowlength=1.4,linecolor=white]{<-}(-7.25,2.25)(-6.75,2.25)
%--------------------------------------------------------------------------------------
%--------------------- Leitung Abgang Kondensator zu Pumpe und Dampferzeuger ----------
%--------------------------------------------------------------------------------------
\rput(-10,1.5){\pumpeA{Speisewasser}}
\psline[style=WasserleitungSP](-7.5,1.9)(-7.5,1.325)(-9.5,1.325)%
\pcline[linewidth=0.5pt,arrowscale=1,arrowinset=0.04,arrowsize=0.08,arrowlength=1.4,linecolor=white]{->}(-7.5,2.1)(-7.5,1.6)
\psline[style=WasserleitungSP](-10.5,1.675)(-11.5,1.675)(-11.5,3)(-15.52,3)%
\pcline[linewidth=0.5pt,arrowscale=1,arrowinset=0.04,arrowsize=0.08,arrowlength=1.4,linecolor=white]{->}(-11.5,2)(-11.5,2.5)
\pcline[linewidth=0.5pt,arrowscale=1,arrowinset=0.04,arrowsize=0.08,arrowlength=1.4,linecolor=white]{->}(-12.75,3)(-13.25,3)
%--------------------------------------------------------------------------------------
%--------------------- Boden f\"{u}r Turbine und Generator ---------------------------
%--------------------------------------------------------------------------------------
\psframe[linestyle=none,framearc=.0,fillstyle=gradient,gradmidpoint=0.2,gradangle=0,gradend=gray!10,gradbegin=black!80](-11,5.7)(-3,6)%

%--------------------------------- Turbine -------------------------------------------------------
\psframe[linestyle=none,fillstyle=gradient,linecolor=black,gradmidpoint=0.3,%
gradangle=0,gradbegin=black!90,gradend=cyan!40](-10.8,6.4)(-3.1,6.6)%  Achse
\pspolygon[fillstyle=gradient,gradmidpoint=0.25,gradangle=0,gradend=cyan!15,gradbegin=black!60](-10.5,6)(-6.75,6)(-6.75,6.75)(-7,7)(-10.25,7)(-10.5,6.75)
\psline(-7,6)(-7,7)\psline(-10.25,6)(-10.25,7)
\psline[style=LeitungND](-7.5,5.69)(-7.5,4.1)%  Dampfleitung von Turbine zum Kondensator
\pcline[linewidth=0.5pt,arrowscale=1,arrowinset=0.04,arrowsize=0.08,arrowlength=1.4,linecolor=white]{->}(-7.5,5)(-7.5,4.5)

%--------------------------------- Generator ------------------------------------------------------
\psframe[linestyle=solid,fillstyle=gradient,linecolor=black,gradmidpoint=0.25,%
gradangle=0,gradbegin=black!60,gradend=cyan!15,framearc=0.2](-5.7,6.3)(-3.3,6.7)%
\psframe[linestyle=solid,fillstyle=gradient,linecolor=black,gradmidpoint=0.25,%
gradangle=0,gradbegin=black!60,gradend=cyan!15,framearc=0.2](-5.6,6.1)(-3.4,6.9)%
\psframe[linestyle=solid,fillstyle=gradient,linecolor=black,gradmidpoint=0.25,%
gradangle=0,gradbegin=black!60,gradend=cyan!15,framearc=0.2](-5.5,6.0)(-3.5,7.)%

%--------------------------------- Transformator ------------------------------------------------------
\psframe[linewidth=1.2pt,fillstyle=gradient,gradmidpoint=0,gradangle=0,gradend=orange!50,gradbegin=orange!40](-3.1,8.0)(-0.9,10.5)%
\pspolygon[linewidth=1.0pt,fillstyle=gradient,gradmidpoint=0,gradangle=0,gradend=brown!60!black!60,
gradbegin=brown!70!black!90](-3.2,10.5)(-3.1,10.9)(-0.9,10.9)(-0.8,10.5)%
\psframe[linestyle=none,framearc=.0,fillstyle=gradient,gradmidpoint=0.2,gradangle=0,gradend=gray!10,gradbegin=black!80](-3,8.0)(-1,8.2)%
\psframe[linestyle=none,framearc=.0,fillstyle=gradient,gradmidpoint=0,gradangle=0,gradend=black!70,gradbegin=black!40](-2.95,8.4)(-1.05,9.6)%
\psframe[framearc=0,fillstyle=gradient,gradmidpoint=0,gradangle=0,gradend=brown!80!black!90,gradbegin=brown!60](-2.8,8.2)(-1.2,9.8)%
\psline[linewidth=2.5pt,linearc=0.2]{c-}(-2.9,8.8)(-3.8,8.8)(-3.8,7)%  untere Stromleitung von Trafo zum Generator
\psline[linewidth=2.5pt,linearc=0.2]{c-}(-2.9,9)(-4,9)(-4,7)%  mittlere Stromleitung von Trafo zum Generator
\psline[linewidth=2.5pt,linearc=0.2]{c-}(-2.9,9.2)(-4.2,9.2)(-4.2,7)%  obere Stromleitung von Trafo zum Generator
%------------------------- Anschl\"{u}sse Abgang Transformator -----------------------------------------
\pnode(-2.4,11.13){ATL}
\pnode(-2.0,11.13){ATM}
\pnode(-1.6,11.13){ATR}

\def\TrafoAn{%
\psline[linewidth=1.8pt]{-c}(0,0.05)(0,0.23)
\psframe[fillstyle=solid,fillcolor=white,linewidth=0.6pt,framearc=0.15](-0.08,0)(0.08,0.2)
\psline[linewidth=1.8pt]{-c}(0,1.07)(0,1.33)
\psline[linewidth=1.2pt](0,0.2)(0,1.1)
\psframe[fillstyle=solid,fillcolor=white,linewidth=0.6pt,framearc=0.15](-0.08,1.1)(0.08,1.3)
}
\rput(-2.4,9.8){\TrafoAn}
\rput(-2.0,9.8){\TrafoAn}
\rput(-1.6,9.8){\TrafoAn}
%
}%--------  Ende Maschinenhaus-Definition
%
%--------------------------------- Strom-Mast -------------------------------------------------------
\def\Fuss{%
\pspolygon[linewidth=1.2pt,shadow=true,shadowsize=1.5pt,shadowcolor=gray!70](-1,0)(-0.65,0.65)(0.65,0.65)(1,0)(0,0.65)
%\pspolygon[linewidth=1.2pt](-1,0)(-0.65,0.65)(0.65,0.65)(1,9)(0,0.65)
%\pspolygon[linewidth=1.2pt](-1,0)(-0.65,0.65)(0.65,0.65)(1,9)(0,0.65)
\psline(-0.8075,0.3575)(-0.45,0.3575)(-0.65,0.65)
\psline(0.8075,0.3575)(0.45,0.3575)(0.65,0.65)
\psarc[fillstyle=solid,fillcolor=gray!80,linestyle=none](-1,-0.08){0.13}{0}{180}
\psarc[fillstyle=solid,fillcolor=gray!80,linestyle=none](1,-0.08){0.13}{0}{180}
\psline[linecolor=black!70,linewidth=2pt](-1.2,-0.1)(1.2,-0.1)
}
\newcommand{\Mittelsegment}{%
\pnode(-0.65,0){St1}
\pnode(-0.5,0.5){St2}
\pnode(0.65,0){St3}
\pnode(0.5,0.5){St4}
\pscustom[linejoin=1,linewidth=1.2pt]{%
\psline(St1)(St2)(St3)(St4)(St1)(St3)
\openshadow[shadowsize=1.5pt,shadowcolor=gray!70]
\closepath%
}
}
%-----------------------------------------
\def\Strommast{%
\Fuss
\rput(0,0.65){\Mittelsegment}
\rput(0,1.18){\psset{xunit=0.76}\Mittelsegment}
\rput(0,1.7){\psset{xunit=0.5776}\Mittelsegment}
\rput(0,2.22){\psset{xunit=0.438976}\Mittelsegment}
\rput(0,2.74){\psset{xunit=0.33362176}\Mittelsegment}
\rput(0,3.26){\psset{xunit=0.2535525}\Mittelsegment}
%
\psline[linewidth=1.2pt,linejoin=1](-0.1273,3.76)(0,4.1)(0.1273,3.76)(-0.1273,3.76)
%-------------- linker Arm -----------------------------------------------
\psline[linewidth=1.2pt,linejoin=1](-0.16,3.26)(-2.2,2.9)(-0.18,2.9)
\psline[linewidth=1.2pt,linejoin=1](-1.4,2.9)(-1.15,3.07)(-1,2.9)(-0.7,3.15)(-0.5,2.9)(-0.16,3.26)
%-------------- rechter Arm -----------------------------------------------
\psline[linewidth=1.2pt,linejoin=1](0.16,3.26)(2.2,2.9)(0.18,2.9)
\psline[linewidth=1.2pt,linejoin=1](1.4,2.9)(1.15,3.07)(1,2.9)(0.7,3.15)(0.5,2.9)(0.16,3.26)
\rput(1.0,2.77){%
\psline[linewidth=1.8pt]{c-}(0,-0.13)(0,0.13)
\psframe[fillstyle=solid,fillcolor=white,linewidth=0.6pt,framearc=0.18](-0.075,-0.1)(0.075,0.1)
\pnode(0,-0.13){SMR1}
}
%
\rput(1.9,2.77){%
\psline[linewidth=1.8pt]{c-}(0,-0.13)(0,0.13)
\psframe[fillstyle=solid,fillcolor=white,linewidth=0.6pt,framearc=0.18](-0.075,-0.1)(0.075,0.1)
\pnode(0,-0.13){SMR2}
}
%
\rput(-1.0,2.77){%
\psline[linewidth=1.8pt]{c-}(0,-0.13)(0,0.13)
\psframe[fillstyle=solid,fillcolor=white,linewidth=0.6pt,framearc=0.18](-0.075,-0.1)(0.075,0.1)
\pnode(0,-0.13){SML1}
}
%
\rput(-1.9,2.77){%
\psline[linewidth=1.8pt]{c-}(0,-0.13)(0,0.13)
\psframe[fillstyle=solid,fillcolor=white,linewidth=0.6pt,framearc=0.18](-0.075,-0.1)(0.075,0.1)
\pnode(0,-0.13){SML2}
}
}
%----------------------------------------------------------------------------------------------
%----------------------------------------------------------------------------------------------
\def\Reaktorgebaeude{%
%----------------------------------------------------------------------------------------------
%------------------------------- Abluftkamin -------------------------------------
%----------------------------------------------------------------------------------------------
\pspolygon[linecolor=gray!40,linewidth=2pt,fillstyle=solid,fillcolor=gray!10](-14,-2)(-14.4,14)(-15.6,14)(-16,-2)
%----------------------------------------------------------------------------------------------
%----------------------------- \"{A}u{\ss}eres Geb\"{a}ude -----------------------------------
%----------------------------------------------------------------------------------------------
\pscustom[linecolor=gray!40,linewidth=4pt,fillstyle=solid,fillcolor=gray!10]{%
\psarcn[liftpen=0](-19,4){5.5}{180}{0}%
\psline(-13.5,-2)(-24.5,-2)
\closepath%
}
%----------------------------------------------------------------------------------------------
%----------------------------- Inneres Geb\"{a}ude -----------------------------------
%----------------------------------------------------------------------------------------------
\pscircle[linecolor=gray!80,linewidth=2pt,fillstyle=solid,fillcolor=gray!30](-19,4){5}

%----------------------------------------------------------------------------------------------
%----------------------------- Dampferzeuger -----------------------------------
%----------------------------------------------------------------------------------------------
\rput(-16.5,0){%
\pscustom[dimen=inner,linestyle=none,fillstyle=gradient,gradmidpoint=0,gradangle=0,gradbegin=Speisewasser,gradend=DampfHD]{%
\psellipticarc(0,6.5)(1.25,0.5){0}{180}
\psline(-1.25,5)(1.25,5)
\closepath%
}
\pscustom[dimen=inner,linestyle=none,fillstyle=solid,fillcolor=Speisewasser]{%
\psline(-1.25,5)(-1,4.5)(-1,2.5)(1,2.5)(1,4.5)(1.25,5)
\closepath%
}
\pscustom[dimen=inner,linewidth=1.2pt,fillstyle=none]{%
\psellipticarc(0,6.5)(1.25,0.5){0}{180}
\psline(-1.25,5)(-1,4.5)(-1,2.5)(1,2.5)(1,4.5)(1.25,5)
\closepath%
}
\pscustom[dimen=inner,linewidth=1.2pt,fillstyle=solid,fillcolor=DampfHD]{%
\psline(-1,2.5)(-1,2)
\psarc(0,2){1}{180}{270}
\psline(0,1)(0,2.5)
\closepath%
}
\pscustom[dimen=inner,linewidth=1.2pt,fillstyle=solid,fillcolor=DampfMD]{%
\psline(0,2.5)(1,2.5)(1,2)
\psarcn(0,2){1}{0}{-90}
\closepath%
}
\psline[style=LeitungND,linearc=0.3](-0.3,2.5)(-0.3,4.3)(0.3,4.3)(0.3,2.5)
\psline[style=LeitungND,linearc=0.6](-0.6,2.5)(-0.6,4.6)(0.6,4.6)(0.6,2.5)
%------------------- Farbverlauf in den Rohren durch den Dampferzeuger -----------------------------
\psframe[linestyle=none,fillstyle=gradient,gradmidpoint=0,gradangle=0,gradbegin=DampfHD,gradend=DampfND](-0.338,2.45)(-0.257,3.9)
\psframe[linestyle=none,fillstyle=gradient,gradmidpoint=0,gradangle=0,gradbegin=DampfHD,gradend=DampfND](-.638,2.45)(-.557,3.9)
\psframe[linestyle=none,fillstyle=gradient,gradmidpoint=0,gradangle=0,gradbegin=DampfMD,gradend=DampfND](0.262,2.45)(0.343,3.9)
\psframe[linestyle=none,fillstyle=gradient,gradmidpoint=0,gradangle=0,gradbegin=DampfMD,gradend=DampfND](0.562,2.45)(0.643,3.9)
%------------------- Aufsteigende Dampfblasen im Dampferzeuger ----------------------------------
\psRandom[dotsize=3pt,dotstyle=o,randomPoints=30](-0.6,4)(0.6,6.2){%
\psframe[linestyle=none](-0.8,3.9)(0.8,6.3)
}
%\psRandom[dotsize=3pt,dotstyle=o,randomPoints=20](-0.85,5)(0.85,6.3){%
%\psframe[linestyle=none](-0.9,4.9)(0.9,6.6)}
%\psRandom[dotsize=3pt,dotstyle=o,randomPoints=20](-0.5,3.9)(0.5,5.5){%
%\psframe[linestyle=none](-0.7,3.6)(0.7,5.6)}
}
%----------------------------------------------------------------------------------------------
%--------------------------------- Reaktordruckgef\"{a}{\ss} (Kern) -----------------------------------
%----------------------------------------------------------------------------------------------
\rput(-21,-1.25){%
%-------- Randlinie der oberen Einm\"{u}ndung f\"{u}r die Steuerst\"{a}be - F\"{u}llung weiter unten ----------
\psframe[linestyle=solid,linewidth=1.2pt,framearc=0.9](-0.5,6.0)(-0.3,8.75)
\psframe[linestyle=solid,linewidth=1.2pt,framearc=0.9](-0.1,6.0)(0.1,8.75)
\psframe[linestyle=solid,linewidth=1.2pt,framearc=0.9](0.3,6.0)(0.5,8.75)
%----------------------------------------------------------------------------------------------
%--------- F\"{u}llung rechteckiges Mittelteil mit Farb\"{u}bergang des Reaktordruckgef\"{a}{\ss}es -----------
\psframe[dimen=inner,linestyle=none,fillstyle=gradient,gradmidpoint=0,gradangle=0,gradbegin=DampfMD,gradend=DampfHD](-1.25,4)(1.25,6)
%----------------------------------------------------------------------------------------------
%------------------ F\"{u}llung oberer ellipsenf\"{o}rmiger Teil des Druckgef\"{a}{\ss}es ---------------------
\pscustom[dimen=inner,linestyle=none,fillstyle=solid,fillcolor=DampfHD]{%
\psellipticarc(0,6.5)(1.25,0.5){0}{180}
\psline(-1.25,6)(1.25,6)
\closepath%
}
%----------------------------------------------------------------------------------------------
%------------------ F\"{u}llung unterer Teil des Druckgef\"{a}{\ss}es -------------------------------------
\pscustom[dimen=inner,linestyle=none,fillstyle=solid,fillcolor=DampfMD]{%
\psarc(0,4){1.25}{180}{0}
\psline(1.25,4)(-1.25,4)
\closepath%
}
%----------------------------------------------------------------------------------------------
%------------------ Zeichnen der Randlinie des Druckgef\"{a}{\ss}es -----------------------------------
\pscustom[dimen=inner,linewidth=1.2pt]{%
\psellipticarc(0,6.5)(1.25,0.5){0}{180}
\psarc(0,4){1.25}{180}{0}
\closepath%
}
%----------------------------------------------------------------------------------------------
%------------- F\"{u}llung der oberen Einm\"{u}ndung f\"{u}r die Steuerst\"{a}be - Randlinie weiter oben ----------
\psframe[dimen=outer,linestyle=none,framearc=0.9,fillstyle=solid,fillcolor=DampfHD](-0.5,6.0)(-0.3,8.75)
\psframe[dimen=outer,linestyle=none,framearc=0.9,fillstyle=solid,fillcolor=DampfHD](-0.1,6.0)(0.1,8.75)
\psframe[dimen=outer,linestyle=none,framearc=0.9,fillstyle=solid,fillcolor=DampfHD](0.3,6.0)(0.5,8.75)
%----------------------------------------------------------------------------------------------
%------------------------ Brennst\"{a}be ----------------------------------------------------------
\psline[linewidth=6pt,linecolor=green!40!black!90]{-c}(0.2,3.5)(0.2,4.5)
\psline[linewidth=6pt,linecolor=green!40!black!90]{-c}(0.6,3.5)(0.6,4.5)
\psline[linewidth=6pt,linecolor=green!40!black!90]{-c}(-0.2,3.5)(-0.2,4.5)
\psline[linewidth=6pt,linecolor=green!40!black!90]{-c}(-0.6,3.5)(-0.6,4.5)
%----------------------------------------------------------------------------------------------
%-------------------------- Steuerst\"{a}behalter -------------------------------------------------
\psline[linewidth=2.5pt]{c-c}(-0.5,7.15)(-0.5,7.35)
\psline[linewidth=2.5pt]{c-c}(-0.3,7.15)(-0.3,7.35)
\psline[linewidth=2.5pt]{c-c}(-0.1,7.15)(-0.1,7.35)
\psline[linewidth=2.5pt]{c-c}(0.1,7.15)(0.1,7.35)
\psline[linewidth=2.5pt]{c-c}(0.3,7.15)(0.3,7.35)
\psline[linewidth=2.5pt]{c-c}(0.5,7.15)(0.5,7.35)
%----------------------------------------------------------------------------------------------
%-------------------------- Steuerst\"{a}be -------------------------------------------------------
\rput(-0.4,7.5){%
\psline[linewidth=1.2pt]{c-c}(0,0.3)(0,-2.5)
\psline[linewidth=2.5pt]{c-c}(0,-2.5)(0,-3.6)
}
\rput(0,7.7){%
\psline[linewidth=1.2pt]{c-c}(0,0.3)(0,-2.5)
\psline[linewidth=2.5pt]{c-c}(0,-2.5)(0,-3.6)
}
\rput(0.4,8.3){%
\psline[linewidth=1.2pt]{c-c}(0,0.3)(0,-2.5)
\psline[linewidth=2.5pt]{c-c}(0,-2.5)(0,-3.6)
}
%----------------------------------------------------------------------------------------------
%------------------- Str\"{o}mungs-Umlenkung im Druckgef\"{a}{\ss} links + Pfeil --------------------------
\psline[linewidth=1.2pt](-1.25,4.5)(-0.9,4.5)(-0.9,3.5)
%----------------------------------------------------------------------------------------------
\rput{-40}(-1.03,3.6){%
\pscurve[linewidth=0.5pt,arrowscale=1,arrowinset=0.04,arrowsize=0.08,arrowlength=1.4]{->}(0,0)(0.3,-0.1)(0.6,0)
}
%----------------------------------------------------------------------------------------------
}%   Ende des Setztens mit \rput des Reaktordruckgef\"{a}{\ss}es
%----------------------------------------------------------------------------------------------
%----------------------- Pumpe und Leitungen von Dampferzeuger zu Reaktorkern -----------------
%----------------------------------------------------------------------------------------------
%------------------- rechter Rohr-Anschluss des Dampferzeugers ----------------------------------
\psline[style=LeitungMD,linearc=0.3](-15.522,2)(-15.2,2)(-15.2,1.5)
\rput(-18,1){\pumpeA{DampfMD}}
\psline[style=LeitungMD,linearc=0.65](-15.2,1.5)(-15.2,0.825)(-17.6,0.825)
\psline[style=LeitungMD,linearc=0.2](-18.4,1.175)(-22.5,1.175)(-22.5,3)(-22.23,3)
%----------------------------------------------------------------------------------------------
\psline[style=LeitungHD,linearc=0.2](-19.75,5)(-18,5)(-18,2)(-17.48,2)
%----------------------------------------------------------------------------------------------
\pcline[linewidth=0.5pt,arrowscale=1,arrowinset=0.04,arrowsize=0.08,arrowlength=1.4]{->}(-16.5,0.825)(-17.0,0.825)
\pcline[linewidth=0.5pt,arrowscale=1,arrowinset=0.04,arrowsize=0.08,arrowlength=1.4]{->}(-19.5,1.175)(-20.0,1.175)
\pcline[linewidth=0.5pt,arrowscale=1,arrowinset=0.04,arrowsize=0.08,arrowlength=1.4]{->}(-22.5,1.75)(-22.5,2.25)
\pcline[linewidth=0.5pt,arrowscale=1,arrowinset=0.04,arrowsize=0.08,arrowlength=1.4,linecolor=white]{->}(-19.25,5)(-18.75,5)
\pcline[linewidth=0.5pt,arrowscale=1,arrowinset=0.04,arrowsize=0.08,arrowlength=1.4,linecolor=white]{->}(-18,4)(-18,3.5)
%----------------------------------------------------------------------------------------------
\rput{40}(-17.4,2){%
\pscurve[linewidth=0.5pt,arrowscale=1,arrowinset=0.04,arrowsize=0.08,arrowlength=1.4,linecolor=white]{->}(0,0)(0.3,-0.1)(0.6,0)
}
%----------------------------------------------------------------------------------------------
\rput{-40}(-16.1,2.4){%
\pscurve[linewidth=0.5pt,arrowscale=1,arrowinset=0.04,arrowsize=0.08,arrowlength=1.4,linecolor=black]{->}(0,0)(0.3,-0.1)(0.6,0)
}
%----------------------------------------------------------------------------------------------
%-------------------------------- Druckhalter -------------------------------------------------
%----------------------------------------------------------------------------------------------
\psframe[dimen=outer,linestyle=none,framearc=0.9,fillstyle=solid,fillcolor=DampfHD](-19.3,5.5)(-18.7,7)
\psframe[dimen=outer,linestyle=none,framearc=0.9,fillstyle=solid,fillcolor=white](-19.3,6.5)(-18.7,7)
\psframe[dimen=outer,linestyle=none,framearc=0,fillstyle=solid,fillcolor=white](-19.3,6.4)(-18.7,6.7)
\psframe[linewidth=1.2pt,framearc=0.9](-19.3,5.5)(-18.7,7)
\psline[style=LeitungHD](-19,5.52)(-19,5.0443)
}
%----------------------------------------------------------------------------------------------

\psscalebox{0.5}{%
\rput(13,-4){%
\begin{pspicture}[showgrid=false](-25,-2)(8,15)
\rput(-8.5,15.5){\Huge Kernkraftwerk mit Druckwasserreaktor}
%----------------------------------------------------------------------------
\Reaktorgebaeude
\Maschinenhaus
\Kuehlturm
%----------------------------------------------------------------------------
%-------- Strommast und Verbindung der Leitung zum Trafo --------------------
\rput(-6,10){\Strommast}
\psbezier(ATL)(-4.8,10.5)(-7,11)(SML2)
\psbezier(ATM)(-3.5,11.2)(-4.5,11.5)(SMR1)
\psbezier(ATR)(-2.5,11.2)(-3.5,11.5)(SMR2)
%----------------------------------------------------------------------------
%
%----------------------- Beschriftungen -------------------------------------
%
\rput(-7,8.1){Maschinenhaus}
\rput(-19,10){Reaktorgeb\"{a}ude}
\rput[t](-23.0,5.5){\shortstack[l]{\scriptsize Reaktor-\strut\\[-3pt]
\scriptsize druck-\strut\\[-3pt]\scriptsize gef\"{a}{\ss}\strut}}
\rput[tl](-19.45,2.65){\shortstack[l]{\scriptsize Brenn-\strut\\[-3pt]
\scriptsize elemente\strut}}
\psline(-20.4,2.5)(-19.65,2.5)
\rput(-21.0,7.8){\scriptsize Steuerst\"{a}be}
\rput(-19,7.2){\scriptsize Druckhalter}
\rput(-17.0,7.7){\scriptsize Dampferzeuger}
\rput(-17.0,0.5){\scriptsize Hauptk\"{u}hlkreis}
\rput(-13.3,7.8){\scriptsize Wasserdampf}
\rput(-13,3.3){\scriptsize Speisewasser}
\rput(-8.75,6.5){\scriptsize Turbine}
\rput(-4.5,6.5){\scriptsize Generator}
\rput(-2,9){\scriptsize Trafo}
\rput(-5,13.5){\scriptsize Strommast}
\rput(-5.75,4.3){\scriptsize Kondensator}
\rput(4,-0.5){\scriptsize K\"{u}hlturm}
\rput[r](1.3,2.5){\scriptsize Spr\"{u}hwasser}
\uput{2pt}[110](0,1){\scriptsize Luft}
\rput(4,5.5){\scriptsize feuchte, warme Luft}
\uput{4pt}[90](-0.25,3.5){\scriptsize K\"{u}hlwasser}
\end{pspicture}
}}


\end{document}
