\documentclass[a4paper,12pt]{article}
\usepackage{ngerman} % neue deutsche Rechtschreibung
\usepackage[ngerman]{babel}
\usepackage{amsmath,amssymb}
\usepackage[light]{kpfonts}%  f\"{u}r Mathezeichen
\usepackage{libertine}% f\"{u}r rm und sf
\usepackage[T1]{fontenc}
\usepackage[dvipsnames]{color} %% Farben sind im Dokument LatexGraphik.pdf definiert
\usepackage{pstricks-add}


\begin{document}

\section*{Das Pascalsche Dreieck}

Benannt nach Blaise Pascal (19. Juni 1623 in Clermont-Ferrand; $\dagger$ 19. August 1662 in Paris)

\newcommand*{\kastB}[2][0.2]{\psframebox[linecolor=blue,framearc= #1, framesep=1pt, fillcolor=blue,%
fillstyle=solid]{\color{white}\textsf{#2}}}%
\newcommand*{\kastG}[2][0.2]{\psframebox[linecolor=yellow, framearc= #1, framesep=1pt,%
fillcolor=yellow, fillstyle=solid]{\color{blue}\textsf{#2}}}%
{\psset{arrows=->, nodesep=3pt, linewidth=1pt,rowsep=.4cm, colsep=0.75cm}%
\begin{pspicture}[shift=-1](-0.5,-0.5)(8,9)
\begin{psmatrix}[mnodesize=0pt]
&        &  &  &  &  &  &  & 1&  &  &  &  &  &  & &\\
&        &  &  &  &  &  & 1&  & 1&  &  &  &  &  & &\\
&        &  &  &  &  & 1&  & 2&  & \rnode{A}{\kastB{1}}&  &  &  &  & &\\
&        &  &  &  & 1&  & 3&  & \rnode{B}{\kastB{3}}&  & 1&  &  &  & &\\
&        &  &  & 1&  & 4&  & \rnode{C}{\kastB{6}}&  & 4&  & 1&  &  & &\\
&      &  & 1&  & 5&  & \rnode{D}{\kastB{10}}&  & 10&  & 5&  & 1&  & &\\
&     & 1&  & 6&  & \rnode{E}{\kastB{15}}&  & 20&  & 15&  & 6&  & 1& &\\
   & 1&  & 7&  & 21&  & \rnode{F}{\kastG{35}}&  & 35&  & 21&  & 7&  & 1 & \\
1 & & 8 & & 28 & & 56 & & 70 & & 56 & & 28 & & 8 & & 1 \\
{\psset{linecolor=green,arrowinset=0.1,nodesep=2pt,labelsep=2pt}
\ncLine{->}{A}{B}\nbput{\textcolor{green}{$\scriptscriptstyle +$}}%
\ncLine{->}{B}{C}\nbput{\textcolor{green}{$\scriptscriptstyle +$}}%
\ncLine{->}{C}{D}\nbput{\textcolor{green}{$\scriptscriptstyle +$}}%
\ncLine{->}{D}{E}\nbput{\textcolor{green}{$\scriptscriptstyle +$}}%
\ncLine{->}{E}{F}%
}
%\psline(A)(B)
\end{psmatrix}
\end{pspicture}
}%

Beweise mit Hilfe der vollst\"{a}ndigen Induktion die Gleichung
\begin{equation*}
    \binom{k}{k} + \binom{k+1}{k} + \dotsb + \binom{n}{k} = \binom{n+1}{k+1} \quad \text{f\"{u}r} \quad  n=k,
    k+1,k+2, \dotsc
\end{equation*}
und veranschauliche sie im Pascalschen Dreieck.

\subsection*{L\"{o}sung}

\quad 1. \  $n=k$: \qquad $\dbinom{k}{k} = \dbinom{k+1}{k+1} = 1 $ \qquad OK $\checkmark $\\
Es gilt nun schon
\begin{equation*}%\label{}
  \begin{split}
                    &\binom{k}{k} + \binom{k+1}{k} + \dotsb + \binom{n-1}{k} = \binom{n}{k+1}\\[0.3cm]
  \Rightarrow \quad &\binom{k}{k} + \binom{k+1}{k} + \dotsb + \binom{n-1}{k} + \binom{n}{k}
                  = \binom{n}{k+1} + \binom{n}{k}\\[0.3cm]
                  = &\frac{n!}{(k+1)!\cdot (n-k-1)!} +\frac{n!}{k!\cdot (n-k)!}
                  =\frac{n!\cdot (n-k) + n!\cdot (k+1)}{(k+1)!\cdot (n-k)!}\\[0.3cm]
                  = &\frac{n!\cdot[(n-k) + (k+1)]}{(k+1)!\cdot (n-k)!}
                  = \frac{n!\cdot[n+1]}{(k+1)!\cdot (n-k)!}\\[0.3cm]
                  = &\frac{(n+1)!}{(k+1)!\cdot (n+1-(k+1))!} = \binom{n+1}{k+1}
  \end{split}
\end{equation*}

%



\end{document}
