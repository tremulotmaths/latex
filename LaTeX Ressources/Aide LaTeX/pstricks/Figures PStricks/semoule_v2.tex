\documentclass{article}
\usepackage{pst-node}
\usepackage{pst-electricfield}
\makeatletter
\pst@addfams{pst-semoule}
%% [q1 x1 y1] [q2 x2 y2]
\define@key[psset]{pst-semoule}{QXY}{\def\psk@semouleQXY{#1}}
\psset[pst-semoule]{QXY=[1 -2 0][-1 2 0]}
\define@key[psset]{pst-semoule}{randomPoints}[2500]{\def\psk@semoulerandomPoints{#1}}
%% nombre points
\psset[pst-semoule]{randomPoints=2500}

\def\pssemoule{\pst@object{pssemoule}}
\def\pssemoule@i(#1)(#2){{% (xmin,ymin)(xmax,ymax)
  \use@par
  \pst@getcoor{#1}\pst@tempA % (xmin,ymin)
  \pst@getcoor{#2}\pst@tempB % (xmax,ymax)
  \begin@SpecialObj
  \addto@pscode{
  /semenceGazon {1 0.2 scale 0 0 5 0 360 arc closepath} def
  \pst@tempA\space /yMin exch \pst@number\psyunit div def
    /xMin exch \pst@number\psxunit div def
    \pst@tempB\space /yMax exch \pst@number\psyunit div def
    /xMax exch \pst@number\psxunit div def
    /dy yMax yMin sub def
    /dx xMax xMin sub def
% les distances au point consid�r�
    /Radius {xP i getX sub yP i getY sub Pyth} def
% tableau des charges et positions
    /QXY [\psk@semouleQXY] def
% extraction des donnees = qi, xi, yi, Ni
   /NoQ QXY length 1 sub def % nombre de charges -1
% le tableau des charges
   /Qcharges [ % les charges
   0 1 NoQ {/i exch def
	/qi QXY i get def
	 qi 0 get}
   for
     ] def
% le tableau des abscisses
    /xCoor [
     0 1 NoQ {/i exch def
	/qi QXY i get def
	 qi 1 get}
    for
    ] def
% le tableau des ordonn�es
    /yCoor [
     0 1 NoQ {/i exch def
	/qi QXY i get def
	 qi 2 get}
     for
    ] def
  /getX { xCoor exch get } def
  /getY { yCoor exch get } def
  /getQ { Qcharges exch get } def
  /getRandReal { rand 2147483647 div } def
     rrand srand
  \psk@semoulerandomPoints {
    /xP getRandReal dx mul xMin add def
    /yP getRandReal dy mul yMin add def
% le calcul du champ
  0 0
  0 1 NoQ { /i ED
    i getQ xP i getX sub mul Radius 3 exp Div add exch
    i getQ yP i getY sub mul Radius 3 exp Div add exch
  } for
  /Ex ED  /Ey ED
  /Angle Ey Ex atan def
  gsave
  xP \pst@number\psxunit mul yP \pst@number\psyunit mul translate
 Angle rotate
semenceGazon
\pst@usecolor\psfillcolor true
 fill
 grestore
 gsave
   xP \pst@number\psxunit mul yP \pst@number\psyunit mul translate
 Angle rotate
semenceGazon
 stroke
  grestore
  } repeat
}
  \end@SpecialObj
}}
\makeatother

\begin{document}
\begin{center}
\begin{pspicture*}[showgrid](-5,-5)(5,5)
\psframe*[linecolor=yellow!20](-5,-5)(5,5)
\psset[pst-electricfield]{N=30}
\pssemoule[fillstyle=solid,fillcolor=yellow!50,linecolor={[rgb]{0.5 0 0}},linewidth=0.02,QXY={[2 2 0][-2 -2 0]}](-5,-5)(5,5)
\psElectricfield[Q={[2 2 0][-2 -2 0]},linecolor=red]%
\end{pspicture*}
\end{center}

\begin{center}
\begin{pspicture*}[showgrid](-5,-5)(5,5)
\pssemoule[fillstyle=solid,fillcolor=yellow!50,linecolor={[rgb]{0.5 0 0}},linewidth=0.02,QXY={[2 2 0][2 -1 1.732][2 -1 -1.732]}](-5,-5)(5,5)
\psset[pst-electricfield]{N=10}
\psElectricfield[Q={[2 2 0][2 -1 1.732][2 -1 -1.732]},linecolor=red]%
\end{pspicture*}
\end{center}

\begin{center}
\begin{pspicture*}[showgrid](-5,-5)(5,5)
\psframe*[linecolor=yellow!20](-5,-5)(5,5)
\rput{90}(0,0){%
\pssemoule[linewidth=0.04,randomPoints=2500,QXY={[1 0 2][1 0 -2]}](-5,-5)(5,5)
\psElectricfield[Q={[1 0 2][1 0 -2]}, linecolor=red,radius=5pt,N=15,points=1000,Pas=0.025,posArrow=0.1]%
}
\end{pspicture*}
\end{center}
\end{document}
semences de gazon ou grains de riz
http://www.sciences.univ-nantes.fr/sites/genevieve_tulloue/Elec/Champs/champE.html
http://www.rpn.ch/lbc/spip/article.php3?id_article=34
http://www.rpn.ch/lbc/spip/article-album.php3?id_article=34&debut_image=3
http://www.rpn.ch/lbc/spip/article-album.php3?id_article=34&debut_image=4 