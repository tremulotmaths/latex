\documentclass{article}
\usepackage{pst-node}
\usepackage{pst-electricfield}
\makeatletter
\pst@addfams{pst-vectorE}
%% [q1 x1 y1] [q2 x2 y2]
\define@key[psset]{pst-vectorE}{QXY}{\def\psk@vectorQXY{#1}}
\psset[pst-vectorE]{QXY=[1 -2 0][-1 2 0]}
\define@key[psset]{pst-vectorE}{K}{\def\psk@vectorEK{#1 }}
%% coefficient de longueur
\psset[pst-vectorE]{K=3}
\def\psvectorE{\pst@object{psvectorE}}
\def\psvectorE@i(#1){{% (x,y)
  \use@par
% le point o� on calcule le champ
\pst@getcoor{#1}\pst@tempA%
\pnode(!
    \pst@tempA
    /yP exch \pst@number\psyunit div def
    /xP exch \pst@number\psxunit div def
    xP yP
    ){Point}
\pnode(!
    /getX { xCoor exch get } def
    /getY { yCoor exch get } def
    /getQ { Qcharges exch get } def
% les distances au point consid�r�
    /Radius {xP i getX sub yP i getY sub Pyth} def
% tableau des charges et positions
    /QXY [\psk@vectorQXY] def
% extraction des donnees = qi, xi, yi, Ni
        /NoQ QXY length 1 sub def % nombre de charges -1
% le tableau des charges
       /Qcharges [ % les charges
 0 1 NoQ {/i exch def
	/qi QXY i get def
	 qi 0 get}
 for
 ] def
% le tableau des abscisses
    /xCoor [
     0 1 NoQ {/i exch def
	/qi QXY i get def
	 qi 1 get}
    for
    ] def
% le tableau des ordonn�es
    /yCoor [
     0 1 NoQ {/i exch def
	/qi QXY i get def
	 qi 2 get}
     for
    ] def
% le calcul du champ
  0 0
  0 1 NoQ { /i ED
    i getQ xP i getX sub mul Radius 3 exp Div add exch
    i getQ yP i getY sub mul Radius 3 exp Div add exch
  } for
  /Ex ED  /Ey ED
  /K \psk@vectorEK def
         0 0){datas}
%\psdots[linecolor=red](!xA yA)(!xB yB)
%\psdot(!xP yP)
\psline{->}(!xP yP)(! xP Ex K mul add yP Ey K mul add)
}}
\makeatother
\SpecialCoor

\begin{document}
\begin{center}
\begin{pspicture*}[showgrid](-5,-5)(5,5)
\psset[pst-electricfield]{N=10}
\psElectricfield[Q={[2 2 0][2 -1 1.732][2 -1 -1.732]},linecolor=red]%
\pnode(0.6,-0.8){P}\psdot(P)
\psvectorE[linecolor=blue,QXY={[2 2 0][2 -1 1.732][2 -1 -1.732]}](P)
\pnode(0.45,2){P}\psdot(P)
\psvectorE[linecolor=blue,QXY={[2 2 0][2 -1 1.732][2 -1 -1.732]}](P)
\pnode(-2,-0.6){P}\psdot(P)
\psvectorE[linecolor=blue,QXY={[2 2 0][2 -1 1.732][2 -1 -1.732]}](P)
\end{pspicture*}
\end{center}

\begin{center}
\begin{pspicture*}[showgrid](-5,-5)(5,5)
\rput{90}(0,0){\psElectricfield[Q={[1 0 2][1 0 -2]}, linecolor=red,radius=5pt,N=15,points=1000,Pas=0.025,posArrow=0.1]%
                \pnode(-1.3,-0.75){P}\psdot(P)
                \psvectorE[linecolor=blue,QXY={[2 0 2][2 0 -2]}](P)}
\end{pspicture*}
\end{center}
\end{document}

/NArray { % holds the coordinates and on top of stack the showpoints boolean
  /showpoints ED
  counttomark 2 div dup cvi /n ED  	% n 2 div on stack
  n eq not { exch pop } if		% even numbers of points? delete one
  ] aload /Points ED
  showpoints not { Points aload pop } if
} def

/BOC { IC CC x2 y2 x1 y1 ArrowA CP 4 2 roll x y curveto } def
/NC { CC x1 y1 x2 y2 x y curveto } def
/EOC { x dx sub y dy sub 4 2 roll ArrowB 2 copy curveto } def

/OpenCurve {
  NArray n 3 lt
    { n { pop pop } repeat }
    { BOC /n n 3 sub def n { NC } repeat EOC } ifelse
} def

