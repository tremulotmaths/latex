\documentclass[12pt,oneside]{report}
%%%%%%%%%%%%%%%%%%%%%%%%%%%%%%%%%%%%%%%%%%%%%%%%%%%%%%%%%%%%%%%%%%%%%%%%%%%%%%%
\input{preambule_2016}
%%%%%%%%%%%%%%%%%%%%%%%%%%%%%%%%%%%%%%%%%%%%%%%%%%%%%%%%%%%%%%%%%%%%%%%%%%%%%%%

%entête classique

\fancypagestyle{garde_tete}{% 
%\fancyhead[C]{\small\textbf{\seconde} \hfill \small \textbf{Année 2015-2016}}
\renewcommand{\headrulewidth}{0cm}}

\newcommand{\tete}{
\thispagestyle{garde_tete}
\chapitre{Courbes et pstricks}
\noindent 
\vspace{-1em}
}

%%%%%%%%%%%%%%%%%%%%%%%%%%%%%%%%%%%%%%%%%%%%%%%%%%%%%%%%%%%%%%%%%%%%%%%%%%%%%%%

%%%%%%%%%%%%%%%%%%%%%%%%%%%%%%%%%%%%%%%%%%%%%%%%%%%%%%%%%%%%%%%%%%%%%%%%%%%%%%%

%%%%%%%%%%%%%%%%%%%%%%%
%% DEBUT DU DOCUMENT %%
%%%%%%%%%%%%%%%%%%%%%%%

\begin{document}
\selectlanguage{french}
\setlength\parindent{0mm}
\tete 		%entête classique

\renewcommand \footrulewidth{0.2pt}%
\renewcommand \headrulewidth{0pt}%
\pagestyle{fancy}
\fancyhf{}
\pieddepage{\seconde}{\thepage / \pageref{LastPage}}{Année 2016-2017}

%%%%%%%%%%%%%%%%%%%%%%%%%%%%%%%%%%%%%%%%%%%%%%%%%%%%%%%%%%%%
\begin{spacing}{1.2}
%%%%%%%%%%%%%%%%%%%%%%%%%%%%%%%%%%%%%%%%%%%%%%%%%%%%%%%%%%%%

Dans le préambule du document, il faudra appeler :

$\star$ pstricks-add, pour le tracé des courbes : \verb!\usepackage{pstricks-add}! ;

$\star$ pst-eucl, pour la gestion des intersections des courbes : \verb!\usepackage{pst-eucl}!.

\section{Repères}

\begin{center}
\begin{pspicture}(-2,-2)(5,5)
\psgrid[subgriddiv=0,gridlabels=5pt,gridwidth=0.5pt,griddots=5](-2,-2)(5,5)
\psaxes[labels=none]{->}(0,0)(-2,-2)(5,5)
\end{pspicture}
\end{center}

\begin{center}
\begin{pspicture}(-2,-2)(5,2)
\psset{trigLabels=true,labelFontSize=\small}
\psaxes{->}(0,0)(-2,-2)(5,2)
\end{pspicture}
\end{center}


\begin{center}
\begin{pspicture}(-7,-1.25)(7,1.25)
\psset{algebraic=true}
\psaxes[trigLabels=true,trigLabelBase=2,dx=\psPiH,xunit=\psPi]{->}(0,0)(-2.2,-1.25)(2.2,1.25)
\psplot[linecolor=red,linewidth=1.5pt]{-7}{7}{sin(x)}
\psplot[linecolor=blue,linewidth=1.5pt]{-7}{7}{cos(x)}
\psline[linewidth=1.5pt]{->}(0,1)(1.57,1)
\psline[linewidth=1.5pt,linecolor=green]{->}(-\psPi,-1)(\psPi,-1)
\end{pspicture}
\end{center}


\section{Courbes classiques}

\begin{center}
\begin{pspicture}(-2.5,-1)(2.5,5)
\psset{xunit=1 cm, algebraic=true}
\psaxes{->}(0,0)(-2.5,-1)(2.5,5)
\psplot[linecolor=red,linewidth=1.5pt]{-2}{2}{x^2}
\end{pspicture}
\end{center}

\begin{center}
\begin{pspicture}(-2.5,-1)(2.5,8)
\psset{xunit=1 cm,algebraic=true}
\def\f{x*x}
\def\g{\f+3}
\psaxes{->}(0,0)(-2.5,-1)(2.5,8)
\psplot[linecolor=red,linewidth=1.5pt]{-2}{2}{\f}
\psplot[linecolor=blue,linewidth=1.5pt]{-2}{2}{\g}
\end{pspicture}
\end{center}

\begin{center}
\begin{pspicture}(-3,-1)(3,6)
\psset{xunit=1 cm,algebraic=true}
\def\f{x*x+1}
\def\g{1/(\f)}
\psaxes{->}(0,0)(-3,-1)(3,6)
\psplot[linecolor=red,linewidth=1.5pt]{-2.5}{2.5}{\f}
\psplot[linecolor=blue,linewidth=1.5pt]{-2.5}{2.5}{\g}
\end{pspicture}
\end{center}

\section{Courbes paramétrées}

\begin{center}
\begin{pspicture}(-1,-2)(2,2)
\psset{xunit=1.5cm,yunit=1.5cm}
\psaxes(0,0)(-1,-2)(2,2)
\parametricplot[linecolor=red,linewidth=0.7pt]{0}{360}{
%x(t)
t cos 1 add t cos mul %retour à la ligne
%y(t)
t cos 1 add t sin mul
}
\end{pspicture}
\end{center}

Si on utilise l'instruction \verb!algebraic=true! et la commande 

\verb!\parametricplot[par]{tmin}{tmax}{x(t)|y(t)}!, la variable $t$ sera alors exprimée en radians. L'exemple ci-dessous en est une illustration.

\begin{center}
\begin{pspicture}(-1,-2)(2,2)
\psset{xunit=1.5cm,yunit=1.5cm,algebraic=true}
\psaxes(0,0)(-1,-2)(2,2)
\parametricplot[linecolor=red,linewidth=0.7pt]{0}{6.28}{(cos(t))^3|(sin(t))^3}
\end{pspicture}
\end{center}

\section{Aires}

Il est possible de représenter l'aire comprise entre la courbe d'une fonction, l'axe des abscisses et les deux droites d'équations $x=a$ et $x=b$.

Pour cela, il s'agit d'utiliser la commande

\verb!\pscustom[par]{segment(a,0)(a,f(a)) courbe segment(b,f(b))(b,0)}!

On fera attention au dernier segment tracé puisque la première extrémité du segment part de la courbe.


\begin{center}
\psset{xunit=1cm, yunit=0.5cm}
\begin{pspicture}(-2,-6)(2,10)
\psaxes(0,0)(-2,-6)(2,10)
\def\f{2 x 3 exp sub}
\psplot{-2}{2}{\f}
\pscustom[fillstyle=hlines,linestyle=solid,linewidth=0.5pt]
{
\psline(-1,0)(-1,3) % ligne verticale: (a,0)(a,f(a))
\psplot{-1}{1}{\f} % courbe de f sur [a;b]
\psline(1,1)(1,0) % ligne verticale: (b,f(b))(b,0)
}
\end{pspicture}
\end{center}

On pourra aussi représenter l'aire comprise entre deux courbes. Pour cela, il faudra utiliser la commande \verb!\pscustom[par]{courbe du bas courbe du haut}!.

\begin{center}
\psset{xunit=1cm, yunit=1cm,algebraic=true}
\begin{pspicture}(-3,-1)(4,7)
\psaxes(0,0)(-3,-1)(4,7)
\def\f{x*x-2*x+2}
\def\g{-x*x+6}
\psplot{-1.5}{3}{\f}
\psplot{-2}{2.2}{\g}
\pscustom[fillstyle=hlines,linestyle=solid,linecolor=white,linewidth=0.0pt]
{% linecolor=white,linewidth=0.0pt permet de ne pas avoir de trait en diagonales 
% reliant les deux  points d'intersections
\psplot{-1}{2}{\f} % courbe du bas
\psplot{-1}{2}{\g} % courbe du haut
}
\end{pspicture}
\end{center}


\section{Intersection de deux courbes}

Pour obtenir les points d'intersections de deux courbes, il est nécessaire de déclarer dans le préambule du document, le package pst-eucl par le code \verb!\usepackage{pst-eucl}!.

C'est la seule partie où la \textbf{notation RPN} est nécessaire.

De plus, il est possible que la recherche de l'intersection ne puisse aboutir puisque celle-ci utilise l'algorithme de NEWTON.

Pour obtenir l'intersection entre une courbe et une droite (AB), on utilisera le code 

\verb!\pstInterFL[par]{fonction}{A}{B}{abscisse}{nom_du_point}!. 

L'abscisse donnée n'est pas forcément l'abscisse précise du point d'intersection mais juste un \og positionnement\fg{}.

\begin{center}
\begin{pspicture}(-2.5,-5)(2.5,8)
\psset{xunit=1 cm}
\def\f{x 3 exp 3 add}
\psaxes{->}(0,0)(-2.5,-5)(2.5,8)
\psplot[linecolor=blue,linewidth=1.5pt]{-2}{2}{\f}
\pstGeonode[PointSymbol=none,PointName=none](0,2){U}
\pstGeonode[PointSymbol=none,PointName=none](-1,0){V}
\pstLineAB[nodesep=-3]{U}{V}
\pstInterFL[PointSymbol=pentagon,PosAngle=170]{\f}{U}{V}{0}{M}
\pstInterFL[PointSymbol=pentagon,PosAngle=170]{\f}{U}{V}{-1.5}{N}
\pstInterFL[PointSymbol=pentagon,PosAngle=170]{\f}{U}{V}{2}{P}
\end{pspicture}
\end{center}

Pour obtenir l'intersection entre deux courbes, on utilisera le code 

\verb!\pstInterFF[par]{fonction_1}{fonction_2}{abscisse}{nom_du_point}!. 

L'abscisse donnée n'est pas forcément l'abscisse précise du point d'intersection mais juste un \og positionnement \fg{}.

\begin{center}
\begin{pspicture}(-2.5,-5)(2.5,8)
\psaxes{->}(0,0)(-2.5,-5)(2.5,8)
\psset{xunit=1 cm}
\def\f{x 2 exp}
\def\g{3 x 1 add 2 exp sub}
\psplot[linecolor=blue,linewidth=1.5pt]{-2}{2}{\f}
\psplot[linecolor=red,linewidth=1.5pt]{-2}{2}{\g}
\pstInterFF[PointSymbol=pentagon,PosAngle=170]{\f}{\g}{-1.5}{M}
\pstInterFF[PointSymbol=x]{\f}{\g}{1}{N}
\end{pspicture}
\end{center}


\section{Tableau de relation entre la notation RPN et la notation classique}

\begin{center}
\begin{tabular}{|*{3}{M{3cm}|}}
\hline
Nom&Syntaxe&notation classique\\
\hline
add&x y add&$x+y$\\
\hline
sub&x y sub&$x−y$\\
\hline
mul&x y mul&$x\times y$\\
\hline
exp&x y exp&$x^y$\\
\hline
div&x y div&$\dfrac{x}{y}$\\
\hline
neg&x neg&$−x$\\
\hline
sqrt&x sqrt&$\sqrt{x}$\\
\hline
abs&x abs&\verb!\abs(x)!\\
\hline
cos&x cos&$\cos(x)$\\
\hline
sin&x sin&$\sin(x)$\\
\hline
atan&x atan&\verb!\atan(x)!\\
\hline
ln&x ln&$\ln(x)$\\
\hline
\end{tabular}
\end{center}

Pour la fonction exponentielle, utiliser le nombre e en le remplaçant par 2.71828.

\newpage
\section{Autres exemples}

\begin{center}
\psclip{
\psframe[linestyle=none](-5,-7)(5,7)
}
\multido{\i=0+2}{36}{%
    \pstVerb{/kA \i\space def
             /A1 kA 5 mul def
             /A2 A1 5 add def}%
 \pswedge*(0,0){9}{!A1}{!A2}}%
\endpsclip
\end{center}


%%%%%%%%%%%%%%%%%%%%%%%%%%%%%%%%%%%%%%%%%%%%%%%%%%%%%%%%%%%%
\end{spacing}
%%%%%%%%%%%%%%%%%%%%%%%%%%%%%%%%%%%%%%%%%%%%%%%%%%%%%%%%%%%%
%%%%%%%%%%%%%%%%%%%%%
%% FIN DU DOCUMENT %%
%%%%%%%%%%%%%%%%%%%%%
\end{document}