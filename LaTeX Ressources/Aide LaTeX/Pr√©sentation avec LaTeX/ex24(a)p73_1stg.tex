\documentclass{beamer}
%Pour la francisation
\usepackage[T1]{fontenc}
\usepackage{lmodern}
\usepackage[frenchb]{babel}

%simplification notation norme \norme{}
\newcommand{\norme}[1]{\left\Vert #1\right\Vert}

\newcommand{\euro}{\texteuro{}}

%Ensembles de nombres
\newcommand{\R}{\mathbb{R}}
\newcommand{\N}{\mathbb{N}}
\newcommand{\D}{\mathbb{D}}
\newcommand{\Z}{\mathbb{Z}}
\newcommand{\Q}{\mathbb{Q}}
\newcommand{\C}{\mathbb{C}}

%simplification de la notation de vecteur \vect{}
\newcommand{\vect}[1]{\mathchoice%
{\overrightarrow{\displaystyle\mathstrut#1\,\,}}%
{\overrightarrow{\textstyle\mathstrut#1\,\,}}%
{\overrightarrow{\scriptstyle\mathstrut#1\,\,}}%
{\overrightarrow{\scriptscriptstyle\mathstrut#1\,\,}}}

%Rep�res
\def\Oij{$\left(\text{O}~;~\vect{\imath},~\vect{\jmath}\right)$}
\def\Oijk{$\left(\text{O}~;~\vect{\imath},~ \vect{\jmath},~ \vect{k}\right)$}
\def\Ouv{$\left(\text{O}~;~\vect{u},~\vect{v}\right)$}


%
\usetheme{Warsaw}

%%%%%%%%%%%%%%%%%%%%%%%%%%%%%%%%%%%%%%%%%%%%%%%%%%%%%%%%%%%%%%

  \title{Correction de la question a) de l'ex 24 p 73}
  \author{D. Tr�mulot}\institute{Lyc�e Jean Pierre Timbaud}

  \begin{document}

%%%%%%%%%%%%%%%%%%%%%%%%%%%%%%%%%%%%%%%%%%%%%%%%%%%%%%%%%%%%%%
  \begin{frame}
  \titlepage
  \end{frame}
%%%%%%%%%%%%%%%%%%%%%%%%%%%%%%%%%%%%%%%%%%%%%%%%%%%%%%%%%%%%%
  \begin{frame}
  \begin{itemize}
  
  \item<1-5> $v$ est la suite arithm�tique de premier terme $v_0$ telle que $v_1 = 4$ et $v_2=-8$.

  \item<2-5> Notons $a$ la raison de la suite $v$.\\
On a alors : $v_2=v_1+a$, donc $a=v_2-v_1=-8-4=-12$.

  \item<3-5> Par cons�quent, comme $v_1=v_0+a$, on obtient $v_0=v_1-a=4-(-12)=4+12=16$.

  \item<4-5> $v$ est donc la suite arithm�tique de premier terme $v_0=16$ et de raison $a=-12$.

  \item<5> Par cons�quent, pour tout $n \in \N$ :\\
$v_n=v_0+n\times a=16+n \times (-12)=-12n+16$
  
  \end{itemize}
  \end{frame}

%%%%%%%%%%%%%%%%%%%%%%%%%%%%%%%%%%%%%%%%%%%%%%%%%%%%%%%%%%%%

	\begin{frame}
	\begin{itemize}

	\item<1-5> Pour tout $n \in \N$ :\\
$v_n=-12n+16$

	\item<2-5> La somme cherch�e est :\\
$v_0+v_1+v_2+...+v_{48}+v_{49}=50 \times \dfrac{v_0+v_{49}}{2}$

	\item<3-5> Or, $v_{49}=-12\times 49+16=-572$.

	\item<4-5> Donc,
$v_0+v_1+v_2+...+v_{48}+v_{49}=50 \times \dfrac{16+(-572)}{2}=-13900$

	\item<5> La somme des 50 premiers termes de la suite $v$ vaut $-13900$.

	\end{itemize}
	\end{frame}

%%%%%%%%%%%%%%%%%%%%%%%%%%%%%%%%%%%%%%%%%%%%%%%%%%%%%%%%%%%
  \end{document}
