\documentclass[french,10pt]{report}
\input preambule_2014

\begin{document}
\begin{tikzpicture}
    \coordinate (A) at (2,1);
    \coordinate (B) at ($(A) + (3,1)$);
    \coordinate (D) at (0,-2);
    \coordinate (C) at ($(D) + (3,1)$);
    \draw (A)--(B)--(C)--(D)--cycle; % oh le beau parallélogramme
\end{tikzpicture}

%D'ailleurs, ($(B)-(A)$) donne le vecteur AB donc on peut aussi faire comme ça :

\begin{tikzpicture}
    \coordinate (A) at (2,1);
    \coordinate (B) at (7,2);
    \coordinate (D) at (0,-2);
    \coordinate (C) at ($(D) + (B) - (A)$);
    \draw (A)--(B)--(C)--(D)--cycle; % oh le 2e beau parallélogramme
\end{tikzpicture}



\end{document} 