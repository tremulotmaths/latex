\documentclass[%handout,%%%%%%%permet de rendre le doc imprimable en supprimant les pauses
french,onlymath]{beamer}

%%%%%%%%%%%%%%%%%%%%%%%%%%%%%%%%%%%%%%%%%%%%%%%%%%%%%%%%%%%%%%%%%%%%%%%%%%%%%%
\input{preambule_presentation_2015_2016}
%%%%%%%%%%%%%%%%%%%%%%%%%%%%%%%%%%%%%%%%%%%%%%%%%%%%%%%%%%%%%%%%%%%%%%%%%%%%%%
%Titre présentation Beamer
\title{AP du 18/09}  
\author{\seconde}\institute{Lycée Robert Doisneau}
\date{}
%%%%%%%%%%%%%%%%%%%%%%%%%%%%%%%%%%%%%%%%%%%%%%%%%%%%%%%%%%%%%%%%%%%%%%%%%%%%%%%%%

%%%%%%%%%%%%%%%%%%%%%%%
%% DEBUT DU DOCUMENT %%
%%%%%%%%%%%%%%%%%%%%%%%

\begin{document}
\selectlanguage{english}


%%%%%%%%%%%%%%%%%%%%%%%Page 1%%%%%%%%%%%%%%%%%%%%%%%%%%%%%%%%%%%%%%%%
%\begin{spacing}{1.2}

%%%%%%%%%%%%%%%%%%%%%%%%%%%%%%%%%%%%%%%%%%%%%%%%%%%%%%%%%%%%
\begin{frame}
 \titlepage
\end{frame}
%%%%%%%%%%%%%%%%%%%%%%%%%%%%%%%%%%%%%%%%%%%%%%%%%%%%%%%%%%%%%
\begin{frame}{Calculs numériques}

On considère les fonctions suivantes définies respectivement sur $\R\backslash\left\{-\sqrt{3}\pv \sqrt{3}\right\}$,  $\R\backslash\left\{\dfrac{-2}{3}\right\}$ et   $\R_+$ par :
\[f(x)=6-\frac{2x}{x^2-3}\quad\text{et}\quad g(x)=\frac{x-\frac{2}{3}}{x+\frac{2}{3}}\quad\text{et}\quad h(x)=5\sqrt{x+3x^2}-\sqrt{20}+80\sqrt{x}\]

\begin{enumerate}[\bf 1.]

\item Déterminer les images de $-1$ et de $\dfrac{1}{3}$ par $f$

\item Déterminer l'image de $\dfrac{1}{5}$ par $g$.

\item Déterminer l'image de $5$ par $h$ (\textit{On donnera le résultat sous la forme $a\sqrt{b}$, où $a$ et $b$ sont des entiers}).

\end{enumerate}
\end{frame}
%%%%%%%%%%%%%%%%%%%%%%%%%%%%%%%%%%%%%%%%%%%%%%%%%%%%%%%%%%%%%%%%%%
\begin{frame}{Intervalles}

Traduire, si possible, à l'aide d'un intervalle les conditions suivantes :

\begin{enumerate}[\bf 1.]
\item $x>\dfrac{1}{5}$

\item $1<x\leq 5$

\item $x\in \intervalleff{-1}{5}\cup\intervallefo{2}{+\infty}$

\item $x\in \intervalleoo{\dfrac{-5}{2}}{+\infty}\cap \intervalleff{-20}{2}$

\end{enumerate}



\end{frame}
%%%%%%%%%%%%%%%%%%%%%%%%%%%%%%%%%%%%%%%%%%%%%%%%%%%%%%%%%%%%%%%%%
\begin{frame}{Ensemble de nombres}

Indiquer parmi les ensembles $\N, \Z, \D, \Q$ et $\R$ les ensembles auxquels appartiennent les nombres suivants :

\begin{enumerate}[\bf 1.]
\item $-5,25$

\item $\dfrac{1}{3}$

\item $\sqrt{5}$

\item $\dfrac{-12}{4}$

\item $1,5\times 10^8$

\end{enumerate}

\end{frame}
%%%%%%%%%%%%%%%%%%%%%%%%%%%%%%%%%%%%%%%%%%%%%%%%%%%%%%%%%%%%

%\end{spacing}

%%%%%%%%%%%%%%%%%%%%%%%%%%%%%%%%%%%%%%%%%%%%%%%%%%%%%%%%%%%%
%%%%%%%%%%%%%%%%%%%%%
%% FIN DU DOCUMENT %%
%%%%%%%%%%%%%%%%%%%%%
\end{document}