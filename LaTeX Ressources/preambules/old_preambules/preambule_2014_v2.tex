%___________________________
%===    Configurations 12.09.2013
%------------------------------------------------------
%packages permettant d'augmenter le nombre de registres de dimension et donc d'éviter les erreurs de compilation dûs aux packages tikz, pstricks and compagnie
\usepackage{etex}
%___________________________
%===    Pour le français
%------------------------------------------------------
\usepackage[utf8x]{inputenc}
\usepackage[T1]{fontenc}

%===    Polices d'écriture
%------------------------------------------------------
\usepackage{frcursive} % Pour l'écriture cursive
%\cursive{texte}
\usepackage[upright]{fourier}% l'option permet d'avoir les majuscules droites dans les formules mathématiques
\usepackage[scaled=0.875]{helvet}

%___________________________
%===    Les couleurs
%------------------------------------------------------
\usepackage[dvipsnames,table]{xcolor}
%


%___________________________
%===   Entête, pied de page
%------------------------------------------------------

\usepackage{lscape} %permet le format paysage du document
\usepackage{xspace} % création automatique d'espaces dans les commandes

\setlength{\parindent}{0pt} %pas d'alinéa

\usepackage{fancyhdr}
%



\usepackage{enumerate} %permet la modif de la numérotation et de poursuivre une numérotation en cours avec \begin{enumerate}[resume]
\usepackage{enumitem}
\setenumerate[1]{font=\bfseries,label=\arabic*\degres)} % numérotation 1°) 2°) ...
%\setenumerate[2]{font=\itshape,label=(\alph*)} % sous-numérotation (a) (b) ...
\setenumerate[2]{font=\bfseries,label=(\alph*)} % sous-numérotation (a) (b) ...

\usepackage{lastpage} % permet d'afficher le nombre total de pages après DEUX compilations.

%___________________________
%===    Réglages et Commandes Maths
%------------------------------------------------------
%redéfinition de fractions, limites, sommes, intégrales, coefficients binomiaux en displaystyle, limites de suites
\usepackage{amssymb,mathtools}
\everymath{\displaystyle} %permet l'écriture jolie des limites et autres...

%\usepackage{bm} % pour l'écriture en gras des formules mathématiques avec \bm

\usepackage{cancel} % pour les simplifications de fractions
\renewcommand\CancelColor{\color{red}}

\usepackage[autolanguage,np]{numprint}
%permet les espacement pour les nombres décimaux avec \np{3,12456} en environnement maths ou pas

%
\usepackage{dsfont} %écriture des ensemble N, R, C ...


%
\usepackage{mathrsfs}   % Police de maths jolie caligraphie







%___________________________
%===    Pour les tableaux
%------------------------------------------------------
\usepackage{array}
\usepackage{longtable}
\usepackage{tabularx,tabulary}
\usepackage{multirow}
\usepackage{multicol}

%Pour multicol : Pour ajouter un titre numéroté qui apparaisse sur toute la largeur de la page, il faut utiliser l'option [\section{Titre.}] juste après \begin{multicols}{nb-col}.
%Remarques :
%Pour qu'une ligne de séparation apparaisse entre les colonnes, il faut utiliser : \setlength{\columnseprule}{1pt}.
%Pour redéfinir la largeur de l'espace inter-colonnes, il faut utiliser \setlength{\columnsep}{30pt}.

%Pour remonter le texte, dans chaque colonne vers le haut : \raggedcolumns qui se tape :\begin{multicols}{2}\raggedcolumns...\columnbreak...\columnbreak\end{multicols}

\setlength\columnseprule{0pt}

\usepackage{slashbox} %traits obliques dans les cellules

%___________________________
%===    Divers packages
%------------------------------------------------------
\usepackage{bclogo}
\usepackage{textcomp}
\usepackage{eurosym}%avec \EUR{3,12}
\usepackage{ulem} % Pour soiligner simple \ul
					%Pour souligner double : \uuline
                      % Pour souligner ondulé : \uwave
                      % Pour barrer horizontal : \sout
                      % Pour barrer diagonal : \xout
\usepackage{tikz,tkz-base,tkz-fct,tkz-euclide,tkz-tab,tkz-graph,tikz-3dplot}
\usetkzobj{all}
\usetikzlibrary{babel,intersections,calc,shapes,arrows,plotmarks,lindenmayersystems,decorations,decorations.markings,decorations.pathmorphing,
decorations.pathreplacing,patterns,positioning,decorations.text}
\usepackage{pstricks}
%pst-plot,pst-text,pstricks-add,pst-eucl,pst-all}


%INTERLIGNES
\usepackage{setspace}
%s'utilise avec \begin{spacing}{''facteur''}
%   […]
%\end{spacing}


\usepackage{multido} %faire des boucles dans les commandes



%divers cadres
\usepackage{fancybox} % par exemple \ovalbox{}

%caractères spéciaux  (symbole divers) avec la commande \ding{230} par exemple
\usepackage{pifont}

%___________________________
%===    Quelques raccourcis perso
%------------------------------------------------------


%checked box
\newcommand{\checkbox}{
\makebox[0pt][l]{$\square$}\raisebox{.15ex}{\hspace{0.1em}$\checkmark$}
}

%QRcode, codebarre
\usepackage{pst-barcode}
%\begin{pspicture}(2,2)
%	\psbarcode{http://www.latex-howto.be}{eclevel=M}{qrcode}
%\end{pspicture}


%Texte en filigrane
\usepackage{watermark}
%On utilise ensuite les commandes \watermark, \leftwatermark, \rightwatermark ou \thiswatermark qui permettent de définir un filigrane sur toutes les pages, les pages paires, les pages impaires ou juste une page
%Exemple : \thiswatermark {
%\begin{minipage}{0.95\linewidth}
%\vspace{25cm}
%\begin{center}
%\rotatebox{55}{\scalebox{8}{\color[gray]{0.7}\LaTeX}}
%\end{center}
%\end{minipage}
%}

%QCM
\usepackage{alterqcm}					%%Permet de créer des QCM
%\begin{alterqcm}
%\AQquestion{Question}{{Proposition 1},{Proposition 2},{Proposition 3}}
%\end{alterqcm}

%\dingsquare %carré avant V ou F
%\dingchecksquare %carré validé devant V ou F

%___________________________
%===    ALGORITHMES
%------------------------------------------------------

% Mise en forme des algorithmes
\usepackage[french,boxed,titlenumbered,lined,longend]{algorithm2e}
  \SetKwIF {Si}{SinonSi}{Sinon}{si}{alors}{sinon\_si}{alors}{fin~si}
 \SetKwFor{Tq}{tant\_que~}{~faire~}{fin~tant\_que}
 \SetKwFor{PourCh}{pour\_chaque }{ faire }{fin pour\_chaque}
 \SetKwInput{Sortie}{Sortie}
  \SetKwInput{Entree}{Entrée}
\newcommand{\Algocmd}[1]{\textsf{\textsc{\textbf{#1}}}}\SetKwSty{Algocmd}
  \newcommand{\AlgCommentaire}[1]{\textsl{\small  #1}}

%Exemple à faire...ça peut être sympa

\usepackage{marvosym} %différents symboles






%___________________________
%===    SOMMAIRE DANS LES CHAPITRES
%------------------------------------------------------

%\usepackage{minitoc}

%___________________________
%===    TABLEUR
%------------------------------------------------------
\usepackage{pas-tableur}
\usepackage{xstring}
\usepackage{xkeyval}


%___________________________
%===    touches calculatrices
%------------------------------------------------------

\usepackage{tipfr}


\usepackage{babel}
\FrenchFootnotes
%\frenchbsetup{StandardLists=true}%frenchb ne s'occupera pas des listes
\frenchbsetup{SuppressWarning,CompactItemize=false}
\DecimalMathComma %supprime l'espace après la virgule dans un nombre


%___________________________
%===    HYPERLIENS
%------------------------------------------------------
\usepackage[colorlinks=true,linkcolor=black,filecolor=blue,urlcolor=blue,bookmarksnumbered]{hyperref} 