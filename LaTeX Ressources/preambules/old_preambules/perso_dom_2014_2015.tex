%___________________________
%===   Redéfinition des marges par défaut
%------------------------------------------------------
%\usepackage[textwidth=18.6cm]{geometry}%à mettre dans le preambule perso
%\pagestyle{fancy}%à mettre dans le preambule perso


\setlength\paperheight{297mm}
\setlength\paperwidth{210mm}
\setlength{\evensidemargin}{0cm}% Marge gauche sur pages paires
\setlength{\oddsidemargin}%{0cm}%
{-0.5cm}% Marge gauche sur pages impaires
\setlength{\topmargin}{-2cm}% Marge en haut
\setlength{\headsep}{0.5cm}% Entre le haut de page et le texte
\setlength{\headheight}{0.7cm}% Haut de page
\setlength{\textheight}{25.2cm}% Hauteur de la zone de texte
\setlength{\textwidth}{17cm}% Largeur de la zone de texte


% Environnement enumerate
\renewcommand{\theenumi}{\bf\textsf{\arabic{enumi}}}
\renewcommand{\labelenumi}{\bf\textsf{\theenumi.}}
\renewcommand{\theenumii}{\bf\textsf{\alph{enumii}}}
\renewcommand{\labelenumii}{\bf\textsf{\theenumii.}}
\renewcommand{\theenumiii}{\bf\textsf{\roman{enumiii}}}
\renewcommand{\labelenumiii}{\bf\textsf{\theenumiii.}}


\usetikzlibrary{shadows,trees}


%definition des couleurs
\definecolor{fondpaille}{cmyk}{0,0,0.1,0}%\pagecolor{fondpaille}
\definecolor{gris}{rgb}{0.7,0.7,0.7}
\definecolor{rouge}{rgb}{1,0,0}
\definecolor{bleu}{rgb}{0,0,1}
\definecolor{vert}{rgb}{0,1,0}
\definecolor{deficolor}{HTML}{2D9AFF}
\definecolor{backdeficolor}{HTML}{EDEDED}%{036DD0}%dégradé bleu{666666}%dégradé gris
\definecolor{theocolor}{HTML}{036DD0}%F4404D%rouge
\definecolor{backtheocolor}{HTML}{D3D3D3}
\definecolor{methcolor}{HTML}{008800}%12BB05}
\definecolor{backmethcolor}{HTML}{FFFACD}
\definecolor{backilluscolor}{HTML}{EDEDED}
\definecolor{sectioncolor}{HTML}{221E1E}%{B2B2B2}%vert : {HTML}{008800}%{HTML}{2D9AFF}
\definecolor{subsectioncolor}{HTML}{221E1E}%{B2B2B2}%vert : {HTML}{008800}%{rgb}{0.5,0,0}
\definecolor{engcolor}{HTML}{D4D7FE}
\definecolor{exocolor}{rgb}{0,0.6,0}
\definecolor{exosoltitlecolor}{rgb}{0,0.6,0}
\definecolor{titlecolor}{rgb}{1,1,1}

%commande pour enlever les couleurs avant impression
\newcommand{\nocolor}
{\pagecolor{white}
\definecolor{gris}{rgb}{0.7,0.7,0.7}
\definecolor{rouge}{rgb}{0,0,0}
\definecolor{bleu}{rgb}{0,0,0}
\definecolor{vert}{rgb}{0,0,0}
\definecolor{deficolor}{HTML}{B2B2B2}
\definecolor{backdeficolor}{HTML}{EEEEEE}%{036DD0}%dégradé bleu{666666}%dégradé gris
\definecolor{theocolor}{HTML}{B2B2B2}
\definecolor{backtheocolor}{HTML}{EEEEEE}
\definecolor{methcolor}{HTML}{B2B2B2}
\definecolor{backmethcolor}{HTML}{EEEEEE}
\definecolor{backilluscolor}{HTML}{EEEEEE}
\definecolor{sectioncolor}{HTML}{B2B2B2}
\definecolor{subsectioncolor}{HTML}{B2B2B2}
\definecolor{engcolor}{HTML}{EEEEEE}
\definecolor{exocolor}{HTML}{3B3838}
\definecolor{exosoltitlecolor}{rgb}{0,0,0}
\definecolor{titlecolor}{rgb}{0,0,0}
}



%___________________________
%===    Exercice résolu
%------------------------------------------------------
%
%#1 : énoncé
%#2 : solution
\newcounter{exosol}
\newcommand{\exosol}[2]{
\stepcounter{exosol}
\begin{tikzpicture}[node distance=0 cm]
\node[fill=backilluscolor,rounded corners=2pt,anchor=south west] (illus) at (0,-0.02)
{\it \textbf{\textcolor{exosoltitlecolor}{Exercice résolu \arabic{exosol}~:~}}};
\node[fill=backilluscolor,rounded corners=2pt,anchor=north west]at(0,0)
{\parbox{\columnwidth-10pt}{#1\par\medskip{\it \textbf{\textcolor{exosoltitlecolor}{Solution~:~}}}\par#2 }};
\end{tikzpicture}
\bigskip
}

\newcommand{\suite}[1]{
\begin{tikzpicture}[node distance=0 cm]
\node[fill=backilluscolor,rounded corners=2pt,anchor=north west]at(0,0)
{\parbox{\columnwidth-10pt}{{\it \textbf{\textcolor{exosoltitlecolor}{Suite de la solution~:}}}\par#1}};
\end{tikzpicture}
\bigskip
}



%%%%%%%%%%%%%%%%%%%%%%%%%%%%%%%%%%%%%%%%%%%%%%%%%%%%%%%%%%%%%%%%%%%%%%%%%%%%%%%
%Encadrés pour Propriétés, Théorème, Définitions, exemples, exercices

\usepackage{environ}%pour pouvoir utiliser la commande \NewEnviron

%___________________________
%===    Propriété avec ou sans s et avec ou sans titre
%------------------------------------------------------
%
\NewEnviron{Prop}[2][]{
\begin{tikzpicture}[node distance=0 cm]
\node[fill=theocolor,rounded corners=5pt,anchor=south west] (theorem) at (0,0)
{\textcolor{titlecolor}{Propriété#1~:~#2}};
\node[draw,drop shadow,color=theocolor,very thick,fill=backtheocolor,rounded corners=5pt,anchor=north west] at(0,-0.02)
{\black\parbox{\columnwidth-12pt}{\BODY}};
\end{tikzpicture}
\bigskip
}


%___________________________
%===    Théorème avec ou sans titre
%------------------------------------------------------
%
\NewEnviron{Thm}[1][]{
\begin{tikzpicture}[node distance=0 cm]
\node[fill=theocolor,rounded corners=5pt,anchor=south west] (theorem) at (0,0)
{\textcolor{titlecolor}{Théorème~:~#1}};
\node[draw,drop shadow,color=theocolor,very thick,fill=backtheocolor,rounded corners=5pt,anchor=north west] at(0,-0.02)
{\black\parbox{\columnwidth-12pt}{\BODY}};
\end{tikzpicture}
\medskip
}


%___________________________
%===    Règle(s) avec ou sans s et avec sans titre
%------------------------------------------------------
%
\NewEnviron{Regle}[2][]{
\begin{tikzpicture}[node distance=0 cm]
\node[fill=theocolor,rounded corners=5pt,anchor=south west] (theorem) at (0,0)
{\textcolor{titlecolor}{Règle#1~:~#2}};
\node[draw,drop shadow,color=deficolor,very thick,fill=backdeficolor,rounded corners=5pt,anchor=north west] at(0,-0.02)
{\black\parbox{\columnwidth-12pt}{\BODY}};
\end{tikzpicture}
\medskip
}

%___________________________
%===    Définition avec ou sans s et avec sans titre
%------------------------------------------------------
%
\NewEnviron{Defi}[2][]{
\begin{tikzpicture}[node distance=0 cm]
\node[fill=theocolor,rounded corners=5pt,anchor=south west] (theorem) at (0,0)
{\textcolor{titlecolor}{Définition#1~:~#2}};
\node[draw,drop shadow,color=deficolor,very thick,fill=backdeficolor,rounded corners=5pt,anchor=north west] at(0,-0.02)
{\black\parbox{\columnwidth-12pt}{\BODY}};
\end{tikzpicture}
\medskip
}

%___________________________
%===    Méthode avec ou sans s et avec sans titre
%------------------------------------------------------
%
\NewEnviron{Methode}[2][]{
\begin{tikzpicture}[node distance=0 cm]
\node[fill=theocolor,rounded corners=5pt,anchor=south west] (theorem) at (0,0)
{\textcolor{titlecolor}{Méthode#1~:~#2}};
\node[draw,drop shadow,color=methcolor,very thick,fill=backmethcolor,rounded corners=5pt,anchor=north west] at(0,-0.02)
{\black\parbox{\columnwidth-12pt}{\BODY}};
\end{tikzpicture}
\medskip
}


%___________________________
%===    Redéfinition de la commande \chapter{•}
%------------------------------------------------------
%
\makeatletter

\renewcommand{\@makechapterhead}[1]{
\begin{tikzpicture}
\node[fill=theocolor,rectangle,rounded corners=5pt]{%
\begin{minipage}{\linewidth}
\begin{center}
\vspace*{9pt}
\textcolor{titlecolor}{\Large \textsc{\textbf{Chapitre \thechapter \ : \ #1}}}
\vspace*{9pt}
\end{center}
\end{minipage}
};\end{tikzpicture}
}

\makeatother


%___________________________
%===    Exemple avec ou sans s et avec ou sans titre
%------------------------------------------------------
%
\NewEnviron{Exemple}[2][]{
\begin{tikzpicture}[node distance=0 cm]
\node[draw,drop shadow,color=methcolor,very thick,fill=backmethcolor,rounded corners=5pt,anchor=north west] at(0,-0.02)
{\black\parbox{\columnwidth-12pt}{\textbf{Exemple#1~:~#2}\\
\BODY}};
\end{tikzpicture}
\medskip
}

%___________________________
%===    Remarque avec ou sans s
%------------------------------------------------------
%
\NewEnviron{Rmq}[1][]{
\textbf{\large{Remarque#1 :}}\par
\BODY
\medskip
}

%___________________________
%===    Remarques numérotées R1, R2, etc...
%------------------------------------------------------
%
\newcounter{rem}\newcommand{\rem}{\refstepcounter{rem}\textbf{R \therem \ :}\xspace}

%___________________________
%===    Exercices du contrôle numérotés
%------------------------------------------------------
%
\newcounter{exercice}
\NewEnviron{Exercice}[1][]{
\refstepcounter{exercice}\textbf{\large{Exercice \theexercice \ :}}\hfill \textbf{#1}\par
\BODY
\medskip
}

%___________________________
%===    Exercices non numérotés
%------------------------------------------------------
%
\NewEnviron{Exo}[1][]{
\textbf{\large{Exercice #1 \ :}}\par
\BODY
\medskip
}

%___________________________
%===    Démonstration
%------------------------------------------------------
\NewEnviron{Demo}{%
\textit{\textbf{Démonstration.}}\par
\BODY
\strut\hfill$\square$
\medskip
}

%___________________________
%===    Commandes perso
%------------------------------------------------------
%
%\Leftrightarrow
\newcommand{\Lr}{\Leftrightarrow}

%Ancienne commande chapitre
\newcommand{\chapitre}[1]{
\begin{tikzpicture}
\node[fill=theocolor,rectangle,rounded corners=5pt]{%
\begin{minipage}{\linewidth}
\begin{center}
\vspace*{9pt}
\textcolor{titlecolor}{\Large \textsc{\textbf{#1}}}
\vspace*{9pt}
\end{center}
\end{minipage}
};
\end{tikzpicture}
\bigskip
}

%Pour les fiches : commande de Cécile
\newcommand{\Fiche}[2]{%
\begin{tikzpicture}
	\node[draw, color=blue,fill=white,rectangle,rounded corners=5pt]{%
	\begin{minipage}{\linewidth}
		\begin{center}
			\vspace*{9pt}
			\textcolor{blue}{\Large \textsc{\textbf{Fiche~#1\ :\ #2}}}
			\vspace*{7pt}
		\end{center}
	\end{minipage}
	};
\end{tikzpicture}
}%

\pagecolor{white}%couleur du fond de page

\renewcommand{\Pointilles}{%
\makebox[\linewidth]{\dotfill}
}



%%%%%%%%%%%%%%%%%%%%%%%%%%%%%%%%%%%%%%%%%%%%%%%%%%%%%%%%%%%%%%%%%%%%%%%%%%%%%%%