%___________________________
%===    Configurations perso 04.12.2014
%------------------------------------------------------
\usepackage{etex}
\usepackage[utf8]{inputenc}
\usepackage[T1]{fontenc}
\usepackage[frenchstyle,partialup]{kpfonts}
\usepackage[zerostyle=d]{newtxtt}
\renewcommand{\sfdefault}{txss}
\usepackage{aurical}%font lukas \Fontlukas
\usepackage{tipfr}
\usepackage{xspace}
\usepackage{geometry,atbegshi}
\geometry{a4paper, height = 25.5cm,hmargin=2.5cm,marginparwidth=2cm,headheight=20pt,headsep=16.5pt,bottom=2cm,footskip=30pt,footnotesep=30pt}
\usepackage{titlesec}
\usepackage{totpages}
\usepackage{fancyhdr}
\pagestyle{fancy}
\newcommand\RegleEntete[1][0.4pt]{\renewcommand{\headrulewidth}{#1}}
\renewcommand{\headrulewidth}{0pt}
\newcommand{\entete}[3]{\lhead{#1} \chead{\sffamily\textbf{#2}} \rhead{#3}}
\newcommand{\pieddepage}[3]{\lfoot{#1} \cfoot{#2} \rfoot{#3}}
\fancypagestyle{plain}{ \fancyhead{} \renewcommand{\headrulewidth}{0pt}}

\entete{}{\color{\MaCouleur} \textbullet~\leftmark~\textbullet}{}
\pieddepage{}{\color{\MaCouleur}$\stackrel{***}{\thepage}$}{}

\usepackage[dvipsnames,table]{xcolor}
\definecolor{midblue}{rgb}{0.145,0.490,0.882}
\newcommand\MaCouleur{midblue}

\usepackage[dotinlabels]{titletoc}

\usepackage{booktabs,tabularx,environ,marvosym,multirow,multicol,humanist}
% booktabs -> pour les tableaux : toprule, midrule, bottomrule...
% environ : pour les environnements personnalisés
% humanist -> font \hminfamily

\usepackage{mathtools,amssymb}
\everymath{\displaystyle}
\usepackage[autolanguage,np]{numprint}
\usepackage{xlop}
\usepackage{cancel}
\renewcommand\CancelColor{\color{red}}
\usepackage{dsfont}

\usepackage[normalem]{ulem} % Pour souligner double : \uuline
                      % Pour souligner ondulé : \uwave
                      % Pour barrer horizontal : \sout
                      % Pour barrer diagonal : \xout

\usepackage[right]{eurosym}

\usepackage{tkz-tab,pas-tableur,tkz-fct}
\usetikzlibrary{calc,shapes,arrows,shadows,backgrounds,decorations,decorations.footprints,decorations.pathmorphing,decorations.text,patterns,intersections,babel,through}
\usepackage[tikz]{bclogo}

\usepackage{enumitem}
\setenumerate[1]{font=\bfseries,label=\arabic*\degres)}
\setenumerate[2]{font=\itshape,label=(\alph*)}
\setitemize[1]{label=$\ast$}

\usepackage{babel}
\DecimalMathComma
\frenchbsetup{SuppressWarning,CompactItemize=false}
\FrenchFootnotes

%___________________________
%===    Commandes raccourcis
%------------------------------------------------------
\newcommand\seconde{2\up{nde}\xspace}
\newcommand\premiere{1\up{ère}\xspace}
\newcommand\terminale{T\up{le}\xspace}
\newcommand\stmg{\bsc{Stmg}}
\newcommand\sti{\bsc{Sti2d}}

\newcommand{\intervalle}[4]{\left#1 #2\mathpunct{};#3\right#4}
\newcommand{\intervalleff}[2]{\intervalle{[}{#1}{#2}{]}}
\newcommand{\intervalleof}[2]{\intervalle{]}{#1}{#2}{]}}
\newcommand{\intervallefo}[2]{\intervalle{[}{#1}{#2}{[}}
\newcommand{\intervalleoo}[2]{\intervalle{]}{#1}{#2}{[}}

\newcommand\Arc[1]{\ensuremath{\wideparen{#1}}}

\newcommand{\C}{\mathds C}
\renewcommand{\Re}{\mathfrak{Re}}
\renewcommand{\Im}{\mathfrak{Im}}
\newcommand{\R}{\mathds R}
\newcommand{\Q}{\mathds Q}
\newcommand{\Z}{\mathds Z}
\newcommand{\N}{\mathds N}
\newcommand\Ind{\mathds 1} %= fonction indicatrice
\newcommand\p{\mathds P} %= probabilité
\newcommand\E{\mathds E} % Espérance
\newcommand\V{\mathds V} % Variance
\newcommand{\vect}[1]{\ensuremath{\overrightarrow{#1}}}

\newcommand\abs[1]{\ensuremath{\left\lvert #1 \right\rvert}}

\newcommand\e{\ensuremath{\text{\textup{e}}}}

\let\leq\leqslant
\let\geq\geqslant

%___________________________
%===    Gestion des espaces
%------------------------------------------------------
\newcommand{\pv}{\ensuremath{\: ; \,}}
\newlength{\EspacePV}
\setlength{\EspacePV}{1em plus 0.5em minus 0.5em}
\newcommand{\qq}{\hspace{\EspacePV} ; \hspace{\EspacePV}}
\newcommand{\qetq}{\hspace{\EspacePV} \text{et} \hspace{\EspacePV}}
\newcommand{\qLq}{\hspace{\EspacePV} \Leftrightarrow \hspace{\EspacePV}}
\newcommand{\qRq}{\hspace{\EspacePV} \Rightarrow \hspace{\EspacePV}}
\newcommand{\qLRq}{\hspace{\EspacePV} \Leftrightarrow \hspace{\EspacePV}}

\renewcommand*{\hrulefill}[1][0.3mm]{\leavevmode \leaders \hrule height #1 \hfill \kern 0pt} % filet horizontal à épaisseur modifiable


\usepackage[pdfborder={0 0 0},bookmarksnumbered,pdfpagelabels]{hyperref} 