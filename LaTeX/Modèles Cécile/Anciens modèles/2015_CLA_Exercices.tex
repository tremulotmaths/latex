\documentclass[french,12pt]{report}
\input preambule_2015
\usepackage{xargs}
\usepackage{environ}
%___________________________
%===    Redéfinition des marges
%------------------------------------------------------
%



%___________________________
%===    Redéfinition de la commande \chapter{•}
%------------------------------------------------------
%

\makeatletter
\renewcommand{\@makechapterhead}[1]{%
        {\color{blue}
        \begin{flushright}
            \bfseries\fontsize{25}{20}\selectfont\textsc{
            #1}
            
        \end{flushright}
        \vspace{-0.5cm}
            \rule{\linewidth}{1pt}}
       

}

\makeatother
%\makeatletter
%
%\renewcommand{\@makechapterhead}[1]{
%\begin{tikzpicture}
%\node[draw, color=blue,fill=white,rectangle,rounded corners=5pt]{%
%\begin{minipage}{\linewidth}
%\begin{center}
%\vspace*{9pt}
%\textcolor{blue}{\Large \textsc{\textbf{Chapitre \thechapter \ :}}} \par
%\textcolor{blue}{\Large \textsc{\textbf{ \ #1}}}
%\vspace*{7pt}
%\end{center}
%\end{minipage}
%};\end{tikzpicture}
%}

%___________________________
%===    Redéfinition des sections
%------------------------------------------------------
%

        \makeatletter
        \newcommand{\sectioncolor}{blue} %Couleur titre de section
        \newcommand{\ssectioncolor}{MidnightBlue} %Couleur titre de sous-section
        \newcommand{\sssectioncolor}{RoyalBlue} %Couleur titre de sous-sous-section
        %
        %Coloration des titres
        %---------------------
        \renewcommand{\section}{%Commande définie dans le fichier article.cls
            \@startsection%
            {section}%
            {1}%
            {0pt}%
            {-3.5ex plus -1ex minus -.2ex}%
            {2.3ex plus.2ex}%
            {\color{\sectioncolor}\normalfont\Large\bfseries}} %Aspect du titre
        \renewcommand\subsection{%
            \@startsection{subsection}{2}
            {0.5cm}% %décalage horizontal
            {-3.5ex\@plus -1ex \@minus -.2ex}%
            {1ex \@plus .2ex}%
            {\color{\ssectioncolor}\normalfont\large\bfseries}}
        \renewcommand\subsubsection{%
            \@startsection{subsubsection}{3}
            {1cm}% %décalage horizontal
            {-3.25ex\@plus -1ex \@minus -.2ex}%
            {1ex \@plus .2ex}%
            {\color{\sssectioncolor}\normalfont\normalsize\bfseries}}
        %

\setcounter{secnumdepth}{3}


 %___________________________
%===    Redéfinition des numérotation des paragraphes
%------------------------------------------------------
%
\renewcommand\thesection{\Roman{section}}
\renewcommand\thesubsection{\arabic{subsection}.}
\renewcommand\thesubsubsection{\alph{subsubsection}.}
%

%_________________________
%===    Environnements de cours
%------------------------------------------------------



%___________________________
%===    Définitions
%------------------------------------------------------

\newenvironment{Defi}[2][]{%
\hspace*{-0.25cm}
\begin{tikzpicture}
\path[left color=lightgray] (0,0) arc[radius=0.25, start angle =270, end angle =90] -- (10,0.5) |- (0,0);
\draw (0,0.25) node[right=-4pt] {\bfseries\color{red} Définition#1 : \textit{\textmd{#2}}};
\end{tikzpicture}\nopagebreak[4]\par  \hspace*{-0.25cm} {\color{red}$\ulcorner$} \par \vspace{-0.25cm} }
{\hspace{\linewidth} {\color{red}  $\lrcorner$ }\par}


%%\newcounter{defi}[chapter]
%\NewEnviron{Defi}[2][]%
%{\medskip
%\begin{tikzpicture}
%\node[draw,color=red,fill=white,rectangle,rounded corners=5pt ]
%{\black\parbox{\linewidth}{\textcolor{red}{Définition#1 :} \textit{#2}\par \BODY}};
%\end{tikzpicture}
%}%

%____________________
%===    Propriétés
%------------------------------------------------------
%\newcounter{propri}[chapter]
\newenvironment{Prop}[2][]{%
%\refstepcounter{propri}
\hspace*{-0.25cm}
\begin{tikzpicture}
\path[left color=lightgray] (0,0) arc[radius=0.25, start angle =270, end angle =90] -- (10,0.5) |- (0,0);
\draw (0,0.25) node[right=-4pt] {\bfseries\color{red} Propriété#1 : \textit{\textmd{#2}}};
\end{tikzpicture}\nopagebreak[4]\par  \hspace*{-0.25cm} {\color{red}$\ulcorner$} \par \vspace{-0.25cm} }
{\hspace{\linewidth} {\color{red}  $\lrcorner$ }\par}




%___________________________
%===    Théorèmes
%------------------------------------------------------
%\newcounter{thm}[chapter]
\newenvironment{Thm}[2][]{%
%\refstepcounter{thm}
\hspace*{-0.25cm}
\begin{tikzpicture}
\path[left color=lightgray] (0,0) arc[radius=0.25, start angle =270, end angle =90] -- (10,0.5) |- (0,0);
\draw (0,0.25) node[right=-4pt] {\bfseries\color{red} Théorème#1 : \textit{\textmd{#2}}};
\end{tikzpicture}\nopagebreak[4]\par  \hspace*{-0.25cm} {\color{red}$\ulcorner$} \par \vspace{-0.25cm} }
{\hspace{\linewidth} {\color{red}  $\lrcorner$ }\par}

%___________________________
%===    Démonstration
%------------------------------------------------------
\NewEnviron{Demo}[1][]%
{\begin{tikzpicture}
\node[fill=gray!10,rounded corners=2pt,anchor=south west] (illus) at (0,0)
{\hfill \textbf{\textcolor{ForestGreen!50}{Démonstration#1}}};
\node[fill=gray!10,rounded corners=2pt,anchor=north west]at(0,0)
{\parbox{\linewidth}{\BODY \par 
\hfill$\square$}};
\end{tikzpicture}
\medskip
}

%___________________________
%===    Exemples
%------------------------------------------------------
\newcounter{exemple}[chapter]
\NewEnviron{Exemple}%
{
\refstepcounter{exemple}
\psframebox[fillstyle=solid,fillcolor=Yellow!30,linewidth=0.4pt,linecolor=Yellow!30,linearc=0.05,cornersize=absolute]{

    \begin{minipage}{\linewidth}
\textit{Exemple~\theexemple~:} \par 
\BODY
\end{minipage}
\medskip
}
}



\NewEnviron{Exemple*}[1][]%
{
\psframebox[fillstyle=solid,fillcolor=Yellow!30,linewidth=0.4pt,linecolor=Yellow!30,linearc=0.05,cornersize=absolute]{

    \begin{minipage}{\linewidth}
\textit{Exemple#1:} \par 
\BODY
\end{minipage}
\medskip
}
}

%___________________________
%===    Méthodes
%------------------------------------
\NewEnviron{Methode}[1] []%
{\begin{bclogo}[noborder=true, arrondi = 0.1, logo = \bccrayon, barre = snake, tailleOndu=2,marge=0]{\normalsize Méthode#1}
   \BODY
\end{bclogo}
\medskip
}%

%___________________________
%===    Remarques
%------------------------------------------------------

\NewEnviron{Rmq}[1] []%
{\begin{bclogo}[noborder=true, arrondi = 0.1, logo = , barre = snake, tailleOndu = 2 ,marge=0]{\normalsize Remarque#1 :}
   \BODY
\end{bclogo}
\medskip
}%

%___________________________
%===   Exercices
%------------------------------------------------------
\NewEnviron{Exo}[1][]
{\textbf{Exercice~#1 :} \par
\BODY
\medskip
}
\newcounter{exos}[chapter]
\NewEnviron{Exercice}[1][]
{
\refstepcounter{exos}
\textbf{Exercice~\theexos :} ~#1 \par
\BODY
\medskip
}

%===   Commandes
%------------------------------------------------------

\newcommand{\Fiche}[2]{%
\begin{tikzpicture}
	\node[draw, color=blue,fill=white,rectangle,rounded corners=5pt]{%
	\begin{minipage}{\linewidth}
		\begin{center}
			\vspace*{9pt}
			\textcolor{blue}{\Large \textsc{\textbf{Fiche~#1 :}}}\par
			\textcolor{blue}{\Large \textsc{\textbf{#2}}}
			\vspace*{7pt}
		\end{center}
	\end{minipage}
	};
\end{tikzpicture}
}%

\newcommand{\Livre}[1]{%
\begin{bclogo}[noborder=true, arrondi = 0.1, logo =\bcbook , barre = none , tailleOndu = 2 ,marge=0]{\normalsize Dans le livre :}
   #1
\end{bclogo}
\medskip
}%

%Dans un repère
\newcommand{\Dsrepere}[1]{%
\begin{bclogo}[noborder=true, arrondi = 0.1, logo =\bccrayon , barre = line , tailleOndu = 0 ,marge=0]{\normalsize Dans un repère :}
   #1
\end{bclogo}
\medskip
}%

\newcommand\voc[1]{\textbf{#1}\xspace}

%\dominitoc %pour pouvoir créer un sommaire du chapitre en cours avec \minitoc

%produit saclaire 
\newcommand{\Pdtscalaire}[2]{$\vect{#1} \cdot \vect{#2}$}
\input{commandes_2015}

\newgeometry{includeheadfoot,headsep=0.2cm,top=0.5cm,bottom=0.8cm,right=1.5cm,left=1.5cm}
\pagestyle{fancy}
\fancyhf{}


\begin{document}
\ReglePied

\fancypagestyle{garde}{
\pieddepage{2015-2016 - CLA}{}{\thepage / \pageref{LastPage}}}

\pieddepage{2015-2016 - CLA}{Thème ? : Titre }{\thepage / \pageref{LastPage}}



\chapter{Thème ? : Calcul numérique \\ \Large Exercices }
\thispagestyle{garde}


\bigskip

\begin{Exercice}
Calculer les expressions suivantes et donner les résultats sous forme de fractions irréductibles.\bigskip

$A=\dfrac{6}{8}$\hfill
$B=\dfrac{2}{5}+\dfrac{3}{4}$\hfill
$C=\dfrac{2}{5}\times\dfrac{3}{4}$\hfill
$D=\dfrac{\dfrac{2}{5}}{\dfrac{3}{10}}$\hfill\bigskip

\end{Exercice}


\medskip
\begin{Exercice}

Calculer les expressions suivantes en respectant les règles de priorités de calculs et donner les résultats sous forme de fractions irréductibles.\bigskip

$E=\dfrac{1}{15}-\left( 2-\dfrac{7}{3}\right) $\hfill
$F=\dfrac{1-\dfrac{4}{3}}{1+\dfrac{4}{3}}$\hfill
$G=\dfrac{2}{5}-\dfrac{1}{5}\times\dfrac{4}{3}$\hfill\bigskip

$H=\dfrac{5}{8}\times\dfrac{-4}{3}-\dfrac{2}{3}$\hfill
$I=\dfrac{\dfrac{3}{7}}{\dfrac{4}{21}}-\dfrac{5}{2}$\hfill
$J=3-\dfrac{2}{3}\times\dfrac{5-2}{8-2}$\hfill\bigskip

\end{Exercice}

\begin{Exercice}
Pour chaque phrase, écrire l'expression mathématique associée puis calculer : \par 
\begin{tabbing}
\hspace{10cm}\=\kill
le carré de 4 : \ldots \ldots  = \ldots \ldots   \>l'opposé du cube de 5 : \ldots \ldots  = \ldots \ldots  \\ 
l'opposé du carré de 4 :  \ldots \ldots  = \ldots \ldots \> le cube de 5 :	\ldots \ldots  = \ldots \ldots \\
le carré de l'opposé de 4 :  \ldots \ldots  = \ldots \ldots \> le cube de l'opposé de 5 : 	 \ldots \ldots  = \ldots \ldots
\end{tabbing} 

\end{Exercice}
\medskip

\begin{Exercice}
Compléter par = ou $\neq$ \par 
\begin{center}
\begin{tabular}{ccccccccccc}
$(-7)^2$& \ldots \ldots & $7^2$ & ~& $3^5$ &\ldots \ldots& $-3^5$& ~ & $-(-2)^3$ &\ldots \ldots & $2^3$\\ 
$-5^2$&\ldots \ldots &$(-5)^2$ & ~ & $5^4$ &\ldots \ldots &$(-5)^4$& ~ & $-6^4$ &\ldots \ldots &$-(-6)^4$ \\ 
$4^3$ & \ldots \ldots & $(-4)^3$ & ~& $(-5)^3$ & \ldots \ldots &$-5^3$& ~ & $-1^7$ & \ldots \ldots & $(-1)^7$ \\ 
\end{tabular} 

\end{center}
\end{Exercice}

\medskip
\begin{Exercice}
Écrire sous la forme d’une seule puissance : \par 
\begin{tabbing}
\hspace{8,5cm}\=\kill
$10^4\times 10^{-5}$ = \> $\dfrac{10^6}{10^{–7}}$ =\\
$\dfrac{5^4}{5^{– 2}} $= \> $(3^{–1})^4 $ = \\
$(2 \times (– 3))^4$ = \> $\dfrac{(–20)^3}{10^3}$ = \\[2pt]
$(10^{-3})^{-2}$ = \> $\dfrac{10^5 \times 10^{-3}}{(10^9)^{-2}}$ = \\ [2pt]
$\dfrac{30^7}{6^7} $ = \> $\dfrac{2^5 \times (–3)^5}{(– 6)^2 \times (– 6)^3}$ =  

\end{tabbing}



\end{Exercice}

\medskip
\begin{Exercice}
Compléter le tableau ci-dessous :\par 
\begin{center}
\begin{tabular}{|l|c|c|c|c|c|c|}
\hline 
nombre & 12 milliards & & $0,15\times 10^{-7}$  & $314~159\times 10^{-5}$ &  & $0,000~420$ \\ 
\hline 
écriture scientifique & &  $7,3\times 10^4$ &  &  & $1,63\times 10^{-4}$ &  \\ 
\hline 
\end{tabular} 

\end{center}
\end{Exercice}



\begin{Exercice}
Calculer et donner le résultat sous forme scientifique.\par 
$a = 0,007 \times 0,00~005$ \hfill $b = 31~000 \times 0,0~002 $ \hfill $c = \dfrac{1,8 \times 10^2 \times 10^{- 4}}{0,9\times 10^5} $
\end{Exercice}


\begin{Exercice}
\begin{enumerate}[label=\arabic*.]

\item Le micromètre est la millionième partie du mètre. 
\begin{enumerate}[label=\alph*)]
	\item Compléter : \par 
	1 $\mu$m = \ldots \ldots m \hspace{4cm} 1 $\mu$m = \ldots \ldots mm
	
	\item Le diamètre de l'atome de fer est de $0,000~234$ micromètre. \par 
	Parmi les écritures suivantes, entourer celles qui donnent ce diamètre en micromètres et préciser celle qui correspond à l'écriture scientifique .\par 

$2,34 \times 10^{– 4}$\hfill 	$234 \times 10^{–5}$ \hfill	$0,234\times 10{– 3}$ \hfill $	23,4 \times 10^{- 6}$ \par 

		
	\end{enumerate}
 \item En astronomie, pour exprimer les distances entre les planètes, on utilise l’unité astronomique (U.A.) comme unité de longueur (c’est la distance entre la Terre et le Soleil) \par 
1 U.A. = $149~600~000$ km.
\begin{enumerate}[label=\alph*)]
	\item Parmi les écritures suivantes, entourer celles qui correspondent à 1 U.A. \par 

$149,6 \times 10^9$  m \hfill	$1,496 \times 10^{– 8}$  km \hfill 	$1,496 \times 10^5$  km \hfill	$1,496 \times 10^8 $ km
	
	\item  Parmi les réponses entourées, quelle est celle qui correspond à l'écriture scientifique de 1 U.A.?
	
		
	\end{enumerate}

\end{enumerate}


\end{Exercice}
\begin{Exercice}
Calculer les expressions suivantes. \par 

$A = \sqrt{0,64}$ \hfill $B = \sqrt{\dfrac{9}{25}}$ \hfill $C = \sqrt{36+64}$	\hfill $D = (\sqrt{3} + \sqrt{2})(\sqrt{3} – \sqrt{2}) – (2\sqrt{5})^2$ \par 


\end{Exercice}

\begin{Exercice}
Écrire les nombres suivants sous la forme $a\sqrt{b}$ où $a$ et $b$ sont des entiers.\par 
$E = 3\sqrt{8} + 5\sqrt{32}$  \hspace{3cm} $F = 5\sqrt{18} + 2\sqrt{50} – 3\sqrt{98}$


\end{Exercice}

\begin{Exercice}
Écrire les nombres suivants sous la forme $\dfrac{a\sqrt{b}}{c}$ où $a$, $b$ et $c$ sont des entiers.\par 
$G = \dfrac{3}{\sqrt{5}}$ \hspace{4.3cm} $H= \sqrt{\dfrac{21}{8}}$
\end{Exercice}

\end{document}

