\documentclass[french,12pt]{report}
\input preambule_2015
\usepackage{xargs}
\usepackage{environ}
%___________________________
%===    Redéfinition des marges
%------------------------------------------------------
%



%___________________________
%===    Redéfinition de la commande \chapter{•}
%------------------------------------------------------
%

\makeatletter
\renewcommand{\@makechapterhead}[1]{%
        {\color{blue}
        \begin{flushright}
            \bfseries\fontsize{25}{20}\selectfont\textsc{
            #1}
            
        \end{flushright}
        \vspace{-0.5cm}
            \rule{\linewidth}{1pt}}
       

}

\makeatother
%\makeatletter
%
%\renewcommand{\@makechapterhead}[1]{
%\begin{tikzpicture}
%\node[draw, color=blue,fill=white,rectangle,rounded corners=5pt]{%
%\begin{minipage}{\linewidth}
%\begin{center}
%\vspace*{9pt}
%\textcolor{blue}{\Large \textsc{\textbf{Chapitre \thechapter \ :}}} \par
%\textcolor{blue}{\Large \textsc{\textbf{ \ #1}}}
%\vspace*{7pt}
%\end{center}
%\end{minipage}
%};\end{tikzpicture}
%}

%___________________________
%===    Redéfinition des sections
%------------------------------------------------------
%

        \makeatletter
        \newcommand{\sectioncolor}{blue} %Couleur titre de section
        \newcommand{\ssectioncolor}{MidnightBlue} %Couleur titre de sous-section
        \newcommand{\sssectioncolor}{RoyalBlue} %Couleur titre de sous-sous-section
        %
        %Coloration des titres
        %---------------------
        \renewcommand{\section}{%Commande définie dans le fichier article.cls
            \@startsection%
            {section}%
            {1}%
            {0pt}%
            {-3.5ex plus -1ex minus -.2ex}%
            {2.3ex plus.2ex}%
            {\color{\sectioncolor}\normalfont\Large\bfseries}} %Aspect du titre
        \renewcommand\subsection{%
            \@startsection{subsection}{2}
            {0.5cm}% %décalage horizontal
            {-3.5ex\@plus -1ex \@minus -.2ex}%
            {1ex \@plus .2ex}%
            {\color{\ssectioncolor}\normalfont\large\bfseries}}
        \renewcommand\subsubsection{%
            \@startsection{subsubsection}{3}
            {1cm}% %décalage horizontal
            {-3.25ex\@plus -1ex \@minus -.2ex}%
            {1ex \@plus .2ex}%
            {\color{\sssectioncolor}\normalfont\normalsize\bfseries}}
        %

\setcounter{secnumdepth}{3}


 %___________________________
%===    Redéfinition des numérotation des paragraphes
%------------------------------------------------------
%
\renewcommand\thesection{\Roman{section}}
\renewcommand\thesubsection{\arabic{subsection}.}
\renewcommand\thesubsubsection{\alph{subsubsection}.}
%

%_________________________
%===    Environnements de cours
%------------------------------------------------------



%___________________________
%===    Définitions
%------------------------------------------------------

\newenvironment{Defi}[2][]{%
\hspace*{-0.25cm}
\begin{tikzpicture}
\path[left color=lightgray] (0,0) arc[radius=0.25, start angle =270, end angle =90] -- (10,0.5) |- (0,0);
\draw (0,0.25) node[right=-4pt] {\bfseries\color{red} Définition#1 : \textit{\textmd{#2}}};
\end{tikzpicture}\nopagebreak[4]\par  \hspace*{-0.25cm} {\color{red}$\ulcorner$} \par \vspace{-0.25cm} }
{\hspace{\linewidth} {\color{red}  $\lrcorner$ }\par}


%%\newcounter{defi}[chapter]
%\NewEnviron{Defi}[2][]%
%{\medskip
%\begin{tikzpicture}
%\node[draw,color=red,fill=white,rectangle,rounded corners=5pt ]
%{\black\parbox{\linewidth}{\textcolor{red}{Définition#1 :} \textit{#2}\par \BODY}};
%\end{tikzpicture}
%}%

%____________________
%===    Propriétés
%------------------------------------------------------
%\newcounter{propri}[chapter]
\newenvironment{Prop}[2][]{%
%\refstepcounter{propri}
\hspace*{-0.25cm}
\begin{tikzpicture}
\path[left color=lightgray] (0,0) arc[radius=0.25, start angle =270, end angle =90] -- (10,0.5) |- (0,0);
\draw (0,0.25) node[right=-4pt] {\bfseries\color{red} Propriété#1 : \textit{\textmd{#2}}};
\end{tikzpicture}\nopagebreak[4]\par  \hspace*{-0.25cm} {\color{red}$\ulcorner$} \par \vspace{-0.25cm} }
{\hspace{\linewidth} {\color{red}  $\lrcorner$ }\par}




%___________________________
%===    Théorèmes
%------------------------------------------------------
%\newcounter{thm}[chapter]
\newenvironment{Thm}[2][]{%
%\refstepcounter{thm}
\hspace*{-0.25cm}
\begin{tikzpicture}
\path[left color=lightgray] (0,0) arc[radius=0.25, start angle =270, end angle =90] -- (10,0.5) |- (0,0);
\draw (0,0.25) node[right=-4pt] {\bfseries\color{red} Théorème#1 : \textit{\textmd{#2}}};
\end{tikzpicture}\nopagebreak[4]\par  \hspace*{-0.25cm} {\color{red}$\ulcorner$} \par \vspace{-0.25cm} }
{\hspace{\linewidth} {\color{red}  $\lrcorner$ }\par}

%___________________________
%===    Démonstration
%------------------------------------------------------
\NewEnviron{Demo}[1][]%
{\begin{tikzpicture}
\node[fill=gray!10,rounded corners=2pt,anchor=south west] (illus) at (0,0)
{\hfill \textbf{\textcolor{ForestGreen!50}{Démonstration#1}}};
\node[fill=gray!10,rounded corners=2pt,anchor=north west]at(0,0)
{\parbox{\linewidth}{\BODY \par 
\hfill$\square$}};
\end{tikzpicture}
\medskip
}

%___________________________
%===    Exemples
%------------------------------------------------------
\newcounter{exemple}[chapter]
\NewEnviron{Exemple}%
{
\refstepcounter{exemple}
\psframebox[fillstyle=solid,fillcolor=Yellow!30,linewidth=0.4pt,linecolor=Yellow!30,linearc=0.05,cornersize=absolute]{

    \begin{minipage}{\linewidth}
\textit{Exemple~\theexemple~:} \par 
\BODY
\end{minipage}
\medskip
}
}



\NewEnviron{Exemple*}[1][]%
{
\psframebox[fillstyle=solid,fillcolor=Yellow!30,linewidth=0.4pt,linecolor=Yellow!30,linearc=0.05,cornersize=absolute]{

    \begin{minipage}{\linewidth}
\textit{Exemple#1:} \par 
\BODY
\end{minipage}
\medskip
}
}

%___________________________
%===    Méthodes
%------------------------------------
\NewEnviron{Methode}[1] []%
{\begin{bclogo}[noborder=true, arrondi = 0.1, logo = \bccrayon, barre = snake, tailleOndu=2,marge=0]{\normalsize Méthode#1}
   \BODY
\end{bclogo}
\medskip
}%

%___________________________
%===    Remarques
%------------------------------------------------------

\NewEnviron{Rmq}[1] []%
{\begin{bclogo}[noborder=true, arrondi = 0.1, logo = , barre = snake, tailleOndu = 2 ,marge=0]{\normalsize Remarque#1 :}
   \BODY
\end{bclogo}
\medskip
}%

%___________________________
%===   Exercices
%------------------------------------------------------
\NewEnviron{Exo}[1][]
{\textbf{Exercice~#1 :} \par
\BODY
\medskip
}
\newcounter{exos}[chapter]
\NewEnviron{Exercice}[1][]
{
\refstepcounter{exos}
\textbf{Exercice~\theexos :} ~#1 \par
\BODY
\medskip
}

%===   Commandes
%------------------------------------------------------

\newcommand{\Fiche}[2]{%
\begin{tikzpicture}
	\node[draw, color=blue,fill=white,rectangle,rounded corners=5pt]{%
	\begin{minipage}{\linewidth}
		\begin{center}
			\vspace*{9pt}
			\textcolor{blue}{\Large \textsc{\textbf{Fiche~#1 :}}}\par
			\textcolor{blue}{\Large \textsc{\textbf{#2}}}
			\vspace*{7pt}
		\end{center}
	\end{minipage}
	};
\end{tikzpicture}
}%

\newcommand{\Livre}[1]{%
\begin{bclogo}[noborder=true, arrondi = 0.1, logo =\bcbook , barre = none , tailleOndu = 2 ,marge=0]{\normalsize Dans le livre :}
   #1
\end{bclogo}
\medskip
}%

%Dans un repère
\newcommand{\Dsrepere}[1]{%
\begin{bclogo}[noborder=true, arrondi = 0.1, logo =\bccrayon , barre = line , tailleOndu = 0 ,marge=0]{\normalsize Dans un repère :}
   #1
\end{bclogo}
\medskip
}%

\newcommand\voc[1]{\textbf{#1}\xspace}

%\dominitoc %pour pouvoir créer un sommaire du chapitre en cours avec \minitoc

%produit saclaire 
\newcommand{\Pdtscalaire}[2]{$\vect{#1} \cdot \vect{#2}$}
\input{commandes_2015}

\newgeometry{includeheadfoot,headsep=0.2cm,top=0.5cm,bottom=0.8cm,right=1.5cm,left=1.5cm}
\pagestyle{fancy}
\fancyhf{}


\begin{document}
\ReglePied

\fancypagestyle{garde}{
\pieddepage{2015-2016 - CLA}{}{\thepage / \pageref{LastPage}}}

\pieddepage{2015-2016 - CLA}{Thème 7 : Statistiques }{\thepage / \pageref{LastPage}}



\chapter{Thème 7 \\ Statistiques}
\thispagestyle{garde}


\bigskip

\section{Vocabulaire}
\begin{Defi}[s]{}
    Une \voc{étude statistique} a pour but d'obtenir une information, appelée \voc{caractère}, sur une population à partir de \voc{données} recueillies sur un \voc{échantillon} de cette population.

    Le caractère étudié peut être :
        \begin{description}
            \item[quantitatif :] les valeurs du caractère s'expriment avec des nombres (ex : températures, pointures, salaires\ldots) ;
            \item[qualitatif :] les valeurs ne s'expriment pas par des nombres (ex : couleurs, type d'essence\ldots) ;
            \item[discret :] les valeurs du caractère sont isolés (ex : notes\ldots) ;
            \item[continu :] les valeurs sont regroupées par classes (ou intervalles de nombre) (par ex : durée, distance parcourue \ldots).
        \end{description}
\end{Defi}

\textbf{Pour la suite du cours,} $S$ une série statistique à une variable de type quantitatif.




\section{Fréquences, effectifs et fréquences cumulés}
\begin{Defi}[s]{}
    Soit $a$ une valeur d'une série statistique $S$.\par
    
    La \voc{fréquence} de $a$ est le quotient de l'effectif de $a$ par l'effectif total. On peut l'exprimer en pourcentages.
    
    L'\voc{effectif cumulé croissant} associé à $a$ est la somme des effectifs de toutes les valeurs inférieures ou égales à $a$ dans la série $S$, c'est-à-dire le nombre d'individus de la population pour lesquels le caractère étudié a une valeur inférieure ou égale à $a$.\medskip
    
     La \voc{fréquence cumulée croissante} associée à $a$ est la somme des fréquences de toutes les valeurs inférieures ou égales à $a$ dans la série $S$.
\end{Defi}

\uline{\textbf{Exemple :}}

Un distributeur automatique de café propose des expressos. Une pesée sur 30 expressos a donné les masses suivantes (en grammes) de café utilisé.

\begin{center}
    \begin{tabular}{|*{10}{c|}}
        \hline
            81 & 82 & 85 & 83 & 83 & 82 & 87 & 84 & 85 & 84 \\
        \hline
            84 & 81 & 83 & 86 & 84 & 80 & 80 & 79 & 87 & 85 \\
        \hline
            81 & 82 & 85 & 87 & 79 & 80 & 86 & 89 & 83 & 89 \\
        \hline
    \end{tabular}
\end{center}

Compléter le tableau suivants :

\begin{center}
    \begin{tabular}{|>\raggedright m{3cm}|*{11}{c|}}
        \hline
            Masse en g & 79 & 80 & 81 & 82 &83 &84 & 85 & 86 & 87 & 88 & 89 \\
        \hline
            Effectif &  &  & &  &  &  &  &  &  &  &  \\
           %  Effectif & 2 & 3 & 3 & 3 & 4 & 4 & 4 & 2 & 3 & 0 & 2 \\
        \hline
         ECC$^*$ &  &  & &  &  &  &  &  &  &  &  \\
          \hline
        Fréquences en \% &  &  & &  &  &  &  &  &  &  &  \\
         \hline
          FCC$^{**}$ &  &  & &  &  &  &  &  &  &  &  \\
          \hline
    \end{tabular}
    
    
\end{center}
\hspace{3cm} $^*$ \textit{Effectifs cumulés croissants} 
    
\hspace{3cm} $^{**}$ \textit{Fréquences Cumulées Croissantes}

\section{Représentation d'une série statistique}
\subsection{Diagramme en bâtons}

{\itshape Le diagramme en bâton permet d'avoir un aspect visuel du tableau précédent. Il est constitué de segments de droite verticaux dont les hauteurs sont égales aux effectifs ou aux fréquences de chaque valeur. Sur l'axe des abscisses sont reportées les valeurs de la série.}

\uline{\textbf{Exemple :}}

Représenter le diagramme en bâtons de la distribution des effectifs de l'exemple précédent.

\begin{center}
    \begin{tikzpicture}[>=latex',scale=0.75]
    \draw[<->] (0,5) node[above] {\footnotesize Effectif} |- (12,0) node[below=12pt] {\footnotesize Masse en g};
    \foreach \x in {1,...,4} \draw[line width=0.2pt, dashed] (0,\x) node[left] {\small $\x$} -- (12,\x);
    \foreach \x in {79,80,...,89} \draw ({\x-78},0) node[below] {\small $\x$};
    %\foreach \x/\y in {79/2,80/3,81/3,82/3,83/4,84/4,85/4,86/2,87/3,88/0,89/2} \draw[red, line width=1pt] ({\x-78},0) -- ({\x-78},\y);
    \end{tikzpicture}
\end{center}

\subsection{Diagramme circulaire}



{\itshape Le diagramme circulaire permet de comparer les proportions des différentes valeurs.\par
On utilise la proportionnalité pour connaître la mesure de l'angle. }


\uline{\textbf{Exemple :}}

Le tableau suivant donne la répartition des chômeurs en 2013 en fonction de leur tranche d'âge :
\begin{center}
    \begin{tabular}{|>\raggedright m{6cm}|c|c|c|c|}
        \hline
            Âge en année entre & 15 et 24 & 25 et 49 & 50 et 64 & 65 et + \\
        \hline
            Taux de chômage en \% & 23,2 & 58,7 & 17,8 & 0,3 \\
        \hline
         & &  & &  \\
        \hline
    \end{tabular}
\end{center}

Représenter le diagramme circulaire associé au tableau précédent.

\newpage

\section{Caractéristique de dispersion}
\begin{Defi}{}
    L'\voc{étendue} de $S$ est la différence entre la plus grande et la plus petite des valeurs de $S$.
\end{Defi}


\section{Caractéristiques de position}




\subsection{Calculer une moyenne}

\begin{Defi}{}
    Les valeurs de $S$ sont notées $x_1, x_2, \ldots, x_p$ d'effectif respectif $n_1$, $n_2, \dots, n_p$.\par
    L'effectif total est noté $N$ ($N = n_1 + n_2 + \dots n_p$).\par
    La \voc{moyenne} de $S$, notée $\overline{x}$, est donnée par :
    \[\overline{x} = \frac{n_1x_1 + n_2x_2 + n_3x_3 + \cdots + n_px_p}{N}.\]

    Lorsque la série $S$ est définie par des valeurs $x_1, x_2, \ldots, x_p$ de fréquence respective $f_1$, $f_2, \dots, f_p$, alors la moyenne est donnée par :
    \[\overline x = f_1x_1 + f_2x_2 + \dots + f_px_p.\]
\end{Defi}


\subsection{Calculer une médiane}
\begin{Defi}{}
    \textbf{Une} \voc{médiane} $M$ est un nombre réel tel qu'au moins la moitié ($50\%$) des valeurs de la série sont inférieures ou égales à $M$.
\end{Defi}


\begin{center}
    \begin{tikzpicture}
        \draw (0,0)--(5,0) node[midway] {|} node[midway,above=5pt] {$M$};
        \draw (0,0) node {|} node[above=2pt] {\scriptsize $\min$}; \draw (5,0) node {|} node[above=2pt] {\scriptsize $\max$};
%        \draw[color=white,decorate,decoration={brace,raise=0.25cm}] (2.45,0) -- (0.05,0) node[below=0.5cm,pos=0.5] {au moins $50\%$};
%        \draw[color=white,decorate,decoration={brace,raise=0.25cm}] (4.95,0) -- (2.55,0) node[below=0.5cm,pos=0.5] {au moins $50\%$};
    \end{tikzpicture}
\end{center}

\subsection{Quartiles}
\begin{Defi}[s]{}
    Soit $S$ une série statistique à une variable quantitative discrète ordonnée dans l'ordre croissant.
        \begin{itemize}
            \item Le \voc{premier quartile} $Q_1$ de $S$ est le plus petit élément $a$ de $S$ tel qu'au moins $25\%$ des données soient inférieures ou égales à $a$.
            \item Le \voc{troisième quartile} $Q_3$ de $S$ est le plus petit élément $b$ de $S$ tel qu'au moins $75\%$ des données soient inférieures ou égales à $b$.
        \end{itemize}
\end{Defi}


\begin{center}
    \begin{tikzpicture}
        \begin{small}
        \draw (0,0)--(5,0) node[midway] {|} node[midway,above=5pt] {$M$} node[pos=0.75,above=5pt] {$Q_3$} node[pos=0.25,above=5pt] {$Q_1$};
        \draw (0,0)--(5,0) node[pos=0.25] {|}; \draw (0,0)--(5,0) node[pos=0.75] {|};
        \end{small}
        \begin{scriptsize}
        \draw (0,0) node {|} node[above=2pt] {$\min$}; \draw (5,0) node {|} node[above=2pt] {$\max$};
%        \draw[color=white,decorate,decoration={brace,raise=0.25cm}] (1.2,0) -- (0.05,0) node[below=0.4cm,pos=0.5] {au moins $25\%$};
%        \draw[color=white,decorate,decoration={brace,raise=0.25cm}] (4.95,0) -- (3.8,0) node[below=0.4cm,pos=0.5] {au moins $25\%$};
%        \draw[color=white,decorate,decoration={brace,raise=0.25cm}] (3.7,0) -- (1.3,0) node[below=0.4cm,pos=0.5] {au moins $50\%$};
%        \draw[color=white,decorate,decoration={brace,raise=0.75cm}] (3.7,0) -- (0.05,0) node[below=0.9cm,pos=0.5] {au moins $75\%$};
        \end{scriptsize}
    \end{tikzpicture}
\end{center}

\end{document}

