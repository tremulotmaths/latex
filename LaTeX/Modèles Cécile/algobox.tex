\documentclass[12pt,french]{report}
\input preambule_17-18
\input{perso_cecile_2017}
\input{commandes_17-18}

\usepackage{algobox}


\newgeometry{includeheadfoot,headsep=0.2cm,top=0.5cm,bottom=0.8cm,right=1.5cm,left=1.5cm}
\pagestyle{fancy}
\fancyhf{}

\ReglePied

\fancypagestyle{garde}{
\entete{\small\textbf{Nom,  Prénom :} \makebox[5cm]{\dotfill}} {}{\small \textbf{2nde9} }

\pieddepage{2017-2018 }{}{\thepage / \pageref{LastPage}}}

\pieddepage{2017-2018 - 2nde9}{DS 1 }{\thepage / \pageref{LastPage}}



\newsavebox\algorithme



\begin{document}


\begin{lrbox}{\algorithme}
\begin{minipage}{0.5\linewidth}
\begin{verbatim}
1     LIRE a
2     LIRE b
3    c PREND_LA_VALEUR (a+1)*(a+1)
4    d PREND_LA_VALEUR 3*b
5    r PREND_LA_VALEUR (c-d)/2
6    AFFICHER r
\end{verbatim}
\end{minipage}
\end{lrbox}

\begin{question}[subtitle={(4 points)}]
On considère l'algorithme suivant extrait de algobox :

\begin{center}
\usebox\algorithme
\end{center}

\begin{algobox}

\;   \DEBUTALGORITHME
\;   \LIRE a
\;    \LIRE b
\;     c \PRENDLAVALEUR (a+1)*(a+1)
\;     d \PRENDLAVALEUR 3*b
\;    r \PRENDLAVALEUR (c-d)/2
\;    \ AFFICHER r
\;  \FINALGORITHME
\end{algobox}

\begin{enumerate}[label=\arabic*.]
	\item Indiquer sur votre copie les variables à déclarer.
	\item Déterminer dans ces instructions la ou les lignes correspondant à chacune des trois étapes suivantes : entrée, traitement , sortie.
	
\item  Proposer un algorithme qui fournit le même résultat mais en utilisant au maximum 3 variables.

\end{enumerate}
\end{question}

\begin{algobox}

\;   \DEBUTALGORITHME
\;   \LIRE a
\;    \LIRE b
\;     c \PRENDLAVALEUR (a+1)*(a+1)
\;     d \PRENDLAVALEUR 3*b
\;    r \PRENDLAVALEUR (c-d)/2
\;    \ AFFICHER r
\;  \FINALGORITHME
\end{algobox}

%\printsolutions[byID={1 }]

\end{document}
