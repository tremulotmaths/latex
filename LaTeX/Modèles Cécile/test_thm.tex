\documentclass[french,12pt]{report}
\input{preambule_2015}
\input{commandes_2015}

\usepackage{linegoal}
\newlength{\Long}

\newenvironment{Thm}[2][]{%
%avant
\hspace*{-0.25cm}
\begin{tikzpicture}
\path[left color=lightgray] (0,0) arc[radius=0.25, start angle =270, end angle =90] -- (10,0.5) |- (0,0);
\draw (0,0.25) node[right=-4pt] {\bfseries\color{midblue} Théorème#1 : \textit{\textmd{#2}}};
\end{tikzpicture}\nopagebreak[4]\par\smallskip}
%après
{\setlength{\Long}{\linegoal}\addtolength{\Long}{-9pt}
\tikz{\path[left color=white, right color=lightgray] (0,0) -- (\Long,0) arc[radius=0.15, start angle = -90, end angle = 90] -| (0,0);}
\par\bigskip}




\begin{document}
\begin{Thm}{}
Théorème court.
\end{Thm}


\begin{Thm}{}
Théorème un peu plus long avec un peu plus de texte qui remplit presque une ligne.
\end{Thm}


\begin{Thm}{}
Théorème un peu plus long avec un peu plus de texte qui remplit complètement une ligne tout entière.
\end{Thm}

\begin{Thm}{}
Un théorème très très long avec plein de trucs hyper intéressant écrit dedans comme par exemple une liste d'hypothèses que l'on écrit ci-dessous :
\begin{itemize}
    \item hyp1
    \item hyp2
\end{itemize}

Et alors voilà l'énoncé du théorème qui peut être très long aussi avec plein de trucs intéressant et généralement ponctué par une conclusion à apprendre par c{\oe}ur.
\end{Thm}

\begin{Thm}{}
Un théorème très très long avec plein de trucs hyper intéressant écrit dedans comme par exemple une liste d'hypothèses que l'on écrit ci-dessous :
\begin{itemize}
    \item hyp1
    \item hyp2
\end{itemize}

Et alors voilà l'énoncé du théorème qui peut être très long aussi avec plein de trucs intéressant et généralement ponctué par une conclusion mathématique à apprendre par c{\oe}ur :
\[\int_a^b f(x) dx = F(b) - F(a)\]
\end{Thm}

\end{document}