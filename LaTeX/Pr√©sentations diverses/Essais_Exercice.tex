\documentclass[10pt,french]{book}

\input preambule_2013

\entete{\today}{Avez-vous bien appris ?}{Sujet A}
\RegleEntete

\pieddepage{\seconde $\np{1356}$}{Page \thepage\ sur un total de \pageref{LastPage}.}{Prix : \EUR{$\np{35468}$}}

\usepackage{skull} % pour utiliser la commande \skull -> voir à la fin

\newenvironment{exoB}[1]%
{\refstepcounter{exo}
$\triangleright$ \textbf{Exercice \theexo.} [#1]\par}%
{\[*\]}

\newenvironment{exoC}[1]%
{\refstepcounter{exo}
\hspace*{\stretch{1}}\textbf{Exercice \theexo.} [#1]\hspace*{\stretch{1}}\par}%
{\[*\]}

\newenvironment{exoD}[1]%
{\refstepcounter{exo}
\hspace*{\stretch{1}}\textbullet \textbf{Exercice \theexo.} [#1]\hspace*{\stretch{3}}\par}%
{\[*\]}

\newenvironment{exoE}[1]%
{\refstepcounter{exo}
{\fontfamily{augie}\fontsize{10}{8}\selectfont Exercice \theexo. [#1]}\par}%
{\[*\]}

\newenvironment{exoF}[1]%
{\refstepcounter{exo}
\psovalbox{\textbf{Exercice \theexo.}}\hfill \psovalbox{#1}\par}%
{\[*\]}

\newenvironment{exoG}[1]%
{\refstepcounter{exo}
\pfr{\textbf{Exercice \theexo.}} [#1]\par}%
{\[*\]}

\newenvironment{exoH}[1]%
{\refstepcounter{exo}
\begin{tikzpicture}[x=1mm,y=1mm]
\draw[inner color=red,outer color=yellow](0,0)ellipse(10 and 5) node {\color{blue}\textbf{Exercice \theexo.}};
\draw[ball color=white]++(17,0)circle(5) node{\small\textbf{#1}} ;
\end{tikzpicture}\par}
{\[*\]}

\begin{document}

\begin{center}
\psframebox[shadow=true,shadowcolor=gray!75,shadowsize=5pt,% gestion de l'ombre
framearc=-0.5,%arrondi de l'encadrement valeur=0 : rectangle. 0 < valeur < 1 : coin de + en + arrondi. valeur >=1 : demi-cercle. Valeur négative : voir résultat.
fillstyle=gradient,gradmidpoint=0.8,gradangle=20,gradbegin=red!80!yellow!40,gradend= white]{% gestion du dégradé
\parbox[c][30pt]{0.5\linewidth}{% c = texte centré dans la boîte, 30pt = hauteur de la boîte, 0.5\linewidth = longueur de la boîte
\begin{center}
\Large\bfseries
\uuline{Devoir sur table}
\end{center}}}
\end{center}\bigskip

\begin{exo}
    Racontez vos vacances... Mais uniquement la partie intéressante.
\end{exo}

\begin{exoB}{3 pts}
    Expliquez en 20 lignes pourquoi votre professeur de maths est vraiment le meilleur.
\end{exoB}

\begin{exoC}{3 pts}
    C'est en forgeant que l'on devient forgeron. Doit-on jouer dans une fanfare pour devenir fanfaron ?
\end{exoC}

\begin{exoD}{3 pts}
    Le menuiser a-t-il souvent la gueule de bois ?
\end{exoD}

\begin{exoE}{3 pts}
    Considérons un fan de sudoku. Que possède-t-il au nord ?
\end{exoE}

\begin{exoF}{3 pts}
    Une femme qui s'épile est-elle l'opposé d'une femme qui s'efface ?
\end{exoF}

\begin{exoG}{3 pts}
    Est ce que Pikachu s'appelle Pikachu parce qu'il n'arrête pas de dire Pikachu, ou est ce qu'il n'arrête pas de dire Pikachu parce qu'il s'appelle Pikachu? 
\end{exoG}

\begin{exoH}{3 pts}
    Quelle note espérez-vous obtenir à ce contrôle ?
\end{exoH}

{\Large\[\skull \qquad \skull \qquad \skull\]}

\end{document} 