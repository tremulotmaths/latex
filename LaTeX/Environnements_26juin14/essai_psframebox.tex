\documentclass[french,10pt]{book}
\input preambule_2014
%\documentclass[french,10pt]{book}
%\input preambule_2014

\usepackage{xargs}
\usepackage{environ}
%___________________________
%===    Redéfinition des marges
%------------------------------------------------------
%
%\usepackage[textwidth=18.6cm]{geometry}
\usepackage{geometry}
%\geometry{textwidth=18.6cm}

%\geometry{margin=1.5cm}
%\pagestyle{fancy}
%\fancyhf{}
%___________________________
%===    Redéfinition de la commande \chapter{•}
%------------------------------------------------------
%
\makeatletter

\renewcommand{\@makechapterhead}[1]{
\begin{tikzpicture}
\node[draw, color=blue,fill=white,rectangle,rounded corners=5pt]{%
\begin{minipage}{\linewidth}
\begin{center}
\vspace*{9pt}
\textcolor{blue}{\Large \textsc{\textbf{Chapitre \thechapter \ :}}} \par
\textcolor{blue}{\Large \textsc{\textbf{ \ #1}}}
\vspace*{7pt}
\end{center}
\end{minipage}
};\end{tikzpicture}
}

%___________________________
%===    Redéfinition des sections
%------------------------------------------------------
%

        \makeatletter
        \newcommand{\sectioncolor}{blue} %Couleur titre de section
        \newcommand{\ssectioncolor}{MidnightBlue} %Couleur titre de sous-section
        \newcommand{\sssectioncolor}{RoyalBlue} %Couleur titre de sous-sous-section
        %
        %Coloration des titres
        %---------------------
        \renewcommand{\section}{%Commande définie dans le fichier article.cls
            \@startsection%
            {section}%
            {1}%
            {0pt}%
            {-3.5ex plus -1ex minus -.2ex}%
            {2.3ex plus.2ex}%
            {\color{\sectioncolor}\normalfont\Large\bfseries}} %Aspect du titre
        \renewcommand\subsection{%
            \@startsection{subsection}{2}
            {0.5cm}% %décalage horizontal
            {-3.5ex\@plus -1ex \@minus -.2ex}%
            {1ex \@plus .2ex}%
            {\color{\ssectioncolor}\normalfont\large\bfseries}}
        \renewcommand\subsubsection{%
            \@startsection{subsubsection}{3}
            {1cm}% %décalage horizontal
            {-3.25ex\@plus -1ex \@minus -.2ex}%
            {1ex \@plus .2ex}%
            {\color{\sssectioncolor}\normalfont\normalsize\bfseries}}
        %

\setcounter{secnumdepth}{3}


 %___________________________
%===    Redéfinition des numérotation des paragraphes
%------------------------------------------------------
%
\renewcommand\thesection{\Roman{section}}
\renewcommand\thesubsection{\arabic{subsection}.}
\renewcommand\thesubsubsection{\alph{subsubsection}.}
%

%_________________________
%===    Environnements de cours
%------------------------------------------------------



%___________________________
%===    Définitions
%------------------------------------------------------

%\newcounter{defi}[chapter]
\NewEnviron{Defi}[2][]%
{\medskip
\begin{tikzpicture}
\node[draw,color=red,fill=white,rectangle,rounded corners=5pt ]
{\black\parbox{\linewidth}{\textcolor{red}{Définition#1 :} \textit{#2}\par \BODY}};
\end{tikzpicture}
}%
%ou différrence ? ... ds tous les cas pb si texte avt def cadre sort de la page

%\NewEnviron{Defi}[2][]%
%{\medskip
%\begin{tikzpicture}
%\node[draw, color=red,fill=white,rectangle,rounded corners=5pt]{%
%\begin{minipage}{\linewidth}
%\textcolor{red}{Définition#1 :} \textit{#2}\par \BODY
%
%\end{minipage}
%};\end{tikzpicture}
%}%
%___________________________
%===    Propriétés
%------------------------------------------------------
%\newcounter{propri}[chapter]
\NewEnviron{Prop}[2][]%
{
%\refstepcounter{propri}
\medskip
\begin{tikzpicture}
\node[draw,color=red,fill=white,rectangle,rounded corners=5pt ]
{\black\parbox{\linewidth}{\textcolor{red}{Propriété#1 :} \textit{#2}\par \BODY}};
\end{tikzpicture}
}%


%___________________________
%===    Théorèmes
%------------------------------------------------------
%\newcounter{thm}[chapter]
\NewEnviron{Thm}[2][]%
{
%\refstepcounter{thm}
\medskip
\begin{tikzpicture}
\node[draw,color=red,fill=white,rectangle,rounded corners=5pt ]
{\black\parbox{\linewidth}{\textcolor{red}{Théorème#1 :} \textit{#2}\par \BODY}};
\end{tikzpicture}
}%

%___________________________
%===    Démonstration
%------------------------------------------------------
\NewEnviron{Demo}[1][]%
{\begin{tikzpicture}
\node[fill=gray!10,rounded corners=2pt,anchor=south west] (illus) at (0,0)
{\hfill \textbf{\textcolor{ForestGreen!50}{Démonstration#1}}};
\node[fill=gray!10,rounded corners=2pt,anchor=north west]at(0,0)
{\parbox{\linewidth}{\BODY \par 
\hfill$\square$}};
\end{tikzpicture}
\medskip
}

%___________________________
%===    Exemples
%------------------------------------------------------
\newcounter{exemple}[chapter]
\NewEnviron{Exemple}%
{
\refstepcounter{exemple}
\psframebox[fillstyle=solid,fillcolor=Yellow!30,linewidth=0.4pt,linecolor=Yellow!30,linearc=0.05,cornersize=absolute]{

    \begin{minipage}{\linewidth}
\textit{Exemple~\theexemple~:} \par 
\BODY
\end{minipage}
\medskip
}
}



\NewEnviron{Exemple*}[1][]%
{
\psframebox[fillstyle=solid,fillcolor=Yellow!30,linewidth=0.4pt,linecolor=Yellow!30,linearc=0.05,cornersize=absolute]{

    \begin{minipage}{\linewidth}
\textit{Exemple#1:} \par 
\BODY
\end{minipage}
\medskip
}
}

%___________________________
%===    Méthodes
%------------------------------------
\NewEnviron{Methode}[1] []%
{\begin{bclogo}[noborder=true, arrondi = 0.1, logo = \bccrayon, barre = snake, tailleOndu=2,marge=0]{\normalsize Méthode#1}
   \BODY
\end{bclogo}
\medskip
}%

%___________________________
%===    Remarques
%------------------------------------------------------

\NewEnviron{Rmq}[1] []%
{\begin{bclogo}[noborder=true, arrondi = 0.1, logo = , barre = snake, tailleOndu = 2 ,marge=0]{\normalsize Remarque#1 :}
   \BODY
\end{bclogo}
\medskip
}%

%___________________________
%===   Exercices
%------------------------------------------------------
\NewEnviron{Exo}[1][]
{\textbf{Exercice~#1 :} \par
\BODY
\medskip
}
\newcounter{exos}[chapter]
\NewEnviron{Exercice}[1][]
{
\refstepcounter{exos}
\textbf{Exercice~\theexos :} ~#1 \par
\BODY
\medskip
}

%===   Commandes
%------------------------------------------------------

\newcommand{\Fiche}[2]{%
\begin{tikzpicture}
	\node[draw, color=blue,fill=white,rectangle,rounded corners=5pt]{%
	\begin{minipage}{\linewidth}
		\begin{center}
			\vspace*{9pt}
			\textcolor{blue}{\Large \textsc{\textbf{Fiche~#1 :}}}\par
			\textcolor{blue}{\Large \textsc{\textbf{#2}}}
			\vspace*{7pt}
		\end{center}
	\end{minipage}
	};
\end{tikzpicture}
}%

\newcommand{\Livre}[1]{%
\begin{bclogo}[noborder=true, arrondi = 0.1, logo =\bcbook , barre = none , tailleOndu = 2 ,marge=0]{\normalsize Dans le livre :}
   #1
\end{bclogo}
\medskip
}%

%Dans un repère
\newcommand{\Dsrepere}[1]{%
\begin{bclogo}[noborder=true, arrondi = 0.1, logo =\bccrayon , barre = line , tailleOndu = 0 ,marge=0]{\normalsize Dans un repère :}
   #1
\end{bclogo}
\medskip
}%
%\dominitoc %pour pouvoir créer un sommaire du chapitre en cours avec \minitoc

%produit saclaire 
\newcommand{\Pdtscalaire}[2]{$\vect{#1} \cdot \vect{#2}$}

\usepackage{manfnt}

\begin{document}
\renewcommand{\footrulewidth}{0.5pt}

\pieddepage{2014-2015 - 2nde}{Fiche 2 : Inéquations}{\thepage}

\Fiche{2}{Inéquations}

\setlength{\columnseprule}{0pt} % largeur de la ligne entre les deux colonnes

\begin{Exemple*}[s]
    \begin{multicols}{2}
        \begin{enumerate}
            \item Résoudre sur $\R$ $f(x) \geq -1$\par
                \begin{center}
                        \begin{tikzpicture}[scale=0.7,line cap=round,line join=round,>=triangle 45]
                            \draw [color=black,dash pattern=on 2pt off 2pt, xstep=0.7cm,ystep=0.7cm] (-3,-3.5) grid (4,2.5);
                            \draw[->,color=black] (-3,0.0) -- (4,0.0);
                            \foreach \x in {-2,-1,1,2,3}
                            \draw[shift={(\x,0)},color=black] (0pt,2pt) -- (0pt,-2pt) node[below] {\footnotesize $\x$};
                            \draw[->,color=black] (0.0,-3.5) -- (0.0,2.5);
                            \foreach \y in {-3,-2,-1,1,2}
                            \draw[shift={(0,\y)},color=black] (2pt,0pt) -- (-2pt,0pt) node[left] {\footnotesize $\y$};
                            \draw[color=black] (0pt,-10pt) node[right] {\footnotesize $0$};
                            \clip(-3,-3.5) rectangle (4,2.5);
                            \draw[smooth,samples=100,domain=-2.8198332081141952:5.71510293012773] plot(\x,{0.6666666666666666*(\x)^(3.0)-2.0*(\x)^(2.0)-0.6666666666666666*(\x)+1.0});
                            \draw [line width=1.5pt,color=red] (-1.0,0.0)-- (1.0,0.0);
                            \draw [line width=1.5pt,color=red,domain=3.0:5.71510293012773] plot(\x,{0});
                            \draw [line width=1.5pt,color=blue,domain=-2.8198332081141952:5.71510293012773] plot(\x,{-1});
                            \draw (2.5,2) node[anchor=north west] {$\calig{C}_f$};
                            \begin{scriptsize}
                                \draw [line width=1.2pt,color=black] (-1.0,-1.0)-- ++(-1.5pt,-1.5pt) -- ++(3.0pt,3.0pt) ++(-3.0pt,0) -- ++(3.0pt,-3.0pt);
                                \draw [line width=1.2pt,color=black] (1.0,-1.0)-- ++(-1.5pt,-1.5pt) -- ++(3.0pt,3.0pt) ++(-3.0pt,0) -- ++(3.0pt,-3.0pt);
                                \draw [line width=1.2pt,color=black] (3.0,-1.0)-- ++(-1.5pt,-1.5pt) -- ++(3.0pt,3.0pt) ++(-3.0pt,0) -- ++(3.0pt,-3.0pt);
                            \end{scriptsize}
                    \end{tikzpicture}
                \end{center}

On lit les abscisses des points de $\calig{C}_f$ d'ordonnée supérieure ou égale à $-1$\par
L'ensemble des solutions de $f(x) \geq -1$ est $\intervalleff{-1}{1} \cup \intervallefo{3}{+\infty}$.

            \item Résoudre sur $\R$ $f(x) \leq g(x)$. \par
                \begin{center}
                    \begin{tikzpicture}[scale=0.7,line cap=round,line join=round,>=triangle 45]
                        \draw [color=black,dash pattern=on 2pt off 2pt, xstep=0.7cm,ystep=0.7cm] (-3,-3.5) grid (4,2.5);
                        \draw[->,color=black] (-3,0.0) -- (4,0.0);
                        \foreach \x in {-2,-1,1,2,3}
                        \draw[shift={(\x,0)},color=black] (0pt,2pt) -- (0pt,-2pt) node[below] {\footnotesize $\x$};
                        \draw[->,color=black] (0.0,-3.5) -- (0.0,2.5);
                        \foreach \y in {-3,-2,-1,1,2}
                        \draw[shift={(0,\y)},color=black] (2pt,0pt) -- (-2pt,0pt) node[left] {\footnotesize $\y$};
                        \draw[color=black] (0pt,-10pt) node[right] {\footnotesize $0$};
                        \clip(-3,-3.5) rectangle (4,2.5);
                        \draw[smooth,samples=100,domain=-2.8198332081141952:5.71510293012773] plot(\x,{0.6666666666666666*(\x)^(3.0)-2.0*(\x)^(2.0)-0.6666666666666666*(\x)+1.0});
                        \draw [line width=1.5pt,color=red] (0,0.0)-- (3,0.0);
                        \draw [line width=1.5pt,color=red,domain=-3:-1] plot(\x,{(0});
                        \draw [smooth,samples=100,color=blue,domain=-2.8198332081141952:5.71510293012773] plot(\x,{-0.67*(\x)^(2)+1.33*(\x)+1});
                        \draw (2.5,2) node[anchor=north west] {$\calig{C}_f$};
                        \draw [color=blue] (2.7,-2.5) node[anchor=north west] {$\calig{C}_g$};
                            \begin{scriptsize}
                                \draw [line width=1.2pt,color=black] (-1.0,-1.0)-- ++(-1.5pt,-1.5pt) -- ++(3.0pt,3.0pt) ++(-3.0pt,0) -- ++(3.0pt,-3.0pt);
                                \draw [line width=1.2pt,color=black] (0,1)-- ++(-1.5pt,-1.5pt) -- ++(3.0pt,3.0pt) ++(-3.0pt,0) -- ++(3.0pt,-3.0pt);
                                \draw [line width=1.2pt,color=black] (3.0,-1.0)-- ++(-1.5pt,-1.5pt) -- ++(3.0pt,3.0pt) ++(-3.0pt,0) -- ++(3.0pt,-3.0pt);
                            \end{scriptsize}
                    \end{tikzpicture}
                \end{center}

        On lit les abscisses des points de $\calig{C}_f$ situés en dessous ou sur la courbe $\calig{C}_g$\par
        L'ensemble des solutions de $f(x) \leq g(x)$ est $\intervalleof{-\infty}{-1} \cup \intervalleff{0}{3}$.
        \end{enumerate}
    \end{multicols}
\end{Exemple*}



\begin{Exemple*}
    \begin{tikzpicture}[scale=0.7]
        \tkzTabInit[nocadre,espcl=1.5]{$x$/0.75,\small signe de $x-2$/1.3,\small signe de  $-x-3$/1.3,\small signe du  produit/1.3}{$-\infty$,$-3$,$2$,$+\infty$}
        \tkzTabLine{,-,t,-,z,+}
        \tkzTabLine{,+,z,-,t,-}
        \tkzTabLine{,-,z,+,z,-}
    \end{tikzpicture}
\end{Exemple*}

\begin{tikzpicture}[scale=0.7]
\tkzTabInit[nocadre,espcl=1.5]{$x$/0.75,\small signe de $x-2$/1.3,\small signe de  $-x-3$/1.3,\small signe du  produit/1.3}{$-\infty$,$-3$,$2$,$+\infty$}
\tkzTabLine{,-,t,-,z,+}
\tkzTabLine{,+,z,-,t,-}
\tkzTabLine{,-,z,+,z,-}
\end{tikzpicture}

\psframebox[fillstyle=solid,fillcolor=Yellow!30,linewidth=0.4pt,linecolor=Yellow!30,framearc=0.05]{
    \textit{Exemple :}
}
\par\vspace*{-0.25\baselineskip}
\psframebox[fillstyle=solid,fillcolor=Yellow!30,linewidth=0.4pt,linecolor=Yellow!30,framearc=0.05]{
    \begin{minipage}{\linewidth}
        \begin{tikzpicture}
            \tkzTabInit[nocadre,espcl=1.5]{$x$/0.75,\small signe de $x-2$/1.3,\small signe de  $-x-3$/1.3,\small signe du  produit/1.3}{$-\infty$,$-3$,$2$,$+\infty$}
            \tkzTabLine{,-,t,-,z,+}
            \tkzTabLine{,+,z,-,t,-}
            \tkzTabLine{,-,z,+,z,-}
        \end{tikzpicture}
    \end{minipage}
}



\psframebox[fillstyle=solid,fillcolor=Yellow!30,linewidth=0.4pt,linecolor=Yellow!30,]{

    \begin{minipage}{\linewidth}
    \textit{Exemple :}\par 
        \begin{tikzpicture}
            \tkzTabInit[nocadre,espcl=1.5]{$x$/0.75,\small signe de $x-2$/1.3,\small signe de  $-x-3$/1.3,\small signe du  produit/1.3}{$-\infty$,$-3$,$2$,$+\infty$}
            \tkzTabLine{,-,t,-,z,+}
            \tkzTabLine{,+,z,-,t,-}
            \tkzTabLine{,-,z,+,z,-}
        \end{tikzpicture}
    \end{minipage}
}


\psframebox[fillstyle=solid,fillcolor=Yellow!30,linewidth=0.4pt,linecolor=Yellow!30,framearc=0.05]{

    \begin{minipage}{\linewidth}
    \textit{Exemple :}\par 
Test arrondi en fct de la longueur du texte.Test arrondi en fct de la longueur du texte.Test arrondi en fct de la longueur du texte.Test arrondi en fct de la longueur du texte.Test arrondi en fct de la longueur du texte.Test arrondi en fct de la longueur du texte.Test arrondi en fct de la longueur du texte.Test arrondi en fct de la longueur du texte.Test arrondi en fct de la longueur du texte.Test arrondi en fct de la longueur du texte.
    \end{minipage}
}
\end{document} 