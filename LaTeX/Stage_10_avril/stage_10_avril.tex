\documentclass[12pt,french]{article}

\usepackage[frenchb]{babel}
\usepackage[latin1]{inputenc}
\usepackage[T1]{fontenc}
\usepackage{amsmath,amsfonts,amssymb}
\usepackage{tabularx}
\usepackage{geometry}
\geometry{margin=1cm}
\usepackage{hyperref}
\usepackage{tikz,tkz-tab,tkz-graph,tikz-3dplot}
\usetikzlibrary{calc,shapes,arrows,plotmarks,lindenmayersystems,decorations,decorations.markings,decorations.pathmorphing,
decorations.pathreplacing,patterns,positioning}
\usepackage{pstricks,pst-plot,pst-text,pstricks-add,pst-eucl,pst-all}

%\setlength{\textwidth}{19cm}
%\setlength{\textheight}{27.5cm}
%\setlength{\voffset}{-2.5cm}
%\setlength{\parindent}{0pt}

\begin{document}


%%%%%%%%%%%%%%%%%%%%%%%%% TEXTE %%%%%%%%%%%%%%%%%%%%%%%
Lien utile : pour tableau variation avec code simple : \par 

\url{http://www.altermundus.fr/pages/downloads/TKZdoc-tab.pdf}

\vspace{1cm}

\textbf{Ecrire des �galit�s}\par 

\begin{minipage}{0.3\linewidth}
\begin{align*} %$ inutile
(a+b)^3&=(a+b)(a+b)^2\\
&=(a+b)(a+b)(a+b)
\end{align*}
\end{minipage}\hfill \hspace{2cm}
\begin{minipage}[t]{0.5\linewidth}
\begin{verbatim}
\begin{align*} %$ inutile
(a+b)^3&=(a+b)(a+b)^2\\
&=(a+b)(a+b)(a+b)
\end{align*}
\end{verbatim}
\end{minipage}


\begin{minipage}{0.3\linewidth}
\begin{eqnarray*} %$ inutile 3 colonnes alignement autour �gal
(a+b)^3&=&(a+b)(a+b)^2\\
&=&(a+b)(a+b)(a+b)\\
(a+b)(a+b)^2 &=& 
\end{eqnarray*}
\end{minipage}\hfill \hspace{2cm}
\begin{minipage}[t]{0.5\linewidth}
\begin{verbatim}
\begin{eqnarray*} 
%$ inutile 3 colonnes alignement autour �gal
(a+b)^3&=&(a+b)(a+b)^2\\
&=&(a+b)(a+b)(a+b)\\
(a+b)(a+b)^2 &=& 
\end{eqnarray*}
\end{verbatim}
\end{minipage}

\newpage

\textbf{Tableau de variations}\par 

\vspace{1cm}
\begin{minipage}{0.5\linewidth}
\begin{tikzpicture}
\tkzTabInit[nocadre,espcl=2]{$x$/0.75,signe de \\ $f(x)$/1.5,Variations de \\ $f(x)$/1.75}{$-3$,$-1$,$1$,$4$}
\tkzTabLine{2,+,z,-,z,+}
\tkzTabVar{-/$2$,+/$4$,-/$-3$,+/$10$}
	\end{tikzpicture}
\end{minipage} \hfill
\begin{minipage}{0.5\linewidth}
\begin{verbatim}
\begin{tikzpicture}
\tkzTabInit[nocadre,espcl=2]
{$x$/0.75,signe de \\ $f(x)$/1.5,variation de \\ 
$f(x)$/1.5}{$-3$,$-1$,$1$,$4$}
\tkzTabLine{2,+,z,-,z,+}
\tkzTabVar{-/$2$,+/$4$,-/$-3$,+/$10$}
	\end{tikzpicture}
\end{verbatim}
\end{minipage}



\begin{minipage}{0.4\linewidth}
\begin{tikzpicture}[scale=0.8]
\tkzTabInit[nocadre,espcl=2]{$x$/0.75,signe de \\ $f(x)$/1.5,Variations de \\ $f(x)$/1.5}{$-3$,$-1$,$1$,$4$}
\tkzTabLine{2,+,z,-,z,+}
\tkzTabVar{-/$2$,+/$4$,R,-/$10$}
	\end{tikzpicture}
\end{minipage} \hfill
\begin{minipage}{0.6\linewidth}
\begin{verbatim}
\begin{tikzpicture}[scale=0.8]
\tkzTabInit[nocadre,espcl=2] 
(espcl largeur des colonnes entre les valeurs de x)
{$x$/0.75,\footnotesize signe de \\  
(0.75 hauteur ligne)
$f(x)$/1.75,\footnotesize Variation   de \\ 
$f(x)$/1.75}{$-3$,$-1$,$1$,$4$}
\tkzTabLine{2,+,z,-,z,+}
(z pour z�ro avec pointill�s)
\tkzTabVar{-/$2$,+/$4$,R,-/$10$}
(- pour �crire en bas ; + pour �crire en ht ; 
R comme rien)
	\end{tikzpicture}
\end{verbatim}
\end{minipage}
	

\begin{minipage}{0.5\linewidth}
\begin{tikzpicture}
\tkzTabInit[nocadre,espcl=2]{$x$/0.75,signe de \\ $f(x)$/1.5,Variations de \\ $f(x)$/1.5}{$-3$,$-1$,$1$,$4$}
\tkzTabLine{t,+,z,-,d,+}
\tkzTabVar{-/$2$,+/$4$,-D-/$1$/$3$,+/$10$}
	\end{tikzpicture}
\end{minipage} \hfill
\begin{minipage}{0.5\linewidth}
\begin{verbatim}
\begin{tikzpicture}
\tkzTabInit[nocadre,espcl=2]{$x$/0.75,signe de \\ $f(x)$/1.5,Variations de \\ $f(x)$/1.5}{$-3$,$-1$,$1$,$4$}
\tkzTabLine{t,+,z,-,d,+}
\tkzTabVar{-/$2$,+/$4$,-D-/$1$/$3$,+/$10$}
	\end{tikzpicture}
\end{verbatim}
\end{minipage}

\newpage

\textbf{Tableau de signes}\par 
\vspace{1cm}
\begin{minipage}{0.5\linewidth}
\begin{tikzpicture}
\tkzTabInit[nocadre,espcl=2]{$x$/0.75,Signe de \\ $x+2$/1.5,Signe de \\ $x^2-1$/1.5,signe du \\ produit/1.5}{$-\infty$,$-2$,$-1$,$1$,$+\infty$}
\tkzTabLine{,-,z,+,t,+,t,+}
\tkzTabLine{,+,t,+,z,-,z,+}
\tkzTabLine{,-,z,+,z,-,z,+}
%\tkzTabline
	\end{tikzpicture}
\end{minipage} 
\begin{verbatim}
\begin{tikzpicture}
\tkzTabInit[nocadre,espcl=2]{$x$/0.75,Signe de \\ $x+2$/1.5,Signes de \\ $x^2-1$/1.5,signe du \\ produit/1.5}{$-\infty$,$-2$,$-1$,$1$,$+\infty$}
\tkzTabLine{,-,z,+,t,+,t,+}
\tkzTabLine{,+,t,+,z,-,z,+}
\tkzTabLine{,-,z,+,z,-,z,+}
%\tkzTabline
	\end{tikzpicture}
\end{verbatim}

\end{document}
