% Snowflakes with TikZ
% Author: JLDiaz
\documentclass{article}
\usepackage[textwidth=6cm]{geometry}% just for the preview
\usepackage{tikz}
%%%<
\usepackage{verbatim}
\usepackage[active,tightpage]{preview}
\PreviewEnvironment{center}
\setlength\PreviewBorder{10pt}%
%%%>
%\begin{comment}

%\end{comment}
\usetikzlibrary{lindenmayersystems}

\pgfdeclarelindenmayersystem{A}{
    \rule{F -> FF[+F][-F]}
}

\pgfdeclarelindenmayersystem{B}{
    \rule{F -> ffF[++FF][--FF]}
}

\pgfdeclarelindenmayersystem{C}{
    \symbol{G}{\pgflsystemdrawforward}
    \rule{F -> F[+F][-F]FG[+F][-F]FG}
}

\pgfdeclarelindenmayersystem{D}{
    \symbol{G}{\pgflsystemdrawforward}
    \symbol{H}{\pgflsystemdrawforward}
    \rule{F -> H[+HG][-HG]G}
    \rule{G -> HF}
}

\tikzset{
    type/.style={l-system={#1, axiom=F,order=3,step=4pt,angle=60},
      blue, opacity=0.4, line width=.5mm, line cap=round   
    },
}

\newcommand\drawsnowflake[2][scale=0.2]{
    \tikz[#1]
    \foreach \a in {0,60,...,300}  {
    \draw[rotate=\a,#2] l-system;
    };
}

\begin{document}
%\begin{center}
%Title: Snowflakes with TikZ
%Tags: Fractals;Decorative drawings
%Author: JLDiaz
%Slug: snowflakes
%
%One of the easiest approach for drawing snowflakes with TikZ
%is using Lindenmayer systems.
%
%This code defines Lindenmayer rules for drawing a single "arm"
%of the snowflake. Symmetry should be crucial. Then this arm is
%repeated rotated 60 degrees each time to produce the complete snowflake.
%
%By changing the Lindenmayer rules, the angle and the line width, you
%can get a lot of nice variations.
%
%You may notice some scaling, this is just for getting similar
%dimensions, because we have different line widths.
%
%This code was written by JLDiaz and published on TeX.SE.
%\end{center}

\begin{center}
\foreach \width in {.2,.4,...,.8} 
{  \drawsnowflake[scale=0.3]{type=A, line width=\width mm} }

\foreach \width in {.2,.4,...,.8} 
{  \drawsnowflake[scale=0.38]{type=A, l-system={angle=90}, line width=\width mm} }    

\foreach \width in {.2,.4,...,.8} 
{  \drawsnowflake[scale=0.3]{type=B, line width=\width mm} }

\foreach \width in {.2,.4,...,.8} 
{  \drawsnowflake{type=B, l-system={angle=30}, line width=\width mm} }

\drawsnowflake[scale=0.24]{type=C, l-system={order=2}, line width=0.2mm}
\drawsnowflake[scale=0.25]{type=C, l-system={order=2}, line width=0.4mm}
\drawsnowflake[scale=0.25]{type=C, l-system={order=2,axiom=fF}, line width=0.2mm}
\drawsnowflake[scale=0.32]{type=C, l-system={order=2,axiom=---fff+++F}, line width=0.2mm}

\drawsnowflake[scale=0.38]{type=D, l-system={order=4,angle=60,axiom=GF}, line width=0.7mm}
\drawsnowflake[scale=0.38]{type=D, l-system={order=4,angle=60,axiom=GfF}, line width=0.7mm}
\drawsnowflake[scale=0.38]{type=D, l-system={order=4,angle=60,axiom=FG}, line width=0.7mm}
\drawsnowflake[scale=0.38]{type=D, l-system={order=4,angle=60,axiom=FfG}, line width=0.7mm}
\end{center}
\end{document}