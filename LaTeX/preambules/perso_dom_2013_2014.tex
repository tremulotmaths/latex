\setlength{\oddsidemargin}{-0.5cm}% Marge gauche sur pages impaires


% Environnement enumerate
\renewcommand{\theenumi}{\bf\textsf{\arabic{enumi}}}
\renewcommand{\labelenumi}{\bf\textsf{\theenumi.}}
\renewcommand{\theenumii}{\bf\textsf{\alph{enumii}}}
\renewcommand{\labelenumii}{\bf\textsf{\theenumii.}}
\renewcommand{\theenumiii}{\bf\textsf{\roman{enumiii}}}
\renewcommand{\labelenumiii}{\bf\textsf{\theenumiii.}}



\usetikzlibrary{shadows,trees}


%definition des couleurs
\definecolor{fondpaille}{cmyk}{0,0,0.1,0}%\pagecolor{fondpaille}
\definecolor{gris}{rgb}{0.7,0.7,0.7}
\definecolor{rouge}{rgb}{1,0,0}
\definecolor{bleu}{rgb}{0,0,1}
\definecolor{vert}{rgb}{0,1,0}
\definecolor{deficolor}{HTML}{2D9AFF}
\definecolor{backdeficolor}{HTML}{EDEDED}%{036DD0}%dégradé bleu{666666}%dégradé gris
\definecolor{theocolor}{HTML}{036DD0}%F4404D%rouge
\definecolor{backtheocolor}{HTML}{D3D3D3}
\definecolor{methcolor}{HTML}{008800}%12BB05}
\definecolor{backmethcolor}{HTML}{FFFACD}
\definecolor{backilluscolor}{HTML}{EDEDED}
\definecolor{sectioncolor}{HTML}{B2B2B2}%vert : {HTML}{008800}%{HTML}{2D9AFF}
\definecolor{subsectioncolor}{HTML}{B2B2B2}%vert : {HTML}{008800}%{rgb}{0.5,0,0}
\definecolor{engcolor}{HTML}{D4D7FE}
\definecolor{exocolor}{rgb}{0,0.6,0}
\definecolor{exosoltitlecolor}{rgb}{0,0.6,0}
\definecolor{titlecolor}{rgb}{1,1,1}

%commande pour enlever les couleurs avant impression
\newcommand{\nocolor}
{\pagecolor{white}
\definecolor{gris}{rgb}{0.7,0.7,0.7}
\definecolor{rouge}{rgb}{0,0,0}
\definecolor{bleu}{rgb}{0,0,0}
\definecolor{vert}{rgb}{0,0,0}
\definecolor{deficolor}{HTML}{B2B2B2}
\definecolor{backdeficolor}{HTML}{EEEEEE}%{036DD0}%dégradé bleu{666666}%dégradé gris
\definecolor{theocolor}{HTML}{B2B2B2}
\definecolor{backtheocolor}{HTML}{EEEEEE}
\definecolor{methcolor}{HTML}{B2B2B2}
\definecolor{backmethcolor}{HTML}{EEEEEE}
\definecolor{backilluscolor}{HTML}{EEEEEE}
\definecolor{sectioncolor}{HTML}{B2B2B2}
\definecolor{subsectioncolor}{HTML}{B2B2B2}
\definecolor{engcolor}{HTML}{EEEEEE}
\definecolor{exocolor}{HTML}{3B3838}
\definecolor{exosoltitlecolor}{rgb}{0,0,0}
\definecolor{titlecolor}{rgb}{0,0,0}
}




%Exercices résolus dans le cours
%#1 : énoncé
%#2 : solution
\newcounter{exosol}
\newcommand{\exosol}[2]{
\stepcounter{exosol}
\begin{tikzpicture}[node distance=0 cm]
\node[fill=backilluscolor,rounded corners=2pt,anchor=south west] (illus) at (0,-0.02)
{\it \textbf{\textcolor{exosoltitlecolor}{Exercice résolu \arabic{exosol}~:~}}};
\node[fill=backilluscolor,rounded corners=2pt,anchor=north west]at(0,0)
{\parbox{\columnwidth-10pt}{#1\par\medskip{\it \textbf{\textcolor{exosoltitlecolor}{Solution~:~}}}#2 }};
\end{tikzpicture}
\bigskip
}

\newcommand{\suite}[1]{
\begin{tikzpicture}[node distance=0 cm]
\node[fill=backilluscolor,rounded corners=2pt,anchor=north west]at(0,0)
{\parbox{\columnwidth-10pt}{{\it \textbf{\textcolor{exosoltitlecolor}{Suite de la solution~:}}}\par#1}};
\end{tikzpicture}
\bigskip
}



%%%%%%%%%%%%%%%%%%%%%%%%%%%%%%%%%%%%%%%%%%%%%%%%%%%%%%%%%%%%%%%%%%%%%%%%%%%%%%%
%Encadrés pour Propriétés, Théorème, Définitions, exemples, exercices

\usepackage{environ}

%\newcounter{propr}
%\newenvironment{propr}{\refstepcounter{propr}\begin{bclogo}[%noborder=true , 
%couleur = white , arrondi = 0.1 ,logo = \bccrayon , barre = snake , tailleOndu = 1.5 , ombre = true , epOmbre = 0.15 , couleurOmbre = black!30]{Propriété \thepropr}}{\end{bclogo}}

\NewEnviron{Prop}[1][]{
\begin{tikzpicture}[node distance=0 cm]
\node[fill=theocolor,rounded corners=5pt,anchor=south west] (theorem) at (0,0)
{\textcolor{titlecolor}{Propriété~:~#1}};
\node[draw,drop shadow,color=theocolor,very thick,fill=backtheocolor,rounded corners=5pt,anchor=north west] at(0,-0.02)
{\black\parbox{\columnwidth-12pt}{\BODY}};
\end{tikzpicture}
\bigskip
}

%\newcommand{\propr}[2]{
%\begin{tikzpicture}[node distance=0 cm]
%\node[fill=theocolor,rounded corners=5pt,anchor=south west] (theorem) at (0,0)
%{\textcolor{titlecolor}{Propriété~:~#1}};
%\node[draw,drop shadow,color=theocolor,very thick,fill=backtheocolor,rounded corners=5pt,anchor=north west] at(0,-0.02)
%{\black\parbox{\columnwidth-12pt}{#2}};
%\end{tikzpicture}
%\bigskip
%}

%\newenvironment{propr}[1][]{\begin{bclogo}[noborder=true , 
%couleur = white , arrondi = 0.1 , barre = none , ombre = true , epOmbre = 0.15 , couleurOmbre = black!30]{Propriété {\bf \xspace : #1}}}{\end{bclogo}}
%
%
%\newcounter{theo}
%\newenvironment{theo}{\refstepcounter{theo}\begin{bclogo}[%noborder=true , 
%couleur = white , arrondi = 0.1 ,logo = \bcbook , barre = snake , tailleOndu = 1.5 , ombre = true , epOmbre = 0.15 , couleurOmbre = orange]{Théorème \thetheo}}{\end{bclogo}}

%Théorème
\NewEnviron{Thm}[1][]{
\begin{tikzpicture}[node distance=0 cm]
\node[fill=theocolor,rounded corners=5pt,anchor=south west] (theorem) at (0,0)
{\textcolor{titlecolor}{Théorème~:~#1}};
\node[draw,drop shadow,color=theocolor,very thick,fill=backtheocolor,rounded corners=5pt,anchor=north west] at(0,-0.02)
{\black\parbox{\columnwidth-12pt}{\BODY}};
\end{tikzpicture}
\bigskip
}



%\newcounter{defi}
%\newenvironment{defi}{\refstepcounter{defi}\begin{bclogo}[%noborder=true , 
%couleur = white , arrondi = 0.1 ,logo = \bcinfo , barre = snake , tailleOndu = 1.5 , ombre = true , epOmbre = 0.15 , couleurOmbre = blue]{Définition \thedefi}}{\end{bclogo}}
\NewEnviron{Defi}[1][]{
\begin{tikzpicture}[node distance=0 cm]
\node[fill=theocolor,rounded corners=5pt,anchor=south west] (theorem) at (0,0)
{\textcolor{titlecolor}{Définition~:~#1}};
\node[draw,drop shadow,color=deficolor,very thick,fill=backdeficolor,rounded corners=5pt,anchor=north west] at(0,-0.02)
{\black\parbox{\columnwidth-12pt}{\BODY}};
\end{tikzpicture}
\bigskip
}

%\newcounter{methode}
%\newenvironment{methode}{\refstepcounter{methode}\begin{bclogo}[%noborder=true , 
%couleur = white , arrondi = 0.1 ,logo = \bccrayon , barre = snake , tailleOndu = 1.5 , ombre = true , epOmbre = 0.15 , couleurOmbre = red]{Méthode \themethode}}{\end{bclogo}}
\NewEnviron{Methode}[1][]{
\begin{tikzpicture}[node distance=0 cm]
\node[fill=theocolor,rounded corners=5pt,anchor=south west] (theorem) at (0,0)
{\textcolor{titlecolor}{Méthode~:~#1}};
\node[draw,drop shadow,color=methcolor,very thick,fill=backmethcolor,rounded corners=5pt,anchor=north west] at(0,-0.02)
{\black\parbox{\columnwidth-12pt}{\BODY}};
\end{tikzpicture}
\bigskip
}


%chapitres
\makeatletter

\renewcommand{\@makechapterhead}[1]{
\begin{tikzpicture}
\node[fill=theocolor,rectangle,rounded corners=5pt]{%
\begin{minipage}{\linewidth}
\begin{center}
\vspace*{9pt}
\textcolor{titlecolor}{\Large \textsc{\textbf{Chapitre \thechapter \ : \ #1}}}
\vspace*{9pt}
\end{center}
\end{minipage}
};\end{tikzpicture}
}

\makeatother

\NewEnviron{Exemple}[1][]{
\begin{tikzpicture}[node distance=0 cm]
\node[draw,drop shadow,color=methcolor,very thick,fill=backmethcolor,rounded corners=5pt,anchor=north west] at(0,-0.02)
{\black\parbox{\columnwidth-12pt}{\textbf{Exemple~:~#1}\\
\BODY}};
\end{tikzpicture}
\bigskip
}

\newcounter{exemple}\newcommand{\exemple}{\refstepcounter{exemple}\textbf{Exemple \theexemple \ :}\xspace}
\newcounter{exercice}\newcommand{\exercice}{\refstepcounter{exercice}\textbf{\large{Exercice \theexercice \ :}}\xspace}

\newcounter{probleme}\newcommand{\probleme}{\refstepcounter{probleme}\textbf{\large{Problème \theprobleme \ :}}\xspace}
\newcounter{remarque}\newcommand{\remarque}{\refstepcounter{remarque}\textbf{Remarque \theremarque \ :}\xspace}

\NewEnviron{Rmq}{
\textbf{\large{Remarque :}}\par
\BODY
\bigskip
}

\newcounter{rem}\newcommand{\rem}{\refstepcounter{rem}\textbf{R \therem \ :}\xspace}

%Exercices sans numérotation automatique
%\newcommand{\Exercice}[1]{\textbf{\large{Exercice #1 :}\xspace}}
%Exercices du contrôle numérotés
\newcounter{exercice}
\NewEnviron{Exercice}{
\refstepcounter{exercice}\textbf{\large{Exercice \theexercice \ :}}\par
\BODY
\bigskip
}

%Exercices non numérotés
\NewEnviron{Exo}[1][]{
\textbf{\large{Exercice #1 \ :}}\par
\BODY
\bigskip
}


%\Leftrightarrow
\newcommand{\Lr}{\Leftrightarrow}

\pagecolor{white}

%Sommaire dans les chapitres
\usepackage{minitoc}

%%%%%%%%%%%%%%%%%%%%%%%%%%%%%%%%%%%%%%%%%%%%%%%%%%%%%%%%%%%%%%%%%%%%%%%%%%%%%%%