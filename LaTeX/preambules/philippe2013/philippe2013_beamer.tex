\usepackage[utf8]{inputenc}
\usepackage[T1]{fontenc}
\usepackage[frenchb]{babel}
\uselanguage{French}
\languagepath{French}

\newcommand{\Cyan}[1]{{\color{cyan} #1}}
\newcommand{\Yellow}[1]{{\color{yellow} #1}}
\newcommand{\Magenta}[1]{{\color{magenta} #1}}
\newcommand{\bleu}[1]{{\color{blue} #1}}
\newcommand{\rouge}[1]{{\color{red} #1}}
\newcommand{\verte}[1]{{\color{green} #1}}
\newcommand{\Purple}[1]{{\color{purple} #1}}
\newcommand{\Orange}[1]{{\color{orange} #1}}
\newcommand{\violet}[1]{{\color{violet} #1}}

\usepackage{mathpazo}
\usepackage[euler-digits]{eulervm} %-> police maths
\usefonttheme{serif}
\usepackage{eurosym,textcomp}

\usepackage{dsfont} %écriture des ensemble N, R, C ...
\newcommand{\C}{\mathds C}
\newcommand{\R}{\mathds R}
\newcommand{\Q}{\mathds Q}
\newcommand{\D}{\mathds D}
\newcommand{\Z}{\mathds Z}
\newcommand{\N}{\mathds N}
\newcommand\Ind{\mathds 1} %= fonction indicatrice
\newcommand\p{\mathds P} %= probabilité
\newcommand\E{\mathds E} % Espérance
\newcommand\V{\mathds V} % Variance

%___________________________
%===    Raccourcis classe
%------------------------------------------------------
\newcommand\seconde{2\up{nde}\xspace}
\newcommand\premiere{1\up{ère}\xspace}
\newcommand\stmg{\bsc{Stmg}}
\newcommand\sti{\bsc{Sti2d}}
%
\usepackage{mathrsfs}   % Police de maths jolie caligraphie
\newcommand{\calig}[1]{\ensuremath{\mathscr{#1}}}
\newcommand\mtc[1]{\ensuremath{\mathcal{#1}}}

\usepackage{tikz,tkz-tab}
\usetikzlibrary{calc,shapes,arrows,plotmarks,lindenmayersystems,decorations,decorations.pathreplacing}

\usepackage[np]{numprint}
\setlength{\parindent}{0pt}

\usetheme{Copenhagen}
\usecolortheme{orchid}
\useoutertheme{shadow,tree}

\newtheorem{Prop}[theorem]{Propriété}
\newtheorem{Rmq}[theorem]{Remarque}
\newtheorem{Rmqs}[theorem]{Remarques}

%Numérotation des sections
\renewcommand{\thesection}{\Roman{section} -}
\renewcommand{\thesubsection}{\Alph{subsection})} %lettres majuscules
\renewcommand{\thesubsubsection}{\arabic{subsubsection}.}

%Permettre la même numérotation beamer
\newcounter{MonPetitCompteur}
\setbeamertemplate{section in toc}{%
\edef\MaDefTemp{\noexpand\setcounter{MonPetitCompteur}{\inserttocsectionnumber}}\MaDefTemp%
\Roman{MonPetitCompteur} - \inserttocsection\par}

\setbeamertemplate{subsection in toc}{%
\leavevmode\leftskip=1.5em%
\edef\MaDefTemp{\noexpand\setcounter{MonPetitCompteur}{\inserttocsubsectionnumber}}\MaDefTemp%
\Alph{MonPetitCompteur}) \inserttocsubsection\par}

\setbeamertemplate{subsubsection in toc}{%
\leavevmode\leftskip=5em%
\edef\MaDefTemp{\noexpand\setcounter{MonPetitCompteur}{\inserttocsubsubsectionnumber}}\MaDefTemp%
\arabic{MonPetitCompteur}. \inserttocsubsubsection\par}

\AtBeginSection{
   \begin{frame}%{\textcolor{red}{\shadowbox{Plan}}}
   \Large%
   \tableofcontents[currentsection,currentsubsection,subsubsectionstyle=hide]
   \end{frame}}

\AtBeginSubsection{
  \begin{frame}%{\textcolor{red}{\shadowbox{Plan}}}
   \Large%
 \tableofcontents[currentsection,currentsubsection,subsubsectionstyle=hide]
   \end{frame}}

\AtBeginSubsubsection{
\begin{frame}%{\textcolor{red}{\shadowbox{Plan}}}
\Large%
\tableofcontents[currentsection,currentsubsection,subsubsectionstyle=show/shaded]
\end{frame}}

\newcommand{\intervalleff}[2]{\left[#1\,;#2\right]}
\newcommand{\intervallefo}[2]{\left[#1\,;#2\right[}
\newcommand{\intervalleof}[2]{\left]#1\,;#2\right]}
\newcommand{\intervalleoo}[2]{\left]#1\,;#2\right[}

%___________________________
%===    Gestion des espaces
%------------------------------------------------------
\newcommand{\pv}{\ensuremath{\: ; \,}}
\newlength{\EspacePV}
\setlength{\EspacePV}{1em plus 0.5em minus 0.5em}
\newcommand{\qq}{\hspace{\EspacePV} ; \hspace{\EspacePV}}
\newcommand{\qetq}{\hspace{\EspacePV} \text{et} \hspace{\EspacePV}}
\newcommand{\qLq}{\hspace{\EspacePV} \Leftrightarrow \hspace{\EspacePV}}
\newcommand{\qRq}{\hspace{\EspacePV} \Rightarrow \hspace{\EspacePV}}
\newcommand{\qLRq}{\hspace{\EspacePV} \Leftrightarrow \hspace{\EspacePV}}

%___________________________
%===    Quelques raccourcis perso
%------------------------------------------------------
\newcommand\pfr[1]{\psframebox[linecolor=red]{#1}}
\newcommand\coef[1][]{c{\oe}fficient#1\xspace}
\newcommand{\vect}[1]{\ensuremath{\overrightarrow{#1}}}
\newcommand\abs[1]{\ensuremath{\left\vert #1 \right\vert}}
\newcommand\Arc[1]{\ensuremath{\wideparen{#1}}}

\setbeamertemplate{navigation symbols}{} 