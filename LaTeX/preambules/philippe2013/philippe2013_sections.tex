%===
%===    Commandes "philippe2013_sections" version du 08.07.2013
%===

%___________________________
%===    Mise en forme des sections
%------------------------------------------------------


\makeatletter
\renewcommand{\thesection}{\color{red}\Roman{section}.\hspace{-0.5em}}
\renewcommand{\section}{%
    \@startsection{section}%
    {1}%
    {0pt}%
    {-15pt \@plus -1ex \@minus -.2ex}%
    {10pt \@plus .2ex \@minus .2ex}%
    {\normalfont\Large\bfseries\color{red}}%
}
\makeatother


\makeatletter
\renewcommand{\thesubsection}{\color{OliveGreen}\Alph{subsection}.\hspace{-0.5em}}
\renewcommand{\subsection}{%
    \@startsection{subsection}%
    {2}%
    {10pt}%
    {-12pt \@plus -1ex \@minus -.2ex}%
    {10pt \@plus .2ex \@minus.2ex}%
    {\normalfont\large\bfseries\color{OliveGreen}}%
}
\makeatother

\makeatletter
\newcounter{sss}[subsection]
\renewcommand{\thesss}{\arabic{sss}.\hspace{-0.5em}}
\renewcommand{\subsubsection}%
            {\@startsection{subsubsection}%
            {3}%
            {1em}%
             {-1ex\@plus -1ex \@minus -.2ex}%
             {0.7ex \@plus .2ex}%
             {\hspace*{20pt}\refstepcounter{sss}\normalfont\bfseries\thesss}}
\makeatother

%\renewcommand{\thechapter}{\Roman{chapter}}
%\renewcommand{\thesection}{\Roman{section}.}
%\renewcommand{\thesubsection}{\Alph{subsection}.} 