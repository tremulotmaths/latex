\usepackage{xargs}
\usepackage{environ}
%___________________________
%===    Redéfinition des marges
%------------------------------------------------------
%



%___________________________
%===    Redéfinition de la commande \chapter{•}
%------------------------------------------------------
%

\makeatletter
\renewcommand{\@makechapterhead}[1]{%
        {\color{blue}
        \begin{flushright}
            \bfseries\fontsize{25}{20}\selectfont\textsc{
            #1}
            
        \end{flushright}
        \vspace{-0.5cm}
            \rule{\linewidth}{1pt}}
       

}

\makeatother
%\makeatletter
%
%\renewcommand{\@makechapterhead}[1]{
%\begin{tikzpicture}
%\node[draw, color=blue,fill=white,rectangle,rounded corners=5pt]{%
%\begin{minipage}{\linewidth}
%\begin{center}
%\vspace*{9pt}
%\textcolor{blue}{\Large \textsc{\textbf{Chapitre \thechapter \ :}}} \par
%\textcolor{blue}{\Large \textsc{\textbf{ \ #1}}}
%\vspace*{7pt}
%\end{center}
%\end{minipage}
%};\end{tikzpicture}
%}

%___________________________
%===    Redéfinition des sections
%------------------------------------------------------
%

        \makeatletter
        \newcommand{\sectioncolor}{blue} %Couleur titre de section
        \newcommand{\ssectioncolor}{MidnightBlue} %Couleur titre de sous-section
        \newcommand{\sssectioncolor}{RoyalBlue} %Couleur titre de sous-sous-section
        %
        %Coloration des titres
        %---------------------
        \renewcommand{\section}{%Commande définie dans le fichier article.cls
            \@startsection%
            {section}%
            {1}%
            {0pt}%
            {-3.5ex plus -1ex minus -.2ex}%
            {2.3ex plus.2ex}%
            {\color{\sectioncolor}\normalfont\Large\bfseries}} %Aspect du titre
        \renewcommand\subsection{%
            \@startsection{subsection}{2}
            {0.5cm}% %décalage horizontal
            {-3.5ex\@plus -1ex \@minus -.2ex}%
            {1ex \@plus .2ex}%
            {\color{\ssectioncolor}\normalfont\large\bfseries}}
        \renewcommand\subsubsection{%
            \@startsection{subsubsection}{3}
            {1cm}% %décalage horizontal
            {-3.25ex\@plus -1ex \@minus -.2ex}%
            {1ex \@plus .2ex}%
            {\color{\sssectioncolor}\normalfont\normalsize\bfseries}}
        %

\setcounter{secnumdepth}{3}


 %___________________________
%===    Redéfinition des numérotation des paragraphes
%------------------------------------------------------
%
\renewcommand\thesection{\Roman{section}}
\renewcommand\thesubsection{\arabic{subsection}.}
\renewcommand\thesubsubsection{\alph{subsubsection}.}
%

%_________________________
%===    Environnements de cours
%------------------------------------------------------



%___________________________
%===    Définitions
%------------------------------------------------------

\newenvironment{Defi}[2][]{%
\hspace*{-0.25cm}
\begin{tikzpicture}
\path[left color=lightgray] (0,0) arc[radius=0.25, start angle =270, end angle =90] -- (10,0.5) |- (0,0);
\draw (0,0.25) node[right=-4pt] {\bfseries\color{red} Définition#1 : \textit{\textmd{#2}}};
\end{tikzpicture}\nopagebreak[4]\par  \hspace*{-0.25cm} {\color{red}$\ulcorner$} \par \vspace{-0.25cm} }
{\hspace{\linewidth} {\color{red}  $\lrcorner$ }\par}


%%\newcounter{defi}[chapter]
%\NewEnviron{Defi}[2][]%
%{\medskip
%\begin{tikzpicture}
%\node[draw,color=red,fill=white,rectangle,rounded corners=5pt ]
%{\black\parbox{\linewidth}{\textcolor{red}{Définition#1 :} \textit{#2}\par \BODY}};
%\end{tikzpicture}
%}%

%____________________
%===    Propriétés
%------------------------------------------------------
%\newcounter{propri}[chapter]
\newenvironment{Prop}[2][]{%
%\refstepcounter{propri}
\hspace*{-0.25cm}
\begin{tikzpicture}
\path[left color=lightgray] (0,0) arc[radius=0.25, start angle =270, end angle =90] -- (10,0.5) |- (0,0);
\draw (0,0.25) node[right=-4pt] {\bfseries\color{red} Propriété#1 : \textit{\textmd{#2}}};
\end{tikzpicture}\nopagebreak[4]\par  \hspace*{-0.25cm} {\color{red}$\ulcorner$} \par \vspace{-0.25cm} }
{\hspace{\linewidth} {\color{red}  $\lrcorner$ }\par}




%___________________________
%===    Théorèmes
%------------------------------------------------------
%\newcounter{thm}[chapter]
\newenvironment{Thm}[2][]{%
%\refstepcounter{thm}
\hspace*{-0.25cm}
\begin{tikzpicture}
\path[left color=lightgray] (0,0) arc[radius=0.25, start angle =270, end angle =90] -- (10,0.5) |- (0,0);
\draw (0,0.25) node[right=-4pt] {\bfseries\color{red} Théorème#1 : \textit{\textmd{#2}}};
\end{tikzpicture}\nopagebreak[4]\par  \hspace*{-0.25cm} {\color{red}$\ulcorner$} \par \vspace{-0.25cm} }
{\hspace{\linewidth} {\color{red}  $\lrcorner$ }\par}

%___________________________
%===    Démonstration
%------------------------------------------------------
\NewEnviron{Demo}[1][]%
{\begin{tikzpicture}
\node[fill=gray!10,rounded corners=2pt,anchor=south west] (illus) at (0,0)
{\hfill \textbf{\textcolor{ForestGreen!50}{Démonstration#1}}};
\node[fill=gray!10,rounded corners=2pt,anchor=north west]at(0,0)
{\parbox{\linewidth}{\BODY \par 
\hfill$\square$}};
\end{tikzpicture}
\medskip
}

%___________________________
%===    Exemples
%------------------------------------------------------
\newcounter{exemple}[chapter]
\NewEnviron{Exemple}%
{
\refstepcounter{exemple}
\psframebox[fillstyle=solid,fillcolor=Yellow!30,linewidth=0.4pt,linecolor=Yellow!30,linearc=0.05,cornersize=absolute]{

    \begin{minipage}{\linewidth}
\textit{Exemple~\theexemple~:} \par 
\BODY
\end{minipage}
\medskip
}
}



\NewEnviron{Exemple*}[1][]%
{
\psframebox[fillstyle=solid,fillcolor=Yellow!30,linewidth=0.4pt,linecolor=Yellow!30,linearc=0.05,cornersize=absolute]{

    \begin{minipage}{\linewidth}
\textit{Exemple#1:} \par 
\BODY
\end{minipage}
\medskip
}
}

%___________________________
%===    Méthodes
%------------------------------------
\NewEnviron{Methode}[1] []%
{\begin{bclogo}[noborder=true, arrondi = 0.1, logo = \bccrayon, barre = snake, tailleOndu=2,marge=0]{\normalsize Méthode#1}
   \BODY
\end{bclogo}
\medskip
}%

%___________________________
%===    Remarques
%------------------------------------------------------

\NewEnviron{Rmq}[1] []%
{\begin{bclogo}[noborder=true, arrondi = 0.1, logo = , barre = snake, tailleOndu = 2 ,marge=0]{\normalsize Remarque#1 :}
   \BODY
\end{bclogo}
\medskip
}%

%___________________________
%===   Exercices
%------------------------------------------------------
\NewEnviron{Exo}[1][]
{\textbf{Exercice~#1 :} \par
\BODY
\medskip
}
\newcounter{exos}[chapter]
\NewEnviron{Exercice}[1][]
{
\refstepcounter{exos}
\textbf{Exercice~\theexos :} ~#1 \par
\BODY
\medskip
}

%===   Commandes
%------------------------------------------------------

\newcommand{\Fiche}[2]{%
\begin{tikzpicture}
	\node[draw, color=blue,fill=white,rectangle,rounded corners=5pt]{%
	\begin{minipage}{\linewidth}
		\begin{center}
			\vspace*{9pt}
			\textcolor{blue}{\Large \textsc{\textbf{Fiche~#1 :}}}\par
			\textcolor{blue}{\Large \textsc{\textbf{#2}}}
			\vspace*{7pt}
		\end{center}
	\end{minipage}
	};
\end{tikzpicture}
}%

\newcommand{\Livre}[1]{%
\begin{bclogo}[noborder=true, arrondi = 0.1, logo =\bcbook , barre = none , tailleOndu = 2 ,marge=0]{\normalsize Dans le livre :}
   #1
\end{bclogo}
\medskip
}%

%Dans un repère
\newcommand{\Dsrepere}[1]{%
\begin{bclogo}[noborder=true, arrondi = 0.1, logo =\bccrayon , barre = line , tailleOndu = 0 ,marge=0]{\normalsize Dans un repère :}
   #1
\end{bclogo}
\medskip
}%

\newcommand\voc[1]{\textbf{#1}\xspace}

%\dominitoc %pour pouvoir créer un sommaire du chapitre en cours avec \minitoc

%produit saclaire 
\newcommand{\Pdtscalaire}[2]{$\vect{#1} \cdot \vect{#2}$}