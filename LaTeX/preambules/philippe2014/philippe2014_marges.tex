%___________________________
%===    Définition des dimensions de la page
%------------------------------------------------------

\setlength\paperheight{297mm}
\setlength\paperwidth{210mm}
\setlength\parindent{0mm} % longueur de l'alinea
\setlength{\evensidemargin}{1cm}% Marge gauche sur pages paires
\setlength{\oddsidemargin}{1cm}% Marge gauche sur pages impaires
\setlength{\marginparwidth}{2.65cm} % taille réservé pour les notes de marges
\setlength{\marginparsep}{0.5cm} % espace entre les notes de marge et le texte
\setlength{\headsep}{2.5pt}% Entre le haut de page et le texte
\setlength{\headheight}{15pt}% Haut de page
\setlength{\textheight}{26.75cm}% Hauteur de la zone de texte
\setlength{\textwidth}{14.5cm}% Largeur de la zone de texte


%___________________________
%===    Rectangles colorés dans la marge
%------------------------------------------------------

% Paramétrage rectangles et texte (zone modifiable)
\newcommand{\RWidth}{4cm} % largeur rectangle
\newcommand{\RHOffset}{-5mm} % Position par rapport à la marge
                            % (Si négatif, on est en pleine page)
\newcommand{\RColor}{\MaCouleur!25} % couleur rectangle
\newcommand{\TColor}{black} % couleur texte
\newcommand{\vcorrection}{2.8cm} % décalage verticale texte
\newcommand{\hcorrection}{0cm} % décalage horizontale texte

% Zone à ne pas modifier (normalement)
\newcommand\RectangleDroite[2]{%
 % Rectangle à droite pour les pages impaires
 \leavevmode%
 \hbox to0pt{\hss#1}%
 \setbox0=\hbox{%
   \hsize=0pt
   \vbox to0pt{%
     \vss
     \hbox to0pt{%
       \color{\RColor}%
       % Pour le rectangle de droite, le calcul du décalage est un
       % poil lourd !
       \hspace*{%
         \dimexpr
           \paperwidth-\textwidth-\oddsidemargin-1in-\hoffset
           -\RWidth-\RHOffset
       }%
       % Rectangle
       \vrule width\RWidth
              height\dimexpr\headheight+\topmargin+1in+\voffset+5mm\relax
              depth\dimexpr \paperheight +1cm\relax
       % On replace le point de référence horizontalement
       \dimen0=\RWidth
       \dimen0=-0.7\dimen0
       \advance\dimen0 \hcorrection
       \vrule width\dimen0
              height0pt
       % Texte
       \color{\TColor}%
       \hbox to0pt{%
         \hss
         \raisebox{\dimexpr-0.5\paperheight+\vcorrection}{%
           \vtop to0pt{%
             \vss
             \rotatebox{-90}{\Large\bfseries #2}%
             \vss
           }%
         }%
       }%
       \hss
     }%
   }%
 }%
 \dp0=0pt
 \ht0=0pt
 \box0
}
\newcommand\RectangleGauche[2]{%
 % Rectangle à gauche pour les pages paires
 \leavevmode%
 \hbox to0pt{#1\hss}%
 \setbox0=\hbox{%
   \hsize=0pt
   \vbox to0pt{%
     \vss
     \hbox to0pt{%
       \hss
       \color{\RColor}%
       \vrule width\RWidth
              height\dimexpr\headheight+\topmargin+1in+\voffset+5mm\relax
              depth\dimexpr \paperheight +1cm\relax
       % Texte
       \color{\TColor}%
       \hbox to0pt{%
         % On replace le point de référence horizontalement
         \dimen0=\RWidth
         \dimen0=-0.5\dimen0
         \advance\dimen0 -\hcorrection
         \kern\dimen0
         \raisebox{\dimexpr-0.5\paperheight+\vcorrection}{%
           \vtop to0pt{%
             \vss
             \rotatebox{90}{\Large\bfseries #2}%
             \vss
           }%
         }%
         \hss
       }%
       % Pour le rectangle de gauche, le calcul du décalage est un
       % peu plus court !
       \hspace*{%
         \dimexpr
           \evensidemargin+1in+\hoffset-\RWidth-\RHOffset
       }%
     }%
   }%
 }%
 \dp0=0pt
 \ht0=0pt
 \box0
}

\fancypagestyle{plain}{%
 \fancyhf{}
 \fancyfoot[C]{}
 %\fancyhead[RO]{\RectangleDroite{}{}}
 \fancyhead[LO]{\RectangleGauche{}{}} % Normalement inutile mais bon...
 \renewcommand\headrulewidth{0pt}
}
\fancypagestyle{normal}{%
 \fancyhead[C]{{\color{\MaCouleur} \textbullet~\leftmark~\textbullet}}
 \fancyfoot[C]{\tikz{\draw[\MaCouleur,fill=\MaCouleur!25] (0,0) circle (10pt) node{\footnotesize \thepage};}}
 %\fancyhead[RO]{\RectangleDroite{}{}}
% \fancyhead[LE]{}
% \fancyhead[RE]{}
 \fancyhead[LO]{\RectangleGauche{}{}}
 \renewcommand\headrulewidth{0pt}
}
\pagestyle{normal} % style par défaut

%------------------------------------------------------------------------------ 