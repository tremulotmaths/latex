%Les couleurs

\newcommand{\rouge}[1]{{\color{red} #1}}
\definecolor{midblue}{rgb}{0.145,0.490,0.882} %Couleur Philippe 


%
\renewcommand{\headrulewidth}{0pt}% pas de trait en entête
\newcommand\RegleEntete[1][0.4pt]{\renewcommand{\headrulewidth}{#1}}%commande pour ajouter un trait horizontal en entête

\renewcommand{\footrulewidth}{0pt}
\newcommand\ReglePied[1][0.5pt]{\renewcommand{\footrulewidth}{#1}}%commande pour ajouter un trait horizontal en pied de page

\setlength{\columnseprule}{0pt}
\newcommand\RegleColonne[1][0.4pt]{\setlength{\columnseprule}{#1}}%commande pour ajouter un trait vertical entre les colonnes

\newcommand{\entete}[3]{\lhead{#1} \chead{#2} \rhead{#3}}
\newcommand{\pieddepage}[3]{\lfoot{#1} \cfoot{#2} \rfoot{#3}}

%___________________________
%===    Raccourcis classe
%------------------------------------------------------
\newcommand\seconde{2\up{nde}\xspace}
\newcommand\premiere{1\up{ère}\xspace}
\newcommand\terminale{T\up{le}\xspace}
\newcommand\stmg{\bsc{Stmg}}
\newcommand\sti{\bsc{Sti2d}}
\newcommand\bat{BAT 1\xspace}
\newcommand\BAT{BAT 2\xspace}
\newcommand\tesspe{TES Spécialité\xspace}


% MATH

%Intervalles

\newcommand{\intervalleff}[2]{\left[#1\,;#2\right]}
\newcommand{\intervallefo}[2]{\left[#1\,;#2\right[}
\newcommand{\intervalleof}[2]{\left]#1\,;#2\right]}
\newcommand{\intervalleoo}[2]{\left]#1\,;#2\right[}

%Ensembles 

\newcommand{\C}{\mathds C}
\newcommand{\R}{\mathds R}
\newcommand{\Q}{\mathds Q}
\newcommand{\D}{\mathds D}
\newcommand{\Z}{\mathds Z}
\newcommand{\N}{\mathds N}

%fonction exponentielle
\newcommand{\e}{\text{e}}

%Intégrale d droit
\newcommand{\dd}{\,\text{d}}

%Nombres complexes
\newcommand{\ii}{\,\text{i}}

% Raccourcis pour police pour écriture des noms de fonctions, ...

\newcommand{\calig}[1]{\ensuremath{\mathscr{#1}}}
\newcommand\mtc[1]{\ensuremath{\mathcal{#1}}}

%Gestion des espaces
%
\newcommand{\pv}{\ensuremath{\: ; \,}}

\newlength{\EspacePV}
\setlength{\EspacePV}{1em plus 0.5em minus 0.5em}

\newcommand{\qq}{\hspace{\EspacePV} ; \hspace{\EspacePV}}

\newcommand{\qetq}{\hspace{\EspacePV} \text{et} \hspace{\EspacePV}}

\newcommand{\qouq}{\hspace{\EspacePV} \text{ou} \hspace{\EspacePV}}

\newcommand{\qLq}{\hspace{\EspacePV} \Leftarrow \hspace{\EspacePV}}

\newcommand{\qRq}{\hspace{\EspacePV} \Rightarrow \hspace{\EspacePV}}

\newcommand{\qLRq}{\hspace{\EspacePV} \Leftrightarrow \hspace{\EspacePV}}

%simplification notation norme \norme{}
\newcommand{\norme}[1]{\ensuremath{\left\lVert #1\right\rVert}}

%simplification de la notation de vecteur \vect{}
\newcommand{\vect}[1]{\ensuremath{\overrightarrow{#1}}}

%Repères
\newcommand\Oij{\ensuremath{\left(O\pv\vect{\imath}\ ,\ \vect{\jmath}\right)}}

\newcommand\Oijk{\ensuremath{\left(O\pv\vect{\imath}\ ,\ \vect{\jmath}\ ,\ \vect{k}\right)}}

\newcommand\Ouv{\ensuremath{\left(O\pv\vect{u}\ ,\ \vect{v}\right)}}

\newcommand\OIJ{\ensuremath{\left(O\pv I\ ,\ J\right)}}


% valeur absolue
\newcommand\abs[1]{\ensuremath{\left\lvert #1 \right\rvert}}%valeur absolue

%Arc
\newcommand\Arc[1]{\ensuremath{\wideparen{#1}}}%arc de cercle

%symbole pour variable aléatoire qui suit une loi
\newcommand{\suit}{\hookrightarrow} %commande beurk Dom 

%Tableaux
%colonnes centrées verticalement et horizontalement permettant d'écrire des paragraphes de largeur fixée du type M{3cm}
\newcolumntype{M}[1]{>{\centering\arraybackslash}m{#1}}%cellule centrée horizontalement et verticalement
%\arraybackslash permet de continuer à utiliser \\ pour le changement de ligne

%Pointillés sur toute la ligne
\newcommand{\Pointilles}[1][1]{%
\multido{}{#1}{\makebox[\linewidth]{\dotfill}\\[1.5\parskip]
}}
%commandes : \Pointilles ou \Pointilles[4] pour 4 lignes

%Raccourcis 

\newcommand\pfr[1]{\psframebox[linecolor=red]{#1}}
\newcommand\coef[1][]{c{\oe}fficient#1\xspace}

%Rond entourant une lettre avec pour arguments la couleur de fond, puis la lettre
\newcommand\rond[2][red!20]{\tikz[baseline]{\node[fill=#1,anchor=base,circle]{\bf #2};}} 


%Ecrire card en écriture normale :
\newcommand{\card}{\text{card}\xspace}